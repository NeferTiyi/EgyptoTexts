% This file was converted to LaTeX by Writer2LaTeX ver. 1.0.2
% see http://writer2latex.sourceforge.net for more info
\documentclass[a4paper]{article}
\usepackage[utf8]{inputenc}
\usepackage[T3,T1]{fontenc}
\usepackage[french]{babel}
\usepackage[noenc]{tipa}
\usepackage{tipx}
\usepackage[geometry,weather,misc,clock]{ifsym}
\usepackage{pifont}
\usepackage{eurosym}
\usepackage{amsmath}
\usepackage{wasysym}
\usepackage{amssymb,amsfonts,textcomp}
\usepackage{color}
\usepackage{array}
\usepackage{supertabular}
\usepackage{hhline}
\usepackage{hyperref}
\hypersetup{pdftex, colorlinks=true, linkcolor=blue, citecolor=blue, 
            filecolor=blue, urlcolor=blue, pdftitle=, pdfauthor=, 
            pdfsubject=, pdfkeywords=}
\usepackage[pdftex]{graphicx}
% Page layout (geometry)
\setlength\voffset{-1in}
\setlength\hoffset{-1in}
\setlength\topmargin{2cm}
\setlength\oddsidemargin{2cm}
\setlength\textheight{25.7cm}
\setlength\textwidth{17.001cm}
\setlength\footskip{0.0cm}
\setlength\headheight{0cm}
\setlength\headsep{0cm}
% Footnote rule
\setlength{\skip\footins}{0.119cm}
\renewcommand\footnoterule{%
  \vspace*{-0.018cm}%
  \setlength\leftskip{0pt}%
  \setlength\rightskip{0pt plus 1fil}%
  \noindent\textcolor{black}{\rule{0.25\columnwidth}{0.018cm}}%
  \vspace*{0.101cm}%
}
% Pages styles
\makeatletter
\newcommand\ps@Standard{
  \renewcommand\@oddhead{}
  \renewcommand\@evenhead{}
  \renewcommand\@oddfoot{}
  \renewcommand\@evenfoot{}
  \renewcommand\thepage{\arabic{page}}
}
\makeatother
\pagestyle{Standard}
\setlength\tabcolsep{1mm}
\renewcommand\arraystretch{1.3}
% Non-floating captions
\makeatletter
\newcommand\captionof[1]{\def\@captype{#1}\caption}
\makeatother

\title{L’image du roi en \kmt et Mésopotamie}
\author{Evelyne Faivre-Martin}
\date{15 mai - 12 juin 2012}

\begin{document}
%%%%%%%%%%%%%%%%%%%%%%%%%%%%%%%%%%%%%%%%%%%%%%%%%%%%%%%%%%%%%%%%%%%%%%%%

\maketitle

\part{Conférence \no1}

\chapter{introduction}


Aujourd'hui, nous allons poser les choses, c'est à dire voir comment 
à partir d'une certaine idéologie royale, on voit apparaître un type 
d'images. Sans approfondir,  il faut quand même marquer les grandes 
caractéristiques de la royauté mésopotamienne, et les grandes 
caractéristiques de la royauté égyptienne.

À partir du deuxième cours, nous nous intéresserons à l'apparition 
de l'iconographie royale et nous ne traiterons que de la fin du IV\ieme 
millénaire pour croiser certains objets, les regarder en chronologie 
la plus juste possible.
Entre Uruk et Nagada, on peut s'aligner et on pourra ainsi constater 
que des points communs sont ainsi à remarquer, mais très tôt également 
que des caractéristiques spécifiques à chacune de ces deux civilisations 
s'affirment et finalement dès les premières images qui sont parfois 
extrêmement succinctes.

On s'intéressera ensuite, durant les trois séances suivantes, à la 
thématique liées à la relation privilégiée du souverain avec son dieu, 
et là encore il faut toujours s'interroger sur sa nature. Ce lien 
privilégié qui fait de ce roi un constructeur, et nous verrons donc la 
mise en image de ce roi constructeur.

Nous verrons également comment il est administrateur des biens du 
dieu, et de ce fait un défenseur de ces biens. Et cela nous conduira 
à envisager l'image du face à face entre le roi et le dieu, l'image 
de l'action en faveur du dieu, et l'image du conflit au nom du dieu.

Et tout cela fait donc à peu près quatre millénaires d'images, sur ces 
civilisations assez vastes ; aussi il faudra se limiter et ne prendre 
que quelques exemples fondamentaux et faire une lecture plus posée de 
certaines images absolument fondamentales.

Des  images du roi en deux ou trois dimensions, et la littérature qui
elle aussi servira de support.

Nous avons donc deux territoires vastes, qui se sont développés de 
manière parallèle, et qui pour le moment ne semblent pas avoir été 
tellement en contact les uns avec les autres. De ces contacts, on ne 
sait trop comment cela se passait et on a donc toujours considéré que 
le lien était le monde palestinien, mais ce n'est pas si évident car 
finalement on a une connaissance assez floue de l'archéologie de la 
péninsule arabe et de la possibilité de la circulation par bateau via 
le golfe persique.

On a certes des idées et l'on sait qu'il y a eu des points de contact 
possibles avec le monde phénicien, à partir du moment où ces deux 
civilisations s'y sont tournés. Le monde mésopotamien vers l'est pour 
y trouver du bois et des marchandises (ils n'ont rien) et l'\kmt pour 
avoir certaines matières dont l'étain.

Seulement les recherches récentes dans le monde mésopotamien montrent
que ces gens se tournaient également de l'autre côté. On va donc partir 
du principe d'un développement par l'est.

Nous sommes donc dans deux civilisations où l'on voit apparaître les 
deux plus anciens systèmes d'écriture de l'histoire de l'humanité et 
on constate dans ces deux civilisations  qu'il y a apparition de 
l'écriture et apparition de la royauté.

Effectivement, en \kmt nous aurons une unification partant de la 
\HE et se diffusant vers le nord, avec toute cette inconnue que 
représente l'annexion de la \BE, et ce au tournant des IV\ieme et 
III\ieme millénaires. Et on arrive finalement à un royaume unifié, 
organisé autour du Nil, et dirigé par un seul roi (sauf durant les 
périodes troublées que l'on appelle périodes intermédiaires). Nous 
n'étudierons pas ces périodes, on se limite aux périodes où l'\kmt 
n'a qu'un seul roi.

La Mésopotamie est un pays qui a deux fleuves (Tigre et Euphrate) et 
une géographie très différente. Elle voit apparaître une royauté au 
IV\ieme millénaire, mais c'est un monde morcelé, et va fonctionner 
dans un premier temps sur la base des cités-états, ce que l'on pourrait 
appeler des principautés (une ville autour d'un micro-royaume). On peut 
constater qu'avec l'apparition du phénomène de guerre (qui n'est pas 
attesté avant 2600\,BC), ces cités entrent en conflit et forment des 
royaumes un peu plus importants en s'absorbant et de là on arrivera 
finalement, aux II\ieme et I\ier millénaires, à ce que le monde 
mésopotamien fonctionne en deux royaumes, au centre et au Sud, Babylone, 
et au nord, le royaume assyrien.

Cela veut dire que même quand nous aurons le grand royaume de Babylone, 
au nord, il existe toujours le royaume assyrien, il n'y a jamais eu 
dans le monde mésopotamien, une seule unité politique.

Mais ce qui est intéressant c'est que ces mésopotamiens fonctionnent 
de la même façon. Certes il existe des particularités entre Babylone 
et l'Assyrie, on ne peut pas dire que le principe de la souveraineté 
soit le même entre Babylone et Assur, car les babyloniens sont 
intellectuellement supérieurs par rapport aux assyriens (les 
babyloniens ont repris à leur compte la tradition culturelle 
sumérienne, et c'est une lacune que les assyriens n'arriveront jamais 
à combler).

Donc déjà là nous avons quand même deux principes qui s'affirment très 
tôt, et on constate que le monde égyptien, comme le monde mésopotamien, 
sont des sociétés basées sur des croyances polythéistes, des religions 
appelées préhistoriques etque dans ces deux systèmes les rois dirigent 
au nom des dieux.

Certes sur quatre millénaires nous aurons des variantes religieuses, 
(par exemple dans le monde mésopotamien en fonction des périodes où des 
endroits où l'on se trouve on ne désigne pas les choses exactement de 
la même façon, c'est à dire que les assyriens par exemple considèrent 
que les rois sont leur dieu Asur, et que celui que nous appelons roi 
est son représentant sur terre. Il y a en effet cette ambiguïté dans le 
monde de la Mésopotamie, car certains textes parlant d'un dieu utilise 
le mot roi.

\section{Chronologie comparée}

Il faut mettre en regard les grandes périodes (elles sont données un peu
à la louche), et il faut se référer à Nagada qui est contemporain
d'Uruk , et la période Thinite qui correspond à la
période d'Uruk final et le début des dynasties
archaïques, dynasties qui seront finalement contemporaines à
l'Ancien Empire :}


{}-  l'empire d'Akad est finalement
contemporain de la fin de la V et VI dynastie}


{}- la période néo sumérienne est contemporaine de la période
intermédiaire}


{}- le Moyen Empire est contemporain à la première dynastie de Babylone 
et à cette époque les assyriens forment un royaume paysan renfermé sur
lui même}


et il y a un moment où il n'y a rien dans le monde
mésopotamien, alors que l'\kmt est à
l'apogée de sa civilisation, c'est à
dire une période où le Pharaon est fort et période de grandes
production d'images}


Cette période en Mésopotamie correspond à ce que les historiens
appelaient autrefois en référence au monde grec,  les siècles obscurs
du monde mésopotamien, pour bien monter qu'à
l'époque où la Mésopotamie est dominée par les
Kassites, paradoxalement c'est une époque importante
au niveau culturel, copie de textes, archivage des textes,
transcription de documents, et c'est donc une époque
fondamentale pour les linguistes, mais c'est une
royauté fermée sur elle même, qui ne produit aucune image, aucun objet}


On peut utiliser le terme de bronze récent pour cette époque et cela
correspond à l'époque de la richesse de toutes les
cités états de la  Syrie Palestine}


Et contrairement au moment où le monde mésopotamien va reprendre de
l'ampleur avec les grands empires conquérants néo
sumériens et néo babyloniens, cela correspond à une période
intermédiaire en \kmt de la basse époque, où il y a un répertoire
artistique intéressant mais qui évolue de façon particulière}


Finalement, nous aurons vraiment un développement parallèle entre ces
deux mondes jusqu'à la fin du bronze moyen au XVI
siècle avant JC}


Bronze Ancien, Bronze moyen, nous avons les deux civilisations qui se
développent un peu à la même vitesse, et ensuite on a ce balancier.}


\textbf{LE MONDE MESOPOTAMIEN}}


C'est un monde qui d'une part
correspond à une archéologie récente et très vaste. Mais attention ce
monde ne correspond pas uniquement à l'iraq et il faut
regarder sur une carte qui présente la géographie actuelle et on voit
que la partie haute Euphrate qui est un territoire syrien actuellement,
et se référer aux cours des deux fleuves, où les mésopotamiens se sont
installés  et il y a des sites anciens dans l'actuel
Turquie. Pour l'Iran, il n'y  a pas
grand chose, car il y avait la limite naturelle formée par la montagne 
ZAGROS. (de toute façon l'Iran était une zone paria
dans le monde antique (à vérifier)}


On peut dire aussi que les sites mésopotamiens sont vraiment liés aux
zones des fleuves}


Attention, dans l'antiquité la côte entrait davantage à
l'intérieur des terres ,les deux fleuves se jetaient
indépendamment dans le golfe persique, et donc UR était un ensemble
portuaire important et qu'au IV et début du III
millénaire le sud a été fondamental dans le développement de cette
région. }


C'est un monde sédentaire et un monde de nomades. Les
historiens ont toujours voulu opposer deux aspects culturels et
retrouver d'ailleurs dans certains royaumes ou dans
certaines caractéristiques des spécificités liées aux deux mondes. On
développe encore l'idée du sumérien plus intellectuel
et le fait que si dans le monde sumérien des conflits se développèrent
vers - 2600, c'est au moment où beaucoup de
populations nomades , sémites, venaient de s'installer
, }


Dans le monde mésopotamien, que nous dit-on  à propos de la royauté ?
Les textes nous donnent des informations avec d'une
part un texte que l'on appelle : \textbf{la liste
royale sumérienne, }dont on a différentes copies dans différents musées
du monde, (copie sur tablette, sur de l'argile, qui
pouvait être aussi un objet de fondation)}


La liste royale sumérienne est un texte qui démarre , mis en place dans
le courant du III millénaire, et raconte l'histoire de
la partie sud du monde mésopotamien, depuis les origines de
l'humanité, sachant que pour le monde mésopotamien,
dans le monde de la création, il y avait un premier temps, qui était le
temps parfait, le temps des dieux. A ce moment les dieux étaient
divisés en deux groupes : les dieux suprêmes et les dieux de rang
inférieur qui serviront les grands dieux jusqu'au jour
où ils se révoltent et refusent de continuer à servir les grands dieux.
Là il y a différentes versions selon les endroits et les époques et ces
versions deviennent sanglantes dans les sources babyloniennes du II
millénaire, avec le meneur qui sera mis à mort.}


Dans les sources du III millénaire , c'est un peu
différent et c'est là que les grands dieux demandent à
la déesse mère de prendre de l'argile et de fabriquer
sur son tour de potier une créature à leur image, mais une créature sur
laquelle on aura une prise, cette créature c'est
l'homme dans le sens humanité (il n'y
a pas de premier homme ou de première femme ) c'est
l'humain qui est ainsi créé et le récit dit que cet
humain, on lui perce le bras et le sang coule, contrairement à une
divinité mais surtout les dieux garderont pour eux
l'immortalité et laisseront la mort aux humains}


Et c'est dans ce contexte de récit que
l'on apprend qu'il y aura un
évènement fondamental, dans cette histoire de
l'humanité c'est la descente de la
royauté sur terre, que l'on attribue au dieu ENLIL,
(seigneur du vent)}


En en sumérien veut dire maitre seigneur, et IL veut dire vent, air, }


ENLIL est donc un dieu associé à l'air,
l'espace, au vent, et dans le panthéon mésopotamien il
est le numéro 2, après le dieu du ciel}


on attribue à ENLIL une descendance, mais surtout c'est
lui qui gère les affaires de la terre}


Donc ENLIL est le créateur de la royauté, car il constata que cette
humanité était absolument incontrable et qu'il fallait
qu'elle soit dominée et il inventa la royauté
qu'il lança du ciel sur la tête des hommes, afin
d'organiser cette création. }


L'humanité a toujours fondamentalement agacée ENLIL, 
et il va même les punir}


Dans le récit de ENLIL dans l'EKUR à Nippur qui est son
grand temple :}


\includegraphics[width=13.965cm,height=3.699cm]{FaivreMartin5conf-img/FaivreMartin5conf-img1.jpg}



C'est la résidence terrestre de ce dieu, où il avait un
grand temple dans la ville de Nippur, et on a ce \textit{passage qui
nous dit :}}


\textit{{\textquotedbl}Sans ENLIL de la grande montagne, aucune ville
n'aurait été construite, aucun habitat
n'aurait été érigé, aucun enclos habitable,
n'aurait été construit, aucune bergerie
n'aurait été établie, aucun roi
n'aurait été élevé, aucun seigneur ne serait  né aucun
grand prêtre, aucune grande prêtresse n'accomplirait
le culte} {\textquotedbl} }


On voit ainsi une référence rurale permanente, ils commencent à parler
du bétail avant les hommes et cela traduit un ordre de pensée. Il est
évident que dans un texte égyptien, on ne classerait pas les choses
dans cet ordre là.}


Les mésopotamiens sont des gens très pragmatique et ils sont toujours
très proches des contingences  terre à terre, liées à leur économie,
qui est une économie agricole.}


Donc sans ENLIL, aucun roi n'aurait été élevé}


Cette humanité braillarde agace ENLIL, qui depuis que
l'homme est sur terre l'empêche de
dormir , il ne trouve plus le sommeil à cause du bruit de
l'humanité; alors il décide d'un
geste d'agacement d'envoyer le déluge
et nous aurons ainsi le deuxième temps}


Et dans la \textbf{liste royale sumérienne,} on nous parle des rois
{\textquotedbl} ceux d'avant le déluge{\textquotedbl}
et il y ensuite ceux {\textquotedbl} d'après le déluge
{\textquotedbl}}


et cela veut dire qu'après le déluge
c'est le moment d'entrer dans les
temps historiques, et on nous dit qu'ENLIL ré envoie
la royauté sur terre, c'est à dire que nous avons une
deuxième descente, nouvelle descente de la royauté sur terre}

\includegraphics[width=3.457cm,height=6.491cm]{FaivreMartin5conf-img/FaivreMartin5conf-img2}
\captionof{figure}[liste sumérienne]{liste sumérienne}



Cette idée est assez intéressante et il faut garder à
l'esprit cette division : rois des temps anciens, et
rois de l'époque historique}


On peut même se demander si cela ne pourrait par être lié aux rois avant
l'écriture et aux rois postérieurs à
l'écriture}


dans l'esprit de la culture sumérienne ce
qu'il présente comme étant mythologie, ce texte du
déluge, ait un fondement finalement par rapport aux rois des temps
anciens, que l'on a oublié, dont les lignées se sont
éteintes, par rapport à de nouvelles royautés qui se sont mises en
place dès le III millénaire}


La royauté est un cadeau des dieux aux hommes,  C'est
donc à la fin du IV millénaire que l'on va voir
apparaître les premières représentations de roi  et les rois seront
présents dans l'art jusqu'à la fin de
la civilisation mésopotamienne}


En général on considère la fin du monde mésopotamien en - 539, car il
est complètement chamboulé par l'invasion perse, et
les perses ont une mentalité tout à fait différentes, un rapport au
divin tout à fait différent et donc un rapport à la souveraineté tout
aussi différent}


Donc on n'intègre pas les perses dans
l'histoire de la Mésopotamie ancienne et on
s'arrête donc à la prise de Babylone en - 539..}


Ce qui est notable, c'est la représentation par
allusion, qui est extrêmement fréquente, comme nous pourrons le
constater : comment magnifier le souverain sans avoir à la représenter
? : il suffit de développer certains thèmes et on sait très bien que
tout cela  a été initié par le roi. }


Après, il faut voir à quoi servent ces images}


\textbf{Epoque d'URUK} : }


C'est la période de formation, et correspond grosso
modo à la période de Nagada en \kmt}


Les dynasties archaïques , ce sont ces fameux rois après le déluge,
c'est à dire le moment où nous avons
l'écriture, donc le moment où nous avons des textes,
et où dans les textes nous avons un mot qui désigne le roi et des
images associées à ce mot et cela signifie que pour la première fois un
homme, la représentation d'un humain, qui aurait à
côté de lui le mot que l'on sait être le titre
correspondant au mot roi et nous pourrons alors avec certitude
qu'il est souverain, car malheureusement la plupart
des sources des listes royales que nous avons sont des copies tardives
. C'est le problème général avec les annales, plus
c'est ancien, plus c'est oublié et
c'est farfelu (on peut voir des noms qui ne sont que
des jeux de mots).}


Les dynasties archaïques sont importantes car elles posent un principe
de société, qui est le dieu , le roi son vicaire humain et les hommes.
Déjà c'est quelque chose d'essentiel,
et cela signifie un fonctionnement par un intermédiaire, il
n'y a pas de lien direct entre les hommes et leurs
dieux.  Dans l'ensemble, il y a même plutôt un lien de
crainte. A partir du moment où  l'on sait
qu'ENLIL ne supporte pas les hommes, il a déjà détruit
une fois l'humanité par le déluge, il y a toujours la
crainte d'une nouvelle punition.}


Les dynasties archaïques posent finalement toutes les bases qui seront
admirées et se perpéturont à travers les millénaires}


Et on voit bien qu'une première dynastie de Babylone,
militaire, du début du deuxième millénaire s'appuie
complètement sur ce vieux fond culturel sumérien.
C'est donc fondamental, de bien voir comment les
choses se mettent en place.}


\textbf{L'empire d'AKAD }}


C'est  la première fois que les hommes vont avoir la
volonté de créer non pas un royaume, mais un empire,
c'est à dire qu'ils vont aller
guerroyer au delà d'où on pouvait aller avant, et
aller vers le Nord, le Sud, l'Est et dominer
complètement le  monde des deux fleuves}


Mais AKAD, c'est une utopie car ils étaient incapable
de gouverner ces territoires auxquels ils aspiraient, car il
n'existait pas de structures d'état
qui en étaient capable. Ils n'avaient pas
l'organisation que les égyptiens avaient à la même
époque, car tout simplement c'étaient des gens qui
fonctionnaient sur la base de micro royaume dans le sud.}


Mais AKKAD est fondamental, même si c'est une période
pour laquelle les sources sont maigres. Mais c'est la
première fois qu'il est attesté qu'un
roi ait été divinisé de son vivant, et cela pour la royauté
mésopotamienne, c'est quelque chose
d'absolument ponctuel}


\textbf{Le roi est un homme}, il a une nature humaine et rien de plus. 
C'est un humain à la tête d'autres
humains et il est le vicaire des dieux sur terre. Il
n'a aucun élément surnaturel en lui, le couronnement
ne lui donne aucun pouvoir surnaturel. C'est
d'autant plus frustrant que le quatrième roi de la
dynastie d'Akkad NARAM Sîn (petit fils de Sargon) se
fait diviniser car devant son nom on met le déterminatif de
l'étoile, et cela implique une divinisation. Ces gens
régnaient à partir d'une ville qui est la seule grande
capitale du monde mésopotamien, que l'on ne sait pas
situer et tant que l'on n'aura pas
les archives royales de l'époque, on ne pourra pas
réellement étudier cette période}


Après AKKAD on notera également quelques divinisations de roi, de son
vivant, notamment sous la 3ème dynastie d'UR, et
quelques unes dans la dynastie babylonienne et après plus jamais;}


On sait donc que cela a existé, mais on ne sait pas concrètement comment
cela se faisait, existe t il un temple où l'on rendait
une culte du roi ?}


Akkad après avoir duré  150 ans s'effondre, mais là
encore on ne sait pas comment.}


Il y aura un moment de renaissance sumérienne, mais dans une logique
différente qui sera de renaître à partir d'un royaume
unifié démembré. C'est à dire que ce ne sont pas les
minuscules cités états qui revoient le jour. On voit bien que la
troisième dynastie d'UR correspond en gros au Sud
c'est  à dire que cela englobe les régions de Sumer et
de Babylone}


A partir du deuxième millénaire, les deux royaumes qui seront rivaux
jusqu'à la fin  : Babylone au centre et au sud et
Akkad au nord ; sauf que durant tout le deuxième millénaire on
n'entend plus parler des assyriens, ils sont dominés
par d'autres, et il ne se réveillent que vers -1200
avant JC, dans le système boule de neige du passage des peuples de la
mer et durant le premier millénaire il y aura un conflit permanent
entre le Nord et le Sud pour l'Hégémonie.}


En Mésopotamie, ce sont donc des royaumes, Zone de Sumer, zone de
Babylone, Akkad les  assyriens sont vraiment  au Nord et autour du
Tigre. C'est ainsi que nous n'avons
aucun texte assyrien avant le 9ème siècle qui nous parle de
l'Euphrate, ils ne partaient pas aussi loin, et ils
allaient plutôt vers le Nord}


Les premières sources dans les listes royales sumériennes qui parlent du
roi, leur donnent le titre de EN, en langue sumérienne cela signifie
Maître, Seigneur, mais il semblerait que ce titre puisse correspondre à
prêtre et c'est la raison pour laquelle les historiens
dans le temps parlaient du roi prêtre, car cela correspondait à une
formule française correspondant à la traduction d'un
mon sumérien }


Nous avons un deuxième titre  LUGAL (?) qui en sumérien veut dire Homme,
et gal en sumérien veut dire rateau et quand on a la place de bien
écrire, le rateau est dessiné au dessus de la tête de
l'homme.}


On a beaucoup discuté pour tenter de comprendre à quoi correspondaient
tous ces titres et finalement nous sommes incapables
d'en faire quelque chose de cohérent, Quand on lit
tout ce qui a été dit sur le sujet, on s'aperçoit que
pour le moment on ne comprend pas la logique. On voit que dans certains
petits royaumes sumériens (ex à l'époque des cités
éclatées) du Sud on va appeler le roi En, et à trente km de là  le roi
s'appellera LUGAL. Mais on ne sait pas pourquoi, car
nous n'avons pas compris la logique de la chose.}


Et il y a un troisième titre, qui est ENSI, qui lui aussi est ambigu,
car plus tard il pourra avoir la notation de gouverneur, donc le ENSI
serait dépendant d'un Lugal ?  On voit bien en tout
cas que les premiers titres associés au roi , sont assez ambigus, et
montrent bien le lien entre le pouvoir et le dieu. Et
d'ailleurs, il y a quelque chose qui peut être
rattaché à cela  : c'est qu'au niveau
de l'archéologie, on constate que dans les phases
anciennes, le palais comme élément architecturé
n'existe pas. Les grands bâtiments des cités états
sont des temples. Il semble bien donc que celui que nous appelons roi,
vive dans le temple et que les entrepôts du royaume soient également
dans le temple.}


C'est à cette époque, autour de 2600 AV JC, que
l'on voit apparaître la filiation, la filiation
humaine, c'est à dire que le roi donne son nom en le
faisant précéder du nom suivi de la mention {\textquotedbl} fils de ..
{\textquotedbl} et le nom qui suit est celui de son père}


Et ceci par les sources écrites prouvent de façon certaine
qu'au moins à partir de - 2600, la royauté est
héréditaire}


est elle héréditaire à partir de ce moment là ? (dynastique archaïque
III) , on  ne le sait pas. On pense qu'elle devient
progressivement héréditaire à partir du III millénaire ; avant on
devait avoir le système du choix parmi les anciens}


Nous avons donc établi un premier point : la fonction royale est
descendue sur terre, cadeau de ENLIL et la première caractéristique  :
l'hérédité : je deviens roi car mon père
l'était. L'hérédité, la filiation est
quelque chose qui se met en  place et qui va durer pendant 3,000 ans
(en effet la plupart des rois sont fils de ... )}


Mais il y a une deuxième raison pour devenir roi, ce sont les grandes
qualités : c'est à dire que les textes développent
l'idée que certains humains étaient destinés à devenir
roi à partir de leur haute qualité }


Le premier à s'en référer est Sargon (premier fondateur
de l'empire d'Akkad) et
n'a cessé de vanter les mérites , Il a renversé le
souverain, en réalité c'est un usurpateur, et
d'ailleurs on ne connait pas son nom de naissance,
mais il se fait couronner sous le nom de SARGON SHARRUKIN et Sharrum
veut dire roi dans la langue d'Akkad et Kin est un
adjectif qui veut dire vrai, légitime,  C'est donc un
usurpateur qui se fait couronner sous le nom de roi légitime.}

\includegraphics[width=10.76cm,height=10.83cm]{FaivreMartin5conf-img/FaivreMartin5conf-img3.jpg}
\captionof{figure}[Tête de Ninive, qui pourrait être celle de
Sargon]{Tête de Ninive, qui pourrait être celle de Sargon}



Après sa mort on a fait de Sargon un personnage merveilleux, mais avec
deux siècles de retard, car tous les textes relatifs à Sargon datent de
la première dynastie de Babylone car cette première dynastie de
Babylone , sont des gens qui n'avaient pas de lien
royal avant et se sont appuyés sur cette fausse filiation . Alors on
montre SARGON comme un personnage merveilleux et on se place comme
étant ses descendants , dotés des mêmes qualités, (et
qu'ils sont ainsi devenus rois du fait de ces mêmes
qualités)}


\textbf{LE ROI EN MESOPOTAMIE EST :}}


\textbf{{}- Vicaire du dieu sur terre,}}


\textbf{{}- gestionnaire et défenseur des biens du dieu }}


\textbf{{}- c'est un homme, un humain tout simplement}}


C'est donc un mortel qui au delà de la mort
n'aura rien de plus que les autres, et cet élément est
très important pour notre sujet}


\textbf{le Roi, vicaire du dieu sur terre} :  }


Pendant 3,000 ans, les images royales le représenteront debout, assis,
les mains jointes dans le signe de la prière, car dans
l'état actuel des choses toutes les statues que nous
connaissons proviennent des temples et sont donc systématiquement
associées à des divinités}


Nous n'avons en effet à ce jour aucun exemple connu de
statue identifiée comme ayant pu être mise à l'entrée
d'un bâtiment officiel, sur une place visible des
gens, ou associée à la sépulture d'un roi}


(mais cela devait arranger les sculpteurs qui ne disposaient pas
d'un matériel extraordinaire et qu'il
était très difficile ainsi de travailler dans la dorrite)}

\includegraphics[width=13.965cm,height=20.142cm]{FaivreMartin5conf-img/FaivreMartin5conf-img4.jpg}
\captionof{figure}[Statue ESHNUNNA (1600)]{Statue ESHNUNNA (1600)}



On voit bien les mains jointes et dans une attitude de prière.}


Le roi étant le vicaire du dieu, il a être représenté dans un face à
face avec des dieux. Ce personnage, nous ne le connaissons jamais par
son nom de naissance. En effet quand on regarde ce que veut dire le nom
de ces rois, ce n'est pas un prénom, le nom du roi est
en effet lié au contexte d'une époque , à sa demande
de protection à l'une des divinités, ce sont donc des
noms de couronnement.}


Et si on reprenait tous les noms des rois depuis 3000 ans, ce que
l'on ne pourrait faire, on verrait peut être quelques
prénoms, mais ce sont surtout des noms de couronnement.}


On peut également s'interroger pour savoir
s'il existe un couronnement, un rituel de sacre, et
dans ce cas à quand il remonte, quel est le plus ancien exemple.  En
réalité, les sources ne sont pas très anciennes, la plus ancienne
attestation connue est la stèle de UR NAMMU, (Philadelphie)}

\includegraphics[width=6.809cm,height=10.089cm]{FaivreMartin5conf-img/FaivreMartin5conf-img5.jpg}
\captionof{figure}[stèle UR NAMMU et détail]{stèle UR NAMMU et détail}
\includegraphics[width=7.761cm,height=6.033cm]{FaivreMartin5conf-img/FaivreMartin5conf-img6.jpg}
\captionof{figure}[stèle UR NAMMU et détail]{stèle UR NAMMU et détail}



UR NAMMU est le fondateur de la troisième dynastie d'Ur
au 22 siècle avant JC, et c'est à cette époque que les
textes parlent d'un cérémonie de sacre qui se passait
dans le sanctuaire d'ENLIL, (ce qui est normal puisque
ENLIL donne la royauté aux hommes) dans son grand sanctuaire de Nippur
qui s'appelle, nous l'avons vu ,
EKUR}


On ne sait pas très bien comment cela se passait, mais on sait que le
roi recevait des choses, même si nous n'avons aucun
récit exhaustif)}

\includegraphics[width=9.202cm,height=7.512cm]{FaivreMartin5conf-img/FaivreMartin5conf-img7.jpg}
\captionof{figure}[Stèle de Shamash (Bristish Muséum)]{Stèle de Shamash
(Bristish Muséum)}



\textit{Le Roi Nabû Appla Iddina est introduit par deux divinités
protectrices auprès du dieu Shamash, assis sur son trône}}


cette stèle date du 9ème siècle avant JC, période où les rois de
Babylone ont beaucoup de mal à se débarrasser des assyriens qui les
dominent, les désignent dans la plupart des cas, et où en réalité il
s'agit de pseudo royautés autonomes, sous le contrôle
des assyriens}


Cette stèle est extrêmement importante au niveau iconographique car elle
nous montre à droite sous un dais, et sur un trône qui peut être
identifié car il y a en gros l'idéogramme de son nom,
Shamas, dieu soleil, et dans les textes de la troisième dynastie
d'Ur, nous savons que la cérémonie du sacre se passait
dans le temple d'ENLIL . C'est donc
le dieu soleil SHAMASH qui remet au roi deux objets : une corde
enroulée sur elle même et un bâton à mesurer, qui sont le symbole de sa
fonction, être garant des normes.}


Avant on pensait qu'il s'agissait
d'une corde et d'un piquet de
fondation et on y voyait l'image du roi arpenteur
prenant les mesures du futur temple; On sait maintenant
qu'il s'agit d'un
bâton à mesurer et en aucun cas  d'un spectre.}


En réalité il s'agit d'objet très
concret : la règle sert à mesure}


On voit le roi arriver près du dieu , tel un dieu secondaire dans une
tenue dont nous reparlerons  : en effet son costume est différent de
celui des deux autres, qui ont des tuniques, alors que le roi  a
toujours l'épaule droite dénudée. On sait
qu'en Mésopotamie, la bonne main
d'usage est la main droite, la main gauche est celle
avec laquelle on fait les choses sales (on écrit , mon mange, on salue,
on prie de la  main droite). Le salut suprême se fait main droite levée
devant la bouche, c'est un signe de respect. Et
d'ailleurs ce sont des traditions qui vont se
perpétuer et on en trouve même référence dans la Bible}


aussi, cette stèle peut très bien représenter une scène de couronnement}


Les textes nous parlent également d'Isthar  du jour
(attention il existe Isthar du jour et Isthar de la Nuit , la
différence est qu'Isthar de la nuit est représentée
avec des ailes et des chouettes)}

\includegraphics[width=7.761cm,height=10.336cm]{FaivreMartin5conf-img/FaivreMartin5conf-img8.jpg}
\captionof{figure}[Isthar de la nuit ( vers {}- 1800)]{Isthar de la nuit
( vers - 1800)}



Isthar est importante à la fin du III et début du II millénaire,
tellement importante d'ailleurs que son nom devient un
nom commun pour signifier déesse}


Isthar donne au roi son trône, sa thiarre et son spectre }


Et nous avons ainsi la liste de cinq objets que le roi recevait lors de
son couronnement et cela se  perpétuera sans doute après des mises en
scènes différentes en fonction des époques et des royaumes
jusqu'à la chute de Babylone}


Ceci va conduire l'iconographie mésopotamienne à
montrer le lien entre le roi et son dieu par une proximité, mais on
s'arrête à cette proximité, il
n'existe aucune familiarité entre le roi et son dieu,
car le roi mésopotamien est un homme.}


Le roi affirme une action , et la légitime par la volonté du dieu et
c'est ce qui se passera durant la période assyrienne,
où on utilise un motif , et là il faut repartir sur la notion du disque
solaire ailé égyptien d'Edifou}


On sait que ce motif, qui est au départ le disque solaire égyptien, a
voyagé par le monde palestinien, le monde hittite, le monde
mésopotamien et même chez les perses et au premier millénaire, les
assyriens (on ne sait pas comment ils en ont eu
l'idée, représentent leur dieu national Assur comme un
buste sortant de l'astre solaire. Les babyloniens
représenteront leur dieu Marduk de la même façon et les perses ensuite
quand ils auront vaincu la Mésopotamie se mettront également à
représenter leur dieu suprême de la même façon}


Cela permet pour les assyriens une composition très pratique en image
pour montrer le niveau céleste de la décision et son application 
terrestre par le roi et cette image a du succès}


Nous constaterons que lié à la personne royale, les femmes de sa famille
(mère, épouse, fille) sont absentes de l'iconographie.
En Mésopotamie, le roi est toujours seul et c'est de
façon tout à fait exceptionnelle que l'on verra la 
présence de ces femmes.}


Et c'est intéressant car si on se réfère aux sources
littéraires, on sait que les femmes en Mésopotamie avait un pouvoir
important. Au moins à partir du II millénaire, le palais est entre
leurs mains, elles ont la clés de l'entrepôt et même
du palais (à certaines époques , à certains endroits, le roi
n'a pas les clés de son palais et doit les demander
aux femmes}


Donc leur absence dans l'iconographie ne signifie pas
qu'elles ont aucun rôle,  Mais on voit que par la
définition même de la royauté en Mésopotamie, elles
n'ont aucun rôle à jouer}


Y a -t-il eu des femmes rois ? à part la fameuse Kubada (?), et ce à la
période dynastique à Ur au moment où la succession par hérédité se met
en place, elle a été roi (on le sait par un texte au British Muséum),
et mais il n'y a aucune autre femme roi attestée}


C'est pourquoi \textit{la stèle
d'ASSARHADDON , avec sa maman}
(Naqi'a) du musée du louvre }

\includegraphics[width=6.842cm,height=7.549cm]{FaivreMartin5conf-img/FaivreMartin5conf-img9.jpg}
\captionof{figure}[stèle d'ASSARHADDON (louvre)]{stèle
d'ASSARHADDON (louvre)}



est particulièrement intéressante. Sa maman était une déportée
babylonienne, et elle va convaincre son fils Assarhddon (qui est un roi
assyrien qui va conquérir l'\kmt)
d'avoir une politique favorable aux ennemis
babyloniens. On peut voir qu'elle est représentée
comme un ennuque, Ce petit morceau a été acheté par le Louvre dans les
années 60, car particulièrement intéressant, et très rare, mais peu
intéressant d'un point de vue de
l'art}


\textbf{Le Roi mésopotamien est le gestionnaire et le défenseur des
biens de son dieu}}


on voit bien que cela marche dans les deux sens}


En effet, et nous le verrons toutes les  premières  images
l'associe à ces deux mondes le temple et son troupeau
et son étable,  et on voit bien que étable, temple, troupeau, sac de
grains, tout cela va ensemble et c'est dans le même
bâtiment  et que toute l'économie passe dans ce fameux
lieu, qui nous appelons temple.}


Ce sont des territoires agricoles et ils ont deux obsessions : que les
champs soient fertiles et que les récoltes soient bonnes, et le
troupeau fécond, ce qui permet de nourrir tout le monde}


Donc , on part du principe que c'est maintenir la
création des dieux, telle qu'elle a été donnée aux
hommes C'est à dire que c'est partir
du principe que le monde terreste a été fait par les dieux pour que
l'humanité y vive et que les hommes doivent la
respecter, et la mettre en valeur, ce qui est une façon de rendre
hommage aux deux}


Mais le territoire agricole appartient aux divinités}


Prenons un exemple ancien, du IV millénaire, le sceau de Berlin, (que
nous verrons plus tard) , et au XII siècle avant JC :}

\includegraphics[width=8.25cm,height=12.79cm]{FaivreMartin5conf-img/FaivreMartin5conf-img10.jpg}
\captionof{figure}[Kuduri Kassite du Roi Méli Shipak II \ Louvre]{Kuduri
Kassite du Roi Méli Shipak II  Louvre}



On voit donc le roi devant une divinité qui est peut être shamash et il
salue le dieu. Le texte devant est effacé , mais derrière il existe
encore et il ne concerne que l'arpentage, mesure de
terrains, (tel champs qui appartient au domaine agricole du dieu, et
qui mesure ... , a produit ceci ...; et est géré comme cela ...)}


Et à partir de là, la destination de la guerre en découle,  en effet à
partir du moment où le roi s'affirme comme étant le
vicaire de son dieu, il défend le territoire de son dieu. Et si on
prend la plus ancienne image sculptée de la guerre, stèle des vautours,
Louvre, date vers 2500 avant JC et dynastie archaïque III}


\textbf{\textit{stèle des vautours}}\textbf{ : }}

\includegraphics[width=15.981cm,height=19.967cm]{FaivreMartin5conf-img/FaivreMartin5conf-img11.jpg}
\captionof{figure}[stèle des vautours]{stèle des vautours}



\textit{Commentaire (hors conférence, site du Louvre)}}


\textbf{\textit{\textcolor[rgb]{0.101960786,0.101960786,0.101960786}{Partiellement
reconstituée à partir de plusieurs fragments trouvés dans les vestiges
de la cité sumérienne de Girsu, cette stèle de victoire constitue le
plus
ancien}}}\textbf{\textit{\textcolor[rgb]{0.101960786,0.101960786,0.101960786}{
}}}\textbf{\textit{\textcolor[rgb]{0.101960786,0.101960786,0.101960786}{document
historiographique connu. Une longue inscription en langue sumérienne
fait le récit du conflit récurrent qui opposait les cités-États
voisines de Lagash et Umma, puis de la victoire
d'Eannatum, roi de Lagash. Son triomphe est illustré
avec un luxe de détails par le remarquable décor en bas-relief qui
couvre les deux faces.}}}}


\textbf{\textit{Un document historique exceptionnel}}}


\textit{Malgré sa conservation lacunaire, cette stèle de grande taille,
sculptée et inscrite sur ses deux }\textit{faces, est un monument
d'une valeur incomparable puisqu'il
s'agit du plus ancien document historiographique
connu. Les fouilles du site de Tello permirent d'en
retrouver plusieurs fragments disséminés parmi les vestiges de
l'ancienne cité sumérienne de Girsu. Cette stèle
commémore, par le texte et l'image, une importante
victoire remportée par le roi de Lagash, Eannatum, sur la cité voisine
d'Umma. Les deux villes entretenaient en effet un état
de guerre récurrent à propos de la délimitation de leur frontière
commune, à l'image de ce que pouvaient être les
relations entre cités-États à l'époque des dynasties
archaïques.[2028?]Petit-fils d'Ur-Nanshe et fondateur
de la Ière dynastie de Lagash, Eannatum régna vers 2450 av. J.-C. et
conduisit sa cité-État à l'apogée de sa puissance.
L'inscription gravée sur }\textit{La Stèle des
vautours}\textit{, d'une ampleur remarquable bien
qu'il n'en subsiste
qu'une petite moitié, exalte les triomphes
d'un souverain placé dès sa naissance sous la
protection divine. Nourri au lait de la déesse Ninhursag et tenant son
nom de la déesse Inanna, c'est du dieu Ningirsu
lui-même qu'il reçut la royauté de Lagash. Assuré du
soutien des divinités par un songe prophétique, Eannatum va
s'engager avec fermeté dans la lutte contre Umma afin
d'imposer son contrôle sur le Gu-edina, territoire
frontalier enjeu de la rivalité entre les deux
cités.[2028?]{\textquotedbl}}\textit{Moi Eannatum, le puissant,
l'appelé de Ningirsu, au pays [ennemi], avec colère,
ce [qui fut] de tout temps, je le proclame ! Le prince
d'Umma, chaque fois qu'avec ses
troupes il aura mangé le Gu-edina, le domaine bien-aimé de Ningirsu,
que [celui-ci] l'abatte }\textit{!{\textquotedbl}.}}


\textbf{\textit{La face {\textquotedbl}historique{\textquotedbl}}}}


\textit{La narration de la campagne militaire contre Umma est illustrée
de manière spectaculaire par des représentations figurées, sculptées
dans le champ de la stèle selon une disposition traditionnelle en
registres. Elles offrent ici la particularité d'être
réparties sur chacune des deux faces en fonction de leur perspective
symbolique. L'une des faces est ainsi consacrée à la
dimension {\textquotedbl}historique{\textquotedbl} et
l'autre à la} \textit{dimension
{\textquotedbl}mythologique{\textquotedbl}, la première rendant compte
de l'action des hommes et la seconde de
l'intervention des dieux. Détermination humaine et
protection divine se conjuguent ainsi pour conduire à la
victoire.[2028?]La face dite {\textquotedbl}historique{\textquotedbl}
montre, au registre supérieur, le souverain de Lagash marchant à la
tête de son armée. Eannatum est vêtu de la jupe à mèches laineuses
appelée kaunakès, recouverte partiellement par une tunique en laine
passant sur l'épaule gauche. Il porte le casque à
chignon, apanage des hauts personnages. Les soldats, casqués eux aussi
et armés de longues piques, s'avancent en formation
serrée, se protégeant mutuellement derrière de hauts boucliers
rectangulaires. L'armée de Lagash triomphante piétine
les cadavres des ennemis qu'une nuée de vautours a
commencé à déchiqueter, scène dont la stèle tire son nom.
L'inscription proclame
:[2028?]{\textquotedbl}}\textit{Eannatum frappa Umma. Il eut vite
dénombré 3 600 cadavres [...]. Moi Eannatum, comme un mauvais vent
d'orage, je déchaînai la tempête
!}\textit{{\textquotedbl}.[2028?]Au deuxième registre est représenté ce
qui semble constituer le défilé de la victoire. Les soldats marchent
alignés sur deux colonnes derrière leur souverain monté sur un char.
Ils tiennent leur pique relevée et la hache de guerre à
l'épaule. Eannatum brandit lui aussi une longue pique
ainsi qu'une harpé à lame courbe, une arme
d'apparat. Il se tient debout sur un char à quatre
roues pourvu d'un haut tablier frontal duquel émergent
des javelots rangés dans un carquois.}}


\textit{Le troisième registre, très fragmentaire, illustre les
cérémonies funéraires qui viennent clôturer
l'engagement militaire. Pour ensevelir les cadavres
amoncelés de leurs camarades, les soldats de Lagash gravissent une
échelle en portant sur la tête un panier rempli de terre. Des animaux,
dont un taureau couché sur le dos et ligoté, sont prêts à être immolés
tandis que l'on accomplit une libation au-dessus de
grands vases porteurs de rameaux végétaux.}}


\textbf{\textit{La face {\textquotedbl}mythologique{\textquotedbl}}}}


\textit{La face dite {\textquotedbl}mythologique{\textquotedbl} illustre
l'intervention divine qui offre la victoire à
Eannatum. Elle est dominée par la figure imposante du dieu Ningirsu,
protecteur de la cité-État de Lagash. Celui-ci tient les troupes
ennemies emprisonnées pêle-mêle dans un gigantesque filet et les frappe
de sa masse d'armes. Instrument de combat par
excellence du dieu, le filet est tenu fermé par
l'emblème d'Imdugud,
l'aigle à tête de lion, attribut de Ningirsu, qui est
représenté les ailes déployées et agrippant deux lions dans ses
serres.}}


\textit{Le reste de la face {\textquotedbl}mythologique{\textquotedbl},
très lacunaire, semble évoquer la présence aux côtés du dieu triomphant
d'une déesse, sans doute Nanshe,
l'épouse de Ningirsu, également associée à
l'aigle léontocéphale. Le registre inférieur laisse
entrevoir le dieu sur un char, en compagnie de la même
déesse.[2028?]L'inscription, après avoir glorifié
l'action victorieuse d'Eannatum, fait
une large place aux serments prêtés par les deux souverains devant les
grandes divinités du }\textit{panthéon. Ayant réintégré le Gu-edina au
sein du territoire de Lagash, Eannatum délimite avec Umma la frontière,
sur laquelle est érigée une stèle. Mais la réussite du projet humain ne
peut s'accomplir que par faveur divine ;
c'est donc elle qui est invoquée afin de garantir la
pérennité du nouvel ordre des choses : {\textquotedbl}}\textit{Que
jamais l'homme d'Umma ne franchisse
la frontière de Ningirsu ! Qu'il n'en
altère pas le talus et le fossé ! Qu'il
n'en déplace pas la stèle ! S'il
franchissait la frontière, que le grand filet d'Enlil,
le roi du ciel et de la terre, par lequel il a prêté serment,
s'abatte sur Umma }\textit{!{\textquotedbl}.}}


On voit donc que c'est très intéressant ,
d'une part cette stèle comporte le plus long texte en
langue sumérien archaïque, et dit, que si le roi représenté ici est
parti en guerre sur son char, contre le petit royaume voisin, 
c'set car Umma n'a pas respecté la
frontière, c'est à dire un canal
d'irrigation Umma et sa troupe ont franchi ce canal,
et sont donc venus sur un territoire qui n'était pas
le leur, territoire qui appartenait au dieu Ennatum}


L'idée est donc que l'on ne peut
tolérer l'idée que l'homme
d'Umma (façon d'appeler le roi
ennemi) vienne revendiquer un terrain }


Au départ c'est une guerre céleste,
c'est le dieu SHAMAH dans le ciel qui revendique un
terrain qui appartient à un autre dieu, mais cette guerre sera vécue à
un niveau humain. et ceci permet de légitimer tout ce que
l'on veut, car on considère que la guerre est une
volonté de la divinité (on le reverra) et cela permettra de mettre en
place tout un discours qui révèle que si l'ennemi a
perdu, et cela va même au delà car le texte dit que celui qui gagne est
béni du dieu, et donc le béni du dieu est l'homme pur
victorieux et l'homme qui a perdu la guerre est un
impie, et donc on le met tout nu on lui saute dessus, on lui fait en
réalité ce que l'on veut. }


Et ce même langage appartient dans les anales assyriennes. En effet
quelque soit le conflit, les assyriens entre le IX et le VII siècle
parlent des autres comme cela. Un roi assyrien fait la guerre au nom de
son dieu, donc d'Assur, et ils ont toujours ce côté
dépréciatif les uns par rapport aux autres, (ce lâche qui fuit devant
nous) et ils ont la certitude d'être redoutables, car
ils font la guerre au nom d'Assur et donc ils ne
peuvent qu'être victorieux , et ils peuvent massacrer
leurs ennemis}


\textbf{Le roi est un mortel sans destinée particulière} : }


Cette conception est embêtante pour nous, car cela veut dire
qu'il n'y a pas
d'art funéraire.}


C'est un mortel, donc on a un homme à la tête des
autres hommes, même si après et parfois il y aura des mises en scène
pour le magnifier,  }


Mais les dieux ne le récompense pas pour ses hauts faits en lui donnant
quelque chose de mieux que les autres; il va comme
n'importe quel autre humain après sa mort dans le
monde des enfers, et en Mésopotamie la vision de l'au
delà n'est pas très drôle}


Aussi dans cette logique, il est normal de ne pas rechercher de
magnifiques tombes royales mésopotamiennes}


Il est évident qu'ils ont dû emmener des choses avec
eux,  des objets précieux, qu'ils avaient un belle
tombe, mais qu'il n'existait pas un
art architecturé de la tombe, il n'y a aucune
construction élaborée}


En 3000 ans,  il n'y a  à ce jour
qu'un seul exemple , ce sont les tombes royales
d'Ur. C'est le seul moment et
seulement huit tombes identifiées comme étant des tombes royales, où
l'on voit qu'il y avait deux grandes
structures à plusieurs pièces, avec des sacrifices humains et une mise
en scène d'un aménagement funéraire}


\textbf{L' \kmt}}


Nous avons là une unité géographique et une unification qui
s'est faite entre le IV et le III millénaire, avec des
récits de la création de la royauté qui elle aussi
s'appuie sur un monde divin}


Et même s'il est évident que des traditions plus
anciennes ont existé, la plus ancienne source résulte des Textes des
Pyramides, car dans ces textes, le roi est décrit comme arrivant du
ciel, (intéressant, car en Mésopotamie, la royauté est tombé du ciel).
Mais le roi égyptien arrive sur terre , après avoir mis MAAT à la place
de l'Isefet, dans l'ile des flammes
(et dans les textes des pyramides, lorsque l'on parle
de l'ile des flammes, c'est la terre
dans son état rudimentaire, sauvage, rustique}


Et à partir du moment où le roi a mis MAAT à la place de
l'Isefet, ce geste fait que la terre devient
habitable}


la principal source qui rattache le roi d'\kmt à son
créateur, le soleil, c'est l'hymne
qui évoque l'adoration matinale du soleil, cet hymne
constitué de 44 vers, et qui découle de cette pensée de
l'ancien empire , exprimée dans les textes des
pyramides, qu'il doit y avoir une origine
rédactionnelle au Moyen Empire , mais que l'on connaît
par des copies, notamment celle de deir el bahari}


Rê a installé le roi sur la terre des vivants,  jamais et à toute
éternité de sorte qu'il juge les hommes, et satisfasse
les dieux, qu'il réalise MAAT et anéantisse
l'Isefet, donne des sacrifices aux dieux et des
offrandes funéraires aux morts immortalisés}


Donc là le point commun , c'est que la royauté est un
cadeau aux hommes : le créateur donne aux hommes le roi pour la terre
des vivants, pour l'éternité}


Le roi d'\kmt doit donc juger les hommes et
satisfaire les dieux; On voit donc dans le fait que le roi juge les
hommes, il y a cette notion d'ordre qui existe
également en Mésopotamie, l'obligation de satisfaire
les dieux,  il a par contre un lien privilégié avec les dieux et une
obligation qui est une spécificité égyptienne :
l'opposition entre Maât et l'Isefet}


Effectivement, en égypte, on sait que l'on part là
encore sur le principe d'une royauté héréditaire (par
les male de la famille), et il y a la possibilité pour une personne qui
n'est pas de la lignée royale de prendre le pouvoir, 
en effet un homme de mérite peut devenir roi, cela est attesté avec bon
nombres de rois, qui ont été appelés usurpateurs, et finalement leur
légitimité est accordée par l'acte de couronnement}


Mais, la différence fondamentale avec le roi mésopotamien,
c'est que l'homme qui subit le rituel
du couronnement en \kmt, devient un netchernéfer,
c'est à dire un  être qui contient en lui une parcelle
divine et c'est le grand bénéficie du couronnement.}


Il va recevoir aussi des insignes régaliens, mais à partir de ce moment
là, il peut toucher le monde des dieux, et avoir en lui une partie
d'essence surnaturelle.}


Exemple THOT qui inscrit le nom du roi KALABASHA}

\includegraphics[width=15.981cm,height=11.395cm]{FaivreMartin5conf-img/FaivreMartin5conf-img12.jpg}
\captionof{figure}[THOT qui inscrit le nom du roi KALABASHA]{THOT qui
inscrit le nom du roi KALABASHA}



(rite de purification pendant le couronnement et Thot écrit non pas le
nom du pharaon mais le nom per aa}


A partir du couronnement, peut s'exprimer cet aspect
surnaturel de la personne royale, qui s'exprimera de
façon différente en fonction des époques. }


On sent très bien la magie des insignes régaliens donnés également au
roi lors de son couronnement}


A l'ancien empire, nous verrons des représentations où
le roi est en contact direct avec les dieux, }


mais contrairement au monde mésopotamien, où l'on voit
le roi humain agir pour la divinité, et éventuellement que la divinité
lui donne quelque chose, il y a beaucoup plus d'aller
et retour dans le monde égyptien.}


On a en effet des scènes de deux divinités qui entourent le roi (et on
peut imaginer deux autres divinités, de l'autre côte
et on fait la boucle avec les quatre points cardinaux) et on est dans
la logique égyptienne, des rituels de protection, pour purifier le roi.
Cela est inenvisageable en Mésopotamie, par le fait que le roi
n'a rien de divin}


La notion de royauté liée l'hérédité,
c'est par les sources écrites que nous la connaissons,
(Pierre de Palerme) et c'est la légitimation par le
simple fait que l'on est fils de }


Le roi en \kmt est mis en image dès le IV millénaire, et il y a la
mise en place d'une iconographie royale, donc avec
ceux d'avant l'écriture, ceux sans
nom}


En \kmt il n'y a qu'un seul roi, et
un seul royaume, (sauf durant les périodes intermédiaires), il y a donc
eu l'unification politique et il en découle un seul
roi}


l'\kmt aussi est un pays protégé naturellement par le
désert, alors que la Mésopotamie est une plaine qui de tout temps a été
traversée par des populations différentes}


\textbf{EN \kmt : LE ROI EST :}}


\textbf{{}- fils de dieu sur terre,}}


\textbf{{}-garant de l'ordre de MAAT}}


\textbf{{}- un dieu parfait de son vivant, et un dieu après sa mort}}


\textbf{fils de dieu sur terre : }}


Il est intéressant de voir que très tôt dans le temps, nous allons avoir
des faces à faces entre Pharaon et les dieux, il y a vraiment un lien
charnel, physique qui va être affirmé.}

\includegraphics[width=5.821cm,height=6.668cm]{FaivreMartin5conf-img/FaivreMartin5conf-img13.jpg}
\captionof{figure}[Niouserrê recevant la vie du dieu Anubis (Relevé du
temple funéraire du roi à Abousir)]{Niouserrê recevant la vie du dieu
Anubis (Relevé du temple funéraire du roi à Abousir)}


\includegraphics[width=15.981cm,height=23.883cm]{FaivreMartin5conf-img/FaivreMartin5conf-img14.jpg}
\captionof{figure}[relief du roi Niouserré à Berlin]{relief du roi
Niouserré à Berlin}



Et on voit sur le relief du roi Niouserré à Berlin, une déesse lionne
(on discute encore pour connaître son nom) et elle tient son sein pour
nourrir le roi, comme si elle était sa mère terrestre. Et à ce jour
c'est la plus ancienne scène
d'allaitement d'un roi humain par une
divinité que l'on connaisse. Niouserrê étant un roi de
la V dynastie}


Cette image est toujours présente dans la XVIII dynastie, où la
littérature égyptienne développe le fait qu'à partir
de la 18ème dynastie l'essence divine
s'affirme , en disant {\textquotedbl} il est le fils
du dieu sur terre, il est l'enfant terrestre des dieux
et  à partir de ce moment là on a la triade divine céleste, (père mère
enfant),  et le dieu et la déesse ont donc un enfant sur terre : le
pharaon régnant}


Et sur la rive Ouest de Thèbes consacrée à Hathor,  on le met en image,
roi debout devant Hathor :
\includegraphics[width=15.981cm,height=11.994cm]{FaivreMartin5conf-img/FaivreMartin5conf-img15.jpg}
}


et cette image sera reprise}


On peut même voir le souverain téter le pie de la vache}


et cela va s'exprimer d'une autre
façon car nous aurons la représentation qui existe dès la 18ème
dynastie du souverain encadré par son père et sa mère céleste,
c'est à dire que là le prend véritablement le rôle de
l'enfant divin et au lieu d'avoir le
dieu fils de Ptah et Sekmet on a carrément Ramsès II entre Ptah et
Seckmet}


\includegraphics[width=15.981cm,height=16.374cm]{FaivreMartin5conf-img/FaivreMartin5conf-img16.jpg}



Et ce lien conduit dans l'art dès
l'ancien empire, à des représentations où le dieu
touche le roi et là effectivement ceci est inenvisageable dans le monde
mésopotamien, (on peut voir parfois le roi être touché par un dieu,
mais il s'agit d'un dieu inférieur),
on le sait car les sumériens ont établi des listes de dieux , et ce par
ordre d'importance et il existe des dieux de rang
inférieur, divinité d'ordre secondaire (un peu comme
le dieu personnel du roi qui est une sorte d'ange
gardien, qui peut lui servir d'intercesseur auprès des
divinités suprêmes et peut alors le toucher , le conduire par la main
auprès du dieu supérieur}


Mais dans le monde mésopotamien, nous ne verrons jamais un dieu
supérieur toucher le roi comme s'il était son égal}


Et d'ailleurs dans la proportion des tailles, la
divinité est toujours plus grande en Mésopotamie, et
l'humain plus petit  (cf photo vu en début du cours
avec le Roi Shamash énorme assis sous son dais, et le roi en tout
petit,  alors que cela aurait dû être l'inverse
puisque le roi est debout et donc en toute logique aurait dû être plus
grand)}


En \kmt au contraire, on observe des faces à faces entre le dieu et le
pharaon et ils ont la même taille et cela est fondamental }


Le roi et la divinité ont la même taille, la même couleur , donc la même
essence, ils se touchent et c'est intéressant au
niveau de la gestuel, }


On utilise donc l'image du roi régnant pour montrer
l'image d'un dieu antropomorphe
contemporain}


et à la même époque de Thoutmosis III , on va jouer sur des similitudes
au niveau des des coiffures, car le toi aura à partir de Thoutmosis III
des coiffures de plus en plus exubérantes, avec de plus en plus de
plumes, qui ressemblent beaucoup à la coiffure divine}


Le lien  profond qui existe entre le roi et les dieux vient du fait que
les dieux sont là en permanence pour lui insuffler la vie, et ce
souffle de vie, il ne le reçoit pas uniquement pour lui, mais pour
toute l'humanité qu'il représente, et
nous verrons bien que c'est tout à fait
caractéristique de la mise en image égyptienne, de
l'iconographie égyptienne, on ne représente pas le
dieu en train de faire bénéficier ce souffle de vie à  la population
d'\kmt. L'humain de référence ,
l'humain de contact est pharaon, }


Il y a donc une hiérarchie et le roi recevant ce souffle sera garant de
la MAAT}


L'ordre doit régner sur terre pour que le monde divin
soit stable et si l'ordre ne règne pas sur terre cela
met en péril le monde divin}

\includegraphics[width=8.431cm,height=18.062cm]{FaivreMartin5conf-img/FaivreMartin5conf-img17.jpg}
\captionof{figure}[Hathor accueille Séthi 1er (Louvre)]{Hathor accueille
Séthi 1er (Louvre)}



Et on voit que leurs mains se tiennent et que quelque chose est donné}


et encore une fois, cette thématique ne peut être envisagée dans le
monde mésopotamien ou si le roi lui donne quelque chose , la divinité
pourra lui donner également quelque chose,  mais jamais la divinité ne
lui donnerait quelque chose sans avoir reçu quelque chose avant, sauf
le jour du couronnement où le roi reçoit ses insigne}


IL faut aussi évoquer le fait que pharaon n'est pas
uniquement le fils du dieu sur terre, mais un dieu sur terre et cela
s'exprimera intensément durant la période aramarnienne
}


exemple tombe de Meryre}


Cette période est d'ailleurs assez intéressante, elle
ne dure que 14 ans sur les 3000 ans de l'histoire
égyptienne, mais elle va changer et chambouler beaucoup de chose}


pendentif de Shed Louvre C'est à cette époque
qu'apparait l'iconographie du dieu
SHED, tueur de bête sauvage, et c'est un enfant royal,
il est représenté avec la tresse de l'enfant, }


c'est à la fois un enfant humain et un enfant divin, on
voit donc que les choses sont floues, et ils tue les animaux du désert
en tuant les crocodiles}


on voit donc que la frontière entre le monde humain royal et le monde
divin est ténu}

\includegraphics[width=9.202cm,height=16.485cm]{FaivreMartin5conf-img/FaivreMartin5conf-img18.jpg}
\captionof{figure}[stèle de l'enfant Horus et les
Crocodiles]{stèle de l'enfant Horus et les Crocodiles}



un roi en enfant solaire, alors qu'il est adulte}


On est là à une relecture durant la période de Ramssès de Shed qui est
une divinité tout à fait solaire, et le passage à partir de la 3ème
période intermédiaire , l'image de SHED va
disparaître, car elle est liée à un principe de souveraineté, et il y a
le développement des stèles d'Horus sur les crocodiles
et Horus est représenté en enfant}


Et c'est intéressant de voir comment une image royale,
mise en place à un certain moment pour dire quelque chose va évoluer et
ne sera conservée pour être acceptable que sous la forme
d'une image divine ; car on est revenu à une
orthodoxie religieuse un peu calmée par rapport à la période
amarnienne.}


\textbf{MAAT et le principe de souveraineté} :}


Cela signifie que le roi d'\kmt est garant de MAAT et
c'est fondamental, car MAAT c'est la
volonté des dieux, le roi est leur intermédiaire et il se charge de la
faire respecter, les hommes doivent lui obéir}


Et les textes disent que }


RE se nourrit chaque jour de la MAAT, cela veut dire que si on ne la
respecte pas , c'est le créateur lui même qui est en
danger}


Le roi garant de MAAT, il doit donc lui faire des offrandes chaque jour
dans le temple, mais cela va plus loin , car être garant de MAAT
c'est être garant de tout et ce tout va de la
construction du temple ,à son fonctionnement, son organisation
intérieure, veiller au bon fonctionnement du culte, et ce
jusqu'au fonctionnement du monde paysan, pour que la
boucle soit complète}

\includegraphics[width=15.981cm,height=7.126cm]{FaivreMartin5conf-img/FaivreMartin5conf-img19.jpg}
\captionof{figure}[Linteau de Medamoud (Louvre) XII dynastie]{Linteau de
Medamoud (Louvre) XII dynastie}



On peut voit que Sésostris III est représenté deux fois }


et elle correspond à la plus ancienne scène de culte journalier que nous
connaissons : Sésostris III fait offrande de pain blanc et de gâteau}


On reverra d'autres scènes d'offrandes
dans les temples et tout ceci insiste sur la garantie de la MAAT et que
l'Isfet ne revienne pas et cela se résume notamment à
la représentation du roi à genou présentant les vases globulaires. On
sait très bien que cette image apparaît à la VI dynastie  : le roi
offrant , cela va au delà du roi à genou présentant des vases
(globulaires, à vins ..) c'est le roi en offrande et
cette représentation existe à toutes les époques}



\includegraphics[width=15.981cm,height=15.558cm]{FaivreMartin5conf-img/FaivreMartin5conf-img20.jpg}
\captionof{figure}[Siamon roi en Sphinx (Louvre)]{Siamon roi en Sphinx
(Louvre)}
}


On voit les offrandes faites au dieu}


il y a donc des éléments de décor aussi sur le mobilier du temple, et
également en relief sur les murs}


\textbf{Le roi a une destinée post mortem}}


Si en \kmt tout le monde a une destinée après la mort, pour le roi
c'est un peu spécifique, il a une vie éternelle, mais
il rejoint le monde des dieux, et ce dès les textes des pyramides}


Dans les tombes du Nouvel Empire, on voit des scènes où le roi est face
au dieu (on reste dans la logique de la thématique de 
l'offrande), mais il est dans la phase de transition
où il va vers cet autre monde et où Am Douat lui ouvre les portes et
l'iconographie est spécifique, on nous représente le
voyage des 12 heures de la nuit du soleil car on associe la destinée
post mortem du roi à ce que vit le soleil chaque nuit}


Le soleil meurt le soir et renaît le matin, et ce que le soleil de la
nuit vit chaque nuit, le roi vit cela après sa mort}


On sait aussi que c'est une évocation à mettre en
parallèle avec les rites de l'embaumement}


Cet aspect divin de la personne du roi conduit à une imagerie funéraire
tout à fait spécifique en \kmt, et très importante et on construit en
pierre (en Mésopotamie, c'est en brique), }


Il y a non seulement le décor du temple, mais il y a également le décor
de la tombe, avec cette destinée funéraire spécifique, alors que le roi
mésopotamien , nous avons forcément un corpus d'images
moins importants, partant d'une architecture de
briques, qui certes  a reçu un décor peint,  (il est évident que dès le
III millénaire en Mésopotamie il y a eu des décors peints, mais ils
sont perdus car ils ont été peints à même un enduit et la présence des
deux fleuves ne facilite pas non plus la conservation des choses)}


Mais il ne faudrait surtout pas penser que seuls les assyriens ont fait
des décors muraux avec leur dalle de gypses sculptées (en réalité on
sait qu'ils ont repris une ancienne tradition de
peinture}


Durant tout ce cours, nous aurons forcément un corpus
d'images moindre pour la Mésopotamie que pour
l'\kmt}

\includegraphics[width=15.981cm,height=8.396cm]{FaivreMartin5conf-img/FaivreMartin5conf-img21.jpg}
\captionof{figure}[stèle des vautours]{stèle des vautours}
\includegraphics[width=15.981cm,height=9.878cm]{FaivreMartin5conf-img/FaivreMartin5conf-img22.jpg}
\captionof{figure}[stèle des vautours]{stèle des vautours}


\clearpage\clearpage\setcounter{page}{1}\pagestyle{Standard}

\textbf{Conférence N° 2}}


\textbf{MISE EN PLACE DE l'ICONOGRAPHIE ROYALE AU IV
MILLENAIRE}}


\textbf{MESOPOTAMIE}}


aujourd'hui, nous allons glisser sur des oeufs entre la
période d'Uruk et celle de Nagada sachant que nous
sommes sur de la chronologie pour la période d'Uruk et
pour la période de Nagada}


Mais le problème principal viendra non pas de l'\kmt,
où nous connaissons bien le passage du IV au III millénaire, (sauf pour
les rois de la dynastie 0), mais de la Mésopotamie.}


Nous avons en effet un problème pour les objets qui viennent du Sud de
la Mésopotamie, car beaucoup d'objets qui sont dans
les musées proviennent du marché de l'art en général
et cela pose un problème, car ce sont des datations qui sont proposées
par une approche stylistique , mais au Proche Orient, il
n'y a pas d'historien
d'art, on est archéologue et l'objet
n'est utilisé que pour le texte qu'il
porte, ou pour faire une présentation, présentation par rapport à la
civilisation sociologique ...}


Donc, avoir un discours de l'art sur la discipline du
Proche Orient, est un peu difficile, car les professionnels de cette
région n'ont pas cette approche, et donc on a parfois
des publications assez incroyables , les archéologues
s'intéressent essentiellement à la couche de
stratification de l'objet trouvé}


Et il n'y a pas d'article ou de
publication récente ou approfondie sur l'étude d
'un objet en le comparant à d'autres
et cela est une particularité de cette discipline}


Les objets que nous verrons proviennent du Sud de
l'Iraq et de la Haute \kmt, il y a donc une zone de
désert que sépare ces deux régions. Pourtant certaines images venant de
ces deux pays devront être mises en parallèle}


Chronologie période Uruk : }


Uruk Ancien 4300 - 3800}


Uruk Moyen 3800 - 3400}


Uruk récent  3400 - 3100}


Uruk final  3100 - 2900 (ou Djemdet Nasr)}


On constate une chose, qui est importante dans le monde mésopotamien,
dans la deuxième période de néolithique, entre le VI et le IV
millénaire, c'est l'apparition des
chefferies et c'est cette organisation en chefferies
qui petit à petit va nous conduire à la royauté.}


le IV millénaire a reçu le nom de période d'Uruk,  à
partir donc d'un site archéologique. En effet, Uruk
est le site de référence car c'est le seul site à ce
jour qui donne le passage entre la culture précédente dire Obeid ,
(néolithique Sud de l'Iraq) et la période de
l'apparition de l'urbanisation que
l'on appelle Proto  urbaine et c'est
cette époque proto urbaine qui correspond au IV millénaire et est
contemporaine de Nagada en \kmt et a reçu le nom
d'Uruk.}


Cette période est donc divisée nous venons de le voir en séquence
chronologique}


Attention, si on se réfère à des ouvrages anciens , le découpage
n'est pas le même, on le faisait démarrer un peu plus
tard et on le divisait en trois phases . Mais maintenant on a affiné
les choses et l'Uruk ancien et l'Uruk
 moyen correspondent à deux périodes dont on ne connaît pas grande
chose.}


mais au milieu du IV millénaire , nous avons un peu plus
d'information et on fait maintenant la distinction
entre Uruk récent et Uruk final,}


en effet jusqu'à une époque récente on disait Uruk
récent : 3300- 2900, mais l'étude de la stratification
des couches nous permet de montrer qu'il y a eu une
évolution durant cette période , aux alentours de 3000 et on a décidé
de distinguer un Uruk récent 3400 -3100 et un Uruk Final 3100 - 2900 ,
également appelé Djemdbet Nsar, }


C'est effectivement grâce à des fouilles allemandes,
scientifiques, dans les années 1930 , qu' il a été
établi une stratification en 18 nivaux archéologiques et si on part de
publications d'après la seconde guerre mondiale, on
voit un affinage des choses et que si Uruk ancien et Uruk moyen restent
toujours mal connus, on a pu dans les années 80 distinguer pour la
dernière période un Uruk récent et un Uruk final}


On voit donc le découpage de la période à partir des 18 couches
archéologiques relevées par les archéologues et c'est 
ainsi que l'on arrive à ce découpage, la couche la
plus profonde  Uruk 18 correspondant à la période la plus ancienne.}


Et les couches 6 à 4 correspondent à Uruk récent  3500 - 3100}


couches 3 à 1 correspondant à Uruk final 3100 - 2900 }


Nous allons donc nous intéresser à ce qui caractérise cet Uruk récent ,
niveau 6 à 4, car c'est durant cette période que deux
choses apparaissent :  \textbf{l'écriture et
l'apparition de la royauté.}}


c'est un période , globalement extrêmement inventive,
on a non seulement l'attestation de
l'urbanisme, mais c'est une période
où l'on voit apparaître beaucoup
d'inventions (roue, tour de potier ...) et tout ceci
révèle une société hiérarchisée, et des villes qui sont dirigées par
une élite sociale.}


On peut aussi établir l'extension de cette culture
d'Uruk, partie du sud de l'Iraq, vers
l'est (Suse), le nord l'est et
éventuellement une partie du delta}


Ce sont des choses dont on ne pouvait pas avoir conscience il y a une
cinquante d'années. A la fin des années 30, on fouille
dans la région d'Uruk et on trouve un même matériel
archéologique et on en déduit que cette culture est remontée vers le
nord est. }


Par contre il existe un zone entre les deux fleuves, où on
n'avait jamais travaillé avant les années 80 et donc
il y a eu une diffusion dans des zones nord et sa diffusion en Turquie
a été découverte récemment., (et on attend beaucoup de ces
recherches).}


En effet ces recherches dans le nord pourront peut être compléter notre
corpus d'images du simple site
d'Uruk}


Donc lorsque l'on parle d'Uruk récent
cela correspond au niveau 6 , 5 et 4. pour les nivaux 6 et 5 nous avons
peu de chose, c'est donc surtout du niveau IV que nous
avons des choses et voit bien qu'au niveau IV
l'agglomération s'est
considérablement agrandie et que le nombre d'habitants
a été multiplié par 10 et finalement que cette agglomération est un
centre de pouvoir avec des petits villages autour et une économie
agricole qui en dépend}


Au départ, il semblerait que ce soit deux chefferies qui ont été réunies
pour constituer une seule et même ville et sur le plan religieux cette
information est importante , elle est loin d'être
anodine, car c'est un site archéologique qui aura
toujours deux espaces cultuels fondamentaux : le grand temple
d'Inanna  et le grande temple d'Anu}


Il s'agit donc de deux villages, devenus deux bourgs,
qui ont été réunis pour ne former qu'une seule et même
ville.}


Les architectures sont importantes car on voit apparaître le plan en
trois parties et que les ateliers des lapidaires sont dans ce que nous
appelons le temple et donc dans un espace urbanisé où il
n'y a pas d'architecture 
identifiable comme étant liée au palais,  Nous sommes donc dans un
moment où un grand bâtiment , considéré sans doute comme la maison
terrestre d'une divinité est le lieu de résidence de
son représentant sur terre et de sa famille, et c'est
également le lieu de stockage , d'organisation et de
fabrication.}


Le temple correspond donc à toute une partie de la ville;}


C'est cette \textbf{période d'Uruk V
que vont apparaître les bulles}, }


\includegraphics[width=6.98cm,height=12.642cm]{FaivreMartin5conf-img/FaivreMartin5conf-img23.jpg}



les jetons sont plus anciens, il est possible que les jetons remontent à
la période d'Uruk ancien, et même avant.}


le fait d'enfermer les jetons dans une bulle est
attesté par des morceaux casés , peut être à la fin de
l'Uruk moyen, mais cela se généralise à
L'Uruk récent et on peut constater une multiplication
énorme durant la période d'Uruk V}


la bulle contient des jetons et sert à mémoriser des choses.
Traditionnellement, on a toujours considéré que
c'était le premier pas vers
l'invention de l'écriture, mais peut
être pas. Mais on peut remarquer que les jetons ont des tailles
différentes, ils codifient donc des choses, (mais on ne sait pas quoi).
ils sont dans des bulles d'argile fermées par un
cachet plat. Mais sur la partie inférieure ou supérieur de la bulle, au
moment où la ferme on met des encoches, et on a pu constater que le
nombre d'encoches correspond toujours au nombre de
jetons contenus à l'intérieur Mais ces encoches
n'ont jamais la forme des jetons  (on a des jetons en
forme de bulle, mais on ne va jamais dessiner de bulle ) et cela
indique qu'il y avait une codification}


Et finalement le fait d'inventer
l'écriture ne résulte pas du fait
d'utiliser les jetons, mais le fait de tracer une
mémorisation de ces jetons par un système de code à
l'extérieur des bulles}


L'écriture apparaît dans la phase
d'Uruk IV}


Un des objets caractéristique de cette période est \textbf{les écuelles
grossières}}

\newline


\begin{figure}
\centering
\includegraphics[width=4.725cm,height=5.29cm]{FaivreMartin5conf-img/FaivreMartin5conf-img24.jpg}
\end{figure}

\textbf{\textcolor[rgb]{0.18039216,0.18039216,0.18039216}{Suse II :
époque d'Uruk (3500 - 3100 avant J.-C.)}} }


\textbf{\textcolor[rgb]{0.18039216,0.18039216,0.18039216}{Ecuelles
grossières à bord biseauté avec marque du pouce à la fabrication}} 
(Louvre)}


on retrouve ces écuelles en milliers d'exemplaires ,
elles sont faites par emboutissage au point,  (on tape au point
l'argile pour lui donner cette forme
d'écuelle) , ce sont des objets fabriqués en série,
trouvés en milliers d'exemplaires dans les tombes,
dans les fossés longeant des bâtiments identifiés comme des temples et
dans des décharges;}


Peu importe la fonction de ces écuelles, mais si on trouve ces écuelles
on peut en déduire que nous sommes dans une couche stratification
d'Uruk, elles nous servent donc de marqueur , et nous
les utilisons pour voir l'expansion de la culture
d'Uruk dans l'ensemble du proche
orient, }


Et un certain nombre de documents écrits durant la période archaïque ont
été retrouvé dans ces coupes, }


Le site d'Uruk a livré près de 3600 \textbf{tablettes}
, sur les  niveaux 4 et 3 et cela veut dire que l'on
voit bien que l'invention de
l'écriture se fait à la fin d'Uruk
récent et que le développement de l'écriture se fait
au début de l'Uruk final;}


en effet Uruk IV correspond à la fin de l'Uruk récent
et Uruk III signifie que nous sommes dans l'Uruk
final}


\includegraphics[width=6.35cm,height=8.361cm]{FaivreMartin5conf-img/FaivreMartin5conf-img25.jpg}



(tablette de comptabilité Uruk récent III, 3200 - 3000}


enregistrement d'une livraison de produits céréaliers
pour une fête au temple d'Inanna (musée Pergam)}


Les tablettes nous montrent la différence entre les pictogrammes
extrêmement archaïques, qui pour beaucoup ne peuvent être lus , par
rapport aux autres tablettes qui ont déjà une disposition
d'un tracé de case et dans chaque cas il y a une
information et un pictogramme,  et un élément chiffré qui  permet de
lire chaque case et de savoir que c'est un
enregistrement de produit de céréales pour la déesse Inanna.}


Donc en très peu de temps, ce qui relève d'
l'Uruk récent, on ne le lit pas, et ce qui relève de
l'Uruk final, cela ressemble à ce que nous aurons par
la suite et on peut le lire}


\textbf{Et c'est effectivement dans cette époque
d'Uruk IV que nous allons voir apparaître
l'iconographie du roi} Mais en réalité nous sommes un
peu mal à l'aise car nous manquons de matériaux,
raison pour laquelle il ne faut jamais oublier dans cette discipline
que certaines données peuvent être balayées du jour au lendemain du
fait de nouvelles découvertes archéologiques}


En effet les images du corpus pour l'Uruk récent se
comptent sur les doigts d'une main et ce sont des
objets qui proviennent soit de fouilles très anciennes et donc pas
documentées, soit du marché de l'art et donc sans
documentation associée à cet objet et c'est notamment
le cas des deux statuettes de roi du Louvre, faites en calcaire ,
achetées en 1930 sans aucune information sur leur provenance réelle}


A partir de là effectivement, on peut , mais c'est un
autre travail, regarder ce que l'on a comme fouille
archéologique à cette époque là et finalement arriver à une provenance
supposée, mais qui ne pourra jamais figurer dans un catalogue, où dans
un inventaire de musée.}


Dans les années 30, tous les archéologues travaillaient dans le sur de
l'Iraq jusqu'au moment où les
français iront dans le monde syrien. Donc nul doute que ces deux objets
, d'une trentaine de cm viennent de cette région. Ils
ont d'ailleurs été analysés et on sait
qu'ils sont faits dans un calcaire grossier identique}


\textbf{Statue des deux rois, Louvre}}


\includegraphics[width=6.174cm,height=8.89cm]{FaivreMartin5conf-img/FaivreMartin5conf-img26.jpg}



En les analysant on s'est aperçu que ces deux objets ne
portent aucune trace de peinture et représentent deux personnages
debout, nus , les jambes l'une contre
l'autre et se caractérisent par le même geste,  poings
fermés et rapprochés l'un de l'autre;
}


Ces personnages ont été identifiés comme étant des rois prêtres ,
partant du principe que certes ils sont dépourvus de source écrite et
on ne connaît pas le contexte de leur provenance, mais
qu'ils portent deux attributs sur la tête, que plus
tard nous retrouverons sur une image où il y aura écrit le mot roi , à
côté du personnage qui lui aussi à ces deux attributs : }


ces deux éléments sont le \textbf{bandeau ou boudin}, pour certains
d'ailleurs c'est
d'ailleurs déjà le bonnet que nous reverrons plus tard
avec Gudéa et la \textbf{barb}e, c'est une barbe ronde
qui a une forme caractéristique puisqu'elle passe sous
le menton et elle n'est pas associée à une moustache}


Malheureusement ces deux objets ont toujours été présentés de face
(jamais de profil ou avec un miroir ce qui permettrait de voir leur
dos) . De dos on peut s'apercevoir
qu'il n'y a aucun travail de
sculpture, il n'y a aucun galbe pour marquer les
fesses, }


C'est donc une ronde bosse qui techniquement est
travaillée comme un haut relief. C'est un bloc de
pierre détaché et on voit donc que ces objets étaient faits pour être
vus de face , et ils ont été travaillés dans ce sens (ce qui explique
qu'il n'y ait eu aucun travail pour
le dos)}


Quand on passe à la période du IV millénaire, on verra que le souverain
mésopotamien n'a pas systématiquement des insignes
régaliens portés sur la tête, il peut être représenté de la même taille
que les hommes qui l'entourent, avec le même type de
coiffure , même type de vêtements, et c'est simplement
la présence de l'inscription , toujours associée à sa
tête, au dessus, à côté, qui indique Lugal, ou En, et qui nous indique
que le personnage est le roi}


On a effectivement un cas, où effectivement celui identifié par les
textes comme un roi, porte ce bandeau et cette barbe,
c'est le prêtre roi de Bagdad}


A partir de là, on prend cet élément et on remonte dans le temps et nous
sommes dans la période où l'écriture est en train de
se mettre en place, ou les objets ne portent pas de texte , car pour
eux c'était évident que le personnage représenté était
le roi. Et si on se place à ce niveau l à, on a donc deux objets
uniques au monde, il n'existe aucun article publié, le
seul élément de comparaison est donc le roi de Bagdad, mais qui est
brisé au niveau de la taille et qui présente un niveau de sculpture
plus avancé.}


\textbf{roi Prêtre de Bagdad}}


\includegraphics[width=6.662cm,height=8.952cm]{FaivreMartin5conf-img/FaivreMartin5conf-img27.jpg}



C'est un roi, et il aurait vraiment fallu le trouver en
contexte, dans son temple ou dans son lieu d'origine,
ce qui nous aurait permis d'affirmer que nous sommes
bien dans la période d'Uruk IV, car en plus il y a
deux choses qui doivent attirer notre attention, c'est
que toutes les autres images de roi que nous verrons dans ce cours
datant d'Uruk, il sera toujours habillé  et donc cela
rajoute une problématique à notre sujet. Mais on peut exclure le fait
qu'il s'agirait d'un
faux pour cette époque.}


\textbf{La nudité}, dans la culture sumérienne (donc plus tard) est
associée à certain moment du culte , et c'est
notamment le cas pour le rituel de libation que
l'officiant qu'il soit souverain ou
prêtre, est représenté nu face à la divinité}


ce ne sont pas des gens qui représente la nudité chez
l'adulte de façon volontaire, car ce sont des gens
comme les égyptiens et d'autres civilisations, où le
costume est au contraire utilisé comme valeur sociale, et au contraire
on utilise l'image de la nudité pour signifier
l'humiliation, et donc l'anéanti et
avec tout ce que cela peut avoir comme signification, le vaincu et plus
largement la domination de l'ordre sur le chaos.}


Ces images de nudité sont donc assez troublantes pour des images aussi
anciennes, ou peut être que l'on se trompe avec cette
interprétation basée sur des sources du début du III millénaire ; car
pourquoi ne pas imaginer l'inverse et
qu'à cette époque là, le seul officiant étant le roi,
il est absolument normal de le représenter nu, car
c'est l'évocation de son rôle premier
d'être le représentant du dieu sur la terre et le seul
qui officie.}


en effet on ne sait pas s'il existait une classe
sacerdotale à cette époque}


Mais en tout car c'est remarquable, dans la
civilisation sumérienne pour être mis en évidence}


deuxième point : \textbf{la position des mains} . Mais  peut être est ce
une perte de temps}


On peut se dire  tout simplement, que l'artiste
n'était pas doué. Ce sont des gens qui pendant trois
millénaires vont être représentés devant la divinité les mains jointes
ou les mains en l'air dans le geste de la prière, et
le sculpteur se serait simplifié la vie en évoquant juste le geste de
la prière , en fermant les poings et en les mettant côte à côte}


Cette série de statue (avec celle du Louvre) constituent des objets
fondamentaux}


traditionnellement, dans les manuels , nous verrons cette statue de
Bagdad sans provenance, et daté de l'Uruk récent.}


mais pour Madame FAIVRE MARTIN, si on regarde attentivement cette
oeuvre, on peut remarquer immédiatement qu'elle
n'a pas le même niveau de sculpture que les rois
prêtres du Louvre, elle est plus aboutie et travaillée. Si on regarde
la musculature du personnages, qui est présentée, même si elle est
simpliste, et mis en parallèle avec les statues du Louvre elle donne à
celles ci d'être deux sculptures à peine dégrossies, 
}


dans cette statue de Bagdad on a l'impression
qu'il devait tenir quelque chose, et cela est
intéressant car plus tard dans les années - 2600, on aura des
représentations de gens tenant une petite boîte qui contient
l'offrande que nous apportons à la divinité. Est-ce
déjà cela ? et du coup ce serait également cela pour les deux statues
du Louvre.}


On peut remarquer également que le travail est plus abouti pour le
bandeau, il y a une démarcation entre la calotte et le bandeau, et
derrière la tête on peut voir une masse , comme si on avait une sorte
de \textbf{chignon }(chevelure ramenée en chignon),  La barbe, par
contre est identique , elle entoure le petit menton, et il
n'y a pas de moustache et on a le même type de bouche}


Si on regarde ces oeuvres de profil, on constate que la barbe remonte
bien jusqu'à la base du bandeau, comme si
c'était un seul morceau et du coup ce ne serait plus
une barbe, mais un élément de parure qui aurait été raccordé et cela
expliquerait le fait qu'elle passe bien en arrière du
menton, comme si c'était une sorte de mentonnière,}


On peut voir qu'elle est également incisée de lignes
horizontales}


Le sourcil est fait en une seule ligne continue, les yeux étaient
incrustées et la forme du visage est la même que ceux du Louvre}


On peut aussi se poser la question, s'agit-il
d'une oeuvre inachevée ?, on peut en effet se poser la
question pour les deux statues du Louvre, et on pourrait imaginer
qu'elles proviennent d'un atelier .
Nous n'aurons jamais la réponse à cette question, mais
on peut se la poser}


En résumé, les deux statues du Louvre datent-elles de la même période
que cette de Bagdad, et donc de l'Uruk récent, ou
avons nous un décalage dans le temps, ou une provenance différente ?}


Mais en l'état de nos connaissances, notre corpus
s'arrête là pour les statues en trois dimensions,  et
donc leur datation d'Uruk récent peut être remise en
cause }


Selon Madame FAIVRE MARTIN, on peut se demander du fait de
l'aboutissement du travail qui est différent si ce ne
sont pas des statues d'époque différentes}


\textbf{Stèle de la chasse vers 3300}}


\includegraphics[width=7.615cm,height=10.964cm]{FaivreMartin5conf-img/FaivreMartin5conf-img28.jpg}



Elle fait 80 cm , est en balzate}


elle date Uruk niveau IV, et c'est très important car
c'est le seul objet qui soit bien daté et cela va nous
donner des informations}


C'est la plus ancienne représentation connue dans la
culture d'Uruk du thème du souverain chasseur, ici
c'est la chasse du lion qui est représentée, thème qui
traversera les millénaires et on aura ensuite les fameuses chasses
assyrienne de Ninive., même les perses reprendront ce thème }


On peut observer sur cette stèle deux registres et nous avons deux fois
le personnages , ressemblant au roi prêtre, nous avons bien le bandeau,
la barbe, et l'intérêt de ce bas relief est que nous
bien l'élément qui est sculpté sur la statuette de
Bagdad, à savoir le chignon (attention ce sont des mots, nous
l'appelons chignon, mais en réalité on ne sait pas
exactement ce que c'est)}


\textbf{Un bandeau, une barbe, un chignon, c'est donc
un souverain qui est représenté,} on voit qu'il est
vêtu d'une \textbf{jupe lisse} qui
s'arrête au genou, avec juste
l'indication de la ceinture et
l'homme est pied nu}


On voit donc que l'homme s'attaque à
des lions. Est ce deux moments de chasse qui est représenté ? }


dans la partie supérieure , nous avons un roi prêtre face à un lion
qu'il transperce de sa lance, et en bas un roi prêtre
et son arc (et des gens qui ont travaillé sur les flèches de cette
période en Mésopotamie ont remarqué que sur cette stèle la flèche
représentée n'est pas une flèche très performante pour
s'attaquer à un lion, à une époque où
l'on sait que les mésopotamiens savaient réaliser des
flèches bien plus performante pour chasser le lion)}


Cela nous pose donc un problème s'agit il
d'une chasse réelle ou d'une chasse
symbolique ?}


on peut se poser la question car s'ils avaient voulu
représenter une chasse réelle, on ne voit pas pourquoi ils
n'auraient pas représenter une flèche adéquate}


Il est possible (cet objet a fait couler beaucoup
d'encre et les interrogations sont
d'ailleurs partis dans tous les  sens) , de se
demander également s'il s'agit du
même bonhomme sur la stèle.}


Madame FAIVRE MARTIN remarque qu'il
n'y a aucune ligne de sol , et dans la composition
tout est décalé, il n'y a aucun alignement. Mais
surtout, on voit bien que dans les proportions de ce qui est
représenté, l'arme est énorme, et
c'est un élément que nous reverrons sur
d'autres objets,  (c'est aussi une
caractéristique que nous retrouverons en \kmt)}


En effet pour montrer la supériorité du souverain sur les autres hommes,
et finalement une force qui tient du monde surnaturel, il aura toujours
des armes énormes, par rapport à sa taille }


en effet, si on regarde la taille du bonhomme et celle de
l'arme, il lui est impossible de la tenir et de
s'en servir et cela  nous le reverrons donc. Or il est
évident qu'il ne s'agit pas
d'une erreur du sculpteur, une erreur dans les
proportions car nous ne sommes qu'au III millénaire,
cela de toute façon nous le reverrons plus tard}


Cela signifie qu'il y a des moments où si
l'on veut représenter le souverain
d'une façon particulière, l'arme
qu'il tient en main est énorme, et cela implique donc
qu'il y a déjà une tournure d'esprit
dans ces phases ancienne}


Cet objet est daté d'Uruk IV et il sert de marqueur de
datation, }


\textbf{En effet quand on retrouve cette  image du roi, avec son
bandeau, sa barbe ronde, son petit chignon, sa jupe lisse, nous sommes
dans l'Uruk récent}}


\textbf{LES SCEAUX CYLINDRES}}


ils commencent à apparaître et ils sont liés à
l'apparition de l'écriture à là nous
sommes dans une période charnière}


On a vu dans l'introduction que
l'écriture apparaît durant l'Uruk
récent et se développe durant l'Uruk final}


Visiblement, (on en est même sur) les premiers sceaux cylindres datent
de la période d'Uruk récent, et même de niveau V,
C'est à dire qu'avant les premières
tablettes nous avons déjà des sceaux cylindres, qui sont faciles à
couler sur la bulle.}


mais le problème c'est que pendant longtemps , ils ont
été laissé de côté, et donc s'ils ne proviennent pas
de fouilles allemandes d'Uruk, on ne connaît pas la
couche de stratification dont ils sont issus et donc
c'est par comparaison que nous les daterons (ex
fouilles françaises un peu anarchiques)}


C'est ainsi que si l'on veut dater un
sceau d'Uruk IV, il faut que le personnage ait une
barbe ronde, un bandeau, un chignon, une jupe lisse et il faut
également regarder la taille de sa lance, }


ce sont donc les plus anciennes images attestées des personnes et on
peut les regrouper ensemble, }


exemple d'un sceau où l'on voit des
personnages en haut agenouillé et entravés, dans un autre on a des
personnages entravés (mais en Mésopotamie on les représente totalement
désarticulés , tête en bas, corps de travers, tête bêche et cela
signifie qu'ils sont des gêneurs et donc entravés)}


Or dans les fouilles, nous n'avons trouvé aucune trace
de conflit durant la période d'Uruk, il existe des
traces de conflit visible l'archéologie ou par la
littérature avant les dynasties archaïques, mais vers - 2600.}


Donc nous ne sommes pas dans la commémoration d'un
évènement historique, mais dans l'affirmation 
d'une organisation, et cela si on regarde les images
égyptiennes de l'époque de Nagada, on retrouve
également le souverain, des personnages entravés, sans que
l'on sache à quoi correspondent ces personnages
entravés (mais en \kmt c'est un peu différent car
nous savons qu'à cette période, le Sud de
l'\kmt va absorber le nord}


Mais il faut faire attention car avec notre culture nous voulons
rattacher une image à un fait, or les mésopotamiens ne sont absolument
pas dans cette culture, et l'image commémorative
d'un fait pour eux a une valeur prophylactique}


les sceaux sont frustrant pour nous, car nous sommes à une époque où ils
ont valeur puisqu'il représente une marque de pouvoir,
richesse pour son propriétaire, mais en même temps il
n'y a aucun signe épigraphique sur cet objet (cela
viendra très tard en fait, en effet pour voir le nom
d'une personne sur une statue ou sur un sceau il
faudra attendre - 2600 environ)}


Avant la simple image suffit et le porteur du sceau est un personnage
identifié comme étant important}


Dans l'ensemble les sceaux sont dans un état
déplorables, on remarque que c'est une gravure dans le
creux, et c'est donc plus difficile à faire que du bas
relief ou de la ronde bosse, et que finalement il y a beaucoup plus de
détail dans la représentation d'un sceau}


les images sont plus nombreuses pour la période d'Uruk
final, cela correspond au niveau III , II et I des couches
d'Uruk et à la période de transition, exactement comme
en \kmt, 3100- 2900, cela veut dire que l'on est
exactement contemporain de Nagada III, dynastie o et de la première
dynastie}


Attention dans les ouvrages anciens, la culture de
l'Uruk final n'existe pas car on
parlait d'une culture de \textbf{Djemdet Nasr} et cela
vient du fait qu'en 1928 les archéologues anglais qui
travaillaient à Uruk , ont identifié une culture type Uruk mais
beaucoup plus au Nord et surtout plus aboutie que ceux que les
allemands trouvaient à Uruk et ils ont crée cette culture de Djemdet
Nasr, qui pour eux était contemporaine d'Uruk Récent}


Mais plus tard on s'est rendu compte
qu'il s'agissait de la même culture,
et qu'il y avait eu une mauvaise compréhension des
archéologues anglais et à l'heure actuelle on ne garde
cette expression de Djemdet Nasr que pour la région de
l'affluent du Tigre, car effectivement dans les années
60 on a fouillé dans cette région et on a pu
s'apercevoir qu'à la même époque
qu'Uruk c'était une culture un peu
différente et que cela valait donc la peine qu'elle
ait un nom différent}


Les sceaux sont faits dans des matières différentes
d'abord en calcaire puis on les fera dans des pierres
de luxe, associé  au développement des élites urbaines et royales}


\textbf{Sceau musée de Bagdad  (vers 3000)}}


\includegraphics[width=13.012cm,height=5.808cm]{FaivreMartin5conf-img/FaivreMartin5conf-img29.jpg}



L'iconographie du roi est attesté pour les sceaux de
cette époque  et il a été trouvé dans un temple
d'Uruk, niveau III, et il représente le roi sur un
navire qui avec une image centrale d'un personnage
dans une position proche de ce que nous avons vu
jusqu'à présente, debout, main jointe dans le geste de
la prière, barbe ronde avec une petite pointe au menton,  bandeau
chignon, et une jupe.}


Mais là on voit une particularité , il y a un incision au niveau de la
jupe et elle a un quadrillage ce qui n'est pas attesté
avant Uruk III, (et il n'y a pas
d'erreur pour le moment, ce quadrillage nous sert donc
de marqueur de datation}


A partir d'Uruk III, il y a une mise en scène de la
personne royale, d'une part avec
d'autres personnages , des représentations
architecturées de la rive (on suppose), d'éléments et
d'animaux}


Il faut aussi regarder la représentation du bâtiment, avec ces deux
hampes, motifs très caractéristiques que nous retrouverons}


\textbf{Sceau du Louvre  roi prêtre et son acolyte nourrissant le
troupeau sacré}}


\includegraphics[width=14.041cm,height=5.009cm]{FaivreMartin5conf-img/FaivreMartin5conf-img30.jpg}



Là encore nous retrouvons notre petit personnages  : bandeau, chignon,
barbe , jupe avec même un effet de transparence et on voit derrière le
roi un petit personnage qui porte ce que le roi porte, à savoir des
épis}


si on reconstitue l'ensemble du sceau,  on peut
supposer que le roi fait face à un troupeau qu'il est
en train de nourrir}


De chaque côté de l'image, on voit
l'évocation de deux hampes, que nous avons déjà vu}


Pour analyser les choses, il faut se référer à des textes écrits au
XXVII et XXVI siècles avant Jésus Christ, quand nous avons les premiers
textes concernant le rôle du roi par rapport aux dieux et il est
indiqué qu'il doit nourrir le troupe sacré de la
divinité Inanna, }


et on sait également que dans la culture sumérienne, le troupeau
représente non seulement les animaux sacrés qui étaient élevés dans le
temple, mais aussi l'humanité}


Peut être que dans le monde mésopotamien, ces troupeaux sont une
évocation des hommes et des femmes, alignés sous la forme
d'un troupeau, car les textes sumériens disent que le
roi est garant de tout l'équilibre, et il doit veiller
à ce qu'il 'y ait pas de famine, que
tout le monde puisse manger.}


Monsieur FOREST qui a beaucoup travaillé sur le IV millénaire , était
persuadé que l'on ne se trompait pas en proposant ce
genre de lecture : troupeau sacré, mais aussi évocation de la
population.}


\textbf{Sceau du British Muséum}}


ll a également été acheté sur le marché de l'art, et on
voit une mise en page différente, car le troupeau est disposé de part
et d'autre des hampes.}


Ces hampes c'est l'idéogramme du
roseau, qui sert dans l'écriture cunéiforme à écrire
le nom de la déesse Inanna, déesse maîtresse des troupeaux.}


aussi ces hampes que l'on voit sur les sceaux sont là
pour évoquer une architecture difficile à réaliser , et que les
lapidaires ont donc utiliser l'image de ces deux mats
pour évoquer une architecture difficile à réaliser, et que de toute
évidence à l'entrée sur sanctuaire il y avait des mats
, qui permettaient d'identifier le temple}


\textbf{le sceau Lapis lazulis d'Uruk à Berlin (vers -
3000)}}


\includegraphics[width=15.981cm,height=6.491cm]{FaivreMartin5conf-img/FaivreMartin5conf-img31.png}



c'est le sceau le plus abouti, et il est en marbre et
au dessus il était décoré d'un petit bélier en cuivre,
pour être porté en bandoulière}


Il a été trouvé dans les environs d'Uruk et appartient
à une série de petits sceaux officiels, trouvé dans un dépôt daté
d'Uruk III, à cause du motif de la jupe à chevron(?)}


Ce sceau représente l'image la plus aboutie et est la
plus ancienne représentation, avec un jeu de symétrie parfait , le roi
tient des rinceaux de végétaux où l'on a un motif de
rosettes à huit pétales et ce motif est très précisément dessiné . Il y
a aussi un troupeau représenté en symétrie, les animaux étant dressés
sur leur patte pour tendre la tête vers la nourriture . A
l'arrière des animaux on voit les deux hampes, et
surtout en dessous les deux vases, comme celui retrouvé à Uruk (que
nous verrons tout à l'heure) Mais sur ce vase
d'Uruk on a l'entrée du temple
marquée par deux mats, cela nous permet de dire que dans cette scène du
sceau de Berlin nous sommes à l'intérieur
d'un temple, puisque le roi et les animaux sont
derrière ces hampes}


Cette représentation est unique, il n'y a pas un défilé
d'animaux devant le roi qui porte des épis,  mais au
contraire nous avons un foisonnement d'épis (et Mr
FOREST disait que c'était la première représentation
de l'arbre de vie avec une représentation symétrique
des animaux de part et d'autre , Pourquoi pas ?)}


d'ailleurs en Mésopotamie, cette thématique va
perdurer,  à savoir les animaux dressés sur leurs pattes.}


\textbf{vase d'URUK - Bagad}}


\includegraphics[width=15.981cm,height=15.699cm]{FaivreMartin5conf-img/FaivreMartin5conf-img32.jpg}



il a été trouvé à Uruk, dans le quartier du grand temple
d'inanna, il n'y a donc aucun doute,
c'est un vase cultuel, dans une couche archéologique
de niveau III, trouvé en 1933, 1934}


et c'est important, car nous sommes sur de sa datation
et il peut donc nous servir de marqueur pour dater
d'autres objets similaires où l'on
retrouve la même iconographie ou la même forme de vase}


si on observe ce vase, c'est émouvant car on peut
constater qu'il a subi des restaurations  et la
première date de l'antiquité, donc au moment où il
était encore en usage. Cela signifie donc que c'est un
objet qui est resté longtemps dans le temple}


et de ce fait c'est une pièce unique et fondamentale}


en bas on voit un filet d'eau, grâce à elle les 
plantes vont pousser  et on peut avoir des animaux et ce motif, pendant
trois millénaires nous allons le retrouver}


Pour les mésopotamiens, l'eau est un cadeau des dieux,
qui permet aux plantes de pousser, aux animaux de vivre et ensuite aux
humains de vivre également}


ensuite au niveau supérieur on voit les porteurs
d'offrandes tout nus, et on voit au niveau supérieur
les deux mats, avec de toute évidence la vaisselle qui se trouvait dans
le temple et un individu qui vient accueillir, et il porte une coiffure
avec des choses dressées, est  ce la première pierre à corne de la
divinité ?}


Puisque nous avons les hampes, nous sommes donc dans le temple
d'Inanna, est ce Inana elle même, ou une prêtresse
jouant le rôle d'Inanna. On voit aussi au registre
supérieur un personnage nu et un personnage dont on tient la traîne, ce
qui reste de ce personnage est cassé, mais on peut deviner un motif de
jupe à chevrons}


et donc du fait de ce motif de jupes à chevron nous pouvons en déduire
qu'il s'agit du roi qui arrive avec
son offrande}


et là , s'il y a bien un rituel né à Uruk, spécifique
d'Uruk, et qui a la fin du troisième millénaire
donnera lieu à un fête à l'ensemble du monde
mésopotamien, c'est le mariage sacré.}


le mariage sacré raconte l'union terrestre du roi à la
déesse Inanna , le jour du printemps et de là le cycle des saisons peut
redémarrer et il n'y aura ainsi pas de famine}


Et on sait que le rôle d'Inanna était joué ce jour là
par une prêtresse.}


il est donc tentant : cet objet vient d'Uruk, il
représente un être féminin sortant du temple d'Inanna
qui accueille le roi}


ce pourrait donc être la plus ancienne représentation du mariage sacré
qui ne sera plus représenté ainsi ensuite.}


et là c'est intéressant car nous sommes encore dans une
époque où ils dessinent , ensuite non, car les sources seront
uniquement textuelles, et il n'y aura plus
d'images}


Ce sont des gens qui ont beaucoup de mal à représenter le monde divin,
l'antropomorphisme même }


dans le monde mésopotamien  à toutes les époques,  on sent bien
qu'il y a quelque chose qui fait
qu'ils sont gêner de représenter les  êtres
surnaturels, tellement sublimes, aussi proche d'eux
physiquement et donc la divinité sera évoquée, car ils rechignent à
représenter les divinités fondamentalement importantes}


Donc des choses aussi graves que le mariage sacré, dont tout
l'équilibre dépend, ils ne le mettent pas en image}

{\centering\sffamily
\textbf{l'\kmt}
\par}


Les objets proviennent tous de la Haute \kmt, et majoritairement
notamment du site d'Abydos.}


depuis les années 1980; on sait que ces objets datent de Nagada II,
phase finale, dite Nagada D, et cela permet de dater le fameux couteau
de Gebel el Arak, acheté par le conservateur du Louvre à un antiquaire
du Caire en 1914}


\textbf{COUTEAU DE GEBEL EL ARAK\ \ }}


\includegraphics[width=8.885cm,height=13.326cm]{FaivreMartin5conf-img/FaivreMartin5conf-img33.jpg}



\includegraphics[width=15.981cm,height=15.593cm]{FaivreMartin5conf-img/FaivreMartin5conf-img34.jpg}



la lame était détachée du manche et le remontage s'est
fait entre les deux guerres}


Il y a donc sur l'une des faces de ce manche, une
thématique de la chasse, avec l'image
d'un homme qui peut surprendre car il
n'est pas représenté à l'égyptienne,
sauf dans la règle de la représentation égyptienne de face et de profil
: il a bien le visage de profil, l'oeil et
l'épaule  de face et tout le reste de profil, et cela
a conduit pour certains à dire que ce n'était pas de
l'art égyptien, et de cette petite phrase a coulé
toute une littérature}


Ce qui est ennuyeux c'est qu'il est
très rare de rencontrer quelqu'un qui soit compétant à
la fois dans l'histoire de la Mésopotamie et dans
celle de l'\kmt, }


il y a eu tellement d'ineptie , que Madame FAIVRE
MARTIN donne même l'interdiction de lire certains
manuels sur ce sujet !}


Il est vrai que cet objet est troublant, certains en ont même déduit que
l'\kmt aurait été conquise par la Mésopotamie, mais
en réalité il ne faut pas oublier que les objets circulent plus
facilement que les hommes, et il y a eu forcément des objets
mésopotamiens qui sont venus en \kmt, à Abydos, et ce sans les
mésopotamiens}


de surccroît, cet objet a été trouvé en haute \kmt, et à ce jour on
n'a jamais trouvé la moindre trace
d'une culture d'Uruk}


en réalité, nous ne sommes pas capables d'expliquer le
pourquoi du comment, et il faut donc l'étudier de la
façon la plus intelligente possible}


Quand on dit que ce coteau donne une représentation typique du Proche
Orient, c'est faux, }


en réalité il faut partir du fait que nous avons une image royale
typique de la culture d'Uruk, }


Ce n'est pas cette image qui vient du Proche Orient et
il faut l'étudier en deux temps}


Il y a l'image de l'homme, et là il y
a effectivement une influence d'Uruk, }


mais il y a aussi l'homme et les lions et cela est un
thème}


Pour l'image de l'homme,
c'est vrai que nous sommes embêtés, car nul ne doute
que cet objet vienne du cimetière d'Abydos et donc
cette représentation humaine est troublante car elle correspond à
l'iconographie du Proche Orient. IL y a même la jupe
lisse ce qui nous fait penser à un objet d'Uruk récent
(3500 - 3300) et cela colle avec la période de Nagada II 3500- 3200}


Mais il ne faut pas oublier un objet qui voyage beaucoup, objet de
prestige et donc de cadeau, le sceau cylindre, le sceau est donné
c'est un beau cadeau, c'est un signe
important de pouvoir et de richesse}


Donc un sceau comme tout objet de prestige voyage, (on a bien trouvé en
Bretagne des haches qui provenaient de Russie), et voyage seul, }


en plus il ne faut oublier que nous ne savons pas grand chose sur la
navigation dans le golfe persique à cette époque}


Pour Monsieur FAROUD, cet objet serait l'évocation de
l'autre monde et que le sculpteur égyptien aurait su
que cela représentait des gens d'ailleurs,  et cela
aurait donc été une façon de représenter les gens
d'ailleurs, c'est possible.}


mais il faut s'intéresser au thème de cette image, ce
n'est pas le thème du roi prêtre, mais le thème du
maître des animaux.}


Quand on a un homme qui maîtrise des animaux, ce n'est
pas un prédateur, il les retient mais ne les tue pas et
c'est important. Ce thème apparaît très tôt en
Mésopotamie, dès le Néolithique à Suse, mais là il y a une chose
intéressante et ce sont les anglais qui ont travaillé dessus : }


le Maître des Animaux au Proche Orient, où que l'on
soit, se caractérise par le fait qu'il ne tue pas les
animaux,  mais les maîtrise (lions, caprins , éventuellement des
serpents), mais cet homme est toujours nu, sans attribut aucun, ou
alors avec juste l'évocation de la ceinture }


Donc sur ce couteau nous avons bien une image sumérienne, mais cet
assemblage d'un roi sumérien, dans
l'attitude du maître des animaux , pour le moment ,
est inconnu du répertoire mésopotamien, (du fait qu'il
soit habillé)}


Et c'est en cela que ce coteau est surprenant car dans
les images vues précédemment, notre homme roi prêtre, nous
l'avons vu, nourri les animaux mais nos
l'avons pas vu dompter ou retenir les animaux  et
c'est pour cela qu'il faut procéder à
une étude en deux temps, image de l'homme et thème}


\textbf{En \kmt, à la période de Nagada II on a
d'ailleurs la peinture de la tombe 100
d'Heriakonpolis}}


\includegraphics[width=7.615cm,height=5.533cm]{FaivreMartin5conf-img/FaivreMartin5conf-img35.png}



et là c'est intéressant car nous avons une image que
jamais en  Mésopotamie on associera à un nom propre, mais simplement le
Maître des Animaux, dont la traduction signifie nu, vêtu
d'une ceinture }


mais peut être que dans la culture égyptienne cela signifie autre chose,
pour représenter la terre , il y avait le lion de
l'est et le lion de l'Ouest,  et
qu'il y aurait là une évocation du nil, et donc chacun
des lions correspondrait à une rive et cela traduirait la domination
animale sur les rives est et Ouest (?)}

\clearpage\clearpage\setcounter{page}{1}\pagestyle{Standard}

\textbf{conférence N° 3 Mésopotamie, \kmt}}


On a pu constater que les attributs du roi sumérien étaient assez
limités, de même que Les insignes qui lui sont associés, puisque
finalement, en dehors des éléments arborés qu'il peut
tenir pour nourrir le troupeau , on ne l'a pas vu
tenir un spectre ou une chose comme cela. On ne l'a
pas vu lui même dans un bâtiment, on l'a vu simplement
associé à deux hampes, mais les hampes ce n'est pas
lui, ce sont le temple}


Et finalement, c'est cela qui est intéressant, dans
l'iconographie du roi sumérien, 
c'est plus ce qui est lié aux dieux qui est mis en
valeur, plutôt que le roi lui même et cela va être la différence
importante, entre le monde mésopotamien et le monde égyptien}


Nous avions vu la tombe 100 d'Hériakonpolis, où là il y
avait peut être l'image la plus proche du monde
oriental, car l'homme qui maîtrise les animaux est
vêtu d'une simple ceinture, hasard ?,cela ne se peut,
relation par des objets ? la question reste ouverte.}


Nous allons nous intéresser aujourd'hui \textbf{à
l'apparition de l'iconographie
royale, }en se posant la question de la datation des objets}


Effectivement, depuis la découverte à Abydos des couteaux à manche
d'ivoire, (notamment le couteau de Gébel el Arak
(feuille d'or travaillée au repoussé) , on date ces
objets et ces couteaux de Nagada II D}


parmi ces manches de couteaux, il y en a un qui est également
exceptionnel et qui est au Métropolitan Muséum, et qui présente une
iconographie différente, qui conduit à le dater de la phase Nagada III
B2, c'est à dire de la phase qui correspond à la
dynastie des rois zéro}


Nous avons également un autre manche de couteau à Oxford, qui présente,
mais la lame est très abimée, comme intérêt d'avoir
des éléments comme le serpent et l'éléphant, que
l'on trouvera sur les palettes historiées du début de
Nagada III, mais avec le reste illisible . mais on voit quand même la
bossette, et la représentation d'hommes agenouillés ,
bras ramenés dans le dos, ligotés}


On pourrait donc imaginer, si on accepte un relevé dessin plaçant un
individu, invisible, que la présence de ces individus conduit à penser
que de l'autre côté il y avait une présence humaine,
ou non, (nous ne pouvons le savoir) représentant un chef, sans doute un
souverain de la dynastie zéro, que l'on évoquait par
l'image du vaincu, son pouvoir. en effet, ce
n'est pas la commémoration de la victoire}


Et effectivement cela pose le problème de la datation de ces objets qui
proviennent soit de fouilles anciennes, clandestines, achetés sur le
marché de l'art , et sans aucun contexte archéologique
et du coup on ne peut pas affiner cette évocation de
l'apparition de la représentation royale
(d'ailleurs chaque auteur contredit son voisin)}


En résumé, traditionnellement, on établi que les couteaux en ivoire
datent de Nagada II et donc l'époque des chefferies,
et les palettes à fard, à décor de bas relief sont de Nagada III, parmi
laquelle il y a la dynastie O}


Pour certains auteurs, nous avons d'abord les palettes
à fard, avec un décor uniquement animalier et qu'en
phase II la représentation du chef est faite sous la forme
d'un animal et que ce n'est
qu'en phase III que nous avons
l'apparition du chef sous forme humaine avec Narmer et
le Roi Scorpion.}


Et tout cela marche si nous regardons les palettes historiées, mais si
on regarde les  manches d'ivoire des couteaux, cela ne
marche pas , nous avons alors en effet de
l'antropomorphisme à un moment où cela ne marche pas}


Car sur les manches d'ivoire, nous avons de
l'antropomorphisme à un moment où on est censé
représenter le chef par le lion ou le taureau.}


Donc, on ne peut rien dire de plus, mais ces objets, on ne les a pas
associés au corpus qui a été fait jusqu'à présent.}


Ces hommes agenouillés, et on voit là une mise en image identique à
celle du pays de Sumer, veut finalement montrer une notion
d'ordre, et pour évoquer cette  image on prend celle
de l'homme agenouillé, vaincu}


\textbf{Couteau du Métropolitan Muséum}, (New York),  pièce offerte à ce
musée par Carter, qui lui même  l'avait acheté au
Caire, et donc nous n'avons pas de provenance, dont le
manche est très abimé ; mais on peut remarquer qu'il y
a  d'un côté une représentation architecturée, mais
encore une fois on ne peut utiliser que le relevé dessin car le manche
étant très abimé on ne voit rien}


C'est donc à prendre avec une extrême  prudence, il
faut faire confiance à celui qui a relevé le dessin, or cela est
subjectif.}


Il semble donc au vu du dessin, qu'il y ait eu une
architecture et si cela est vraiment le cas, nous avons un motif qui
rappelle les motifs de natte que nous avons sur les fausses portes , 
un personnage debout (mais là pour Madame FAIVRE MARTIN, cela sent
l'interprétation), et d'autres
personnages accroupis et non attachés et qui correspondent donc à la
catégorie des {\textquotedbl} soumis  par {\textquotedbl}  et non à la
catégorie des {\textquotedbl} dominés par {\textquotedbl} , ce qui
n'est pas pareil. Nous avons aussi des personnages à
barbe qui semble tenir un spectre, ce qui pourrait être le spectre
héqua. Mais cela nous pose alors un problème, car ces personnages sont
chevelus  et une barbiche à une époque où ce sont qui sont ligotés que
l'on représente de cette façon}


Sur l'autre face, également très abimée, il y a la
représentation de bateaux,  avec peut être des personnes,  et selon un
auteur qui aurait fait sa thèse sur le sujet (Mr Wilkinson ?), il y
aurait peut être apparition de la personne royale. Mais attention, il
faut se poser la question de savoir si effectivement le relevé dessin
est correct et ce serait alors la personne portant une couronne blanche
(mise en horizontale car on n'avait pas la place de la
mettre en position verticale)}


Et si on part du principe qu'après le Roi Scorpion,
nous avons le motif de la rosette surmontant le dessin de Scorpion, on
peut en déduire que la rosette correspond au roi (mais attention tout
le monde ne fait pas cette lecture)}


Mais on a l'impression d'un personnage
dans un manteau ample (et si on regarde la peinture  ou la gravure sur
les poteries, dès Nagada I on a pu retrouver de nombreux vases où il y
a des petites incisions qui pourraient évoquer la couronne blanche ou
la couronne rouge}


Cela a t il une autre signification dans le contexte  de la première
moitié du IV millénaire ? Cela est possible, nous ne le savons pas}


Il faut donc décaler ces objets non pas à la période des chefferies,
mais à celle du début de la royauté, et il faut les étaler à la période
de l'apparition de la royauté, donc dynastie O, et
donc à la phase de Nagada III, phase B1, B2, C1, ce qui permet de les
dater aux alentours de 3150, 3000 , 2900, donc à une période
contemporaine d'Uruk Final}


Et là les choses ne marchent plus, mais c'est normal
car il n'y a que trente ans que l'on
travaille sur cette question.}


On a vu le problème par rapport aux manches de couteau, mais il y a une
quantité phénoménale de petits ivoires, qui proviennent
d'Hiérakonpolis , où il y a plein de choses dessinées
et qui sans doute nous donneront des compléments
d'informations sur l'apparition de
ces images}


Il faut donc considérer deux objets : \textbf{LA PALETTE AUX VAUTOURS} 
du British Muséum et \textbf{celle des TAUREAUX} qui appartient au
Louvre, deux objets absolument hors contexte, achetées par les musées ,
donc nous n'avons aucune datation archéologique
possible.}


sont ils contemporains des deux lames vues ci dessus, où au contraire
ces palettes sont elles plus anciennes ? }


Ces deux palettes appartiennent au décor à bas relief, et pourraient
dater de Nagada II  , mais toutes ces palettes à décor de bas relief
pourraient elles appartenir  à Nagada III ? Comment faut il les placer
? Une hypothèse traditionnellement admise  était que les égyptiens
représentaient d'abord leur chef sous la forme
d'un animal et on s'appuyait sur ces
deux objets}


\textbf{PALETTE AUX VAUTOURS}}


\includegraphics[width=7.615cm,height=9.19cm]{FaivreMartin5conf-img/FaivreMartin5conf-img36.jpg}



On peut donc voir un lion de taille importante qui piétine un ennemi
désarticulé et barbu, et nu , et d'autres hommes du
même type sont dévorés par des lions}


Ce que l'on peut observer sur cet objet,
c'est une chose très intéressante, on est à
l'époque où dans l'art égyptien
apparaît la notion de registre,  c'est à dire que
l'on met les choses sur des bandes de décor ( ce que
l'on a finalement, si on observe
l'alignement des prisonniers, hommes agenouillés
vaincus par rapport à la rosette) Il n'y a pas de
ligne de sol incisée, mais les personnages sont bien rangés et la
taille des personnage est adaptée à la forme des objets (cf palette des
taureaux)}


Au contraire ce que l'on observe sur la palette des
vautours, c'est l'opposition entre la
structure verticale avec palmier et girafe et sur une autre face , un
côté anarchique très intéressant car cela correspond en fait à une mise
en image, une façon de représenter une scène que l'on
retrouvera beaucoup plus tard dans l'art égyptien ,
avec les représentations à partir de Ramsès II lorsque
l'on voit les batailles , où là il
n'y a plus de ligne de sol, et des gens dans tous les
sens pour bien montrer la grande agitation.}


C'est donc intéressant car nous sommes avec les Ramsès
à une période où l'on dispose au contraire les choses
en registre, et cela traduit donc une volonté de la part de ces gens
d'exprimer le chaos, car on a bien
l'évocation au niveau de la cupule de la ligne de sol,
bien alignés car ils sont dirigés par des hampes à bras humains, et
finalement l'homme maîtrise le lion et les chiens et
on a l'anarchie du monde animale mais moins agité que
sur la palette  des vautours)}


Cela signifie donc, que dès ces époques, on est déjà dans
l'évocation de l'ordre et du
désordre, de Maât et de l'Isfet et on peut remarquer
que le lion n'est pas centré}


Mais attention, encore une fois, on peut se tromper complètement dans la
lecture d'un objet quand nous ne
l'avons pas en entier (par exemple, si on ne fait la
lecture de la Palette de la Chasse qui est au Louvre, sans utiliser les
morceaux qui sont au British Muséum, la lecture est fausse)}


Donc ici comme il nous maque le haut de l'objet,  il
faut faire très attention, et à la fin du XIX siècle quand cet objet a
été donné au Louvre, il n'a pas été donné au
département des antiquités égyptiennes car cela ne faisait pas
égyptien, mais il a été donné aux antiquités orientales}


et cela résume la réflexion de Monsieur PETRIE au sujet du couteau de
Gebel El Arak, {\textquotedbl} ce n'est pas de
l'art égyptien{\textquotedbl}}


petit aparté : attention pour qu'il y  ait registre, on
n'a pas besoin d'une ligne de sol
bien incisée,  En ce qui concerne les bateaux nous avons des bateaux à
fond plat et des bateaux à fond rond cf tombe N° 100 Hiérakonpolis}


\includegraphics[width=14.605cm,height=4.269cm]{FaivreMartin5conf-img/FaivreMartin5conf-img37.jpg}



(dessin tombe 100 Hiérakonpolis)}


On a trouvé dans le désert égyptien des graffitis représentant des
bateaux à fond plats, dessins qui n'ont pu être faits
que par des égyptiens}


Attention, les égyptiens travaillent davantage l'ivoire
que les mésopotamiens, car c'est une matière première
qui vient d'Afrique et qui donc leur est plus
accessible}


et autre problème liée à la Mésopotamie, résulte du fait que tout ce qui
a été fait en matière organique a pourri avec le temps, les gens ont
toujours habités près du fleuve et ce sont fait enterrer près de
l'endroit où ils habitaient, donc près du fleuve Or
c'est dans les sépultures que nous retrouvons les
objets. Ce qui nous sauve en \kmt, c'est
d'avoir eu la tradition de mettre
l'élite à l'entrée du désert. Et si
les égyptiens n'avaient pas eu cette tradition, il ne
nous resterait pas grand chose. D'ailleurs il suffit
de constater pour en être convaincu, qu'il ne reste
rien des villes anciennes en \kmt, que nous n'avons
aucun cimetière populaire, car nous avons le même problème avec le Nil
qu'en Mésopotamie}


On a effectivement trouvé une présente égyptienne à Gaza, il y a
effectivement des implantations d'égyptiens à Gaza
durant Nagada II, donc au moment où l'\kmt
n'est pas encore unifiée et au moment de la dynastie
O, où l'\kmt est unifiée on peut remarquer
qu'ils retournent dans leur pays
d'origine (après cette période les couches de
stratifications des habitations sont moins grandes)}


Il ne faut pas oublier qu'un grand désert sépare
l'\kmt du Proche orient e que même en Mésopotamie,
déjà, on ne sait pas les contacts qu'ils avaient entre
eux. Aussi, il faut faire très attention quand on parle de contact
entre l'\kmt et le Proche Orient,  de quelle époque
parle-t-on ? et de quelle \kmt parle t on , il faut absolument se
référer à la chronologie , et de l'époque dont on
parle de ces échanges}


\textbf{LA PALETTE AU TAUREAU}}


\includegraphics[width=15.134cm,height=10.583cm]{FaivreMartin5conf-img/FaivreMartin5conf-img38.jpg}



\includegraphics[width=14.57cm,height=21.872cm]{FaivreMartin5conf-img/FaivreMartin5conf-img39.jpg}



\includegraphics[width=14.817cm,height=21.872cm]{FaivreMartin5conf-img/FaivreMartin5conf-img40.jpg}



Sur la palette au taureau, on considère que l'on
retrouve l'idée de représenter un chef , peut être  un
roi, comme un taureau ou un auroch, et il piétine
l'ennemi, de façon intéressante : il lui écrase le
muscle de la cuisse}


Ce qui est perturbant pour l'\kmt, mais ne
l'est pas pour la Mésopotamie, c'est
que dans SUMER il n'y a pas de signe archaïque
d'écriture associés aux images, et du coup on perd
moins de temps, on se contente de lire l'image telle
qu'on l'interprète. En \kmt,
c'est un peu plus compliqué, car finalement sur ces
objets figurent des proto hiéroglyphes et que dans la plus part des cas
on a passé plus de temps à tenté de comprendre la signification de ces
signes, plutôt que de lire l'image elle même}


Ici, nous avons un agencement qui nous montre quelque chose de connu aux
époques historiques, en \kmt, à savoir :
l'association du roi vainqueur d'une
part, le motif d'enceintes crénelées (et ce motif
contenant justement une information transcrite sur ce qui
s'est passé, avec une part
d'interprétation , nom de ville, groupe humains, on ne
sait pas et toutes les propositions de lecture de ces signes ne peuvent
être que des hypothèses}


Le lion se retrouvera sur le trône du roi, le taureau se retrouvera dans
la titulature royale, l'épithète du taureau est
souvent associé à sa majesté dans les époques historiques (taureau
victorieux), La queue dite cérémonielle qui fait partie des insignes
régaliens , qui justement fait partie des éléments qui constituent le
costume royal qui apparaiî sous Nagada III avec la dynastie O, a
toujours été interprétée come une queue de taureau, }


et voilà pourquoi on pense effectivement que cet objet est une image
royale. En tout cas, cet objet devait avoir la même forme que la
palette de Narmer, là c'est la partie supérieure
d'un objet et l'image
d'un taureau piétinant l'ennemi , et
c'est la même sur les deux faces.}


autre exemple, \textbf{petit ivoire découvert à Hiérakonpolis}, ivoire
très intéressant, car il nous donne des informations, des thématiques
complémentaires et notamment un troisième maître des animaux dans
l'art égyptien de cette époque, voilà la troisième
image après celle figurant sur le couteau de Gebel el Arak et la
peinture de la tombe 100, et à ce jour nous n'en
connaissons pas d'autres}


Mais si on prend  cet  ivoire, on voit que l'homme est
torse nu, il a quelque chose de nouer à la ceinture, peut être un
pagne, et ce sont des animaux fantastiques, à très grand cou
(sarcopares ?) qu'il maîtrise et le motif on le
retrouve dans les tombes de deuxième dynastie}


comme le hiéroglyphe servant à écrire le nom d'une
ville, et dans ce hiéroglyphe on a cette représentation de
l'homme tenant des animaux monstrueux, }


souvent lorsque l'on voit ces animaux monstrueux à long
cou, on pense qu'il s'agit
d'un thème sumérien, arrivé dans
l'art égyptien, or il y en dans l'art
égyptien (cf palette de Narmer), pas beaucoup, mais il y en a}


\textbf{Avec la dynastie O, le point très important est
l'apparition du chef comme un homme},  qui montre que
l'on s'éloigne d'une
tradition culturelle préhistorique}


mais la grande frustration, si on prend ce groupe
d'images : animaux, puis le chef est un animal et
enfin le chef est un humain, et ce passage à
l'antropomorphisme est attesté sur trois objets si
l'on ne tient pas compte des manches en ivoire des
couteaux d'Oxford du Métropolitan, provenant du même
déposit, trouvé à Hiérakonpolis, le long d'un mur
datant  du moyen empire (fosse remplie d'objets) donc
1896/4898}


Aussi la seule certitude on est qu'à Hiérakonpolis, qui
est un site important pour la période de Nagada, dynastie Zéro, et ces
objets sont :}


massue de Scorpion}


massue de Narmer,}


palette de Narmer}


Cependant , il y a un autre objet toujours oublié, qui
n'a pas fait l'objet
d'étude de publication, et qui provient du même
déposit à Hiérakonpolis,  qui est aujourd'hui dans le
musée Pétrie Muséum, qui est un objet en calcaire très abimé, et qui
est aussi une massue qui devait à l'origine avoir la
même forme que les palettes de Narmer ou de Scorpion}


Dans les corpus, on oublie toujours cet objet, on ne peut
l'utiliser que pour l'image, car elle
est cassée à l'endroit où il devait y avoir un texte,
mais elle doit dater de la même époque, et elle représente un
personnage qui porte de façon très évidente la couronne rouge, sans
uraeus, et on a bien l'impression (il
n'y a jamais d'uraeus sur les deux
couronnes au moment de leur apparition) que le personnage est assis,
dans un vêtement d'où sortent deux petites mains}


on est toujours tenté d'interpréter ces objets, ces
images par ce que l'on connaît par la suite comme
image dans l'art égyptien ; mais il y avait des
personnages associés portant des objets et c'est une
représentation de Roi en manteau de jubilé. est ce
qu'il n'y a pas de nom de roi par ce
que cette représentation est plus ancienne ? ou pas de  nom car cet
objet est abimé ? On ne le sait pas, on sait juste
qu'il provient du même dépôt  Hiérakonpolis}


Si on observe le relevé dessin fait par Madame Barbara Adams, on voit un
faucon devant le personnage du roi,  mais le fait est que \textbf{le
Sérekh} enfermant le roi du  nom apparaît très tôt dans
l'art égyptien : on le voit sur les vases ou les
poteries , peintes ou incisées, dès Nagada III B, ce qui correspond à
la dynastie Zéro}


Mais les premiers Sérekh sont vides, et le fait de mettre un nom à
l'intérieur apparaît avec le Roi Qua, considéré comme
étant le prédécesseur de Narmer (avec le problème de considérer la
place de Narmer : est il le premier roi de la dynastie Zéro ou le
premier Roi de la première dynastie)}


Cela veut dire qu'ici nous avons un objet avec un
personnage qui porte la couronne blanche sans uraeus, le scorpion , et
le motif de la rosette qui a été proposé comme étant la lecture du mot
roi scorpion. la chose n'est pas enfermée dans un
sérekh. Il est donc intéressant de voir que pour les premiers
souverains de la dynastie zéro,  on identifie le roi par la simple
forme du sérekh, il n'est pas besoin de noter son  nom
et pour le roi Scorpion qui est avant Qua, on écrit son nom à
l'extérieur du sérekh et finalement
c'est à partir de QUA  que l'on 
fusionne sérekh et à l'intérieur nom du roi et à
partir de là les égyptiens vont glisser et faire quelque chose
d'intéressant, qui est le remplacement de
l'image de l'humain par son nom, qui
est un principe tout à fait égyptien, auquel les mésopotamiens pourront
avoir recours, mais de façon un peu différente.}


\textbf{MASSUE DU ROI SCORPION}}


\textcolor{black}{ (environ 3650-3}}


\textcolor{black}{O300 avant notre è}
\includegraphics[width=14.252cm,height=16.087cm]{FaivreMartin5conf-img/FaivreMartin5conf-img41.jpg}
\textcolor{black}{re).}}


\includegraphics[width=15.991cm,height=14.88cm]{FaivreMartin5conf-img/FaivreMartin5conf-img42.png}



Nous avons là une iconographie unique et inconnue par la suite ce qui
est troublant. Nous avons un roi qui est présenté dans une thématique
que nous allons dire agraire, associé au fleuve, aux canaux, à
l'irrigation et le roi a une houe entre les mains.}


On a toujours lu cet objet comme une scène (qui disparaîtra ensuite dans
l'art égyptien) lié justement au roi garant de
l'ordre dans l'agriculture par
l'entretien des canaux d'irrigation}


Il faut se souvenir qu'aux époques historiques, quand
on représente le Rituel de Fondation,  le premier rituel est de
trancher une tranchée avec la grande houe. Avons nous simplement une
scène à conotation agricole ? }


En tout cas, nous sommes dans  la dynastie zéro, nous avons un
personnage plus grand que les autres,  il occupe toute la hauteur
d'un registre et sa couronne touche le registre
supérieur et voit que sa taille correspond à trois registres de décor
derrière lui et à deux registres devant lui. C'est
intéressant, il est centré, représenté de face et de profil et avance
sa jambe gauche, et il porte les insignes régaliens qui perdureront
jusqu' à l'époque romaine. }


On a en effet la couronne blanche hedjet , sans uraeus, un costume
difficile à définir que l'on appelle un corselet à
bretelle et un pagne court attaché à sa ceinture, et
c'est très visible, une chose qui de toute évidence
est la queue cérémonielle que sa majesté conservera par la suite, mais
qui là est vraiment volumineuse, comme si on avait la vrai queue de
l'animal}


Nous avons donc ainsi la preuve qu'à
l''époque Nagada III, dynastie O,
\textbf{des insignes régaliens qui vont perdurer }, apparaissent et
c'est étonnant car ils apparaissent soudainement}


D'autre part , il y a des attributs juxtaposés, les
porteurs d'éventails. C'est à dire 
que non seulement il y a la taille du roi,
l'identification par l'inscription,
les insignes régaliens, et il y a ensuite le décorum autour du
personnage avec le porte éventail, et ces quatre éléments seront
typiques de l'iconographie royale égyptienne}


Cet objet est donc \textbf{le plus ancien objet connu au monde où  le
roi apparaît comme une institution} et en plus nous avons
l'association d'un signe qui donne
son nom par rapport à la massue}


\textbf{MASSUE NARMER}}


\includegraphics[width=8.885cm,height=7.826cm]{FaivreMartin5conf-img/FaivreMartin5conf-img43.jpg}



\includegraphics[width=13.965cm,height=6.625cm]{FaivreMartin5conf-img/FaivreMartin5conf-img44.jpg}



également trouvée à Hiérakonpolis, et elle est plus petite, elle fait 20
cm de haut. Elle est moins compliquée, car plus complète et surtout
elle a une iconographie plus identifiable.}


C'est incroyable de constater qu'avec
Narmer , dernier roi de la dynastie zéro, ou premier roi de la dynastie
Un, , les choses semblent aller très vite, car nous avons la
représentation d'un jubilé, avec le roi sous un dais,
(le dais qui fait marche comme nous l'aurons après
chez Djoser, et partout après), le roi porte la couronne rouge sans
uraeus  et il a en plus une petite barbe sous le menton, manteau
jubilaire  et il tient le spectre et le fouet}


Alors et c'était la question que se posait Barbara
Adams sur cette autre massue trouvée également à Hiérakonpolis, car on
peut voir une ligne incisée et elle pensait que
c'était la représentation d'un dais}


On retrouve les portes éventails que l'on met en
dessous du roi pour montrer qu'ils sont à côté et on a
la représentation , qui pour la première fois va représenter le sérekh
surmonté du faucon, avec le n om du roi en hiéroglyphe et donc une
disposition en deux registres, et en trois registres derrière lui. Nous
avons aussi la représentation du porte sandale et des bornes liées ,
sans doute à la force royale , et des personnages captifs et des
enseignes}


\textbf{PALETTE de NARMER}}


\includegraphics[width=9.843cm,height=16.258cm]{FaivreMartin5conf-img/FaivreMartin5conf-img45.jpg}



\includegraphics[width=11.43cm,height=8.747cm]{FaivreMartin5conf-img/FaivreMartin5conf-img46.jpg}



Nous avons tout ce que nous avons déjà vu, à savoir le roi qui se met
sous la protection du monde divin par la représentation de la déesse
Bath ou Hathor. Au centre  le sérekh avec les deux hiéroglyphes de son
nom, Narmer, }


D'un côté du registre , nous avons le roi en taille 
monumentale, couronne blanche, barbe, corselet à bretelle, pagne,
ceinture, et queue de taureau, pied nu, avançant la jambe gauche et il
a deux attributs dans les mains : attribut non animé  la massue  (hedge
?) qui apparaît sous Nagada III B et l'ennemi}


Pour la première fois,  on voit cette mise en scène dans
l'art égyptien, de l'ennemi tenu par
les cheveux, qui écarte les bras}


et derrière sa majesté, nous avons sur une ligne de sol, le porte
sandale}


la présence du faucon que Madame ADAMS voyait sur la petite massue
cassée du musée Pétrie, les ennemis du delta avec une forêt de papyrus,
et désarticulés, et des signes que nous ne savons pas lire et qui donne
peut être une explication , et qui ressemblent aux hommes que
l'on avait sur le couteau de Gébel el Arak qui
représentent les noyés (à vérifier) }


La disposition est différente de l'autre côté , nous
avons toujours trois registres et c'est le premier
objet où nous avons la couronne blanche d'un côté et
la couronne rouge de l'autre, couronne rouge sans
uraeus, corselet à bretelle, pagne court, ceinture, queue,  massue
hedge, donc les mêmes insignes régaliens, }


et nous avons les mêmes attributs  : porte sandales, là nous avons le
prince héritier, que l'on avait sur la massue, les
enseignes que l'on a toujours associé au roi, les
vaincus la tête entre les jambes et les sarcopares, au cou entrelacé}


Là encore, nous avons un répertoire que l'on retrouve
dans le monde sumérien, qui évoque un monde surnaturel et ce qui est
intéressant en \kmt, c'est que nous allons
l'associer à l'image du roi sur ces
objets datant de Nagada III B, et on ne les trouve plus du tout dans
l'\kmt unifiée , dynastie, 1, 2, Ancien Empire et on
les verra réapparaître durant le Nouvel Empire dans des tombes, ou dans
des scènes de chasse au désert}


Et cela n'est pas anodin, si au Nouvel Empire, nous
voyons réapparaître ces scènes de chasse au désert, cela veut dire
qu'elles sont associées à l'Isfet, à
ce qui est sauvage et qui doit  être dominé}


Il y a donc de forte chance, vu que justement on les tient en laisse,
que se soit cela qui soit évoqué}


et là c'est une image, comme celle de la palette au
taureau du Musée du Louvre, puisque nous avons
l'auroch qui défonce l'enceinte et
piétine}


il est incroyable de voir que ces représentations, finalement, sont les
mêmes et que l'on va avoir par exemple sur les plaques
qui ont été trouvées dans les chambres funéraires de Djeser, 
c'est la même iconographie, costume, corselet,
bretelle, }


Donc on peut en déduire que les choses apparaissent et quand elles
apparaissent  , elles se figent et perdurent au delà du temps}


Sur la ceinture, il devait y avoir des éléments perlés qui tombent sur
le pagne}


dans le détail, on voit bien la massue hedge et un petit spectre et que
le nom n'est pas dans un sérekh, mais un serekh, même
vide, signifie roi}


tant que nous étudions un corpus d'objet provenant du
même dépôt, se posait la question de la datation.
C'est qui est intéressant c'est de
regarder ce qui a pu être trouvé à Abydos, et notamment cette plaque de
Narmer, c'est une étiquette de jarre}


Il n'y a aucun doute le nom du roi est marqué, : nous
avons le sérekh, le nom du roi Narmer et le faucon, mais nous avons
quelque chose de typiquement égyptien , qui est le fait de représenter
le nom du roi comme un oiseau. il attrape l'ennemi
qu'il tient par son aile, ou bras, on ne sait pas
trop, et on voit la massue hedge}


Donc dire qu' à partir du moment où nous sommes dans la
phase III que le roi est représenté de façon antropomorphe, ce
n'est pas complètement vrai. On sait bien
qu'à toutes les époques, on va pouvoir utiliser des
images pour évoquer la présence de la personne royale et là
c'est le hiéroglyphe du roi qui attaque
l'ennemi (on peut voir qu'il y a une
touffe de papyrus sur la tête , comme les ennemis du Nord, que
l'on a sur la palette}


Il y a toujours beaucoup d'informations à retirer des
étiquettes de jarre}


Et si on part du principe que l'on a établi un fait,
c'est aussi bien à l'époque
d'Uruk qu'à la période de Nagada III
B, ce qui est important c'est de représenter le
souverain plus grand, armé, avec des armes de taille importante que
l'on va mettre en image, et donc la domination et
l'affirmation de l'ordre par le biais
d'hommes ligotés , alors que nous
n'avons aucun contexte archéologique qui puisse nous
évoquer de réels conflits à cette époque et qui va conduire dans
l'art égyptien à une mise en image , qui va perdurer
jusqu'à l'époque romaine}


\textbf{PREMIERE DYNASTIE}}


Pour cette période, nous avons de nombreux fragments
d'ivoire trouvés dans différentes tombes et notamment
des morceaux de bois et d'ivoire trouvés dans la tombe
d'AHA , qui est le premier roi de la première dynastie
et l'on retrouve les hommes attachés, ligotés dans le
dos, ou des hommes un genou à terre  attachés par le cou}


il y a en a beaucoup, nul ne doute que ces images étaient censées
représenter l'image du souverain vainqueur}


le même AHA si on prend une plaque provenant de la tombe de sa mère,
NEITH HOTEP, qui provient de Nagada, et nous avons le sérekh surmonté
du faucon, et ce serait alors la plus ancienne attestation du deuxième
titre royal  : NETBY, alors que le roi lui même n'est
pas représenté, ou alors est il évoqué par ce bateau, ce dais et
simplement que fait que le roi soit dans le Dais sur le bateau}


\includegraphics[width=15.981cm,height=13.018cm]{FaivreMartin5conf-img/FaivreMartin5conf-img47.jpg}



Très tôt, les égyptiens marquent , inscrivent les objets, mais pas en
Mésopotamie, il faudra attendre plusieurs siècles, et plus exactement
vers - 2600 / 2500 pour que l'on écrive le nom
d'un roi à côté de son image. Cela signifie
qu'en \kmt, quatre ou cinq siècles avant, la
représentation du Roi est associée à l'inscription de
son nom, et son  nom peut même remplacer son image, comme on le voit
sur ces stèles}


\textbf{IVOIRE DE DEN}}


\includegraphics[width=15.981cm,height=15.981cm]{FaivreMartin5conf-img/FaivreMartin5conf-img48.jpg}



C'est intéressant car il nous place déjà un paysage ,
on peut voir le sol et on le retrouvera plus tard dans les tombes des
Raméssides, Rive Ouest de Thèbes, cette façon de représenter le sol
avec des petits points et on remonte sur le côté pour évoquer la
falaise de sable et la limite de l'horizon dans la
pensée égyptienne.}


Dans cette représentation de DEN, il n'y a aucun doute,
c'est écrit, nous avons la plus ancienne gravure
\textbf{d'un uraeus}, et cet uraeus est sur une coiffe
qui n'est ni la couronne rouge, ni la couronne
blanche, une coiffe de tissus qui n'est pas un némès ,
ni la coiffure hat (?), qui est un némès en boule, mais juste une
coiffure en tissus}


Cet élément qui nous  manquait , on constate qu'il
n'est pas systématique, avant DEN nous ne
l'avons pas,  et en tous cas dans les  périodes
anciennes, nous n'avons jamais ni sur la couronne
blanche, ni sur la couronne rouge et c'est une chose
qui se mettra en place dans le courant de l'Ancien
Empire seulement}


Cette image nous présente le roi torse nu, sans corselet à bretelle}


Autre nouveauté, nous pouvons voir qu'il a un pagne
arrondi, il n'y a pas de languette devant, mais on
peut constater que la queue cérémonielle est moins volumineuse que
celle de Narmer}


remarquons également son attitude, qui ici est très hiératique, les deux
pieds alignés sur la ligne de sol, et les deux bras formant une
diagonale, et l'ennemi tenu par les cheveux}


Sur cet ivoire nous avons un mouvement et pour la première fois, on va 
détacher le pied du sol, pour montrer la puissance du roi, et présenter
des armes énormes, (gourdin, massue hedge) beaucoup trop importantes
pour tuer un ennemi.}


On peut remarquer également que l'ennemi
n'est plus dans la même position  que celle dans la
palette de Narmer, car si on regarde de près on voit
qu'il attrape la jambe du roi par la main}


Nous sommes dans la première dynastie, ce qui signifie que nous sommes
là aux alentours de - 2900 et on a le prototype d'une
mise en image qui nous aurons avec les Toutmossides, sur les pylônes de
Karnac}


\textbf{Etiquette de Jubilé du Roi DEN (British muséum)}}


\includegraphics[width=11.742cm,height=11.742cm]{FaivreMartin5conf-img/FaivreMartin5conf-img49.jpg}



\textbf{étiquette du roi de Den d'après Pétrie}}


\includegraphics[width=12.377cm,height=10.269cm]{FaivreMartin5conf-img/FaivreMartin5conf-img50.jpg}



(dessin de Mr PETRIE fait à partir d'une autre
étiquette de jubilé de DEN, similaire, je crois)}


Cela ajoute à notre corpus, non pas par ce que c'est
une scène de jubilé, car finalement cela reprend ce que nous avons déjà
vu, le dais, la marche, ici couronne blanche avec uraeus,  on voit
également qu'il a un spectre, est le spectre Mérer ? 
ce qui est intéressant ici c'est que
l'on semble voir la double couronne, ce qui est très
rare}


Mais attention, lorsque l'on travaille sur des objets
en bois, ou sur les étiquettes de Jarre, il faut  être très prudent,
car vu l'état de conservation de ces objets, on peut
imaginer}


Ici on voit sur ce fragment l'une des plus anciennes
représentations de la course, mais schématique, ici on voit DEN est
c'est vraiment l'ancêtre de ce que
nous aurons avec Djoser, la course, couronne blanche sans uraeus, barbe
plus longue et spectre}


Là il n'y a aucun doute, cela vient de DEN (le nom est
écrit)  et il est évident qu'il y a une représentation
de la double couronne,  on voit aussi un {\textquotedbl} grand
machin{\textquotedbl} qui pourrait être pris pour un harponneur. }


Et ce qui est très étonnant c'est que finalement en
Mésopotamie à l'époque d'URUK, le roi
est le nourricier des troupeaux sacrés de la divinité,  et en \kmt
c'est l'image du jubilé,
Qu'est ce le jubilé à cette époque de la fin du IV
millénaire ? On ne sait pas très bien à qui il correspond même pour des
époques plus récentes}


Mais c'est tout de même incroyable, si
l'on prend un nombre de représentations royales de
l'époque Thinite, jusqu'à nous,
c'est pratiquement que cela, c'est à
dire que si nous n'avons pas le thème du combat par
rapport à un ennemi, symbolique, c'est le jubilé qui
est représenté , et  il n'y a rien
d'autre}


C'est d'ailleurs pour cela que la
massue du roi Scorpion a autant interpellé le monde, car
c'est le seul objet ou on a le roi avec une houe,
associé à un environnement agricole}


autrement on a le roi en costume de Jubilé, sur le relief de Saqqarah :
est ce le même roi en costume jubilaire, à deux moments, on ne le sait
pas }


On peut avoir les restes d'un faucon (?) et  cela peut
correspondre à l'image que nous avons du roi qui entre
sous un dais, et qui se fait confirmer par des images de divinité le
jour de son jubilé}


Mais on peut aussi se demander si à cette époque, ce sont vraiment des
scènes de jubilé, ne serait ce pas plutôt  cette période des scènes de
couronnement , qui par la suite seront des scènes de jubilé  ? Le
jubilé est une reprise du couronnement, et on a une petite image au
British Muséum d'un roi à couronne blanche sans
uraeus, enveloppé dans le manteau, et sur ce manteau des motifs de vase
incisé,  et l'objet était peint à
l'origine, s'agit - il
d'un objet commémoratif ?}


Troisième roi de la deuxième dynastie : \textbf{NYNETER}}


\includegraphics[width=4.233cm,height=7.126cm]{FaivreMartin5conf-img/FaivreMartin5conf-img51.png}



\includegraphics[width=15.981cm,height=13.192cm]{FaivreMartin5conf-img/FaivreMartin5conf-img52.jpg}



on le voit assis paré, couronne blanche, sans uraeus, avec de la barbe
au menton, brans qui sorte du manteau et semble tenir deux spectres}

\textsf{Image que l'on retrouve également pour le roi
}\textsf{\textbf{KHASEKHEM,
}}\href{http://www.flickr.com/photos/12371736@N00/4613032830/}{\textsf{dernier
roi de la dernière dynastie, , donc dernier roi de
l'époque thinite}}
\includegraphics[width=0.035cm,height=0.041cm]{FaivreMartin5conf-img/FaivreMartin5conf-img53.png}



\includegraphics[width=0.035cm,height=0.041cm]{FaivreMartin5conf-img/FaivreMartin5conf-img54.png}

\includegraphics[width=6.103cm,height=9.807cm]{FaivreMartin5conf-img/FaivreMartin5conf-img55.jpg}

\includegraphics[width=1.762cm,height=3.17cm]{FaivreMartin5conf-img/FaivreMartin5conf-img56.jpg}

\includegraphics[width=6.385cm,height=5.115cm]{FaivreMartin5conf-img/FaivreMartin5conf-img57.jpg}



Statue du OXFORD  et inscription sur la base}


\includegraphics[width=15.946cm,height=16.016cm]{FaivreMartin5conf-img/FaivreMartin5conf-img58.jpg}



statue de Khasekhem au Caire}


Là encore, nous avons le roi sur son trône , couronne blanche, sans
uraeus visible, ne tenant pas de spectre, manteau court, mais petits
plis par rapport aux autres images}


sur le socle ; il y a la représentation des ennemis, donc les ennemis
sont placés en dessous, \textbf{j'ai vaincu, je suis
dessus}}


et on l'avait déjà vu sur les gravures des étiquettes
où les ennemis étaient toujours à un niveau inférieur au roi}


On a la même iconographie sur la statue d'Oxford
couronne blanche sans uraeus, manteau de jubilaire, et même position
des bras, bras gauche ramené sur l'abdomen, et bras
droit posé sur la cuisse et l'évocation des ennemis
sur le socle}


il est troublant de voir que les premières statues égyptiennes pour
l'époque des premières et deuxième dynasties, nous
représente le roi dans ce costume, où nous avons simplement le nom du
roi dans un serekh}


Il y a notamment cette tête de Boston , avec une couronne blanche sans
uraeus, qui ressemble beaucoup  aux deux statues que nous venons de
voir}


\ \ \textbf{Maintenant que nous avons établi
l'iconographie royale , nous allons prendre
}\textbf{un thème et regarder comment il est abordé en Mésopotamie et
en \kmt}}


mais nous n'aurons pas toujours des éléments à mettre
en phase chronologiquement , car il y a des moments où
l' \kmt et la Mésopotamie connaissent des périodes
troublées.}


\ \ \ \ \ \ \textbf{LE ROI EN PRIERE}}


Pour la Mésopotamie, sur une période de 3000 ans, pour le thème du roi
en prière, on peut remarquer que l'on prie les mains
jointes, quelque soit les époques depuis les plus anciennes
représentations et ce jusqu'à
l'arrivée des Perses.}


Le roi est représenté debout ou assis  les mains jointes.
l'idée est : }


\textbf{{}- je suis représenté assis, je prie mon dieu,}}


\textbf{{}- je suis représenté debout, mains jointes, je suis face à
quelque chose qui évoque mon dieu.}}


Je ne serai jamais assis face à mon dieu.}


C'est toujours le personnage le plus important qui est
assis, alors la divinité sera assise et l'humain, fut
il roi, sera debout}


Et il ne faut pas oublier que toutes ces statues royales, quelle que
soit l'époque en Mésopotamie, sont des images de
temple. A ce jour, aucune statue royale n'a été
trouvée dans une tombe. }


Jamais ces images de roi en prière ne sont destinées en Mésopotamie à
une sépulture}


Ces images, de pierre, car nous n'en avons  aucune
attestation en bois ou en métal, sont placées dans  la pièce  qui
précède la chambre de la divinité. et elles sont là généralement pour
perpétuer la prière, l'acte de construction réalisé
par le roi qui est le vicaire humain de la divinité. Il prie les mains
jointes en son nom et au nom de toute l'humanité
qu'il représente.}


\textbf{Voici la main célèbre en diorite de Gudéa, Statue A du musée du
Louvre}}


\includegraphics[width=11.425cm,height=8.574cm]{FaivreMartin5conf-img/FaivreMartin5conf-img59.jpg}



\includegraphics[width=7.549cm,height=13.159cm]{FaivreMartin5conf-img/FaivreMartin5conf-img60.png}

\includegraphics[width=13.647cm,height=10.393cm]{FaivreMartin5conf-img/FaivreMartin5conf-img61.png}



La période de Gudéa  (22ème siècle avant JC) correspond à la période de
la fin de l'Ancien Empire pour
l'\kmt, première période intermédiaire}


Toutes les statues de Gudéa sont en diorite.}


Attention, il ne faut pas oublier que les premiers humains qui vont
tenter de rendre la réalité humaine sont les grecs tant que nous ne
sommes pas dans l'art grec. Nous ne sommes pas encore
dans des expressions artistiques où l'on cherche à
représenter la réalité anatomique , les premiers qui auront cette
conception , encore une fois, sont les grecs}


Ici les textes cunéiformes sont clairs, les oeuvres sont faites pour
être vue de face, elles étaient d'ailleurs collées sur
des banquettes de briques crues (sous les statues on peut voir des
traces de goudron)}


on a donc un sanctuaire et dans l'entrée ces statues
collées}


La statue est faite pour être vue de face, en effet
l'idée est effectivement de représenter le roi les
mains jointes }


mais en sculpture il faut plier les doigts, ce qui est très difficile à
faire,  aussi à partir de la période d'Akkad, on met
les pouces l'un au dessus de l'autre
pour permettre de voir les deux mains en entier, avec les cinq doigts,
d'où cette attitude sculptée un peu irréelle pour les
mains}


On peut remarquer que les sculpteurs mésopotamiens sont moins
performants que les égyptiens,  mais les artistes mésopotamiens doivent
faire venir les pierres en diorite , aux alentour de - 2600 et dans un
premier temps ils ne doivent pas très bien savoir la travailler et
surtout ils n'ont pas les outils nécessaires}


Au louvre, il y a même une statue amusante, le sculpteur
s'apercevant que son oeuvre est ratée, a écrit dessus
le nom du roi et précise qu'il est représenté assis en
prière, alors que jamais on n'indique par écrit
l'attitude du personnage}


Pour ces sociétés  mésopotamiennes, l'\kmt est un
eldorado, car si on prend la période du Nouvel Empire, il y avait sans
doute un bouche à oreille, de roitelets en roitelets}


Mais attention à l'interprétation  : on a beaucoup
utilisé l'image du roi mésopotamien en prière pour
dire que le roi privilégiait la prière sur les conflits mais on peut
lire dans les textes que pour faire des temples le roi ayant besoin de
main d'oeuvre est allé la chercher dans les états
voisins (Elam notamment). il y a donc eu nécessairement des expéditions
militaires pour récupérer cette main d'oeuvre.}


\textbf{Gudéa associé à un plan de Construction (Louvre)}}


\includegraphics[width=9.066cm,height=13.192cm]{FaivreMartin5conf-img/FaivreMartin5conf-img62.png}



C'est le seul objet connu à ce jour du site de Tello,
provenant des fouilles épouvantables faite à cette époque de la fin du
XIX siècle, où l'on allait jusqu'à
demander aux diplomates de fouiller eux mêmes et de rapporter des
objets}


Il faut attendre en effet 1925, pour que Monsieur PARROT  tente de
mettre une certaine discipline dans l'archélogie, }


les fouilles étaient faites de telle manière que Monsieur PARROT voyant
le site de Tello décide de ne pas le fouiller, considérant que  plus
aucune donnée archéologique ne peut être retiré, }


Et c'est très frustrant pour cet adorant, car nous
n'avons aucune donnée archéologique le concernant , et
c'est très dommage car c'est la seule
image mésopotamienne, associée à un plan de temple gravé sur la
tablette qu'il a sur ses genoux et ce texte raconte la
reconstruction du temple pour son dieu et on se plait à rêver que la
tabelette qu'il a sur ses genoux représente ce fameux
temple qu'il a fait construire et que
l'on ne retrouvera jamais , compte tenu des fouilles
qui ont été faites n'importe comment.}


Où les mésopotamiens sont amusants, c'est
qu'ils osnt une notion hiérarchique des choses très
étonnantes, les égyptiens placent les divinités de façon globalement
surnaturellle avec évidemment des dieux fondamentaux, mais en
Mésopotamie il y a réellement un classement hiérarchique noté par
écrit, et cela démare par ANU dieu du ciel, et toute cette liste se
termine par tous les dieux possibles et inimaginables : dieu pioche ,
dieu ange gardien. Et bien évidemment, on ne s'adresse
pas à un dieu secondaire comme à un dieu premier }


 et c'est ainsi par exemple que l'on
peut remarquer que dans les très nombreuses statues de Gudéa que nous
avons, la taille de ces statues dépend du dieu auxquelles elles sont
destinées}


et bien évidemment les grandes statues correspondent aux grands dieux du
royaume Ninger sur et si la statue est pour un temple secondaire, alors
elle sera beaucoup plus petite et cela est typiquement mésopotamien.
Les égyptiens n'ont pas ce même rapport au surnaturel}


\textbf{Statue du fils de Gudéa} : UR NINGIRSU (Louvre et New York)}


\includegraphics[width=3.598cm,height=7.056cm]{FaivreMartin5conf-img/FaivreMartin5conf-img63.jpg}



La tête et le corps n'appartiennent pas au même musée,
Louvre et New York, aussi tous les quatre ans cette statue voyage et va
d'un musée à l'autre}


Nous le voyons les mains jointes dans le geste de la prière et
c'est la plus ancienne statue  royales connue où les
tributaires ne sont pas des prisonniers de guerre, ce sont des hommes
asservis, car sur les pieds (même logique qu'en \kmt
où les vaincus sont en dessous) mais ici ils portent des cadeaux}


\textbf{L'adorant de Larsa :}}


  [Warning: Image ignored] % Unhandled or unsupported graphics:
%\includegraphics[width=2.434cm,height=4.792cm]{FaivreMartin5conf-img/FaivreMartin5conf-img64}
    [Warning: Image ignored] % Unhandled or unsupported graphics:
%\includegraphics[width=3.21cm,height=3.528cm]{FaivreMartin5conf-img/FaivreMartin5conf-img65}
 }


\textcolor[rgb]{0.18039216,0.18039216,0.18039216}{Le personnage qui
porte un bonnet à haut bord, proche de la coiffure royale, est à demi
agenouillé, une main devant la bouche dans l'attitude
de la prière. Sur le socle, il est représenté dans la même position
face à une divinité assise. Une longue inscription indique que la
statuette a été dédiée au dieu Amurru ou Martu, dieu-patron des
Amorrites, par un homme de la ville de Larsa, pour la vie de
Hammurabi.}}


\textcolor[rgb]{0.18039216,0.18039216,0.18039216}{Une petite vasque à
offrandes est fixée à l'avant du socle (extrait du
Louvre)}}


\includegraphics[width=7.297cm,height=9.843cm]{FaivreMartin5conf-img/FaivreMartin5conf-img66.jpg}


\begin{flushleft}
\tablehead{}
\begin{supertabular}{l|l|}
\hhline{~-}
 & \\\hhline{~-}
\end{supertabular}
\end{flushleft}

Cette statue est rarissime , car le roi a un genou à terre , or ce ne
sont pas des gens qui prient à genoux. Si on prend dans
l'orient ancien comme un vaste territoire, ce sont les
gens de la bible, de l'Ancien Testament, qui disent
qu'ils prient à genoux}


Mais jamais les mésopotamiens ne diront que {\textquotedbl} je
m'agenouille pour prier mon dieu{\textquotedbl} et
cette image est sans doute Hammourabi, 5ème roi de Babylone,
contemporain de la fin du Moyen empire}


on doit avoir que cinq statues ainsi, où le roi prie à genou}


sur le socle de cette statue on peut voir un dieu assis, un genou à
terre}


Et cette statue est intéressante car elle nous montre la deuxième
position  de la prière dans le monde mésopotamien : \textbf{prier la
main droite devant la bouche}, et cela se fait de la main droite en
signe de respect . je place ma main devant ma bouche, je me tais, mais
je fais le signe du salut}


Et nous avons les explications, nous sommes dans une société où
l'on nous dit ce qu'il faut faire,
comment il faut être en récitant telle ou telle prière et on  a dans le
monde mésopotamien dès la fin du  III millénaire des prières
incantatoires qui commencent par {\textquotedbl} prière à réciter main
droite devant la bouche ...}


Donc on nous donne le mode d'emploi et
s'il s'agit en général de prières
sérieuses que l'on récite pour se faire pardonner de
son dieu}


Et on retrouve cette référence dans l'ancien Testament,
quand Job, (il y a différents passages qui évoquent la main devant la
bouche) se défend d'avoir pratiquer
l'idôlatrie du culte lunaire et il dit {\textquotedbl}
ma main devant la bouche, c'est encore un péché que
mes juges doivent condamner}


Et donc c'est intéressant car cela signifie que
c'est une tradition ancrée et pas seulement dans le
monde  mésopotamien, on en retrouve cette évocation dans
l'ancien testament avec Job, au moment où job se
défend d'avoir succombé à
l'idolatrie}


Là encore on peut voir le petit godet , servant  à mettre
l'encens que l'on faisait brûler pour
la divinité.}

\clearpage\clearpage\setcounter{page}{1}\pagestyle{Standard}

\textbf{Conférence N° 4 Mésopotamie et \kmt}}


\textbf{(suite de la thématique du Roi en Prière)}}


\textbf{Le face à face Roi Dieu}}


Et de là nous glissons sur deux thématiques qui sont liées :}


{}- le roi constructeur, à partir du moment où le temple est construit
:}


{}- la thématique de l'offrande  }


et nous verrons comment ces deux civilisations ont synthétisé ou résumé
l'image de l'offrande}


Nous avions vu la semaine dernière sur le roi en prière , Ur NINGSU en
prière avec des tributaires sur le socle et non des prisonniers. On
avait ces éléments \textbf{sur l'adorant de Larsa} où
on avait en plus une caractéristique exceptionnelle, à savoir un homme
à genou à terre, ce qui est exceptionnel, c'est une
attitude très rare en Mésopotamie.}


Encore une fois, même si les images artistiques sont peu nombreuses ,
nous avons une autre source très riche et nombreuse, ce sont les textes
cunéiformes, notamment les textes de prière qui donnent des consignes
sur quand et l'état de pureté ou sur
l'attitude qu'il faut avoir (vêtement
ou toute autre chose) au moment de la récitation de telle ou telle
prière, or les mésopotamiens ne disent jamais : mets toi à terre pour
prier ton dieu{\textquotedbl} En effet la gestuelle qui peut être
indiquée dans les textes, ne concernent que la gestuelle des mains}


La particularité de cet adorant, c'est
qu'en plus sur le socle, on retrouve le même motif en
relief, sauf que là il y a une différence le personnage peut être
Hammurabi, mais effectivement du fait de la période, en prière devant
un dieu et c'est là que les choses se rejoignent, }


En effet, on avait évoqué le fait qu'on avait le petit
godet pour les grains d'encens que
l'on offre à la divinité, donc il y aune offrande}


Mais cet objet est classé dans le groupe des rois en prière, et
c'est aussi un face à face roi dieu, car cet objet de
toute évidence était placé dans une anti cella, et dans  la pièce
d'à côté il devait y avoir une statue de culte.}


Donc dans l'esprit des gens, c'était
bien un face à face entre un dieu dans un Saint des Saints et un roi
dans la pièce attenance, ce que l'on a sur le socle.}


et le deuxième point est l'attitude des mains, les deux
mains doivent être réunies ensemble pour la prière, soit jointes, soit
ouvertes et à ce moment là c'est le geste de
l'incantation et les mains sont alors orientées les
paumes vers la divinité (et quand nous sommes en deux dimensions,
peinture ou bas relief ou relief dans le creux - en effet
l'une des sources principales résulte des sceaux
cylindres où on a un relief dans le creux -) et on voit que la main
droite est toujours devant la main gauche. La main droite,
c'est celle de l'alimentation,
prière, celle qui fait les choses pures, la main gauche fait tout ce
qui est sale. On avait déjà vu que la main droite devant la bouche
était un signe de salut, de respect et un signe de prière, de prière
incantatoire en général, prière que l'on récite en
général en cas de gros soucis (mais nous reverrons cette attitude)}


\textbf{\kmt, Attitude du Roi en Prière} : }


Attention, on ne peut pas le faire chronologiquement, car il y a des
périodes où il n'y a rien en Mésopotamie et
d'autres où il n'y a rien en \kmt.
Tant que nous sommes au IV et au III millénaire, cela ne pose pas de
problème et nous avons des images pour ces deux civilisations. Pour le
début du II millénaire, cela reste encore valable, mais pour la
deuxième moitié du II millénaire, nous avons des images égyptiennes
pour le Nouvel Empire mais pas d'images
mésopotamiennes, en dehors d'Uruk classique, Au
premier millénaire, nous aurons une iconographie importante avec les
assyriens et les babyloniens, alors que pour l'\kmt
, ce n'est pas que nous n'en avons
pas, mais ces images égyptiennes n'apportent rien de
nouveau}


Nous allons prendre des  images pour montrer la continuité en
Mésopotamie et en \kmt}


L'attitude de la prière (c'est à dire
sans association à une quelconque offrande), nous ne
l'avons pas dans l'art égyptien avant
la période du Moyen Empire (mais attention, cela ne veut pas dire que
cela n'a pas existé) . Mais si nous prenons ce qui est
parvenu jusqu'à nous de l'époque
thinite ou de l'Ancien Empire, à chaque fois, le roi a
une chose dans les mains en train d'être offertes, ou
il y a un enlacement, une bénédiction, donc ce n'est
pas un simple geste de prière}


Pour le moment de nos connaissance, la plus ancienne attestation est
Sésostris  III}


\textbf{Sésostris III, 5ème roi de la XII dynastie}}


\includegraphics[width=6.98cm,height=12.501cm]{FaivreMartin5conf-img/FaivreMartin5conf-img67.png}



A ce jour , il est le premier roi que nous connaissons dans cette
position, les mains ouverte, posé à plat sur l'avant
du page, et on peut constater que le costume ira de pair avec la
gestuelle car cette attitude sera associée à ce genre de vêtement long
ou court, mais qui présente la partie empesée à
l'avant pour positionner la mains ouvertes à plat.}


On trouvera ensuite cette position assise, mais nous
n'en connaissons pas d'exemple pour
Sésostris III, il faut attendre, son successeur,  à savoir son fils
Amenemhat III, qui nous a laissé les plus anciennes images royales du
roi assis les deux mains ouvertes posées à plat sur les cuisses }


(Madame FAIVRE MARTIN insiste sur ces deux mains, car dans les
représentations antérieures à partir de Djedefrê, de la IV dynastie,
quant le roi est assis, il a la main droite ouverte postée à plat sur
sa cuisse droite, mais le poing droit est toujours fermé tenant
généralement un linge, ce qui ne correspond donc pas à une attitude de
prière}


Cette attitude de prière apparaît donc à la XII dynastie et on continue
à la trouver dans sa {\textquotedbl}forme moyen empire {\textquotedbl}
sur beaucoup de représentation de l'époque
Thoutmoside, en gros des oeuvres qui concernes les règnes
d'Hatshepsout et de son successeur Thoutmosis III, qui
sont des règnes dont l'iconographie reste le plus dans
la tradition antérieure des images classiques. Sachant que
l'évolution se fait davantage dans le choix des
matériaux utilisés et dans la taille des oeuvres réalisées. Et
c'est avec les successeurs de Thoutmosis III , à
savoir Aménophis II et Thoutmosis IV que le répertoire
d'images va s'éloigner de la
tradition du Moyen Empire et nous livrer de nouveaux types }


\textbf{Reine Hatsespsout Métropolitan Muséum}}


\includegraphics[width=6.662cm,height=12.287cm]{FaivreMartin5conf-img/FaivreMartin5conf-img68.jpg}



petit clin d'oeil, par rapport au monde mésopotamien,
où  les filles royales n'ont pas accès au trône, mais
même en \kmt cela reste exceptionnel.}


Ce qui pose problème dans l'iconographie égyptienne, ce
sont les représentations agenouillées, que l'on
connaît finalement depuis les origines. Mais on constate que dans les
sources de l'Ancien Empire, on a encore une fois ,
c'est associé plutôt à la thématique de
l'offrande plutôt qu'à celle de la
prière, et c'est vrai pour les périodes hautes
n'avons pas d'exemple royal, }


On a un exemple , sous la troisième dynastie, avec la statuette de Houni
? à vérifier, où nous avons un homme à genou les mains jointes, (mains
jointes : cela est très rare en \kmt) mais c'est un
particulier.}


Donc pour avoir un roi ne tenant pas un attribut, mais sans représenter
l'attitude de la prière, il faut attendre cette
évolution des images après Thoutmosis IV}


\textbf{Aménophis III de Boston}}


\includegraphics[width=4.757cm,height=11.915cm]{FaivreMartin5conf-img/FaivreMartin5conf-img69.jpg}



\textbf{LE FACE A FACE ROI DIEU}}


Nous avons là un beau répertoire d'images avec pour le
monde  mésopotamien, dans un premier temps le recours à
l'allusion}


Là où les égyptiens auront recours aux textes, vu que le nom remplace
l'image, on peut avoir un texte où on a le nom du roi
et d'un dieux, et nous sommes dans cette même logique}


\textbf{Le monde Mésopotamien }}


il a un rapport au divin tout à fait particulier : le monde des dieux
est sublimé alors que le monde des hommes est assez souvent descendu à
un niveau d'être absolument imparfaits, plein de
défauts, }


Aussi, lorsque nous regardons des images anciennes, on constate bien que
la notion d'antropomorphisme , qui  est bien définie
dans les textes de la création de cette époque (les dieux nous ont crée
à leur image selon les mythes mésopotamiens, mythes qui sont sumériens,
puis repris par les assyriens et babyloniens) et là il y a donc quelque
chose de fondamentalement contradictoire.}


En effet, d'un côté nous avons ces êtres sublimes,
merveilleux à l'intelligence telle que
l'être humain n'a pas la capacité
pour pouvoir les appréhender et immortels, mais avec des défauts et
nous avons aussi ce monde humain, mortel et rempli de défaut.}


aussi, comment peut on dans l'art donner la même
enveloppe , la même image à ces deux choses aussi différentes,
l'homme et la divinité, et on a le sentiment, (même si
one sait pas comment cela se passait dans leur affection) que les
mésopotamiens rechignent à représenter la divinité, ils vont aborder la
divinité par le symbole et on le voit par les premières images
(troupeau, hampe) qui sont là pour évoquer la déesse Ishtar}


Et pour la période des dynasties archaïques au début du III millénaire
(ce qui correspond à la période du début de l'Ancien
Empire pour l'\kmt) jusqu'à la
dynastie I en Mésopotamie ( et dynastie III en \kmt), on continue à
représenter la divinité en l'invoquant par un symbole,
}


Ce qui va changer c'est qu'avec
l'apparition de l'écriture, on aura
dans le texte accompagnant l'image,
l'idéogramme de la divinité,  de
l'être divin, l'étoile qui précèdera
le nom de la divinité }


\textbf{relief aux plumes de Girsu au Louvre}}


\includegraphics[width=8.885cm,height=10.486cm]{FaivreMartin5conf-img/FaivreMartin5conf-img70.jpg}



Bas-relief de la «~figure aux plumes~». Le roi-prêtre, portant une jupe
en filet et un chapeau orné de feuilles ou de plumes, se tient devant
la porte d'un temple symbolisé par deux grandes
masses. L'inscription mentionne le dieu Ningirsu.
Période Dynastique Archaïque ancienne.}


Et c'est la plus ancienne représentation connue au
monde du nom d'un dieu, dieu ce  royaume Ningirsu et
c'est la plus ancienne version connue à ce jour de son
grand temple, qui s'appelle le temple
d'Eninnu, ce qui veut dire la maison des cinquantes,
les dieux étant associés au système comptable (et
c'est ce fameux temple d'Eninnu que
Gudéa va reconstruire quelques siècles plus tard et que
l'on voyait sur le socle de sa statue)}


Ce relief aux plumes mentionne le nom du dieu, mais ne mentionne pas le
nom du personnage représente, on nous dit simplement que le Eninnu fut
reconstruit pour la divinité}


Ce personnage, cependant on l'identifie à un roi et par
la paléographie très archaïque de signes on propose de dater ce relief
de la dynastie archaïque II, c'est à dire vers 2800 -
2700 av JC, et on pense que cet homme est un souverain même
s'il est tout à fait déroutant pas sa coiffure
inhabituelle et qu'il a quelque chose autour du front,
qui le scinde en deux, et que peut rappeler le bandeau, fameux turban
que nous avons déjà vu et la masse des cheveux ramenés à
l'arrière de la tête évoque un peu un chignon quoique
un peu volumineux. On peut remarquer également que tout est un peu
disproportionné dans cette image, et les deux plumes plantées dans le
bandeau sont déroutantes , et sans aucun élément de comparaison
répertorié ailleurs}


Aussi, faute de comparaison, on dit qu'il
s'agit d'une image royale face à une
entrée de sanctuaire. Mais il y a un problème, car si on regarde cette
image, on voit qu'il fait le signe du salut da la main
gauche (on retrouve ce geste en \kmt, et quand le roi fait cela
c'est quand il livre le temple au dieu et cela
correspond à la dernière image du rite de fondation du temple). Sur
cette image on retrouve les deux mats, sauf que l'on
voit mal car c'est un objet où le relief est à peine
levé, (ce que l'on appelle le méplat), et là nous
avons une façade architecturée (attention cet objet
n'a fait l'objet
d'aucune publication)}


Il y a donc un truc troublant que l'on ne retrouve que
sur un objet du Métropolitan  Muséum, qui est un objet qui ne peut
venir que de Oumma, donc cela pose un problème car Oumma  est le
royaume ennemi de Nirgirsu, et le personnage que nous voyons
s'appelle Ushumgal et il est devant le dieu Shara,
dieu du royaume d'Oumma}


\textbf{Ushumagal, devant le dieu Shara Métropolitan Muséum}}


\includegraphics[width=9.031cm,height=11.501cm]{FaivreMartin5conf-img/FaivreMartin5conf-img71.jpg}



mais attention car cet objet a été acheté par le Met à un collectionneur
français , et ils le datent d'une période plus ancien
que celui du Louvre, en le datant de la dynastie I, mais encore un fois
il faut être prudent le site d'Umma
n'a d'ailleurs pratiquement pas été
fouillé. mais cela pose un problème, car à ce jour, il
n'y aucun bas relief du DAI connu au monde, aucun
musée, aucun collectionneur n'en possède, les plus
anciens bas reliefs remontent à DA II, ce qui fait que les américains
vieillissent cet objet d'un ou deux siècles}


Or ces deux royaumes sont à environ 30km l'un de
l'autre, donc très proche et pourtant si on examine
ces deux objets, on ne peut que constater que le relief de Oumma est
plus abouti, plus travaillé et donc avec cette différence de
technicité, il doit  être postérieur au contraire}


On peut remarquer également que sur le relief d'Oumma,
on voit bien la jupe que l'on retrouve sur les reliefs
perforés du dynastique archaïque II,}


donc il conviendrait selon Madame Faivre Martin de dater cet objet de
2800- 2700  et alors il serait complémentaire avec le relief aux plumes
du Louvre}


C'est intéressant, mais cela pose un problème car cela
veut dire qu'il y aurait eu la même iconographie pour
un roi et pour un particulier }


Mais ce qui change également en dehors de la présence ou non de plumes,
c'est l'attitude des mains, qui vient
de toute évidence du fait qu'il y a un cadre sculpté
pour l'objet américain et de la longueur des textes
qui est différent d'un objet à
l'autre}


on remarque le profil droit des personnages regardant vers la droite, ce
qui est normal puisque nous sommes dans une civilisation où
l'on écrit de droite  à gauche, dont ils regardent
vers le texte. On remarque également qu'ils regardent
un bâtiment qui n'a pas deux mats, avec les boules
cela donne une impression très archaïque (c'est ce que
l'on peut trouver sur les sceaux cylindres de la
période d'Uruk). Si on regarde à droite, on peut voir
des petites choses, les deux montants comme pour un linteau, et cela
nous rappelle ce que nous avons en \kmt à savoir un rabattu vertical
, c'est un passage et les poutres sont vues en
perspective, et cela les mésopotamiens comme les égyptiens le font en
rabattu vertical.}


Nous avons aussi le nom du personnage, prêtre de Shara}


il serait intéressant d'avoir un commentaire sur ces
deux objets, et l'on pourrait se poser la question
savoir , du fait que nous n'avons pas le nom
d'un roi sur l'objet du Louvre, si
nous  avons raison d'y voir un roi emplumé, cela nous
montre que nous avons au Louvre un grand prêtre qui se dit en sumérien
Sangah (?) Et à cette époque le grand prêtre peut faire tout ce que
fait le roi, et même des offrandes, car il est en quelque sorte le
premier ministre du roi, c'est donc un personnage très
important}


Il faut aussi remarquer que la coiffure est la même, on a le bandeau et
le chignon, donc avons nous raison au Louvre de
l'appeler Roi, n'est il pas tout
simplement un grand prêtre ? }


En réalité, on utilise le relief au  plume comme étant la représentation
d'un roi, puisque ce serait le seul roi de la dynastie
archaïque II qui existerait au monde}


Il faut donc toujours faire attention au problème
d'affirmation par les collections des musées, car
maintenant il serait impossible de ne plus dire qu'il
s'agit d'un roi et cela nous pose un
problème dans l'iconographie sumérienne,
c'est en effet plus difficile que dans
l'iconographie égyptienne, car le roi sumérien est
représenté de la même taille que les autres gens, voir plus grand (ici
c'est normal car il est représenté avec ses fis et un
échanson  et sans attribut}


\textbf{Ur Nanshe (Louvre)}}


\includegraphics[width=11.425cm,height=10.931cm]{FaivreMartin5conf-img/FaivreMartin5conf-img72.jpg}



\includegraphics[width=15.981cm,height=11.465cm]{FaivreMartin5conf-img/FaivreMartin5conf-img73.png}



Nous reverrons cette image en fin de conférence. Ici il est facile de
reconnaître le roi, sur ce relief d'une part nous
avons le titre de Lugal c'est un roi qui est connu
grâce  une cinquantaine d'inscription , il a été le
fondateur de la première dynastie de Lagash}


Mais si nous n'avions pas le texte, il
n'y aurait aucun moyen de savoir
qu'il s'agit d'un
roi, car il ne porte ni ne tient aucun attribut, il
n'a aucune gestuelle particulière spécifique au
souverain. Il est debout, les mains jointes dans le geste de la prière,
le crane rasé, il porte un vêtement kaunakès, comme ses contemporains,
quelque soit leur place dans le royaume sumérien et cela permet de
comprendre la difficulté pour analyser le relief aux plumes et
finalement ce qui av ait servi à dire que ce relief représentait un
roi, était deux éléments  d'une part le bandeau, (on
voit que Ur-Nanshe a aussi un bandeau autour du front) et
l'autre élément était l'utilisation
de la présence de l'architecture du temples, deux mats
par référence au sceaux de l'époque
d'Uruk.}


Ici lorsque nous avons ces reliefs percés, il faut se souvenir
qu'ils étaient placés par rapport à une architecture,
et une anti cella face à un saint des saints, car si tous ces
personnages sont orientés vers la droite, c'est que
cet objet était placé dans une architecture sur un mur de gauche, donc
il était face à un dieu}


Ce sont en effet des reliefs d'applique commémorative ,
comme nos plaques de marbre que nous mettons dans nos églises encore de
nos jours,  et on pense que ces reliefs perforés devaient être placé
les uns à côté des autres dans un temple.}


Le problème c'est que nous avons un corpus
d'environ 120 stèles perforées et malheureusement
aucun n'a été trouvé dans son emplacement
d'origine.}


Mais il est vrai que dans un face à face entre un roi et un dieu,
finalement c'est  à partir du 3ème millénaire que
cette image va se développer dans la glyptique, dans les sceaux
cylindre plus ou moins bien représentée et si on passe au début du II
millénaire on a cet image qui figure en haut du code
d'Hammurabi.}


\textbf{le code d'Hammurabi : ( louvre)}}


\includegraphics[width=8.567cm,height=14.109cm]{FaivreMartin5conf-img/FaivreMartin5conf-img74.jpg}



Nous sommes au 18ème siècle avant JC et là nous sommes contemporain de
la fin du Moyen Empire, sous la première dynastie de Babylone et donc
dans le face à face, roi dieu, on marque les caractéristiques de
l'être surnaturel, en le plaçant dans une position
céleste : le dieu est dans le ciel, assis sur un trône (en façade de
palais d'ailleurs), et ses pieds reposent sur les
montagnes de la terre (petites billes sous ses pieds sont des image des
montagnes de la terre)}


A partir du moment  où l es divinités sont toutes issues de culture
préhistoriques, les entités divines sont majoritairement masculines, il
y a peu d'entité féminines, et si elles figurent elles
ont générale le rang de padère, compagnes, on parle de mère. Finalement
il n'existe qu'une entité féminine de
poigne en Mésopotamie, c'est Inana pour les sumériens
et Ishtar pour les sémites, qui supplantera plein de petites déesses au
deuxième millénaire}


Et c'est pour cela que l'on voit bien
que ces êtres divins mésopotamiens sont les héritiers des procréateurs
des troupeaux vénérés durant les époques néolithique, représentés très
tôt dans l'art sous la forme du troupeau ou du bélier.
Et on nous dit que l'être divin tire sa force de ses
cornes, et c'est une image qui sera utilisée dans
l'art à partir du III millénaire, justement dans la
dynastie archaïque dont nous avons vu quelques images, on va
différencier l'antropo-morphisme divin et humain, non
seulement par le texte  où le nom de la divinité est précédée de
l'idéogramme de l'étoile (SIN GIR
dieu), mais aussi en mettant sur la tête de l'être
divin \textbf{une tiare à corne}, avec une comptabilité qui se met en
place au III millénaire, qui ne disparaîtra qu'avec
l'arrivée des perses au VI siècle avant JC : deux
cornes est le nombre minimum pour une divinité de rang inférieur, puis
4, 6 et 8 , huit étant le maximum ; et cela on ne sait pas pourquoi}


attention il ne faut pas oublier lorsque l'on regarde
une image que l'on voit les cornes de profil et il
faut donc les multiplier par deux (ici nous voyons quatre cornes de
profil, il y a donc en réalité huit cornes , et donc nous avons affaire
à une divinité majeure du panthéon }


Ce qui est intéressant également dans l'art
mésopotamien, c'est qu'il
n'existe aucune image définie ou fixe pour chaque
dieu, cela peut être n'importe quel dieu supérieur du
panthéon, cela signifie juste dieu important, donc cela peut être Anu,
dieu du ciel, Enlil dieu chargé des affaires de la terre, Ensi, dieu
souterrain, Samash, dieu du soleil. Si nous sommes à Babylone, cela
peut être Marduk, dieu local de Babylone, mais à Babylone
c'est le dieu le plus important}


Bien évidemment pour le code Haednt  b[FDFD?][FDFD?]iestevvhéritierslier
( qe te la pmmurabi, qui représente le plus beau face à face roi
divinité nous aurions aimé avoir le nom de la divinité et
malheureusement l'introduction  de ce code de Loi
mentionne toutes les divinités possibles et inimaginables. }


C'est l'un des plus anciens textes
d'ailleurs, où l'on rattache Marduk,
petit dieu local sans importance aux trois dieux  anciens, Anu, Enlil
et Ensi, en faisant de Marduk le fils d'Enlil (cela
est un bidouillage que nous retrouvons d'ailleurs en
\kmt, quand on fait à l'Ancien Empire ,
d'Osiris fils de Nout pour le rattacher à la
cosmogonie des dieux)}


Pourquoi ce long discours sur les dieux, alors que nous parlons des rois
. Cela pourrait être intéressant de savoir dans ces faces à faces
s'il y a un dieu en particulier qui donne au roi la
fameuse baguette d'arpenteur et la corde , donnés au
moment du couronnement. Lorsque nous avons des textes qui nous parlent
du couronnement, on nous dit que ces objets sont donnés par Ishtar,
mais nous n'avons aucune image
d'Ishtar donnant les attributs au roi, elle est
supplantée par les dieux. De toute façon en Mésopotamie, les entités
divines masculines réussissent toujours à passer avant les entités
féminines. On peut dire qu'ils se touchent presque,
c'est l'anneau qui vient
jusqu'au coude du roi. Ce n'est pas
un monde où il y a un contact charnel entre le roi et les dieux, car le
roi est un homme, et c'est vraiment une spécificité
qui se trouve dans l'image.}


Nous retrouvons l'attitude de la prière avec la main
droite levée devant la bouche, et cela nous permet de comprendre que le
roi priant en Mésopotamie a toujours ce vêtement, qui couvre son bras
gauche dans un vaste drapé, l'épaule droite restant
dénudée. Le bras droit doit être libéré pour faire
l'offrande , ou prier , agir}


Il est évident que les rois mésopotamiens n'étaient pas
habillés comme cela dans les autres moments de la vie, on voit bien
dans d'autres images qu'ils portent
des tuniques à deux manches, comme tout le monde.}


exemple , sur un sceau de la même époque, trouvé à (Tel Amas ?), qui
appartient à l'inventaire des objets perdus en Irak,
on voit le roi suppliant, on reconnaît le roi par la coiffure : bonnet
et bandeau, et on voit son profil gauche regardant vers la gauche, et
on voit qu'il porte dans la main gauche un vase
cultuel, et on voit son bras sur son ventre et là ce
n'est pas une attitude de prière, mais de suppliant,
par rapport à un être divin. Cet être divin a simplement une petite
corne, c'est une lama.}


Ce sont des entité divines, majoritairement féminines , qui sont là
comme intercesseur et qui souvent sont utilisées comme intermédiaire
entre les hommes et les dieux, (fut il roi) et cela traduit une chose
intéressante dans la pensée mésopotamienne ; cette idée
d'intermédiaire, qui sont des être surnaturels de rang
inférieur, qui vont  conduire, guider l'être humain
vers la divinité suprême.}


Et finalement dans les sceaux , on a le plus souvent cette image
d'un dieu trônant assis (dieu principal) , un humain
en attitude de dévotion, parfois conduit par la main par son dieu
personnel, soit une divinité inférieure, et suivi généralement par une
lama}


en plus la lama est pratique car elle ferme la scène}


On voit qu'à partir du II millénaire, on va beaucoup
utiliser les lamas dans l'art des sceaux et
c'est vraiment une mise en page qui va être un
standard pendant environ deux siècles}


Et cette image va être reproduite sur les stèles , notamment image du
British muséum, \textbf{stèle de NABU APLA IDDINA}, roi de Babylone au
9ème siècle avant Jésus Christ}


\includegraphics[width=15.981cm,height=12.982cm]{FaivreMartin5conf-img/FaivreMartin5conf-img75.jpg}


\textsf{Bas-relief représentant le dieu
}\textsf{\textbf{Shamash}}\textsf{ faisant face au roi
}\href{http://fr.wikipedia.org/wiki/Babylonie}{\textsf{\textcolor[rgb]{0.0,0.21960784,0.62352943}{babylonien}}}\textsf{
}\href{http://fr.wikipedia.org/wiki/Nabû-apla-iddina}{\textsf{\textcolor[rgb]{0.0,0.21960784,0.62352943}{Nabû-apla-iddina}}}\textsf{
(888-855 av. J.-C.) introduit par un prêtre et une divinité protectrice
; entre les deux, le disque solaire symbolisant le
dieu}\textsf{\textcolor[rgb]{0.0,0.21960784,0.62352943}{1}}\textsf{.
}\href{http://fr.wikipedia.org/wiki/British_Museum}{\textsf{\textcolor[rgb]{0.0,0.21960784,0.62352943}{British
Museum}}}\textsf{.}


Si on veut retenir une  image d'un roi mésopotamien
devant un dieu suprême, c'est cette composition
qu'il faut avoir en tête (le roi devant le dieu
Shamash)}


Il y a une distance par rapport au code d'Hammurabi,
c'est à dire que plus on avance dans le temps, plus
les mésopotamiens mettent des barrières entre eux et leurs dieux
importants. C'est à dire que nous avons ces
intermédiaires, ces intercesseurs , qui notamment dans
l'imagerie sont de plus en plus présents et
c'est vrai que là c'est révélateur,
nous avons le dieu trônant , avec sa tiare à corne, il est sous son
dais, devant sa table d'offrande.
C'est Shamash, alors il y a aussi son symbole
idéogramme.}


Souvent on dit que le dieu qui tient le roi par la main est son dieu
personnel, et le dieu personnel est très important au II millénaire,
mais il est important également dans les périodes archaïques.
C'est lui qui est dans le ciel et auquel
l'humain va demander : intercède en ma faveur auprès
de tel ou tel dieu important. L'humain de base a un
dieu  personnel, Le roi a aussi son dieu personnel, car même roi, il
n'a aucun contact privilégié avec les dieux et
c'est une spécificité mésopotamienne.}


On le voit donc introduit par son dieu personnel, son bras droit est
dénudé, il fait le geste de la prière main droite devant la bouche, 
(ce roi Nabû apla Iddina avait effectivement intérêt à bien prier car
il connaissait des difficultés avec les assyriens. On voit également
une lama, c'est pratique, elle ferme la scène}


\textbf{ASSURBANIPAL II}  (attention rechercher diapositive)}


On le voit avec cinq divinités, et il  pointe la main droite vers cinq
symboles au dessus de sa tête,  moyens d'évoquer les
dieux qui vivent dans le ciel, et dont les pieds reposent sur la
montagne de la terre.}


On voit la tiare à corne de face, avec une seule corne,  nous sommes en
Asssyrie, (attention dans un autre lieu ou à une autre époque, cela
pourrait être un autre dieu) mais ici chez les assyriens du IX au VII
sicle , l'idéogramme de la corne est le symbole
d'Ashur, dieu national}


le disque avec deux ailes évoque Shamash le dieu soleil (attention
lecture assyrienne du premier millénaire}


nous avons ensuite le croissant de lune avec le dieu Sin}


la représentation de la foudre, éclair dans le ciel,
c'est le dieu de l'orage Adad}


puis nous avons une représentation d'étoile, qui
symbolise  Ishtar, très importante car les assyriens vénèrent une forme
d'Ishtar, qui a partir du IX siècle protège le bras
armé du roi}


On a les images divines qui seront mises face à la personne royale à
l'époque assyrienne et finalement on voit bien
qu'on est dans la logique d'évoquer
la divinité par son symbole ou son nom, plutôt que par son image}


et voilà pourquoi la petite stèle du British Muséum est intéressante,
car ce genre d'image est rare}


Ce qui est caractéristique, c'est que les divinités
auxquelles on se réfère change d'une époque à
l'autre, c'est à dire
qu'au III , II millénaire, l e roi
s'adresse à Anu le dieu du ciel (1), ENLIL responsable
de la terre (2) à Ea (3) et shamas (4) , et Sin (5ème symbole) et
normalement en sixième position vient le dieu local Marduk ou un
autre.}


Ce qui est intéressant également c'est que par la
militarisation de ces royaumes, il va y avoir un renversement de la
divinité qui protège le roi et en numéro un nous allons avoir le dieu
du royaume, MARDUK, Les babyloniens au premier millénaire vont lui
donner la première position, et il vont faire ce dieu protecteur du
roi, un dieu créateur c' qu'il
n'était pas auparavant. Les égyptiens feront de même}


\textbf{Le monde égyptien} : }


On va voir que cela est plus varié et complexe en fonction des époques ,
mais surtout d'une part on va être beaucoup moins figé
dans un panthéon défini, en \kmt il  y a une plus grande richesse de
dieux et déesses qui peuvent mis en face à face avec le roi. Cela
dépend évidemment des époques, des lieux, mais malgré tout au niveau du
répertoire d'images , si dans le monde mésopotamien,
nous sommes sur une dizaine de dieux, en \kmt ils sont toujours plus
nombreux }


Le deuxième point tout à fait différent est le lien charnel , le
contact, l'égalité qui existe en \kmt Si en
Mésopotamie, le roi est un homme à la tête des hommes, et il représente
les hommes, il sert les dieux, il est le serviteur mortel des dieux ;
en \kmt au contraire nous avons un être surnaturel , par son
couronnement , qui à partir de là devient le fils terrestre des dieux.}


\textbf{Stèle de Qahedjet et Horus} (III dynastie) Louvre }


\includegraphics[width=7.691cm,height=11.642cm]{FaivreMartin5conf-img/FaivreMartin5conf-img76.jpg}



Elle a été acheté en 1967 par le Louvre à un antiquaire égyptien au
Caire, c'est la plus ancienne stèle connue du baiser
d'Horus, du dieu dans l'art
égyptien.}


On a en effet un Horus d'Hélipolis qui enlace le roi,
main droite passe derrière l'épaule du roi, main
gauche posée sur le bras du pharaon}


C'est une scène d'enlacement que nous
retrouverons plus tard dans l'Ancien Empire pour des
scènes liées à la famille}


Dans l'art égyptien, il n'y a que dans
les images d'un homme et de sa femme, que
l'on pourra retrouver une image comparable. Il
s'agit donc d'une particularité ,
mais on peut remarquer qu'il n'y a
pas d'enlacement double : c'est le
divin qui enlace  l'humain, l'humain
lui a des insignes liés au pouvoir, montrant qu'il est
le représentant du dieu sur terre.}


On a des images que nous trouvons à chaque époque, }


TAHARQA EDFOU (Louvre)}


\includegraphics[width=15.981cm,height=11.889cm]{FaivreMartin5conf-img/FaivreMartin5conf-img77.jpg}



 Là nous avons un Horus sans couronne on a peu la même gestuelle sauf
que Taharqa fait un geste de salut (c'est un roi de la
XXV dynastie et on le reconnaît à sa coiffure très  particulière 
calotte et deux uraeus (dans les textes assyriens, quand on parle de
Taharqa, les sarrazins disent celui des deux serpents), deux uraeus
l'un correspond à la couronne blanque et
l'autre à la couronne rouge}


Cela va s'exprimer en trois dimensions et dès
l'ancien empire nous aurons de très beaux exemples
avec notamment Mykerinos}


\textbf{exemple triade de Mykerinos}}


\includegraphics[width=7.761cm,height=11.148cm]{FaivreMartin5conf-img/FaivreMartin5conf-img78.jpg}



Les triades ne sont pas des objets si simple à comprendre, il y a un roi
, une forme d'Hathor et un être qui représente une
province d'\kmt. et à cette époque comme il y avait
42 provinces en \kmt, on s'est demandé
s'il y avait eu 42 triades.}


le roi est toujours associé à Hathor, et c'est elle qui
a un contact avec lui, elle lui enlace la taille et place sa main sur
la saignée du bras du roi}


On a une statue  inachevée (musée de Boston) qui représente le roi et
son épouse, dans la même attitude }


\includegraphics[width=5.292cm,height=4.163cm]{FaivreMartin5conf-img/FaivreMartin5conf-img79.jpg}



\textbf{A New York, nous avons une petite sculpture  moins connue de
SAHOURE et du dieu nome de Coptos}}


\includegraphics[width=11.742cm,height=16.91cm]{FaivreMartin5conf-img/FaivreMartin5conf-img80.jpg}



Il est vrai que pour la V dynastie, nous avons davantage de relief que
de statue intact}


et on reste sur le même principe : on a un souverain : assis ou debout,
peu importe, et à côté il y a la personnification d'un
nome : ici Coptos,  On peut remarquer le nome Coptos
n'enlace pas la personne du roi, mais tient un
hiéroglyphe  qui représente le signe de la vie, qu'il
pose sur le trône.}


et c'est intéressant également car habituellement,
lorsque l'on voit le roi recevoir le souffle de vie,
cela lui est donné par un dieu ou une déesse et ici
c'est un nôme}


\textbf{Amenemhat  I , temple} funéraire de Licht (bas relief
Métropolitan Muséum) }


\includegraphics[width=15.776cm,height=3.417cm]{FaivreMartin5conf-img/FaivreMartin5conf-img81.jpg}



C'est une scène de jubilé et nous avons le roi qui est
placé au milieu de ce qui était de toute évidence un linteau, avec une
belle symétrie }


d'un côté Horus suivi de Nekbet }


et de l'autre Anubis suivi de Ouadjet ?}


et là c'est intéressant , on peut effectivement
retrouver le roi centré et deux personnages en miroir notamment dans
certains miroirs assyriens,  mais en \kmt cette composition
n'est pas si fréquente (on la voit au Moyen Empire sur
des linteaux)}


ici on voit que le souffle de vie est donné par Horus, et Anubis au roi
, ce qui conduit en réalité à une scène de Jubilé}


Au nouvel Empire, cela peut s'exprimer
d'une façon moins évidente du premier coup
d'oeil, comme sur ce \textbf{morceau de trône au
Métropolitan de New York, qui appartient à Thoutmosis IV, }qui 
provient de sa tombe}


\includegraphics[width=13.012cm,height=10.111cm]{FaivreMartin5conf-img/FaivreMartin5conf-img82.jpg}



(vérifier cette image, je ne suis pas certaine que cela soit la bonne)}


On voit le roi assis sur un trône représentant un séta métouy ?, qui est
face  à une déesse Hourette, lionne, qui dans sa main gauche tient le
signe de la vie et qui mais on le voit mal, plaçait sa main face à la
couronne du roi. Nous sommes donc une scène de couronnement à un moment
où le roi fait face au dieu qui place la couronne sur sa tête}


en \kmt on voit donc que le face à face roi dieu
s'associe à la thématique de l'enfant
et du baiser, ou alors à la représentation de son accession au pouvoir,
par la scène de couronnement ou du jubilé (qui ont une thématique
identique). On voit la présence de Thot à l'arrière,
car dans le rituel du couronnement, il est là pour inscrire le nom du
roi sur l'arbre shed (et on le reprendra pour la
thématique du jubilé évidemment)}


Ensuite les choses seront rendues avec plus de démonstration artistiques
en fonction des époques on se trouve et nous aurons beaucoup
d'images de Hatchepsout et de Ramsès II}


Dans n'importe quel saint des saints égyptiens, nous
aurons ce face à face roi dieu, et on peut trouver
n'importe quelle divinité du panthéon, ce rôle
n'est pas assigné à une divinité spécifique. Et là
encore une fois, on le voit recevoir le souffle de la vie, et on voit
le roi se rapprocher de la divinité, spécificité égyptienne, en rapport
au fait que le roi égyptien est représenté avec ses parents célestes}


Et on arrive ainsi à la représentation plus aboutie du \textbf{Louvre,
d'Osorkon II} (874/ 850) sur un bloc de granit qui
provient de son temple de Bubastis,}


\includegraphics[width=7.056cm,height=5.644cm]{FaivreMartin5conf-img/FaivreMartin5conf-img83.jpg}



\textbf{\textcolor[rgb]{0.18039216,0.18039216,0.18039216}{scène du
jubilé, an 22 du roi Osorkon II}} }


\textbf{\textcolor[rgb]{0.18039216,0.18039216,0.18039216}{853 avant
J.-C. (22e dynastie)}} }


\textcolor[rgb]{0.18039216,0.18039216,0.18039216}{fragment
d'un portail monumental du temple de Bubastis}}


\textcolor[rgb]{0.18039216,0.18039216,0.18039216}{granite}}


\textcolor[rgb]{0.18039216,0.18039216,0.18039216}{H. : 1,25 m. ; L. :
1,63 m. ; Pr. : 0,55 m.}}


\textcolor[rgb]{0.18039216,0.18039216,0.18039216}{Le roi, en cape de
jubilé, {\textquotedbl}fait une pause dans la demeure{\textquotedbl}
des dieux d'Héliopolis, de droite à gauche et de haut
en bas : Rê, Atoum, ses enfants Chou et Tefnout, leurs enfants Geb et
Nout (la terre et le ciel) et les descendants de ces derniers : Osiris,
Horus, Seth, Isis et}}


\textcolor[rgb]{0.18039216,0.18039216,0.18039216}{Il
s'agit donc d'une scène de jubilé et
on a le détail de cette  procession, on a le dais , édicule provisoire
monté pour ce jour là, le roi qui entre dans cet édicule, grand
manteau, couronne blanche, et deux spectres face à
l'encadré d'Héliopolis, et alors
qu'il ressort de cet édifice dais, il se trouve face à
face avec Bastet, déesse tutélaire des lieux, On a donc
d'abord l''énnéade
et ensuite la déesse tutélaire}}


\textbf{\textcolor[rgb]{0.18039216,0.18039216,0.18039216}{petite
parenthèse :  deux images quand le dieu est un
roi,}}\textcolor[rgb]{0.18039216,0.18039216,0.18039216}{ et cela a pu
arriver de façon très ciblée aussi bien en \kmt
qu'en Mésopotamie}}


\textcolor[rgb]{0.18039216,0.18039216,0.18039216}{Dans le monde
mésopotamien, c'est plus clair, et la plus ancienne
représentation concerne Naram Sim, le roi de la 5ème dynastie
d'Akkad aux alentours de  {}- 2200 , ce qui correspond
à la 6ème dynastie en \kmt, personnage qui régna durant 36 ans, et
son nom signifie {\textquotedbl}aimé de Sin{\textquotedbl} et
c'est un petit fils de Sargon . malheureusement la
capitale de cette empire n'a toujours pas été
retrouvée }}


\textbf{\textcolor[rgb]{0.18039216,0.18039216,0.18039216}{Narâm Sîn
Louvre : }}}


\includegraphics[width=12.171cm,height=12.665cm]{FaivreMartin5conf-img/FaivreMartin5conf-img84.png}



Cette stèle a été retrouvée à Suse, car elle avait été emportée comme
butin de guerre, mais elle pose un problème majeure au restaurateur du
Louvre, car elle est atteinte d'une maladie de la
pierre, qui fait que régulièrement des morceaux de granit tombent}


sinon c'est une stèle exceptionnelle, avec une
iconographie ascendante et nous la reverrons dans le thème le roi et la
guerre, car c'est la première fois que nous avons la
mise en scène d'un roi combattant.}


Ici on voit le roi sur sa montagne et les dieux sont dans le ciel. Ce
qui est troublant c'est que la première petite case en
haut à droite n'écrit pas Lugal Narâm Sîn, mais Digir
(étoile) Narâm Sïn, et donc il est présenté comme un dieu}


et la deuxième chose troublante, relative à l'image, il
a un casque de guerre, tel qu'il existe à
l'époque avec un couvre nuque en cuir, mais sur le
casque deux cornes sont présentées de face.}


C'est le plus ancien roi mésopotamien, qui a été
divinisé de son vivant. On le sait par ailleurs
d'après quelques sources textuelles qui le confirment.
Pourquoi  lui et pas les suivants ? Nous l'ignorons}


Ce qui est troublant également c'est que
l'on n'est pas capable, (même si on
peut voir que nous sommes dans une période ou le principe de
souveraineté évolue), de voir concrètement comment cela
s'est passé. Etait il considéré comme un dieu sur
terre et en cas y avait il un culte qui lui était rendu ? et un culte :
c'est un temple, un personnel qui va avec. Pour le
moment au vu de ce que nous avons, nous ne pouvons pas répondre à cette
question. On peut également s'interroger sur sa tombe,
il devait avoir une tombe digne d'un roi avec un culte
, mais là encore on n'en sait rien}


On a seulement des textes qui au lieu de mettre le mot Lugal , et donc
roi devant son nom, mette le mot dieu et nous savons que cet homme
régnait sur le royaume d'Akkad. On pense que sa
capitale est située dans  le sud de l'Iraq}


On aurait pu supposer qu'après ces rois, ses
successeurs soient aussi considérés comme des dieux, et soient
divinisés,  Mais en réalisé si quelques autres rois seront divinisés,
ce n'est qu'après leur  mort :
Goudéa, Ur Namu, (fondateur de Ur III, mais ensuite cela
s'arrête et aucun autre roi ne sera considéré comme un
dieu dans les textes}


On peut donc remarquer que cette divinisation est exceptionnelle et
correspond à une période très limitée : fin du III début du II
millénaire, et à un champs géographique : rois d'Ur
III et quelques rois de la première dynastie de Babylone. Par contre,
ni le vieux sud mésopotamienne,  ni la zone assyrienne ne connaissent
ce phénomène.}


\textbf{En \kmt} également nous connaît ce genre de chose et on peut
aborder l'aspect roi divinisé après sa mort et roi
divinisé de son vivant}


On est plus à l'aise en \kmt, car on comprend mieux ,
on a soit le roi défunt, (ex Thèbes, qui à l'époque
ramesside rend un culte sur la rive Ouest à Aménophis 1er et  à sa mère
et on le sait par les  images que l'on a retrouvé à
Der Deir El Médineh , avec sa mère Ahmès Nefertari, avec ce détail
humain sur lequel Madame Guillemette a travaillé sur le fait que cette
femme avait la peau noire et tout le questionnement donc sur cette
reine d'\kmt et à sa peau noire, que
l'on a systématiquement sur ses tombes ramessides, et
elle montré que ce n'est pas la reine Ahmès Nefertari
que l'on représente, mais sa statue de culte que
l'on vénère et encense, que l'on
promène en procession  lors des fêtes et que cette statue à force
d'être promenée et encensée a eu le bois qui a fondé
et ce n'est donc pas l'image
d'une femme à la peau noire}


Et de ce fait , nous ne sommes plus face à quelque chose
d'inexplicable, comme on l'avait en
Mésopotamien, car nous avons des images, des lieux de culte et des
textes et on sait finalement que c'était des saints
patrons d'une unité locale comme le saint patrons que
nous aurons à la période médiévale et moderne}


Le deuxième cas de figure , est la divinisation du vivant du roi ,
c'est Aménophis III en tant que dieu lune sur terre,
sa femme étant l'équivalente d'Hathor
sur terre. Et cela continuera à  l'époque ramesside,
avec le culte voué aux colosses de Ramsès II}


Mais encore une fois, ce n'est pas valable sur toute la
période pharaonique, on voit bien que cette divinisation du roi de son
vivant ou de l'image royale correspond à la période du
Nouvel Empire}


\textbf{III LE ROI CONSTRUCTEUR }}


Les dieux demandent au roi de leur construire une maison sur terre}


et d'ailleurs les mésopotamiens sont amusants car
lorsqu'ils parlent du sanctuaire, ils disent la maison
du dieu, et dans les textes anciens on parle même du lit du dieu}


En \kmt, grâce à l'architecture de calcaire et de
grès, on peut avoir des images sur les murs des sanctuaires, en
Mésopotamie du fait de l'architecture en brique nous
n'avons pas d'image (même si on sait
qu'à l'époque il y avait des images
peintes sur des enduits de plâtre. Aussi c'est plus
difficile, il faut partir des textes, et de ce qu'ils
nous disent et interpréter. }


On sait que les rois mésopotamiens ont été de grands constructeurs pour
leur dieu, et selon un principe que l'on a dans toutes
les sociétés, il y a de grands programmes liés à des grands évènements,
tels une victoire, car on veut montrer que la divinité locale  est
protectrice du roi de la souveraineté, (grand travaux pour Amon en
\kmt, ou pour Marduk à Babylone)}


Dès les origines, les rois mésopotamiens, sont représentés comme des
rois bâtisseurs pour leur dieu et la plus ancienne représentation
connue du roi mésopotamien bâtisseur est la plaque BLAU (nom de
l'ancien collectionneur) au British Muséum :}


\includegraphics[width=12.695cm,height=12.695cm]{FaivreMartin5conf-img/FaivreMartin5conf-img85.jpg}



En réalité, il s'agit donc d'une
petite plaque et on ne voit pas grand chose on voit un personnage
représenté en prière, entouré de gens qui enfoncent des piquets et
l'on considère que cette scène représente une scène
d'arpentage pour délimiter un espace cultuel dans le
contexte de la construction d'un temple}


Et on retrouve toujours cette problématique des sumériens à la fin du IV
, III millénaire, avec un personnage qui n'a aucun
insigne particulier, }


Dès les origines au contraire dans l'art égyptien, le
roi sera clairement différencié des autres, et même
s'il n'a pas de couronne,  il a au
moins l'uraeus (amenhat 1er) ou un costume
particulier, ou un attribut dans les mains. Mais
l'attribut de tête est constant dans
l'art égyptien, et on veut donc nous montrer dès
l'origine qu'il
n'est pas comme les autres}


le roi sumérien est un petit roi, mais attention nous ne sommes pas non
plus à la même échelle de royaume, en effet en \kmt le roi au III
millénaire règne sur une bande de 12,000 km, alors
qu'en Mésopotamie (si on prend Ur ou Uruk on aura de
grands royaumes, mais Um est un tout petit royaume de la taille
d'un bourg)}


On donne donc le nom de roi à des gens d'envergure et
de puissance très différentes }


\textbf{Ur NANSHE LOUVRE (roi de Lagash)}}


\includegraphics[width=15.981cm,height=13.406cm]{FaivreMartin5conf-img/FaivreMartin5conf-img86.jpg}



Et nous avons là la plus ancienne représentation connue à ce jour
d'un roi directement constructeur, nous sommes vers -
2500 et il suffit de regarder pour comprendre. On a le roi qui est plus
grand que les autres, profit droit regardant vers la droite, (il y a
donc un point commun avec l'imagerie égyptienne qui
elle aussi a un sens de lecture), et en bas nous avons le même
personnage assis qui regarde vers la gauche}


Et là les mésopotamiens ont fait une chose, qu'ils
auraient du faire plus souvent,  ils ont écrit le nom du roi}


on voit aussi le roi entouré de gens à son service, son épouse, et son
fils}


Il est très rare de voir l'épouse en face de son mari,
introduisant ses fils et on peut remarquer que le fils ainé porte déjà
une sorte de petit bandeau, et que ses frères font le geste de la
prière, et sur sa jupe figure son nom, il s'appelle
Ussur Ga (?), et qu'il règne après son père, (on le
sait par les listes royales) On peut donc en conclure que nous avons là
une représentation du prince héritier, et ce qui est incroyable,
c'est que le roi (toujours sans attribut) porte un
panier sur sa tête, panier en vannerie, d'où dépasse
des pierres (et là on peut dire que comme dans l'art
égyptien, pour représenter ce qui est à l'intérieur du
panier on dessine au dessus son contenu (ici des briques)}


en réalité les professeurs d'histoire de la Mésopotamie
disent que le panier et tellement plein qu'il déborde
nous permettant de voir qu'il est rempli de briques !}


(madame FAIVRE MARTIN a d'ailleurs publié à ce sujet
faisant remarquer que les briques que l'on voit sont
plates d'un côté et bombé de l'autre,
c'est à dire la brique plano convexe que
l'on utilise à cette époque pour la construction.  et
il est donc amusant qu'ils représentent ainsi la
brique.}


et on peut remarquer également que les briques sont disposées en biais,
en arrête de poissons, comme on les maçonne à
l'époque. On peut donc imaginer que cela va au delà de
la simple représentation d'un matériau de
construction, et que l'on veuille évoquer la
représentation de l'architecture d'un
bâtiment}


Les textes, malheureusement, ne nous donnent pas beaucoup
d'informations, ils  nous disent simplement
qu'Ur Nanshe reconstruit le temple de Nin Girus ?}


mais attention ces rois doivent reconstruire très souvent, car
l'architecture étant en briques, il faut sans arrêt
reconstruire du fait notamment de pluies en automne et au printemps}


Mais en réalité l'information va au delà, car en bas on
voit le roi assis tenant un gobelet en forme de calice, que nous avons
retrouvé dans les tombes royales d'Ur et on a toujours
considéré que la pose de la première briques faisait que le roi
organisait ensuite un banquet, mais le texte ne nous parle pas de
banquer il dit que }


{\textquotedbl}Ur Nanshe, les bateaux lourdement chargés de bois, sont
arrivé depuis BARRAIN, (mais il n'y a pas de bois à
cet endroit) , ce texte est le plus ancien , il date de 2550 
mentionnant l'existence de commerce dans le golfe
arabo persique par cabotages }


c'est déjà la preuve que la pierre arrive de la
péninsule amanèse  et un ou deux siècles plus tard, elle viendra de
Aman.}


on a ainsi une première attestation du commerce du bois venant de
l'Indus, on sait qu'il y avait déjà à
cette époque une exploitation intensive du Bois  dans
l'Indus, que les bateaux traversaient le golfe vont à
Baraïm et repartent chez eux (nous avons ces informations par le sceaux
cylindres) }


aussi le banquet peut il également évoquer l'arrivée
par bateau de ces matières indispensables à la construction,  et
c'est aussi une façon de nous dire que le temple était
 magnifique, car il n'était pas fait en simple bois,
mais avec un bois de qualité et de belles poutres qui permettent donc
une belle architecture}


Tout cela va donc ensemble et on voit bien que cette façon de
s'exprimer autour de cette thématique du bois veut
mettre en valeur le fait de l'associer au temple du
dieu}


c'est donc effectivement un objet unique, sur cette
thématique du roi architecte du temple de son dieu}

\clearpage\clearpage\setcounter{page}{1}\pagestyle{Standard}

\textbf{Conférence N° 5 }}


L'évocation du Roi portant une corbeille sur sa tête ,
et cette image va traverses les millénaires et elle restera
l'image du roi bâtisseur et ce jusque dans la Babylone
de Nabuchodonosor (en nous en verrons un exemple}


Cette image peut se retrouver aussi bien sur les bas reliefs , que
transposée sur des objets de cuivre comme des clous de fondation, qui
se développent essentiellement  la fin du III millénaire}


et ce qui est intéressant, c'est que dans cette
iconographie, même quant le roi est représenté simplement vu de dos,
dans l'attitude de la prière (par exemple UR GABA ,
beau père de Gudéa, fondateur de la II dynastie de Lagash, cette
sculpture à la tête cassée est intéressante par le long texte
qu'elle porte dans le dos, en langue cunéiforme et en
langue sumérienne,  et ce souverain évoque les nombreuses constructions
faites pour les dieux de son royaume dès son avènement. Et
c'est une source qui n'a pas de
comparaison connue ailleurs, et qui nous montre que dès
l'avènement d'une nouvelle dynastie
royale, nous ignorons sir UR GABA est un lointain descendant
d'Ur Nanshe et si il a ou non un lien avec la première
et la deuxième dynastie de Lagash)}


C'est intéressant effectivement car on voit ainsi
qu'indépendamment du lien géographique, il y a un lien
cultuel qui se fait par la divinité}


\newline
En effet, ces gens, mêmes s'ils ne sont pas de la même
lignée dans le sens où nous l'entendons, sont malgré
tout de la même lignée car ils sont les représentants du dieu Nin GIRSU
sur terre}


Le roi peut être représenté dans l'attitude du roi
bâtisseur, associé au rituel de fondation, et c'est un
point commun avec l'\kmt, sauf
qu'en Mésopotamie , jamais les rituels de fondation ne
seront mis en image de façon aussi détaillée et ordonnée que dans les
scènes égyptiennes}


Nous verrons effectivement que dans les scènes égyptiennes dès
l'ancien empire  V dynastie, on a certaines images qui
représentent les différentes phases de la fondation
d'un sanctuaire, d'un bâtiment,  en
Mésopotamie on va insister ,on va choisir pour représenter cet
évènement l'arpentage , qui était évoqué
d'ailleurs par les insignes que nous avons vu
précédemment}


\textbf{Stèle de calcaire de Gudéa, II dynastie, dite stèle de la
Musique, Louvre}}


\textbf{et commentaire pris sur le site du Louvre}}


\includegraphics[width=9.49cm,height=13.335cm]{FaivreMartin5conf-img/FaivreMartin5conf-img87.png}



\includegraphics[width=15.24cm,height=17.11cm]{FaivreMartin5conf-img/FaivreMartin5conf-img88.png}
\newline

\includegraphics[width=14.958cm,height=24.694cm]{FaivreMartin5conf-img/FaivreMartin5conf-img89.png}



Nous sommes à la fin de la période sumérienne, au 22ème siècle, avant
Jésus Christ, et contemporain de la fin de l'Ancien
Empire. Malheureusement, elle provient du site mal fouillé de Girsu où
beaucoup de morceaux de stèles ont été retrouvés,
d'évidence dans des sites cultuels, mais en miettes, 
est ce dû à des conflits  ? nous ne le savons pas. Mais faire ramasser
ces miettes de stèles par un diplomate ou un militaire , comme
c'était le cas à la fin du XIX siècle, fait que
beaucoup de miettes sont définitivement perdues et donc on ne pourra
jamais reconstituer ces stèles, car nous avons trop de lacune et ces
miettes proviennent d'objets différents. Le seul gros
morceau est celui là}


en réalité, on le trouvait moche et il est resté longtemps dans les
réserves du Louvre depuis son arrivée à la fin du XIX siècle
jusqu'en 1993 , date à laquelle il a été présenté au
public à l'occasion des nouvelles salles
mésopotamiennes.}


Or à ce jour c'est une scène unique pour cet art
sumérien, car nous avons la représentation de
l'arpenteur, et on a les piquets, la corde un
personnage qui tient le manche, un autre en prière et un troisième qui
clôt la marche,  On peut remarquer que celui qui fait la prière porte
un bonnet à haut rebord et donc même si nom n'est pas
mentionné ,s'il n'y a aucune
inscription, de toute évidence c'est le roi}


Et du fait qu'on a trouvé un fragment du même calcaire
qui porte le nom de Gudéa , et que tous les morceaux retrouvés nous
ramène à la même période,  on peut en déduire que
c'est Gudéa qui est représenté}


C'est une image un peu plus élaborée, mais finalement
c'est le même récit de l'arpentage}


Nous avions vu précédemment l'image de Gudéa avec un
temple sur ses genoux}


\includegraphics[width=14.6cm,height=10.797cm]{FaivreMartin5conf-img/FaivreMartin5conf-img90.png}



et nous avions vu que cette statue avait le plus ancien texte
mésopotamien, d'un songe où le dieu parle au roi
durant son sommeil et lui dit {\textquotedbl}reconstruis moi un temple,
plus beau, plus grand, qu'il n'a
jamais été}


Et il est vrai que finalement les mésopotamiens résument tout cela avec
l'image du roi portant une corbeille sur la tête}


Et là sur cette stèle de la musique, nous avons finalement ce que les
textes nous racontent : c'est à dire que
l'on choisit un terrain, on va le brûler pour chasser
les mauvais esprits, et ensuite les prêtres arpenteurs prendront les
mesures du sol, on va planter les piquets au quatre angles du futur
temples, entre les piquets }


on va tendre une corde et le long de cette corde on va creuser la
tranchée de fondation}


Et là nous sommes dans un type de texte que nous avons finalement en
\kmt, car après nous avons finalement
l'inauguration, c'est à dire que les
deux civilisations ne nous donnent pas le détail intermédiaire et
humain qui concerne la construction du temple}


Attention si le roi apparaît moins grand sur cette stèle , il ne faut
pas en déduire que cette stèle ne respecte pas la taille des
personnages en Mésopotamie,  Mais sur le calcaire , le sculpteur ne
trace aucune ligne de sol, aussi la taille du personnage
n'est pas révélateur, (on avait déjà vu cela sur la
stèle d'Ur-Nanshe où les personnages
n'étaient pas alignés) et ici il faut faire
d'autant plus attention que nous
n'avons qu'un fragment,
d'une stèle visiblement décoré sur deux faces. Nous
verrons à la fin de ce cours une stèle relative à la guerre , où  le
roi d'un côté est représenté très grand, et de
l'autre côté tout petit.}


Ici, nous avons donc un fragment sans autre élément de comparaison et
effectivement le roi est plus petit, ce qui nous donne deux lectures
possibles : on considère que nous sommes en présence
d'une procession et nous aussi un petit personnage qui
ferme la marche, ou l'on considère que
c'est à l'égyptienne et que le
premier est le roi qui va avec la corde et le piquet, et que nous avons
une deuxième étape du rituel qui est représenté, mais en réalité nous
n'en savons rien.}


On peut à la même époque représenter le roi bâtisseur sous une autre
forme, mais c'est la même idée, dans le clou de
fondation, qui apparaissent dans la culture sumérienne des dynasties
archaïques fin DA II. les clous de fondation peuvent être associés à
une tablette, dans les périodes anciennes non, puis cela va sa
systématiser à la fin du troisième millénaire et on sait que le
personnage représenté en porteur de corbeille est le roi lui même. Donc
c'est finalement ce que l'on avait
sur la stèle d'Ur Nanshe}


Il y avait quatre clous de fondation associés à une tablette, chaque
clou étant enterré au pied du piquet, cadrage par rapport au quatre
points cardinaux du temple et les textes nous disent que les piquets
sont les amarres terrestres du temple, comme si
c'était le lointain souvenir d'une
architecture périssable faite de nattes, ou de tapis,
qu'il fallait maintenir au sol avec des piquets. }


et cette image, nous allons la retrouver dans les époques suivantes}


Nous sommes obligés de faire un grand bond en avant car pour la période
du II millénaire nous n'avons que peut
d'objets}


\textbf{Stèle du British Muséum, Assurbanipal au VII siècle}}


\includegraphics[width=13.688cm,height=21.273cm]{FaivreMartin5conf-img/FaivreMartin5conf-img91.png}



Petite stèle qui représente Assurbanipal portant sur sa tête le panier
de briques pour l reconstruction d'un temple à
Babylone,  et c'est intéressant car cela correspond à
l'époque où les rois assyriens, du royaume du nord,
ont réussi pour un temps à dominer Babylone et à chaque fois, cet acte
politique et militaire est concrétisé par une construction religieuse,
puisque nous sommes à un moment où {\textquotedbl}Marduk le dieu de
Babylone , qui pendant le roi précédant avait résidé à Ashur en
présence du père qui l'a créé{\textquotedbl} }


C'est à dire qu'il y a eu un moment où
les divinités et les statues de culte , et nous le reverrons, avaient
été déportées de Babylone à Assur, ce qui était
l'humiliation suprême pour les babyloniens et quand il
y a eu des accords de paix, il y a eu un retour de la statue du culte
(c'est d'ailleurs arrivé plusieurs
fois au VIII et VII siècle. }


A ce moment là, le roi assyrien qui n'a pas envie
d'avoir une révolte des babyloniens de plus et nous le
comprenons d'autant plus que c'est ce
roi qui a dû lâcher l'\kmt conquise par son père
Asarhaddar, à un moment où il était en paix avec Babylone, Son fils
Assurbanipal au contraire subit la révolte des babyloniens, et de ce
fait doit faire revenir son armée basée en \kmt , ce qui permet en
\kmt l'arrivée de la 26ème dynastie.}


C'est donc intéressant et on voit donc un roi qui peut
se présenter humblement devant Marduk, divinité de Babylone, tout en
glissant quand même cette phrase {\textquotedbl} en présence du père
qui l'a créé{\textquotedbl}, c'est à
dire qu'il sous entend un lien avec les divinités de
Babylone et d'Assur}


Bien souvent en Mésopotamie, le temple lui même sera résumé. En effet
comment représenter un temple ? à l'époque de Gudéa,
on avait vu qu'ils avaient fait un plan.
C'est rarissime que les artistes aient tenté de
représenter le bâtiment. Dans les périodes anciennes, nous avions les
deux hampes,  le plan de Gudéa , exemple unique au monde. Souvent on va
évoquer le temple par l'autel, la table sur laquelle
on apportait les offrandes }


\textbf{Image d'autel albâtre montrant Tukulti Ninurta
1er (Berlin)}}


\includegraphics[width=8.112cm,height=10.619cm]{FaivreMartin5conf-img/FaivreMartin5conf-img92.png}



Le roi est donc dans deux postures d'adoration, debout
et à genou, et on a donc un bel exemple de la forme
d'un autel et un rabattu vertical, on a la tablette
avec le stylet qui est l'attribut du dieu\ \ Nabû,
fils de Marduk, dieu de Babylone et dieu des scribes.}


\textbf{Si on regarde en \kmt,} on voit qu'il prend
certains principes identiques. Il y a cette même problématique de
représenter le bâtiment, d'autant plus
qu'en \kmt, les temples sont très vites des
constructions de très grande taille, monumentale.}


et c'est ainsi que l'on a
l'image unique et fascinante sur un temple de Louxor,
qui reprend l'image de la façade du tempe, avec les
deux môles du pylône , avec en plus des mats à oriflamme et en
représentation de profil les statues de Ramsès, qui étaient
effectivement devant cette fameuse façade, le colosse assis et les deux
colosses debout. mais cette image est également unique.}


On peut avoir des morceaux sur des ostraca et des grafitis}


Les égyptiens dans l'ensemble étaient plus narratifs
sur l'iconographie du roi constructeur, et la plus
ancienne image est celle d'Ounas, dont nous avons un
relevé dessin, et où l'on aperçoit le roi agenouillé,
en train de poser une brique de fondation}


Durant le moyen empire, nous n'avons pas
d'image}


Et c'est au nouvel empire que l'on
aura les images du rituel, qui figurent  systématiquement sur les murs
des bâtiments religieux, avec un développement qui ira
jusqu'à l'époque gréco-romaine . et
l'on peut remarquer que nous aurons soit une sélection
d'images ,soit aura la totalité des images.}


Et les thématiques sont les mêmes, mais l'ordre change,
ce qui nous perturbe beaucoup car nous sommes cartésiens et nous aimons
que les choses soient toujours dans le même ordre. Or on sait que dans
un monument égyptien, on va nous mettre à côté, le creusement de la
tranchée de fondation, le rituel de la remise du temple, puis le jet de
natron.}


et les images du Nouvel Empire ou celles de la période Ptolémaïque nous
permettent de voir que c'est un répertoire constant
sur plus de 1500 ans;}


on a tout d'abord le calcule de l'axe
du temple, le rituel du piquet, (et on voit comme au Proche Orient, l e
piquet , le bâton et le cordon), sauf et c'est la
différence entre ces deux mondes : }


{}- en Mésopotamie, les dieux ne descendent pas sur terre pour
construire leur maison, c'est un travail
d'humain,}


{}- alors qu'en \kmt , on va placer le monde divin
comme participant à l'élaboration de cette chose
merveilleuse, }


tout simplement par ce que pour les égyptiens, le temple est un être
vivant, alors que pour les mésopotamiens, c'est une
architecture, une maison}


Il y a donc une différence énorme : }


en \kmt , le temple est considéré comme un enfant venant au monde, un
coeur qui bat, (à  vérifier), idée qui n'existe
absolument pas dans le monde mésopotamien (le cordon est ainsi toujours
évoqué par une forme arrondie et non rectangulaire pour évoquer le fait
que c'était la poche de l'embryon,
qui va s' y développer cf théorie de Mr de TRAUKNER}


Une fois que l'on a planté les piquets et que
l'on a égorgé une oie, pour chaque piquet, on creuse
la tranchée, la fondation jusqu'au Noun, (nappe
phréatique) le roi creuse la tranchée.}


ensuite on verse le sable de fondation, le lit de fondation, mais avant
de verser ce sable, on aura enfoui des clous de fondation et on aura
préparé la première brique}


Là c'est pareil, c'est un rituel
extrêmement élaboré en \kmt, avec une recette qui va évoluer selon
les périodes, et la brique contient plein de choses, notamment des
résines et des matières premières très coûteuses venant de Proche
Orient}


et ce moment est un moment très important, car nous sommes dans une
dimension magique dans ce rituel de fondation du temple, et cette
dimension magique est totalement inexistante dans le monde
mésopotamien}


en \kmt, la divinité à laquelle on destine la construction est
présente et surveille}


Comme dans le monde mésopotamien, il y a l'étape entre
les deux, que l'on ne représente pas, on passe ensuite
directement au moment où le bâtiment est construit (en image les
égyptiens simplifient et tournent la difficulté en  représentant le
temple par sa porte) et on a alors la purification par le natron.
C'est le moment où l'on fait le tour
du temple et on vérifie avant son inauguration}


Vient ensuite le moment où l'on fait le geste de
remettre le temple au dieu}


Et là c'est vrai que les images sont plus détaillées
que celles liées au rite de fondation, car il y a cet aspect
exceptionnel, surnaturel, }


le monde mésopotamien se contente de cette image de la corbeille sur la
tête du roi, car ce n'est rien de plus que 
{\textquotedbl} je construis une nouvelle maison pour le dieu sur
terre, mêmes si ils réalisent une architecture extraordinaire, cette
architecture n'a aucune fonction magique, elle est un
lieu où l'on pourra être en contact avec le
surnaturel, mais ce n'est pas un lieu magique, ce
n'est pas un lieu de danger}


\ \ \textbf{\ \ L'OFFRANDE}}


\textbf{Relief d'Ur Gipar Ku (British Muséum),}  qui
nous montre le rituel de l'offrande et qui est à peu
près de la même époque que celui d'Ur Nanshe\textbf{
}}


En \kmt, dès les origines, on ne représente pas les prêtres  et que
dès les origines jusqu'à la période romaine,
l'officiant devant les dieux sera à toute époque, sans
aucune exception le roi, }


Les rares fois où l'on représente les prêtres,
c'est uniquement et sans exception que
l'on représente une procession lors
d'une fête religieuse}


Dans le monde mésopotamien, nous ne pouvons pas être aussi tranchés,
pour la simple et bonne raison de tout ce que nous avons vu
précédemment. En effet  partir du moment où on a vu sur le relief
d'Ur Nanshe que le roi portant la corbeille contenant
les briques ne porte aucun attribut associé à son pouvoir et que
c'est simplement par ce qu'il est
écrit le mot roi Ur Nanshe que l'ont peut dire que ce
bonhomme avec ce panier sur la tête est un roi  ; quand nous aurons un
personnage avec un panier sur la tête, mais sans aucune inscription, on
ne peut pas savoir qui est le personnage représenté face à la divinité}


Au British Muséum, sur ce relief, on considère que face à la divinité,
c'est un officiant et on est obligé
d'utiliser ce mot car on ne peut pas dire prêtre , car
ce serait trancher sur le fait qu'il ne
s'agit pas du roi. Sans texte, on ne peut pas être
plus précis}


Ce que l'on peut dire sur ce relief,
c'est que nous avons l'être divin
matérialisé par un personnage assis, avec ses cornes et son vase. En
dessous on voit la façade de l'édifice cultuel et nous
voyons qu'il y a une table d'offrande
près de la façade et on voit des choses qui pendouillent du grand vase
à bec, (mais ce vase à bec est plus parlant en haut)}


Cette scène nous est expliquée grâce à des reliefs plus récents et moins
abimés, : on va offrir à la divinité sa nourriture, chaque jour, et en
Mésopotamie, ce sont des dattes. C'est pourquoi en
Mésopotamie, on a des branches de palmiers dattiers qui sont
représentés et la consécration de l'offrande que
l'on fait à la divinité se fait par un rituel de
purification par l'eau et que
l'officiant est nu}


Pour le monde sumérien, il faudrait faire comme pour les sociétés sans
écriture car ils viennent seulement de l'avoir , }


Il y a beaucoup de publications qui disent que cela ne peut pas être le
roi car il est nu. Mais on peut se demander au contraire si le roi ne
se démarque pas des autres en se mettant nu devant la divinité}


a partir du moment où rien n'est écrit, et nous sommes
à une époque où nous n'avons aucun texte explicatif et
c'est bien embêtant, on peut
s'interroger sur la présence des trois personnages
ayant un bonnet, on n'a pas cela ailleurs, leur
présence est donc dérangeante. }


de même ce personnage avec sa tête de face (Madame FAIVRE MARTIN pense
que c'est par ce qu'il a un mouton
sur les épaules et que c'est pour que
l'on voit bien ce mouton et ses pattes}


Si on a des personnages proche du dieu, qui ne sont pas le roi, on peut
en déduire que ce sont des grands prêtres}


Comme toujours en Mésopotamie, nous avons une longue période sans image
et nous allons donc à nouveau faire un bond dans le temps avec cette
stèle de Philadelphie, daté de 2150 environ}


\textbf{Stèle d'Ur Nammu Philadelphie (- 2150) }}


\includegraphics[width=7.297cm,height=10.813cm]{FaivreMartin5conf-img/FaivreMartin5conf-img93.jpg}



Cette stèle est centrée et nous avons une représentation astrale, là on
a un autre face  à face roi dieu,  On voit le roi main droite levée
devant la bouche, et on a une représentation partielle, on le voit en
prière et en dessous nous avons un autre moment de la cérémonie et il
est en offrande}


Et on comprend du coup ce que l'on a vu tout à
l'heure, la divinité est assise les pieds sur les
montagnes, on voit la façade du palais et la divinité tient les
insignes qu'elle donne au roi et on voit le roi vêtu
qui garde l'eau. On voit également les brandes de
dattes et à cette époque ils développent un rameau de végétation qui
pousse grâce à l'eau versé{\textquotedbl}}


Peut être y a t il déjà sur ce relief, pas très bien sculpté, cette
évocation de rameau de végétation, mais on n'en est
pas sur}


on voit également le dieu à corne et les arpenteurs avec leur hache,
Nous sommes là encore contemporain de la stèle de Gudéa et on voit bien
comment sur plusieurs registres on peut nous montrer différents 
moments.}


\textbf{Gudéa, le vase au quatre flots Louvre} et commentaire du Louvre}


\includegraphics[width=16.947cm,height=9.876cm]{FaivreMartin5conf-img/FaivreMartin5conf-img94.png}



Quand nous avons le roi en offrande, il ne faut pas le confondre avec
cette iconographie unique à ce jour de Gudéa  tenant un vase
d'où jaillit les quatre flots d'eau,
car cette statue, pour le moment, n'a aucune
comparaison connues.}


Ce que tient le roi entre ses mains, ce n'est pas un
cadeau pour son dieu, c'est une symbole, qui
normalement est entre les mains des dieux, ou d'un
acolyte du dieu. C'est le vase d'où
sont sortis les quatre fleuves que les dieux ont donné aux hommes au
moment de la création de l'humanité afin
qu'elle puisse survivre; }


C'est donc un attribut que nous devrions trouver entre
les mains des dieux, des héros ou des génies et en comparaison on peut
voir sur le sceau cylindre qui a un siècle de plus , qui provient
d'Akkad, }


\textbf{sceau cylindre : SHARKALISHARRI Louvre  AKKAD}}


\includegraphics[width=15.981cm,height=7.23cm]{FaivreMartin5conf-img/FaivreMartin5conf-img95.jpg}



Et là encore sur ce sceau cylindre nous voyons les héros nus à coiffure
à six boucles, les buffles, et le vase d'où jaillis
les flots}


La sculpture de Gudéa a posé beaucoup d'interrogation ,
elle a été acheté par le Louvre dans les années 1960 / 1970, puisque
c'est la seule statue où l'on voit le
roi mésopotamien ne pas avoir les mains jointes et il tient un objet
qu'il ne devrait pas tenir}


Certains chercheurs dans les années 1980, s'étaient
même demandés s'il ne s'agissait pas
d'un faux; Mais le texte en sumérien
n'est pas une copie d'un autre texte
et cette sculpture était déjà sur le marché de l'art
en 1927, époque où l'on commençait tout juste à
comprendre l'écriture sumérienne. Donc cela ne peut
être un faux}


En réalité, il faut comprendre que Gudéa parle à une déesse qui est la
petite soeur d'Ishtar , à savoir Geshtinanna, qui est
la déesse de l'eau vivifiante  et
c'est un exemple unique dans l'art
mésopotamien, où le roi a dédicacé un ex voto dans le temple, non
seulement pour Gesthinanna, mais cette divinité est également
l'épouse de Ningishizda, dieu personnel de Gudéa}


Cela signifie donc qu'il y a un lieu entre Gudéa et la
femme de son dieu personnel}


et donc de façon exceptionnelle dans la pensée de ces gens, on a
représente le roi tenant l'attribut de la déesse
qu'il doit lui apporter lui même}


c'est donc tout à fait original}


Dans le thème du vase aux eaux jaillissantes, on a toujours quatre filet
d'eau, aussi bien en peinture que sur les sceaux. et
sur les socles des petits vases on peut parfois voir des poissons  qui
remontent, et cela est pour renforcer l'idée que
c'est l'eau de la vie, car elle est
habitée . Donc c'est l'eau de
L'Absou, dans le monde mésopotamien,
l'absou c'est l'eau
de la nappe souterraine, l'eau des origines sur
laquelle flotte la bande de terre, c'est donc
l'eau pure}


Au début du II millénaire, les rois vont créer de nouvelles lignées, et
on aura la première dynastie de Babylone, ces nouvelles royautés
sémites vont reprendre complètement le répertoire sumérien existant ,
se l'approprier, et finalement ce répertoire restera
le même au II et au I millénaire}


c'est ainsi que l'on peut voir sur une
stèle de calcaire provenant de Suse, au Louvre (non trouvée), très
proche de la stèle d'Ur Nammu, mais malheureusement
nous n'avons que la partie supérieure et
c'est dans la partie inférieure que se trouvait le
texte}


nous ne pouvons donc savoir qui est le roi et qui est le dieu; On voit
un morceau de palmier et la divinité. On sait d'après
d'autres textes que l'on offre à la
divinité du lait en même temps que les dattes, }


et dans ce rituel de l'offrande, il y aura forcément de
la viande, comme en \kmt, qui provenait du troupeau élevé dans le
temple, tué dans les abattoirs du temple et tués juste avant
d'être offert à la divinité}


Ces scènes sont rares, toutefois Monsieur PARROT avait découvert une
peinture dans le palais de Mari, qui est une peinture très restaurée,
il avait fat un relevé dessin, et nous avons ainsi un personnage qui ne
peut être que le roi, qui conduit une bête marquée}


\includegraphics[width=14.393cm,height=9.49cm]{FaivreMartin5conf-img/FaivreMartin5conf-img96.png}



nous avons  plusieurs indication , notamment textuel de ces animaux
parés d'éléments d'orfèvrerie, pour
les rendre beaux avant le rituel de l'offrande et leur
présentation à la divinité}


Là on voit une même approche que dans les scènes égyptiennes, car
l'offrande de la viande elle même ne sera jamais
représentée. En effet, l'offrande de tout ce qui est
pétrissable, un peu sale, n'est jamais figuré. IL y a
le reste qui résume finalement tout ce que l'on peut
offrir au dieu en offrande.}


KUDURU KASSITE (photo non trouvée)}


il y a l'encensoir qui est  présent et
c'est encore un point commun aux deux civilisations,
sauf qu'en Mésopotamie, nous ne verrons jamais le roi
procéder lui même à l'encenssement des offrandes}


et toujours la même difficulté, le monde mésopotamien est plus réduit
dans le monde des images, et non va nous montre : le roi assis, le dieu
debout et l'encensoir}


l'encensoir permet de brûler l'encens,
ce qui va purifier l'espace , en chassant les mauvais
esprits et les démons}


\textbf{En \kmt}, là encore les  images sont plus nombreuses :}


Le roi en offrande peut être debout et chose intéressante par rapport au
monde Mésopotamien, il peut être à genou (le roi mésopotamien est
debout devant la divinité ou la table d'offrande. Nous
avions vu que pour le monde mésopotamien, le roi est debout devant la
divinité ou la table d'offrande}


en \kmt on va dès l'Ancien Empire synthétiser la
notion d'offrande par cette image du roi à genou
tenant les vases nou, globulaires ou à vin et on sait bien que cette
image n'est pas réduite à ce qu'elle
représente en lecture première, c'est à dire une
offrande de vin, mais qu'à elle seule, elle résume
tout le rituel de l'offrande et tout ce qui pouvait
être donné au dieu}


\textbf{statuette de Pépi 1er Brooklin }}


\includegraphics[width=9.202cm,height=14.744cm]{FaivreMartin5conf-img/FaivreMartin5conf-img97.jpg}



C'est la première attestation avec une iconographie qui
va perdurer dans les siècles suivants.}


\textbf{Linteau de Médamoud Louvre (sésostris III)}}


\includegraphics[width=15.981cm,height=7.126cm]{FaivreMartin5conf-img/FaivreMartin5conf-img98.jpg}



La plus ancienne scène qui nous montre plus précisément le rituel de
l'offrande est le linteau de Médamoud}


En \kmt, nous manquons de source pour le temple à
l'ancien empire, on a des plans, on sait que ce sont
des bâtiments de petites tailles en briques crues, mais très peu
d'élément de décor et de relief appliqué sont parvenus
jusqu'à nous.}


Au début du II millénaire, avec le Moyen Empire,
l'architecture de pierre (en grès ou en calcaire se
développe) et du coup le répertoire d'image augmente}


Et c'est donc au Moyen Empire, que nous allons avoir ce
répertoire d'images, de scènes qui nous donnent plus
d'information sur le rituel de
l'offrande (avant il existait forcément, mais  non
conservé car pas en pierre)}


Ici il est émouvant de voir Sésostris III, 5ème roi de la XII dynastie
représenté deux fois face au dieu thébain Montou, en train de frapper ,
consacrer le pain blanc, en référence aux offrandes que
l'on donnait à la divinité}


et cette image va se figer car jusqu'à
l'époque romaine, quand on représente le roi offrant
au dieu son repas dans le contexte du culte divin journalier,
c'est avec la tranche de ce pain  à la main que nous
le verrons et avec le gâteau châtt et le reste peut être connu avec les
inventaires , même s'il n'est pas
figuré, (offrandes de légumes, de viande , on sait
qu'il y avait du bétail préparé pour être offert au
dieu}


mais les temples du Nouvel Empire, essentiellement à partir des
thoutmossides  (XVIII dynastie) vont voir se développer les scènes de
représentation d'offrandes (ex Morceau à Boston, où
l'on voit une table où il y a la représentation du
garde manger de la divinité, avec tout ce qui est représenté sur une
table d'offrandes, les encensoirs, et la
représentation en rabattu de tout cela}


tous ces détails nous montrent que ce que l'on figure,
ce sont les mêmes choses que l'on a depuis
l'Ancien Empire, sur la table
d'offrande des défunts, pain triangulaire ou rond, on
retrouve donc les mêmes types de chose qui sont figurés et le roi en
offrande}


Le roi que nous allons voir, figure non pas avec une chose figée dans la
main, le vas, comme on l'avait vu précédemment dans le
monde mésopotamien, mais effectivement à certains moments choisi avec
comme pour le rituel de fondation, la difficulté de classement dans le
rituel lui même de ce qui était fait en premier et après les gestes qui
venaient dans un temps secondaire}


Il y a certaines thématiques qui seront p lus présentes que
d'autres car elles sont fondamentales, comme
l'offrande de la MAAT, qui termine le rituel divin
journalier, le rituel de l'encensement (et là il y a
deux choses, l'offrande de l'encens
et le rituel de l'encensement)}


et donc là on voit le roi égyptien, actif, devant son père divin , il
tient l'encensoir (Louxor on a
l'offrande de l'encensoir)}


Il y a aussi le fait que l'on va représenter le roi de
façon particulière, le roi en offrande peut être un roi sphinx en
attitude d'offrande }


\textbf{Aménophis III New York} :}


\includegraphics[width=9.031cm,height=7.23cm]{FaivreMartin5conf-img/FaivreMartin5conf-img99.jpg}



Et on peut voir l'importance des mains dans la
représentation égyptienne, ces mains tenant le vase nou, sont énormes,
on les voit englober l'ensemble de
l'objet qu'elles tiennent , ce qui
montre la valeur symbolique de l'objet
qu'elles tiennent}


le roi peut  être montré à genou avec un élément emblématique en face de
lui, ce qui va se développer à partir d'Hatchepshout 
et ce sont des objets de taille importante et le roi à genou sera à
l'époque Ramesside dans une lignée amarnienne de plus
en plus poussée (jusqu'à présent, nous avions le roi à
genou, mais le buste droit, alors qu'avec
l'époque amarnienne nous aurons une gestuelle plus
accentuée et le roi à genou se prosterne d'une façon
beaucoup plus écrasée face à son dieu, et cette attitude aura un impact
de le canon ramesside.}


Si on prend l'exemple de Séthi Premier \textbf{
}provenant d'Abydos,  on peut voir une avancée de la
jambe gauche par rapport à la jambe droite, qui indique le mouvement du
dieu vers son roi et le roi offre une attitude de lui même dans
l'offrande et il s'adresse au
symbole. il est face non  pas à une image inventée
d'un être divin, mais à un objet cultuel existant
réellement. Cela est très particulier nous n'avons
jamais cela dans le monde mésopotamien}


A ce jour, nous n'avons jamais trouvé de représentation
où l'on peut se dire que cela est la représentation
d'une vraie statue, d'une vrai image
de culte vénérée par les humains, ce que nous aurons en \kmt  à 
partir du Nouvel Empire}


Et cela ira en s'accentuant à partir de Ramsès II et
ses suiveurs, car on sait que les canons mis en place par Ramsès seront
repris par les autres pharaons et ensuite par les rois de la XX
dynastie , avec cette position très aplatie}


En bas relief et en statuaire, c'est très représentatif
de l'époque Ramesside, ce qui montre bien la trace, la
réminiscence de la pensée amarnienne, laissée entre le roi et son dieu}


Et on constate que même les souverains d'origine
étrangère qui vont régner sur l'\kmt au premier
millénaire vont s'intégrer à ce cadre cultuel}


\textbf{Taharqa face au dieu Hémen}  (roi soudanais, Louvre)}



\begin{figure}
\centering
\includegraphics[width=15.556cm,height=11.158cm]{FaivreMartin5conf-img/FaivreMartin5conf-img100.png}
\end{figure}

et c'est dans l'attitude de
l'offrande du vase à vin (référence de la grande crue 
pour remercier le dieu de cette crue, qui est arrivée à temps}


\ \ \textbf{LE ROI COMBATTANT}}


A partir du moment où le roi est le représentant du dieu sur terre, il
va affirmer cette information pour affirmer à ses voisins
qu'il ont pas intérêt à venir prendre son territoire}


Nous avions vu que dès les premières images royales, le thème des images
associées à celle du roi permettait de l'identifier et
on avait vu en Mésopotamie comme en \kmt, le {\textquotedbl}tout nu
ligoté et désarticulé{\textquotedbl} placé pas loin
d'un bonhomme et cela indiquait qu'il
y avait une représentation du souverain}


l'autre élément que nous avions vu était la
représentation des armes, dont la taille était tout à fait excessive}


Dans le monde sumérien, il n'y a aucune attestation de
guerre, identifiable avec certitude avant les dynasties archaïques III,
qui démarrent - 2600. On associe l'apparition de
conflits entre les petits royaumes sumériens à cette époque avec
l'apparition de la notion d'hérédité
dynastique; }


En effet, dans les textes c'est le moment où si
l'on écrit le nom du roi à côté du personnage,
c'est une chose, mais mettre la filiation, fils de ...
, cela n'existait pas auparavant. Et nous sommes
arrivés à la conclusions que c'était un moment (qui a
dû se faire doucement) qui révèle une volonté
d'institutionnaliser l'hérédité du
sang pour la transmission de la fonction royale}


A partir de ce moment là, la guerre apparaît dans les royaumes
sumériens, qui sont petits et très proches les uns des autres, et le
phénomène de la guerre apparaît. Ce phénomène apparaît donc vers - 2600
et cela ne s'arrêtera plus. Et cela va entraîner des
choses passionnantes :}


c'est le moment où l'on va commencer à
dire : attention la guerre accomplie par ce roi est en réalité une
guerre de son dieu dans le ciel, face à un autre dieu dans le ciel. Ces
dieux se disputent pour un territoire qui est sur terre,  et donc ce
conflit sera vécu par les hommes et chacun se bat pour le dieu de son
royaume}


et donc pour la première fois apparaît que le roi est la bras armé de
son dieu. Et donc s'il obtient la victoire ,
c'est grâce à son dieu . Et il y a ainsi tout un
ensemble de choses qui vont se mettre en place , et qui font que du
coup on va pouvoir montrer l'ennemi du doigt et
s'il a perdu c'est
qu'il n'a pas bien prié son dieu et
on va insister là dessus par la mise en image}


Et pour nous c'est très intéressant}


\textbf{Stèles des vautours} : }


\includegraphics[width=13.189cm,height=5.355cm]{FaivreMartin5conf-img/FaivreMartin5conf-img101.png}



commentaire du Louvre }


\includegraphics[width=15.169cm,height=23.53cm]{FaivreMartin5conf-img/FaivreMartin5conf-img102.png}



\includegraphics[width=15.981cm,height=24.236cm]{FaivreMartin5conf-img/FaivreMartin5conf-img103.png}



\includegraphics[width=15.628cm,height=16.616cm]{FaivreMartin5conf-img/FaivreMartin5conf-img104.png}



C'est la première mise en image de ces récits
accompagné de la plus longue inscription en sumérien, pour cette
époque, - 2500 et le roi représenté est le petit fils
d'Ur Nanshe(que l'on a vu
précédemment avec sa corbeille sur la tête)}


Il y a donc une évolution, Ur Nanshe était représenté de façon tout à
fait pacifique, alors que son petit fils est représenté en combattant}


\newline
Et dans le texte qui accompagne cette stèle, il est dit {\textquotedbl}
comme mon grand père, Ur Nanshe avait combattu ceux
d'Uma, Comme mon père .... }


il se place donc dans une lignée}


Il faut aussi remarquer que l'on ne prononce par le nom
propre de son ennemi, mais on dit le nom de son dieu, et ici il est
donc évoqué le nom du dieu d'Uma qui a voulu
revendiquer le territoire du dieu Nin Girsu, et cela entraîne
inévitablement un conflit entre ces deux royaumes}


On peut remarquer également que la bataille n'est pas
représentée, les ennemis sont déjà morts, nus en tas et on leur saute
dessus à pied joint, ce qui n'est bien évidemment pas
la réalité d'un combat}


Comme pour les égyptiens, les mésopotamiens se réfèrent à
l'image pour dire j'ai vaincu.
L'image du tout nu écrabouillé remonte à une tradition
très ancienne}


Une particularité de la Mésopotamie, est l'importance
du char, et ce dès les époques anciennes}


Jamais en \kmt au III millénaire, le thème du char a existé , et il
faut attendre l'apparition du cheval est sa
domestication. Alors qu'en Mésopotamie , le char est
un objet de prestige,  c'était des chars comme on en
avait à l'époque romaine, à savoir : de gros chars à
roue en bois tiré par des boeufs. Ce n'est donc pas
une charge héroïque, mais le char est important et cela est  à
mémoriser, car on verra sur certains objets que le roi
n'est pas représenté, mais dans un coin de
l'image on voit un char et on sait  que cette
représentation du char équivaut à celle du roi. En effet, soit ils
représentent des petites bonhommes dont l'un est le
roi, soit il représente un char et l'armée, qui est
une infanterie . Il faut remarquer également que le roi  parle de lui à
la première personne {\textquotedbl} j'ai fait ...
{\textquotedbl} car c'est le roi , le bras armé de son
dieu.}


Nous avons également, l'infanterie, la parade,
l'enterrement des morts, autres éléments, qui se
mettent en place, mais on n'évoque jamais les pertes.}


sur les 2500 ans de récit de guerre en Mésopotamie antique, il y a un
seul texte connu pour le moment, où le roi évoque ses pertes, (tablette
au Louvre sur la campagne d'un roi An Ur artou ?), et
encore il indique qu'il ne trouve plus ses quatre
hommes, il ne dit pas qu'ils sont morts, mais
qu'ils sont perdus}


attention, il ne faut pas bien entendu prendre à la lettre ces récits}


Et à partir de là, beaucoup de choses vont se mettre en place, car le
roi se sert de cette armée vaincue, on déporte la population, soldats
qui serviront de main d'oeuvre,  (au retour du combat
on offre au dieu Nin Girsu un lot d'esclaves qui
proviennent de ces soldats capturés et ramenés,  Et assez vite, ils
emmèneront également la population, femmes et enfants des territoires
vaincus . On a des attestations qui révèlent que certains de ces
prisonniers seront échangés contre une rançon}


On insiste toujours sur le fait que le roi a obtenu la victoire grâce à
son dieu, car sur l'autre côté de cette stèle on a une
image que l'on a pris longtemps pour une
représentation du roi, mais c'est sans doute une image
du dieu, car il tient l'arme des dieux, et
qu'il tient les ennemis dans un filet, il
s'agit sans doute d'une
représentation du Dieu Nin Girsu, dieu vainqueur, lui même représenté
en train de mettre les ennemis dans un filet}


C'est bien évidemment très dommage
qu'elle ne soit pas mieux conservées , car on voit le
roi partir  la chasse avec une peau de chèvre, il a son casque et une
lance de telle taille, qu'elle est insoulevable. En
face on ne sait pas qui c'est : peut être le roi
d'Uma ? on voit des mulets, qui viennent
d'Anatolie, et qui sont très résistants, ce sont des
anagres, et un peu plus prestigieux que les boeufs,(  mais attention
dans ces sociétés , un boeuf a une grande valeur)}


est une représentation du roi ou de son dieu ? on a toujours cette
interrogation dans les représentations mésopotamiennes, }


cf casque d'Ur}


il y a trente ans on disait aux élèves à propos de cette stèle, que
c'était le roi qui tenait ses ennemis dans un filet,
maintenant on est plus nuancé}


d'une part si on se réfère aux textes, on voit que cela
n'est peut être pas cela, ce sont des serments pris au
nom du dieu, et derrière il y a un individu de taille plus petite, donc
inférieur, avec une tiare à deux cornes. or il est inimaginable
qu'un être divin puisse être derrière une humain, dans
la conception mésopotamienne, et donc à partir de là on a compris que
cette stèle avait une importance beaucoup plus  importante que
l'on pensait, et qu'elle représente
un dieu.  Et cela est très intéressant car les mésopotamiens rechignent
à représenter le dieu dans son temple, mais là il le représente sur une
stèle commémorant une victoire, stèle qui était peut être à
l'entrée du temple}


\textbf{Etendard d'UR}}


face du banquet}


\includegraphics[width=15.981cm,height=6.985cm]{FaivreMartin5conf-img/FaivreMartin5conf-img105.jpg}



face de la guerre}


\includegraphics[width=12.347cm,height=5.75cm]{FaivreMartin5conf-img/FaivreMartin5conf-img106.jpg}



Il est fait avec de la nacre collé sur du bois avec du goudron.}


C'est souvent la face du banquet qui est représenté, 
mais nous allons nous intéresser à la face représentant la guerre et
elle se lit de bas en haute (comme le vase d'Uruk), on
y voit le char en haut,  les soldats vainqueurs habillés et les ennemis
nus}


Et on peut donc voir le char garé dans un coin, avec le conducteur de
char et un personnage un peu plus grand que les autres et son casque
dépasse du cadre et c'est donc la représentation du
roi,  et les soldats se dirigent}


sur l'autre face, le banquet après une victoire en bas
à gauche on peut voir le butin}


A l'époque d'Akkad, entre2350 - 2150,
les choses vont changer et les choses sont différentes car avec les
rois d'Akkad durant 150 ans on a déjà un royaume, un
royaume qui  part faire des raids, à partir de leur capitale (que nous
n'avons jamais retrouvé)}


Durant cette période, les principes de souveraineté évoluent . On avait
vu lors de la conférence précédente, Naran Sim, premier roi divinisé de
son vivant) et on peut constater au vu des objets
d'art de cette époque, que nous passons à une
dimension de narration qui n'existait pas auparavant}


nous allons rester sur l'idée du tout nu, le vaincu ,
dans une gestuelle plutôt ridicule et l'habillé, pour
magnifier le vainqueur}


Déjà à cette époque, il y a un souci de représenter
l'armement, et l'habillement, que
nous n'avions pas auparavant; un souci de rendre
hommage à l'archer, car on sait très bien que
c'est comme cela qu'ils avaient
réussi à créer leur royaume, c'est
l'époque où l'on a inventé
l'arc composite, fait de plusieurs bois}


On a ainsi une technique militaire , avec une attaque en plusieurs
temps, avec les archers qui ouvrent les brèches et
l'infanterie qui part dans un deuxième temps.}


Du coup, on va nous représenter les batailles dans une composition deux
à deux, cela nous pouvions l'avoir dans les temps
anciens, mais on va représenter le vaincu toujours dénudé, avec un
souci réaliste au niveau de l'anatomie. Ce sont de
vraies petites histoires en image}


\textbf{stèle du Roi Naram Sîm\ \   (Louvre)}}


\includegraphics[width=13.965cm,height=20.038cm]{FaivreMartin5conf-img/FaivreMartin5conf-img107.jpg}



et là, même si on ne connaît rien à l'art mésopotamien,
on peut tout de suite remarquer que cette stèle n'a
rien à voir avec celle des vautours}


Ici il y a une mise en scène dans la composition pour la place du roi au
dessus de la mêlée}


Nous ne sommes plus dans une disposition en registre, avec ces bandes où
nous avons le char, l'infanterie ; là déjà on nous dit
dans le texte d'origine, on nous dit
qu'il a combattu sur une montagne, or cela indique que
le roi, homme de la plaine, a combattu des montagnards, donc à un
endroit qu'il connaissait moins bien que son ennemi et
on insiste sur ce fait pour accentuer sa victoire contre une tribu dans
le mont Zagros}


Et pour  magnifier cette victoire, on va la mettre en scène et
c'est très exceptionnel dans l'art
mésopotamien une mise en scène ainsi, c'est
pratiquement unique}


Il y a dans cette idée d'ascension, car on a ce
registre de chemins et des hommes qui montent vers  le roi, et on a le
roi au dessus, parfaitement centré au dessus de tout le monde.
C'est la première fois que nous avons la
représentation héroïque de la personne royale, et
c'est la première fois que l'on
insiste sur le fait (contrairement aux petits rois sumériens) que le
roi peut faire des choses que les autres ne savent pas faire, à la ils
nous lancent une façon de montrer le roi qui va perdurer 
jusqu'à la  Babylone tardive}


Dans les récits de guerre à ce moment là, puisqu'il est
protégé par son dieu, s'il doit escalader une
montagne, le roi y sera avant tout le monde, s'il doit
traverser une rivière, il le fera sans encombre}


Et cette idée d'ascension n'est pas
montrée par le chemin que nous avons évoqué, mais par la pointe de la
montagne qui est figurée. Or normalement pour les images
mésopotamiennes, les fonds de décor sont plutôt au premier millénaire
et surtout sur cette image on peut voir l'ennemi qui
tombe la tête la première en bas vers le précipice, et roi qui détruit
ses ennemis du pied gauche (c'est le mauvais pied) et
ensuite pour la première fois nous avons la réalité de son armement, on
voit sa hache de guerre, qu'il tient contre son bras
gauche, et si on regarde bien elle est rattachée par une corde à son
poignet (et on sait que c'est une méthode de guerre 
typique de l'époque, ce qui leur permettait de lancer
leur hache sans avoir à se déplacer pour la récupérer}


Il  y a donc aussi une volonté de montrer leur supériorité au niveau de
l'armement, de l'organisation, un roi
qui a des flèches exceptionnelle (taille exceptionnelle de ces flèches)
et cela traduit là encore une volonté de magnifier le roi par un
armement exceptionnel en lui donnant des armes énormes. Même si cela
n'est pas nouveau (on avait vu un sceau où le roi
avait des armes énormes), mais là cette  image va être réutilisée à
partir de maintenant}


Malheureusement, nous n'avons pratiquement aucune image
pour le II millénaire}


nous avons un fragment de stèle du roi Samsi Addu (louvre)}


\includegraphics[width=7.056cm,height=5.609cm]{FaivreMartin5conf-img/FaivreMartin5conf-img108.jpg}



roi d'un royaume entre la Syrie et
l'assyrie}


commentaire du Louvre}


\includegraphics[width=15.981cm,height=4.339cm]{FaivreMartin5conf-img/FaivreMartin5conf-img109.png}



A partir du moment où Sargon fonde cette lignée royale, un royaume
unifié, sans dire {\textquotedbl} je l'ai fait pour
tel ou tel dieu {\textquotedbl}, mais simplement, ' je
l'ai fait, je suis un homme exceptionnel
{\textquotedbl} }


mais en réalité on revient indirectement au dieu, car si je suis
exceptionnel, c'est par ce que les dieux
m'ont fait exceptionnel, ils m'ont
donné des dons exceptionnel,  j'ai été béni dès mon
berceau}


d'où le fait qu'il va y avoir une
légende dorée, qui va se forger autour de Sargon et
c'est très important car dans le monde mésopotamien,
on n'a aucun lien entre ces différentes royautés,
elles résultent de conflits, d'usurpations, et de
nouvelles familles qui arrivent au pouvoir et à partir de Sargon,
c'est incroyable, on va voir qu'à
partir de lui et pendant 2000ans, celui auquel ils se rattacheront sera
SARGON. En effet , ils ne peuvent pas se légitimer  autrement,
puisqu'ils n'ont aucune légitimité
réelle  (ils arrivent au pouvoir par un coup de chance,
d'intelligence) , alors à chaque fois ils prendront
comme modèle Sargon}


et du coup, toutes ces images nouvelles, même si nous
n'en avons pas beaucoup reviennent à celles
d'Akkad, ils n'inventent rien de
plus}


et quand on voit cette stèle de Samsi ADDU guerroyer ,
c'est dans la même lignée : le roi avec sa hache de
guerre, (taille énorme) frappant l'ennemi, écrase
l'ennemi du pied gauche, ennemi désarticulé }


Par contre il y aura une évolution en ce sens où le vaincu finira par
rester habillé. Nous sommes dans un monde où petit à petit le costume
va servir à identifier le groupe humain, donc ce sera mieux de le
laisser vêtu (au départ nudité pour traduire
l'humiliation)}


En  \kmt , c'est très Ramesside de mettre en place
l'anéantissement de l'ennemi}


Et à partir de ce moment là on va voir que les thèmes évoqués permettent
d'évoquer la guerre, comme ce que
l'on trouve à Ninrumd dans les palais (base
d'un trône royal où l'on  voit le roi
assyrien en train de serrer la main du roi babylonien, mais on  place
ce dernier sous les pieds , ce qui indique une position de vassal, et
il est donc figuré comme un vaincu}


stèle de ASSARHADON , British Muséum (non trouvée) VII siècle}


Là les rois qu'il a vaincu sont du monde syrien, et
sont représentés minuscules, en train de supplier ASSARHADON de ne pas
les massacrer}


Effectivement, les assyriens vont avoir une politique de guerre, et ils
vont un peu modifier la manière de la raconter par rapport au monde des
dieux, et revenir sur quelque chose qui est plus ancien,
c'est de  dire : c'est Ashur qui
m'a dit de faire la guerre}


Cela veut dire que tous les assyriens au 9ème, 8ème et 7ème siècle, ne
racontent jamais une guerre sans mentionner : j'ai
fait cette guerre pour promouvoir l'universalité du
culte de mon dieu Assur, le seul dieu que l'on doit
vénérer avant les autres, les autres n'étant que des
dieux secondaires. Tant que vous ne vénérez pas Asur,
j'irai vous massacrer  et c'est ainsi
qu'ils vont agrandir leur royaume}


Il est intéressant que l'on reprend ce discours à
partir du moment où la guerre devient finalement, au départ
c'était pour se protéger, un moyen
d'augmenter le territoire par les conquêtes}


Les assyriens vivent dans un royaume un peu isolé au nord est de
l'irak , c'est une plaine, qui
n'a pas beaucoup de richesse naturelle (minerais,
bois) Et à partir du moment où ils savent qu'ils sont
entourés par des régions plus riches, il vont faire la guerre au nom
d'Assur, pour enrichir leur dieu}


On pourra remarquer que le roi continue à parler à la première personne
du singulier}


Quand il parle à la troisième personne du singulier,
c'est qu'il s'agit
de bataille de moindre importance, et qui en réalité est faite par
l'un de ses généraux;}


le roi lui même n'est jamais représenté en combattant.}


depuis les origines, il y avait cette image du roi qui combat, le roi
assyrien, au contraire, est montré sur son char, et si on regarde il y
a le conducteur du char (le roi) et le porte ombrelle }


Jamais nous ne verrons le roi mésopotamien conduire son char, alors
qu'en \kmt la conduite du char par le roi relève
d'une technique plus héroïque, dynamique. On voit que
le roi égyptien a les rennes autour de sa taille, et généralement il en
profite pour tuer ses ennemis (tir à l'arc, ou avec sa
massue), et cela traduit une volonté de mise en scène impressionnante
et le roi égyptien est montré dans le feu de l'action
; alors qu'en Mésopotamie très vite il ne
l'est plus (il est à coté de son char, à coté des
ennemis déportés et rien d'autre autour)}


magnifier le roi, c'est aussi mettre des choses
merveilleuses autour de lui, batailles, certes mais on peut montrer
l'image des statues de culte de
l'ennemi que l'on déporte aussi en
Mésopotamie, tout comme la population .Les assyriens en effet déportent
les dieux des populations vaincues}


De la même façon nous aurons le roi sur son char sur les bas reliefs,
avec le butin et les déportés, la bataille elle même, mais
c'est plus l'environnement que la
guerre elle même qui est représenté, qui magnifie ASSUR et son
représentant, le roi}


En \kmt nous avons le même type de développement ,
c'est à dire l'image très ancienne du
roi et de son ennemi (palette de Narmer qui est la plus ancienne et qui
pousse la narration un peu plus loin et on a la couronne rouge,, les
ennemis décapités, la tête entre les jambes


\textbf{massacre des ennemis Akhénaton (Boston) }(non trouvé)


En \kmt également, le roi combattant ses ennemis est attesté sous
toutes ses formes (pour le proche orient nous n'avons
aucune représentation en joaillerie et on ne sait si ils en avaient),
par contre nous l'avons en \kmt


A l'époque des Toutmossides, pour mettre encore un peu
plus de mise en scène nous aurons la grappe d'ennemis,
cela les mésopotamiens ne l'ont jamais fait, ils ont
montré le défilé des déportés, mais magnifier le roi en le montrant
seul, gigantesque, tenant une grappe d'hommes
implorant, n'existe pas en Mésopotamie, mais on peut
le voir à Abou Simbel


Comme dans le thème de l'offrande, nous avions vu que
le roi pouvait prendre la forme d'un sphinx  tenant
les vases globulaires, et on avait vu le trône de Thoutmosis IV , on
voit le roi en sphinx piétinant ses ennemis


En \kmt, dans la thématique du roi en guerre , il faut attendre le
nouvel empire, la domestication du cheval, pour voir le char à deux
roue à sillon, pour que le thème du roi sur son char apparaisse dans
les représentations militaires , avec cette particularité égyptienne de
tenir les rennes à la taille afin de conserver l'usage
de ses mains pour tuer l'ennemi


Sous Ramsès, mais nous en avons déjà parlé, l'artiste
volontairement omet de mettre la représentation sous la forme de
registre, pour accentuer la chaos


Le thème du roi sur son char n'est pas uniquement
associé à la guerre et c'est un point commun à ces
deux civilisations, le roi peut être associé à la chasse, et ce sont
même les mêmes animaux qu'ils vont chasser : le lion
et le taureau


On sait par les textes mésopotamiens, que le roi chassait
l'éléphant sauvage en Syrie du Nord et
l'autruche, mais on ne le représente pas avec ces
animaux qui n'ont aucune valeur symbolique dans leur
religion


Attention la comparaison de ce répertoire d'images
entre ces deux civilisations, se fait avec un décalage chronologique,
les images égyptiennes datant de  Ramsès (1300-1100) et les  images
mésopotamiennes sont plus récentes 9 - 8ème siècle


Assurbanipal II chassant le taureau, bas relief de Nimrund, : là il y a
davantage de mise en scène que pour une représentation de guerre, on
veut montrer également l'action des animaux : chevaux
et taureau


En \kmt pour renforcer le caractère extraordinaire , on nous
représente l'univers  , le paysage , le marais dans
lequel sa majesté est allée chasser


dans ces deux civilisations, la chasse va être représentée et utilisée
dans l'art avec la même signification que la guerre,
c'est repousser le chaos


En Mésopotamie : repousser le chaos c'est ce qui
n'est pas mis en place par les dieux, 


En \kmt, c'est différent, mais cela veut dire la même
chose, c'est repousser l'isfet, qui
est le chaos, et tout ce qui n'a rien à voir avec la
perfection des premiers jours


Le Roi à toutes les époques est montré comme le lien direct avec les
dieux, car il est le seul à pouvoir ordonner une bataille contre les
forces perturbatrices 


Ramsès tuant un tout petit lion avec une lance énorme (ostraca)


et Assurbanipal (Louvre) attrapant le lion par sa crinière et le tuant
avec une énorme lance


et on insiste bien sur cette chasse qui a été possible grâce aux dieux
qui ont protégé le roi et lui ont permis d'accomplir
cette mission


(on sait qu'en Mésopotamie, les lions étaient gavés
plusieurs jours avant la chasse, idem pour les égyptiens

\end{document}
