% This file was converted to LaTeX by Writer2LaTeX ver. 1.2
% see http://writer2latex.sourceforge.net for more info
\documentclass[a4paper]{article}
\usepackage{neferalias}
\usepackage{nefertiyi}

%\usepackage[utf8]{inputenc}
%\usepackage[T1]{fontenc}
%\usepackage[french,english]{babel}
%\usepackage{amsmath}
%\usepackage{amssymb,amsfonts,textcomp}
%\usepackage{color}
%\usepackage{array}
%\usepackage{supertabular}
%\usepackage{hhline}
%\usepackage{hyperref}
\hypersetup{pdftex, colorlinks=true, linkcolor=blue, citecolor=blue, filecolor=blue, urlcolor=blue, pdftitle=, pdfauthor=Florence, pdfsubject=, pdfkeywords=}
\usepackage[pdftex]{graphicx}
%% Page layout (geometry)
%\setlength\voffset{-1in}
%\setlength\hoffset{-1in}
%\setlength\topmargin{1.249cm}
%\setlength\oddsidemargin{2.499cm}
%\setlength\textheight{23.551cm}
%\setlength\textwidth{16.002998cm}
%\setlength\footskip{0.0cm}
%\setlength\headheight{1.251cm}
%\setlength\headsep{1.15cm}
%% Footnote rule
%\setlength{\skip\footins}{0.119cm}
%\renewcommand\footnoterule{\vspace*{-0.018cm}\setlength\leftskip{0pt}\setlength\rightskip{0pt plus 1fil}\noindent\textcolor{black}{\rule{0.25\columnwidth}{0.018cm}}\vspace*{0.101cm}}
%% Pages styles
%\makeatletter
%\newcommand\ps@Standard{
%  \renewcommand\@oddhead{\thepage{}}
%  \renewcommand\@evenhead{\thepage{}}
%  \renewcommand\@oddfoot{}
%  \renewcommand\@evenfoot{}
%  \renewcommand\thepage{\arabic{page}}
%}
%\makeatother
%\pagestyle{Standard}
%\setlength\tabcolsep{1mm}
%\renewcommand\arraystretch{1.3}

\newcommand{\DirImg}{../img/FaivreMartin/}

\title{}
\author{Florence}
\date{2012-07-11}

\begin{document}
%%%%%%%%%%%%%%%%%%%%%%%%%%%%%%%%%%%%%%%%%%%%%%%%%%%%%%%%%%%%%%%%%%%%%%%%

%\clearpage\setcounter{page}{1}\pagestyle{Standard}

\textbf{\ \ \ \ ATTRIBUTS EN EGYPTE ET EN MESOPOTAMIE\ \ }

%\begin{figure}
%\centering
%\includegraphics[width=13.964cm,height=3.698cm]{\DirImg FM1_01.jpg}
%\end{figure}
%
%\begin{figure}
%\centering
%\includegraphics[width=3.456cm,height=6.49cm]{\DirImg FM1_02.png}
%\end{figure}
%
%\begin{figure}
%\centering
%\includegraphics[width=10.759cm,height=10.829cm]{\DirImg FM1_03.jpg}
%\end{figure}
%
%\begin{figure}
%\centering
%\includegraphics[width=13.964cm,height=20.141cm]{\DirImg FM1_04.jpg}
%\end{figure}
%
%\begin{figure}
%\centering
%\includegraphics[width=6.808cm,height=10.088cm]{\DirImg FM1_05.jpg}
%\end{figure}
%
%\begin{figure}
%\centering
%\includegraphics[width=7.76cm,height=6.032cm]{\DirImg FM1_06.jpg}
%\end{figure}
%
%\begin{figure}
%\centering
%\includegraphics[width=9.201cm,height=7.511cm]{\DirImg FM1_07.jpg}
%\end{figure}
%
%\begin{figure}
%\centering
%\includegraphics[width=7.76cm,height=10.335cm]{\DirImg FM1_08.jpg}
%\end{figure}
%
%\begin{figure}
%\centering
%\includegraphics[width=6.843cm,height=7.548cm]{\DirImg FM1_09.jpg}
%\end{figure}
%
%\begin{figure}
%\centering
%\includegraphics[width=8.249cm,height=12.791cm]{\DirImg FM1_10.jpg}
%\end{figure}
%
%\begin{figure}
%\centering
%\includegraphics[width=15.98cm,height=19.966cm]{\DirImg FM1_11.jpg}
%\end{figure}
%
%\begin{figure}
%\centering
%\includegraphics[width=15.98cm,height=11.394cm]{\DirImg FM1_12.jpg}
%\end{figure}
%
%\begin{figure}
%\centering
%\includegraphics[width=5.82cm,height=6.667cm]{\DirImg FM1_13.jpg}
%\end{figure}
%
%%\begin{figure}
%%\centering
%%\includegraphics[width=0.528cm,height=0.387cm]{\DirImg FM1_14.png}
%%\end{figure}
%
%\begin{figure}
%\centering
%\includegraphics[width=15.98cm,height=23.882cm]{\DirImg FM1_15.jpg}
%\end{figure}
%
%\begin{figure}
%\centering
%\includegraphics[width=15.98cm,height=11.993cm]{\DirImg FM1_16.jpg}
%\end{figure}
%
%\begin{figure}
%\centering
%\includegraphics[width=15.98cm,height=16.373cm]{\DirImg FM1_17.jpg}
%\end{figure}
%
%\begin{figure}
%\centering
%\includegraphics[width=8.43cm,height=18.061cm]{\DirImg FM1_18.jpg}
%\end{figure}
%
%\begin{figure}
%\centering
%\includegraphics[width=9.201cm,height=16.484cm]{\DirImg FM1_19.jpg}
%\end{figure}
%
%\begin{figure}
%\centering
%\includegraphics[width=15.98cm,height=7.125cm]{\DirImg FM1_20.jpg}
%\end{figure}
%
%\begin{figure}
%\centering
%\includegraphics[width=15.98cm,height=15.557cm]{\DirImg FM1_21.jpg}
%\end{figure}
%
%\begin{figure}
%\centering
%\includegraphics[width=15.98cm,height=8.395cm]{\DirImg FM1_22.jpg}
%\end{figure}
%
%\begin{figure}
%\centering
%\includegraphics[width=15.98cm,height=9.877cm]{\DirImg FM1_23.jpg}
%\end{figure}


CONFERENCE N°1 \ 


Aujourd'hui, nous allons poser les choses, c'est à dire voir comment à partir d'une certaine idéologie royale, on voit
apparaître un type d'images. Sans approfondir, \ il faut quand même marquer les grandes caractéristiques de la royauté
mésopotamienne, et les grandes caractéristiques de la royauté égyptienne.


A partir du deuxième cours, nous nous intéresserons à l'apparition de l'iconographie royale et nous ne traiterons que de
la fin du IV millénaire pour croiser certains objets, les regarder en chronologie, la plus juste possible. Entre Uruk
et Nagada, on peut s'aligner et on pourra ainsi constater que des points communs sont ainsi à remarquer, mais très tôt
également que des caractéristiques spécifiques à chacune de ces deux civilisations s'affirment et finalement dès les
premières images qui sont parfois extrêmement succinctes.


On s'intéressera ensuite , durant les trois séances suivantes, \ à la thématique liées à la relation privilégiée du
souverain avec son dieu, et là encore il faut toujours s'interroger sur sa nature. Ce lien privilégié qui fait de ce
roi un constructeur, et nous verrons donc la mise en image de ce roi constructeur.

Nous verrons également comment il est administrateur des biens du dieu, et de ce fait un défenseur de ces biens, \ Et
cela nous conduira \ à envisager l'image du face à face entre le roi et le dieu, l'image de l'action en faveur du
dieu, et l'image du conflit au nom du dieu.


Et tout cela fait donc à peu près quatre millénaires d'images, sur ces civilisations assez vaste; aussi il faudra se
limiter et ne prendre que quelques exemples fondamentaux et faire une lecture plus posée de certaines images absolument
fondamentales.

Des \ images du roi en deux ou trois dimensions, et la littérature qui elle aussi servira de support, 


Nous avons donc deux territoires vastes, qui se sont développés de manière parallèle, et qui pour le moment ne semblent
pas avoir été tellement en contact les uns avec les autres. De ces contacts, on ne sait trop comment cela se passait et
on a donc toujours considéré que le lien était le monde palestinien, mais ce n'est pas si évident car finalement on a
une connaissance assez floue de l'archéologie de la péninsule arabe et de la possibilité de la circulation par bateau
via le golfe persique

On a certes des idées et l'on sait qu'il y a eu des points de contact possible avec le monde phénicien, à partir du
moment où ces deux civilisations s'y sont tournés. Le monde mésopotamien vers l'est pour y trouver du bois et des
marchandises (il n'ont rien) et l'Egypte pour avoir certaines matières dont l'étain.

Seulement les recherches récentes dans le monde mésopotamien montrent que ces gens se tournaient également de l'autre
côté


On va donc partir du principe d'un développement par l'est

Nous sommes donc dans deux civilisations où l'on voit apparaître les deux plus anciens systèmes d'écriture de l'histoire
de l'humanité et on constate dans ces deux civilisations \ qu'il y a apparition de l'écriture et apparition de la
royauté


Effectivement \textbf{en Egypte,} nous aurons une unification partant de la Haute Egypte et se diffusant vers le Nord,
avec toute cette inconnue que représente l'annexion de la Basse Egypte, et ce au tournant du IV , III millénaire. Et on
arrive finalement à un royaume unifié, organisé autour du Nil, et dirigé par un seul roi (sauf durant les périodes
troublées que l'on appelle périodes intermédiaires) Nous n'étudierons pas ces périodes, on se limite aux périodes où
l'Egypte n'a qu'un seul roi.


En \textbf{Mésopotamie : }c'est un pays qui a deux fleuves (Tigre et Euphrate) et une géographie très différente. Elle
voit apparaître une royauté au IV millénaire, mais c'est un monde morcelé, et va fonctionner dans un premier temps sur
la base des cités états, ce que l'on pourrait appeler des principautés (une ville autour d'un micro royaume). On peut
constater qu'avec l'apparition du phénomène de guerre (qui n'est pas attesté avant - 2600), ces cités entrent en
conflit et forment des royaumes un peu plus importants en s'absorbant et de là on arrivera finalement au II et I
millénaire à ce que le monde mésopotamien fonctionne en deux royaumes, le centre et au Sud \ Babylone, et au nord le
royaume assyrien.

Cela veut dire que même quand nous aurons le grand royaume de Babylone , au Nord, \ il existe toujours le royaume
assyrien, il n'y a jamais eu dans le monde mésopotamien, une seule unité politique.


Mais ce qui est intéressant c'est que ces mésopotamiens fonctionnent de la même façon, certes il existe des
particularités entre Babylone et l'Assyrie, on ne peut pas dire que le principe de la souveraineté soit le même \ entre
Babylone et Assur, car les babyloniens sont supérieurement intellectuels par rapport aux assyriens, (les babyloniens
ont repris à leur compte la tradition culturelle sumérienne, et c'est une lacune que les assyriens n'arriveront jamais
à combler.


Dons déjà là nous avons quand même deux principes qui s'affirment très tôt, et on constate que le monde égyptien , comme
le monde mésopotamien, sont des sociétés basées sur des croyances polythéistes, des religions appelées préhistoriques
et que dans ces deux systèmes \ les rois dirigent au nom des dieux

Certes sur quatre millénaires nous aurons des variantes religieuses, (par exemple dans le monde mésopotamien en fonction
des périodes où des endroits où l'on se trouve on ne désigne pas les choses exactement de la même façon, c'est à dire
que les assyriens par exemple considèrent que les rois sont leur dieu ASUR , et que celui que nous appelons roi est son
représentant sur terre. \ Il y a en effet cette ambiguïté dans le monde de la Mésopotamie, car certains textes parlant
d'un dieu utilise le mot roi


\textbf{Chronologie entre ces deux mondes}


Il faut mettre en regard les grandes périodes (elles sont données un peu à la louche), et il faut se référer à Nagada
qui est contemporain d'Uruk , et la période Thinite qui correspond à la période d'Uruk final et le début des dynasties
archaïques, dynasties qui seront finalement contemporaines à l'Ancien Empire :

- \ l'empire d'Akad est finalement contemporain de la fin de la V et VI dynastie

- la période néo sumérienne est contemporaine de la période intermédiaire

- le Moyen Empire est contemporain à la première dynastie de Babylone \ et à cette époque les assyriens forment un
royaume paysan renfermé sur lui même

et il y a un moment où il n'y a rien dans le monde mésopotamien, alors que l'Egypte est à l'apogée de sa civilisation,
c'est à dire une période où le Pharaon est fort et période de grandes production d'images

Cette période en Mésopotamie correspond à ce que les historiens appelaient autrefois en référence au monde grec, \ les
siècles obscurs du monde mésopotamien, pour bien monter qu'à l'époque où la Mésopotamie est dominée par les Kassites,
paradoxalement c'est une époque importante au niveau culturel, copie de textes, archivage des textes, transcription de
documents, et c'est donc une époque fondamentale pour les linguistes, mais c'est une royauté fermée sur elle même, qui
ne produit aucune image, aucun objet


On peut utiliser le terme de bronze récent pour cette époque et cela correspond à l'époque de la richesse de toutes les
cités états de la \ Syrie Palestine


Et contrairement au moment où le monde mésopotamien va reprendre de l'ampleur avec les grands empires conquérants néo
sumériens et néo babyloniens, cela correspond à une période intermédiaire en Egypte de la basse époque, où il y a un
répertoire artistique intéressant mais qui évolue de façon particulière

Finalement, nous aurons vraiment un développement parallèle entre ces deux mondes jusqu'à la fin du bronze moyen au XVI
siècle avant JC

Bronze Ancien, Bronze moyen, nous avons les deux civilisations qui se développent un peu à la même vitesse, et ensuite
on a ce balancier.


\textbf{LE MONDE MESOPOTAMIEN}


C'est un monde qui d'une part correspond à une archéologie récente et très vaste. Mais attention ce monde ne correspond
pas uniquement à l'iraq et il faut regarder sur une carte qui présente la géographie actuelle et on voit que la partie
haute Euphrate qui est un territoire syrien actuellement, et se référer aux cours des deux fleuves, où les
mésopotamiens se sont installés \ et il y a des sites anciens dans l'actuel Turquie. Pour l'Iran, il n'y \ a pas grand
chose, car il y avait la limite naturelle formée par la montagne \ ZAGROS. (de toute façon l'Iran était une zone paria
dans le monde antique (à vérifier)


On peut dire aussi que les sites mésopotamiens sont vraiment liés aux zones des fleuves

Attention, dans l'antiquité la côte entrait davantage à l'intérieur des terres ,les deux fleuves se jetaient
indépendamment dans le golfe persique, \ et donc UR était un ensemble portuaire important et qu'au IV et début du III
millénaire le sud a été fondamental dans le développement de cette région.


C'est un monde sédentaire et un monde de nomades. Les historiens ont toujours voulu opposer deux aspects culturels et
retrouver d'ailleurs dans certains royaumes ou dans certaines caractéristiques des spécificités liées aux deux mondes.
On développe encore l'idée du sumérien plus intellectuel et le fait que si dans le monde sumérien des conflits se
développèrent vers - 2600, c'est au moment où beaucoup de populations nomades , sémites, venaient de s'installer, 


Dans le monde mésopotamien, que nous dit-on \ à propos de la royauté ? Les textes nous donnent des informations avec
d'une part un texte que l'on appelle : \textbf{la liste royale sumérienne, }dont on a différentes copies dans
différents musées du monde, (copie sur tablette, sur de l'argile, qui pouvait être aussi un objet de fondation)

La liste royale sumérienne est un texte qui démarre , mis en place dans le courant du III millénaire, et raconte
l'histoire de la partie sud du monde mésopotamien, depuis les origines de l'humanité, sachant que pour le monde
mésopotamien, dans le monde de la création, il y avait un premier temps, qui était le temps parfait, le temps des
dieux. A ce moment les dieux étaient divisés en deux groupes : les dieux suprêmes et les dieux de rang inférieur qui
serviront les grands dieux jusqu'au jour où ils se révoltent et refusent de continuer à servir les grands dieux. Là il
y a différentes versions selon les endroits et les époques et ces versions deviennent sanglantes dans les sources
babyloniennes du II millénaire, avec le meneur qui sera mis à mort.

Dans les sources du III millénaire , c'est un peu différent et c'est là que les grands dieux demandent à la déesse mère
de prendre de l'argile et de fabriquer sur son tour de potier une créature à leur image, mais une créature sur laquelle
on aura une prise, cette créature c'est l'homme dans le sens humanité (il n'y a pas de premier homme ou de première
femme ) c'est l'humain qui est ainsi créé et le récit dit que cet humain, on lui perce le bras et le sang coule,
contrairement à une divinité mais surtout les dieux garderont pour eux l'immortalité et laisseront la mort aux humains

Et c'est dans ce contexte de récit que l'on apprend qu'il y aura un évènement fondamental, dans cette histoire de
l'humanité c'est la descente de la royauté sur terre, que l'on attribue au dieu ENLIL, (seigneur du vent)

En en sumérien veut dire maitre seigneur, et IL veut dire vent, air, 

ENLIL est donc un dieu associé à l'air, l'espace, au vent, et dans le panthéon mésopotamien il est le numéro 2, après le
dieu du ciel

on attribue à ENLIL une descendance, mais surtout c'est lui qui gère les affaires de la terre


Donc ENLIL est le créateur de la royauté, car il constata que cette humanité était absolument incontrable et qu'il
fallait qu'elle soit dominée et il inventa la royauté qu'il lança du ciel sur la tête des hommes, afin d'organiser
cette création.

L'humanité a toujours fondamentalement agacée ENLIL, \ et il va même les punir


Dans le récit de ENLIL dans l'EKUR à Nippur qui est son grand temple :


\includegraphics{\DirImg FM1_01.jpg} 


C'est la résidence terrestre de ce dieu, où il avait un grand temple dans la ville de Nippur, et on a ce \textit{passage
qui nous dit :}


\textit{\og Sans ENLIL de la grande montagne, aucune ville n'aurait été construite, aucun habitat n'aurait
été érigé, aucun enclos habitable, n'aurait été construit, aucune bergerie n'aurait été établie, aucun roi n'aurait été
élevé, aucun seigneur ne serait \ né aucun grand prêtre, aucune grande prêtresse n'accomplirait le culte \fg }


On voit ainsi une référence rurale permanente, ils commencent à parler du bétail avant les hommes et cela traduit un
ordre de pensée. Il est évident que dans un texte égyptien, on ne classerait pas les choses dans cet ordre là.

Les mésopotamiens sont des gens très pragmatique et ils sont toujours très proches des contingences \ terre à terre,
liées à leur économie, qui est une économie agricole.


Donc sans ENLIL, aucun roi n'aurait été élevé

Cette humanité braillarde agace ENLIL, qui depuis que l'homme est sur terre l'empêche de dormir , il ne trouve plus le
sommeil à cause du bruit de l'humanité; alors il décide d'un geste d'agacement d'envoyer le déluge et nous aurons ainsi
le deuxième temps


Et dans la \textbf{liste royale sumérienne,} on nous parle des rois {\textquotedbl} ceux d'avant le
déluge{\textquotedbl} et il y ensuite ceux {\textquotedbl} d'après le déluge {\textquotedbl}

et cela veut dire qu'après le déluge c'est le moment d'entrer dans les temps historiques, et on nous dit qu'ENLIL ré
envoie la royauté sur terre, c'est à dire que nous avons une deuxième descente, nouvelle descente de la royauté sur
terre




Cette idée est assez intéressante et il faut garder à l'esprit cette division : rois des temps anciens, et rois de
l'époque historique

On peut même se demander si cela ne pourrait par être lié aux rois avant l'écriture et aux rois postérieurs à
l'écriture

dans l'esprit de la culture sumérienne ce qu'il présente comme étant mythologie, ce texte du déluge, ait un fondement
finalement par rapport aux rois des temps anciens, que l'on a oublié, dont les lignées se sont éteintes, par rapport à
de nouvelles royautés qui se sont mises en place dès le III millénaire


La royauté est un cadeau des dieux aux hommes, \ C'est donc à la fin du IV millénaire que l'on va voir apparaître les
premières représentations de roi \ et les rois seront présents dans l'art jusqu'à la fin de la civilisation
mésopotamienne

En général on considère la fin du monde mésopotamien en - 539, car il est complètement chamboulé par l'invasion perse,
et les perses ont une mentalité tout à fait différentes, un rapport au divin tout à fait différent et donc un rapport à
la souveraineté tout aussi différent

Donc on n'intègre pas les perses dans l'histoire de la Mésopotamie ancienne et on s'arrête donc à la prise de Babylone
en - 539..


Ce qui est notable, c'est la représentation par allusion, qui est extrêmement fréquente, comme nous pourrons le
constater : comment magnifier le souverain sans avoir à la représenter ? : il suffit de développer certains thèmes et
on sait très bien que tout cela \ a été initié par le roi. 

Après, il faut voir à quoi servent ces images


\textbf{Epoque d'URUK} : 


C'est la période de formation, et correspond grosso modo à la période de Nagada en Egypte

Les dynasties archaïques , ce sont ces fameux rois après le déluge, c'est à dire le moment où nous avons l'écriture,
donc le moment où nous avons des textes, et où dans les textes nous avons un mot qui désigne le roi et des images
associées à ce mot et cela signifie que pour la première fois un homme, la représentation d'un humain, qui aurait à
côté de lui le mot que l'on sait être le titre correspondant au mot roi et nous \ pourrons alors avec certitude qu'il
est souverain, car malheureusement la plupart des sources des listes royales que nous avons sont des copies tardives.
C'est le problème général avec les annales, plus c'est ancien, plus c'est oublié et c'est farfelu (on peut voir des
noms qui ne sont que des jeux de mots).


Les dynasties archaïques sont importantes car elles posent un principe de société, qui est le dieu , le roi son vicaire
humain et les hommes. Déjà c'est quelque chose d'essentiel, et cela signifie un fonctionnement par un intermédiaire, il
n'y a pas de lien direct entre les hommes et leurs dieux. \ Dans l'ensemble, il y a même plutôt un lien de crainte. A
partir du moment où \ l'on sait qu'ENLIL ne supporte pas les hommes, il a déjà détruit une fois l'humanité par le
déluge, il y a toujours la crainte d'une nouvelle punition.

Les dynasties archaïques posent finalement toutes les bases qui seront admirées et se perpéturont à travers les
millénaires

Et on voit bien qu'une première dynastie de Babylone, militaire, du début du deuxième millénaire s'appuie complètement
sur ce vieux fond culturel sumérien. C'est donc fondamental, de bien voir comment les choses se mettent en place.


\textbf{L'empire d'AKAD}

C'est \ la première fois que les hommes vont avoir la volonté de créer non pas un royaume, mais un empire, c'est à dire
qu'ils vont aller guerroyer au delà d'où on pouvait aller avant, et aller vers le Nord, le Sud, l'Est et dominer
complètement le \ monde des deux fleuves

Mais AKAD, c'est une utopie car ils étaient incapable de gouverner ces territoires auxquels ils aspiraient, car il
n'existait pas de structures d'état qui en étaient capable. Ils n'avaient pas l'organisation que les égyptiens avaient
à la même époque, car tout simplement c'étaient des gens qui fonctionnaient sur la base de micro royaume dans le sud.

Mais AKKAD est fondamental, même si c'est une période pour laquelle les sources sont maigres. Mais c'est la première
fois qu'il est attesté qu'un roi ait été divinisé de son vivant, et cela pour la royauté mésopotamienne, c'est quelque
chose d'absolument ponctuel


\textbf{Le roi est un homme}, il a une nature humaine et rien de plus. \ C'est un humain à la tête d'autres humains et
il est le vicaire des dieux sur terre. Il n'a aucun élément surnaturel en lui, le couronnement ne lui donne aucun
pouvoir surnaturel. C'est d'autant plus frustrant que le quatrième roi de la dynastie d'Akkad NARAM Sîn (petit fils de
Sargon) se fait diviniser car devant son nom on met le déterminatif de l'étoile, et cela implique une divinisation. Ces
gens régnaient à partir d'une ville qui est la seule grande capitale du monde mésopotamien, que l'on ne sait pas situer
et tant que l'on n'aura pas les archives royales de l'époque, on ne pourra pas réellement étudier cette période

Après AKKAD on notera également quelques divinisations de roi, de son vivant, notamment sous la 3ème dynastie d'UR, et
quelques unes dans la dynastie babylonienne et après plus jamais ;

On sait donc que cela a existé, mais on ne sait pas concrètement comment cela se faisait, existe t il un temple où l'on
rendait une culte du roi ?

Akkad après avoir duré \ 150 ans s'effondre, mais là encore on ne sait pas comment.


Il y aura un moment de renaissance sumérienne, mais dans une logique différente qui sera de renaître à partir d'un
royaume unifié démembré. C'est à dire que ce ne sont pas les minuscules cités états qui revoient le jour. On voit bien
que la troisième dynastie d'UR correspond en gros au Sud c'est \ à dire que cela englobe les régions de Sumer et de
Babylone


A partir du deuxième millénaire, les deux royaumes qui seront rivaux jusqu'à la fin \ : Babylone au centre et au sud et
Akkad au nord ; sauf que durant tout le deuxième millénaire on n'entend plus parler des assyriens, ils sont dominés par
d'autres, et il ne se réveillent que vers -1200 avant JC, dans le système boule de neige du passage des peuples de la
mer et durant le premier millénaire il y aura un conflit permanent entre le Nord et le Sud pour l'hégémonie.


En Mésopotamie, ce sont donc des royaumes, Zone de Sumer, zone de Babylone, Akkad les \ assyriens sont vraiment \ au
Nord et autour du Tigre. C'est ainsi que nous n'avons aucun texte assyrien avant le 9ème siècle qui nous parle de
l'Euphrate, ils ne partaient pas aussi loin, et ils allaient plutôt vers le Nord


Les premières sources dans les listes royales sumériennes qui parlent du roi, leur donnent le titre de EN, en langue
sumérienne cela signifie Maître, Seigneur, mais il semblerait que ce titre puisse correspondre à prêtre et c'est la
raison pour laquelle les historiens dans le temps parlaient du roi prêtre, car cela correspondait à une formule
française correspondant à la traduction d'un mon sumérien

Nous avons un deuxième titre \ LUGAL (?) qui en sumérien veut dire Homme, et gal en sumérien veut dire rateau et quand
on a la place de bien écrire, le rateau est dessiné au dessus de la tête de l'homme.


On a beaucoup discuté pour tenter de comprendre à quoi correspondaient tous ces titres et finalement nous sommes
incapables d'en faire quelque chose de cohérent, Quand on lit tout ce qui a été dit sur le sujet, on s'aperçoit que
pour le moment on ne comprend pas la logique. On voit que dans certains petits royaumes sumériens (ex à l'époque des
cités éclatées) du Sud on va appeler le roi En, et à trente km de là \ le roi s'appellera LUGAL. Mais on ne sait pas
pourquoi, car nous n'avons pas compris la logique de la chose.


Et il y a un troisième titre, qui est ENSI, qui lui aussi est ambigu, car plus tard il pourra avoir la notation de
gouverneur, donc le ENSI serait dépendant d'un Lugal ? \ On voit bien en tout cas que les premiers titres associés au
roi , sont assez ambigus, et montrent bien le lien entre le pouvoir et le dieu. Et d'ailleurs, il y a quelque chose qui
peut être rattaché à cela \ : c'est qu'au niveau de l'archéologie, on constate que dans les phases anciennes, le palais
comme élément architecturé n'existe pas. Les grands bâtiments des cités états sont des temples. Il semble bien donc que
celui que nous appelons roi, vive dans le temple et que les entrepôts du royaume soient également dans le temple.


C'est à cette époque, autour de 2600 AV JC, que l'on voit apparaître la filiation, la filiation humaine, c'est à dire
que le roi donne son nom en le faisant précéder du nom suivi de la mention {\textquotedbl} fils de .. {\textquotedbl}
et le nom qui suit est celui de son père

Et ceci par les sources écrites prouvent de façon certaine qu'au moins à partir de - 2600, la royauté est héréditaire

est elle héréditaire à partir de ce moment là ? (dynastique archaïque III) , on \ ne le sait pas. On pense qu'elle
devient progressivement héréditaire à partir du III millénaire ; avant on devait avoir le système du choix parmi les
anciens


Nous avons donc établi un premier point : la fonction royale est descendue sur terre, cadeau de ENLIL et la première
caractéristique \ : l'hérédité : je deviens roi car mon père l'était. L'hérédité, la filiation est quelque chose qui se
met en \ place et qui va durer pendant 3,000 ans (en effet la plupart des rois sont fils de ... )


Mais il y a une deuxième raison pour devenir roi, ce sont les grandes qualités : c'est à dire que les textes développent
l'idée que certains humains étaient destinés à devenir roi à partir de leur haute qualité

Le premier à s'en référer est Sargon (premier fondateur de l'empire d'Akkad) et n'a cessé de vanter les mérites , Il a
renversé le souverain, en réalité c'est un usurpateur, et d'ailleurs on ne connait pas son nom de naissance, mais il se
fait couronner sous le nom de SARGON SHARRUKIN et Sharrum veut dire roi dans la langue d'Akkad et Kin est un adjectif
qui veut dire vrai, légitime, \ C'est donc un usurpateur qui se fait couronner sous le nom de roi légitime.


Tête de Ninive, qui pourrait être celle de Sargon


\includegraphics{\DirImg FM1_03.jpg} 


Après sa mort on a fait de Sargon un personnage merveilleux, mais avec deux siècles de retard, car tous les textes
relatifs à Sargon datent de la première dynastie de Babylone car cette première dynastie de Babylone , sont des gens
qui n'avaient pas de lien royal avant et se sont appuyés sur cette fausse filiation . Alors on montre SARGON comme un
personnage merveilleux et on se place comme étant ses descendants , dotés des mêmes qualités, (et qu'ils sont ainsi
devenus rois du fait de ces mêmes qualités)

\textbf{LE ROI EN MESOPOTAMIE EST :}


\textbf{- Vicaire du dieu sur terre,}

\textbf{- gestionnaire et défenseur des biens du dieu }

\textbf{- c'est un homme, un humain tout simplement}


C'est donc un mortel qui au delà de la mort n'aura rien de plus que les autres, et cet élément est très important pour
notre sujet


\textbf{le Roi, vicaire du dieu sur terre} : 


Pendant 3,000 ans, les images royales le représenteront debout, assis, les mains jointes dans le signe de la prière, car
dans l'état actuel des choses toutes les statues que nous connaissons proviennent des temples et sont donc
systématiquement associées à des divinités

Nous n'avons en effet à ce jour aucun exemple connu de statue identifiée comme ayant pu être mise à l'entrée d'un
bâtiment officiel, sur une place visible des gens, ou associée à la sépulture d'un roi

(mais cela devait arranger les sculpteurs qui ne disposaient pas d'un matériel extraordinaire et qu'il était très
difficile ainsi de travailler dans la dorrite)


Statue ESHNUNNA (1600)


\includegraphics{\DirImg FM1_04.jpg} 


On voit bien les mains jointes et dans une attitude de prière.


Le roi étant le vicaire du dieu, il a \ être représenté dans un face à face avec des dieux. Ce personnage, nous ne le
connaissons jamais par son nom de naissance, \ En effet quand on regarde ce que veut dire le nom de ces rois, ce n'est
pas un prénom, le nom du roi est en effet lié au contexte d'une époque , à sa demande de protection à l'une des
divinités, ce sont donc des noms de couronnement.

Et si on reprenait tous les noms des rois depuis 3000 ans, ce que l'on ne pourrait faire, on verrait peut être quelques
prénoms, mais ce sont surtout des noms de couronnement.


On peut également s'interroger pour savoir s'il existe un couronnement, un rituel de sacre, et dans ce cas à quand il
remonte, quel est le plus ancien exemple. \ En réalité, les sources ne sont pas très anciennes, la plus ancienne
attestation connue est la stèle de UR NAMMU, (Philadelphie)


\includegraphics{\DirImg FM1_05.jpg} 
\includegraphics{\DirImg FM1_06.jpg} 


stèle UR NAMMU et détail


UR NAMMU est le fondateur de la troisième dynastie d'Ur au 22 siècle avant JC, et c'est à cette époque que les textes
parlent d'un cérémonie de sacre qui se passait dans le sanctuaire d'ENLIL, (ce qui est normal puisque ENLIL donne la
royauté aux hommes) dans son grand sanctuaire de Nippur qui s'appelle, nous l'avons vu , EKUR

On ne sait pas très bien comment cela se passait, mais on sait que le roi recevait des choses, même si nous n'avons
aucun récit exhaustif)


\textit{Stèle de Shamash} : Bristish Muséum


\includegraphics{\DirImg FM1_07.jpg} 


\textit{Le Roi Nabû Appla Iddina est introduit par deux divinités protectrices auprès du dieu Shamash, assis sur son
trône}


cette stèle date du 9ème siècle avant JC, période où les rois de Babylone ont beaucoup de mal à se débarrasser des
assyriens qui les dominent, les désignent dans la plupart des cas, et où en réalité il s'agit de pseudo royautés
autonomes, sous le contrôle des assyriens


Cette stèle est extrêmement importante au niveau iconographique car elle nous montre à droite sous un dais, et sur un
trône qui peut être identifié car il y a en gros l'idéogramme de son nom, Shamas, dieu soleil, et dans les textes de la
troisième dynastie d'Ur, nous savons que la cérémonie du sacre se passait dans le temple d'ENLIL . C'est donc le dieu
soleil SHAMASH qui remet au roi deux objets : une corde enroulée sur elle même et un bâton à mesurer, qui sont le
symbole de sa fonction, être garant des normes.

Avant on pensait qu'il s'agissait d'une corde et d'un piquet de fondation et on y voyait l'image du roi arpenteur
prenant les mesures du futur temple; On sait maintenant qu'il s'agit d'un bâton à mesurer et en aucun cas \ d'un
spectre.

En réalité il s'agit d'objet très concret : la règle sert à mesure

On voit le roi arriver près du dieu , tel un dieu secondaire dans une tenue dont nous reparlerons \ : en effet son
costume est différent de celui des deux autres, qui ont des tuniques, alors que le roi \ a toujours l'épaule droite
dénudée. On sait qu'en Mésopotamie, la bonne main d'usage est la main droite, la main gauche est celle avec laquelle on
fait les choses sales (on écrit , mon mange, on salue, on prie de la \ main droite). Le salut suprême se fait main
droite levée devant la bouche, c'est un signe de respect. Et d'ailleurs ce sont des traditions qui vont se perpétuer et
on en trouve même référence dans la Bible

aussi, cette stèle peut très bien représenter une scène de couronnement


Les textes nous parlent également d'Isthar \ du jour (attention il existe Isthar du jour et Isthar de la Nuit , la
différence est qu'Isthar de la nuit est représentée avec des ailes et des chouettes)


\includegraphics{\DirImg FM1_08.jpg} 


Isthar de la nuit (vers -1800)


Isthar est importante à la fin du III et début du II millénaire, tellement importante d'ailleurs que son nom devient un
nom commun pour signifier déesse

Isthar donne au roi son trône, sa thiarre et son spectre


Et nous avons ainsi la liste de cinq objets que le roi recevait lors de son couronnement et cela se \ perpétuera sans
doute après des mises en scènes différentes en fonction des époques et des royaumes jusqu'à la chute de Babylone


Ceci va conduire l'iconographie mésopotamienne à montrer le lien entre le roi et son dieu par une proximité, mais on
s'arrête à cette proximité, il n'existe aucune familiarité entre le roi et son dieu, car le roi mésopotamien est un
homme.

Le roi affirme une action , et la légitime par la volonté du dieu et c'est ce qui se passera durant la période
assyrienne, où on utilise un motif , et là il faut repartir sur la notion du disque solaire ailé égyptien d'Edfou


On sait que ce motif, qui est au départ le disque solaire égyptien, a voyagé par le monde palestinien, le monde hittite,
le monde mésopotamien et même chez les perses et au premier millénaire, les assyriens (on ne sait pas comment ils en
ont eu l'idée, représentent leur dieu national Assur comme un buste sortant de l'astre solaire. Les babyloniens
représenteront leur dieu Marduk de la même façon et les perses ensuite quand ils auront vaincu la Mésopotamie se
mettront également à représenter leur dieu suprême de la même façon

Cela permet pour les assyriens une composition très pratique en image pour montrer le niveau céleste de la décision et
son application \ terrestre par le roi et cette image a du succès


Nous constaterons que lié à la personne royale, les femmes de sa famille (mère, épouse, fille) sont absentes de
l'iconographie. En Mésopotamie, le roi est toujours seul et c'est de façon tout à fait exceptionnelle que l'on verra la
\ présence de ces femmes.


Et c'est intéressant car si on se réfère aux sources littéraires, on sait que les femmes en Mésopotamie avait un pouvoir
important. Au moins à partir du II millénaire, le palais est entre leurs mains, elles ont la clés de l'entrepôt et même
du palais (à certaines époques , à certains endroits, le roi n'a pas les clés de son palais et doit les demander aux
femmes


Donc leur absence dans l'iconographie ne signifie pas qu'elles ont aucun rôle, \ Mais on voit que par la définition même
de la royauté en Mésopotamie, elles n'ont aucun rôle à jouer


Y a -t-il eu des femmes rois ? \ à part la fameuse Kubada (?), \ et ce à la période dynastique à Ur au moment où la
succession par hérédité se met en place, elle a été roi (on le sait par un texte au British Muséum), et mais il n'y a
aucune autre femme roi attestée


C'est pourquoi \textit{la stèle d'ASSARHADDON , \ avec sa maman} (Naqi'a) du musée du louvre


\includegraphics{\DirImg FM1_09.jpg} 


est particulièrement intéressante. Sa maman était une déportée babylonienne, et elle va convaincre son fils Assarhddon
(qui est un roi assyrien qui va conquérir l'Egypte) d'avoir une politique favorable aux ennemis babyloniens. On peut
voir qu'elle est représentée comme un ennuque, \ Ce petit morceau a été acheté par le Louvre dans les années 60, car
particulièrement intéressant, et très rare, mais peu intéressant d'un point de vue de l'art


\textbf{Le Roi mésopotamien est le gestionnaire et le défenseur des biens de son dieu}


on voit bien que cela marche dans les deux sens

En effet, et nous le verrons toutes les \ premières \ images l'associe à ces deux mondes le temple et son troupeau et
son étable, \ et on voit bien que étable, temple, troupeau, sac de grains, tout cela va ensemble et c'est dans le même
bâtiment \ et que toute l'économie passe dans ce fameux lieu, qui nous appelons temple.


Ce sont des territoires agricoles et ils ont deux obsessions \ : que les champs soient fertiles et que les récoltes
soient bonnes, et le troupeau fécond, ce qui permet de nourrir tout le monde


Donc , on part du principe que c'est maintenir la création des dieux, telle qu'elle a été donnée aux hommes C'est à dire
que c'est partir du principe que le monde terreste a été fait par les dieux pour que l'humanité y vive et que les
hommes doivent la respecter, et la mettre en valeur, ce qui est une façon de rendre hommage aux deux

Mais le territoire agricole appartient aux divinités


Prenons un exemple ancien, du IV millénaire, le sceau de Berlin, (que nous verrons plus tard) , et au XII siècle avant
JC :


\textit{Kuduri Kassite du Roi Méli Shipak II \ Louvre}


\includegraphics{\DirImg FM1_10.jpg} 

On voit donc le roi devant une divinité qui est peut être shamash et il salue le dieu. Le texte devant est effacé , mais
derrière il existe encore et il ne concerne que l'arpentage, mesure de terrains, (tel champs qui appartient au domaine
agricole du dieu, et qui mesure ... , a produit ceci ...; et est géré comme cela ...)


Et à partir de là, la destination de la guerre en découle, \ en effet à partir du moment où le roi s'affirme comme étant
le vicaire de son dieu, il défend le territoire de son dieu. Et si on prend la plus ancienne image sculptée de la
guerre, stèle des vautours, Louvre, date vers 2500 avant JC et dynastie archaïque III


\textbf{\textit{stèle des vautours}}\textbf{ : }


 \includegraphics{\DirImg FM1_11.jpg} 


\textit{Commentaire \ (hors conférence, site du Louvre)}


\textbf{\textit{\textcolor[rgb]{0.101960786,0.101960786,0.101960786}{Partiellement reconstituée à partir de plusieurs
fragments trouvés dans les vestiges de la cité sumérienne de Girsu, cette stèle de victoire constitue le plus ancien
document historiographique connu. Une longue inscription en langue sumérienne fait le récit du conflit récurrent qui
opposait les cités-États voisines de Lagash et Umma, puis de la victoire d'Eannatum, roi de Lagash. Son triomphe est
illustré avec un luxe de détails par le remarquable décor en bas-relief qui couvre les deux faces.}}}

\textbf{\textit{Un document historique exceptionnel}}

\textit{Malgré sa conservation lacunaire, cette stèle de grande taille, sculptée et inscrite sur ses deux faces, est un
monument d'une valeur incomparable puisqu'il s'agit du plus ancien document historiographique connu. Les fouilles du
site de Tello permirent d'en retrouver plusieurs fragments disséminés parmi les vestiges de l'ancienne cité sumérienne
de Girsu. Cette stèle commémore, par le texte et l'image, une importante victoire remportée par le roi de Lagash,
Eannatum, sur la cité voisine d'Umma. Les deux villes entretenaient en effet un état de guerre récurrent à propos de la
délimitation de leur frontière commune, à l'image de ce que pouvaient être les relations entre cités-États à l'époque
des dynasties archaïques.[2028?]Petit-fils d'Ur-Nanshe et fondateur de la Ière dynastie de Lagash, Eannatum régna vers
2450 av. J.-C. et conduisit sa cité-État à l'apogée de sa puissance. L'inscription gravée sur }\textit{La Stèle des
vautours}\textit{, d'une ampleur remarquable bien qu'il n'en subsiste qu'une petite moitié, exalte les triomphes d'un
souverain placé dès sa naissance sous la protection divine. Nourri au lait de la déesse Ninhursag et tenant son nom de
la déesse Inanna, c'est du dieu Ningirsu lui-même qu'il reçut la royauté de Lagash. Assuré du soutien des divinités par
un songe prophétique, Eannatum va s'engager avec fermeté dans la lutte contre Umma afin d'imposer son contrôle sur le
Gu-edina, territoire frontalier enjeu de la rivalité entre les deux cités.[2028?]{\textquotedbl}}\textit{Moi Eannatum,
le puissant, l'appelé de Ningirsu, au pays [ennemi], avec colère, ce [qui fut] de tout temps, je le proclame ! Le
prince d'Umma, chaque fois qu'avec ses troupes il aura mangé le Gu-edina, le domaine bien-aimé de Ningirsu, que
[celui-ci] l'abatte }\textit{!{\textquotedbl}.}

\textbf{\textit{La face {\textquotedbl}historique{\textquotedbl}}}

\textit{La narration de la campagne militaire contre Umma est illustrée de manière spectaculaire par des représentations
figurées, sculptées dans le champ de la stèle selon une disposition traditionnelle en registres. Elles offrent ici la
particularité d'être réparties sur chacune des deux faces en fonction de leur perspective symbolique. L'une des faces
est ainsi consacrée à la dimension {\textquotedbl}historique{\textquotedbl} et l'autre à la} \textit{dimension
{\textquotedbl}mythologique{\textquotedbl}, la première rendant compte de l'action des hommes et la seconde de
l'intervention des dieux. Détermination humaine et protection divine se conjuguent ainsi pour conduire à la
victoire.[2028?]La face dite {\textquotedbl}historique{\textquotedbl} montre, au registre supérieur, le souverain de
Lagash marchant à la tête de son armée. Eannatum est vêtu de la jupe à mèches laineuses appelée kaunakès, recouverte
partiellement par une tunique en laine passant sur l'épaule gauche. Il porte le casque à chignon, apanage des hauts
personnages. Les soldats, casqués eux aussi et armés de longues piques, s'avancent en formation serrée, se protégeant
mutuellement derrière de hauts boucliers rectangulaires. L'armée de Lagash triomphante piétine les cadavres des ennemis
qu'une nuée de vautours a commencé à déchiqueter, scène dont la stèle tire son nom. L'inscription proclame
:[2028?]{\textquotedbl}}\textit{Eannatum }\textit{frappa Umma. Il eut vite dénombré 3 600 cadavres [...]. Moi Eannatum,
comme un mauvais vent d'orage, je déchaînai la tempête !}\textit{{\textquotedbl}.[2028?]Au deuxième registre est
représenté ce qui semble constituer le défilé de la victoire. Les soldats marchent alignés sur deux colonnes derrière
leur souverain monté sur un char. Ils tiennent leur pique relevée et la hache de guerre à l'épaule. Eannatum brandit
lui aussi une longue pique ainsi qu'une harpé à lame courbe, une arme d'apparat. Il se tient debout sur un char à
quatre roues pourvu d'un haut tablier frontal duquel émergent des javelots rangés dans un carquois.}

\textit{Le troisième registre, très fragmentaire, illustre les cérémonies funéraires qui viennent clôturer l'engagement
militaire. Pour ensevelir les cadavres amoncelés de leurs camarades, les soldats de Lagash gravissent une échelle en
portant sur la tête un panier rempli de terre. Des animaux, dont un taureau couché sur le dos et ligoté, sont prêts à
être immolés tandis que l'on accomplit une libation au-dessus de grands vases porteurs de rameaux végétaux.}


\textbf{\textit{La face {\textquotedbl}mythologique{\textquotedbl}}}


\textit{La face dite {\textquotedbl}mythologique{\textquotedbl} illustre l'intervention divine qui offre la victoire à
Eannatum. Elle est dominée par la figure imposante du dieu Ningirsu, protecteur de la cité-État de Lagash. Celui-ci
tient les troupes ennemies emprisonnées pêle-mêle dans un gigantesque filet et les frappe de sa masse d'armes.
Instrument de combat par excellence du dieu, le filet est tenu fermé par l'emblème d'Imdugud, l'aigle à tête de lion,
attribut de Ningirsu, qui est représenté les ailes déployées et agrippant deux lions dans ses serres.}

\textit{Le reste de la face {\textquotedbl}mythologique{\textquotedbl}, très lacunaire, semble évoquer la présence aux
côtés du dieu triomphant d'une déesse, sans doute Nanshe, l'épouse de Ningirsu, également associée à l'aigle
léontocéphale. Le registre inférieur laisse entrevoir le dieu sur un char, en compagnie de la même
déesse.[2028?]L'inscription, après avoir glorifié l'action victorieuse d'Eannatum, fait une large place aux serments
prêtés par les deux souverains devant les grandes divinités du panthéon. Ayant réintégré le Gu-edina au sein du
territoire de Lagash, Eannatum délimite avec Umma la frontière, sur laquelle est érigée une stèle. Mais la réussite du
projet humain ne peut s'accomplir que par faveur divine ; c'est donc elle qui est invoquée afin de garantir la
pérennité du nouvel ordre des choses : {\textquotedbl}}\textit{Que jamais l'homme d'Umma ne franchisse la frontière de
Ningirsu ! Qu'il n'en altère pas le talus et le fossé ! Qu'il n'en déplace pas la stèle ! S'il franchissait la
frontière, que le grand filet d'Enlil, le roi du ciel et de la terre, par lequel il a prêté serment, s'abatte sur Umma
}\textit{!{\textquotedbl}.}


On voit donc que c'est très intéressant , d'une part cette stèle comporte le plus long texte en langue sumérien
archaïque, et dit, que si le roi représenté ici est parti en guerre sur son char, contre le petit royaume voisin,
\ c'set car Umma n'a pas respecté la frontière, c'est à dire un canal d'irrigation Umma et sa troupe ont franchi ce
canal, et sont donc venus sur un territoire qui n'était pas le leur, territoire qui appartenait au dieu Ennatum


L'idée est donc que l'on ne peut tolérer l'idée que l'homme d'Umma (façon d'appeler le roi ennemi) vienne revendiquer un
terrain 

Au départ c'est une guerre céleste, c'est le dieu SHAMAH dans le ciel qui revendique un terrain qui appartient à un
autre dieu, mais cette guerre sera vécue à un niveau humain. et ceci permet de légitimer tout ce que l'on veut, car on
considère que la guerre est une volonté de la divinité (on le reverra) et cela permettra de mettre en place tout un
discours qui révèle que si l'ennemi a perdu, et cela va même au delà car le texte dit que celui qui gagne est béni du
dieu, et donc le béni du dieu est l'homme pur victorieux et l'homme qui a perdu la guerre est un impie, et donc on le
met tout nu on lui saute dessus, on lui fait en réalité ce que l'on veut. 

Et ce même langage appartient dans les anales assyriennes. En effet quelque soit le conflit, les assyriens entre le IX
et le VII siècle parlent des autres comme cela. Un roi assyrien fait la guerre au nom de son dieu, donc d'Assur, et ils
ont toujours ce côté dépréciatif les uns par rapport aux autres, (ce lâche qui fuit devant nous) et ils ont la
certitude d'être redoutables, car ils font la guerre au nom d'Assur et donc ils ne peuvent qu'être victorieux , et ils
peuvent massacrer leurs ennemis


\textbf{Le roi est un mortel sans destinée particulière} : 


Cette conception est embêtante pour nous, car cela veut dire qu'il n'y a pas d'art funéraire.

C'est un mortel, donc on a un homme à la tête des autres hommes, même si après et parfois il y aura des mises en scène
pour le magnifier, 

Mais les dieux ne le récompense pas pour ses hauts faits en lui donnant quelque chose de mieux que les autres; il va
comme n'importe quel autre humain après sa mort dans le monde des enfers, et en Mésopotamie la vision de l'au delà
n'est pas très drôle


Aussi dans cette logique, il est normal de ne pas rechercher de magnifiques tombes royales mésopotamiennes

Il est évident qu'ils ont dû emmener des choses avec eux, \ des objets précieux, qu'ils avaient un belle tombe, mais
qu'il n'existait pas un art architecturé de la tombe, il n'y a aucune construction élaborée

En 3000 ans, \ il n'y a \ à ce jour qu'un seul exemple , ce sont les tombes royales d'Ur. C'est le seul moment et
seulement huit tombes identifiées comme étant des tombes royales, où l'on voit qu'il y avait deux grandes structures à
plusieurs pièces, avec des sacrifices humains et une mise en scène d'un aménagement funéraire



\textbf{L'EGYPTE}


Nous avons là une unité géographique et une unification qui s'est faite entre le IV et le III millénaire, avec des
récits de la création de la royauté qui elle aussi s'appuie sur un monde divin

Et même s'il est évident que des traditions plus anciennes ont existé, la plus ancienne source résulte des Textes des
Pyramides, car dans ces textes, le roi est décrit comme arrivant du ciel, (intéressant, car en Mésopotamie, la royauté
est tombé du ciel). Mais le roi égyptien arrive sur terre , après avoir mis MAAT à la place de l'Isefet, dans l'ile des
flammes (et dans les textes des pyramides, lorsque l'on parle de l'ile des flammes, c'est la terre dans son état
rudimentaire, sauvage, rustique

Et à partir du moment où le roi a mis MAAT à la place de l'Isefet, ce geste fait que la terre devient habitable


la principal source qui rattache le roi d'Egypte à son créateur, le soleil, c'est l'hymne qui évoque l'adoration
matinale du soleil, cet hymne constitué de 44 vers, et qui découle de cette pensée de l'ancien empire , exprimée dans
les textes des pyramides, qu'il doit y avoir une origine rédactionnelle au Moyen Empire , mais que l'on connaît par des
copies, notamment celle de deir el bahari


Rê a installé le roi sur la terre des vivants, \ jamais et à toute éternité de sorte qu'il juge les hommes, et
satisfasse les dieux, qu'il réalise MAAT et anéantisse l'Isefet, donne des sacrifices aux dieux et des offrandes
funéraires aux morts immortalisés


Donc là le point commun , c'est que la royauté est un cadeau aux hommes : le créateur donne aux hommes le roi pour la
terre des vivants, pour l'éternité

Le roi d'Egypte doit donc juger les hommes et satisfaire les dieux; On voit donc dans le fait que le roi juge les
hommes, il y a cette notion d'ordre qui existe également en Mésopotamie, l'obligation de satisfaire les dieux, \ il a
par contre un lien privilégié avec les dieux et une obligation qui est une spécificité égyptienne : l'opposition entre
Maât et l'Isefet


Effectivement, en égypte, on sait que l'on part là encore sur le principe d'une royauté héréditaire (par les male de la
famille), et il y a la possibilité pour une personne qui n'est pas de la lignée royale de prendre le pouvoir, \ en
effet un homme de mérite peut devenir roi, cela est attesté avec bon nombres de rois, qui ont été appelés usurpateurs,
et finalement leur légitimité est accordée par l'acte de couronnement


Mais, la différence fondamentale avec le roi mésopotamien, c'est que l'homme qui subit le rituel du couronnement en
Egypte, devient un netchernéfer, c'est à dire un \ être qui contient en lui une parcelle divine et c'est le grand
bénéficie du couronnement.

Il va recevoir aussi des insignes régaliens, mais à partir de ce moment là, il peut toucher le monde des dieux, et avoir
en lui une partie d'essence surnaturelle.


Exemple THOT qui inscrit le nom du roi KALABASHA


 \includegraphics{\DirImg FM1_12.jpg} 


(rite de purification pendant le couronnement et Thot écrit non pas le nom du pharaon mais le nom per aa


A partir du couronnement, peut s'exprimer cet aspect surnaturel de la personne royale, qui s'exprimera de façon
différente en fonction des époques. 


On sent très bien la magie des insignes régaliens donnés également au roi lors de son couronnement


A l'ancien empire, nous verrons des représentations où le roi est en contact direct avec les dieux, 

mais contrairement au monde mésopotamien, où l'on voit le roi humain agir pour la divinité, et éventuellement que la
divinité lui donne quelque chose, il y a beaucoup plus d'aller et retour dans le monde égyptien.

On a en effet des scènes de deux divinités qui entourent le roi (et on peut imaginer deux autres divinités, de l'autre
côte et on fait la boucle avec les quatre points cardinaux) et on est dans la logique égyptienne, des rituels de
protection, pour purifier le roi. Cela est inenvisageable en Mésopotamie, par le fait que le roi n'a rien de divin


La notion de royauté liée \ l'hérédité, c'est par les sources écrites que nous la connaissons, (Pierre de Palerme) et
c'est la légitimation par le simple fait que l'on est fils de


Le roi en Egypte est mis en image dès le IV millénaire, et il y a la mise en place d'une iconographie royale, donc avec
ceux d'avant l'écriture, ceux sans nom


En egypte \ il n'y a qu'un seul roi, et un seul royaume, (sauf durant les périodes intermédiaires), il y a donc eu
l'unification politique et il en découle un seul roi

l'Egypte aussi est un pays protégé naturellement par le désert, alors que la Mésopotamie est une plaine qui de tout
temps a été traversée par des populations différentes


\textbf{EN EGYPTE : LE ROI EST :}


\textbf{- fils de dieu sur terre,}

\textbf{-garant de l'ordre de MAAT}

\textbf{- un dieu parfait de son vivant, et un dieu après sa mort}


\textbf{fils de dieu sur terre :}


Il est intéressant de voir que très tôt dans le temps, nous allons avoir des faces à faces entre Pharaon et les dieux,
il y a vraiment un lien charnel, physique qui va être affirmé.


\includegraphics{\DirImg FM1_13.jpg} 


Niouserrê recevant la vie du dieu
\href{http://fr.wikipedia.org/wiki/Anubis}{\textcolor[rgb]{0.0,0.21960784,0.62352943}{Anubis}} - Relevé du temple
funéraire du roi à \href{http://fr.wikipedia.org/wiki/Abousir}{\textcolor[rgb]{0.0,0.21960784,0.62352943}{Abousir}}


\includegraphics{\DirImg FM1_15.jpg} 


Et on voit \ sur le relief du roi Niouserré à Berlin, une déesse lionne (on discute encore pour connaître son nom) et
elle tient son sein pour nourrir le roi, comme si elle était sa mère terrestre. Et \ à ce jour c'est la plus ancienne
scène d'allaitement d'un roi humain par une divinité que l'on connaisse. Niouserrê étant un roi de la V dynastie


Cette image est toujours présente dans la XVIII dynastie, où la littérature égyptienne développe le fait qu'à partir de
la 18ème dynastie l'essence divine s'affirme , en disant {\textquotedbl} il est le fils du dieu sur terre, il est
l'enfant terrestre des dieux et \ à partir de ce moment là on a la triade divine céleste, (père mère enfant), \ et le
dieu et la déesse ont donc un enfant sur terre : le pharaon régnant


Et sur la rive Ouest de Thèbes consacrée à Hathor, on le met en image, roi debout devant Hathor :
\includegraphics{\DirImg FM1_16.jpg}


et cette image sera reprise


On peut même voir le souverain téter le pie de la vache


et cela va s'exprimer d'une autre façon car nous aurons la représentation qui existe dès la 18ème dynastie du souverain
encadré par son père et sa mère céleste, c'est à dire que là le prend véritablement le rôle de l'enfant divin et au
lieu d'avoir le dieu fils de Ptah et Sekmet on a carrément Ramsès II entre Ptah et Sekhmet


\includegraphics{\DirImg FM1_17.jpg} 


Et ce lien conduit dans l'art dès l'ancien empire, à des représentations où le dieu touche le roi et là effectivement
ceci est inenvisageable dans le monde mésopotamien, (on peut voir parfois le roi être touché par un dieu, mais il
s'agit d'un dieu inférieur), on le sait car les sumériens ont établi des listes de dieux , et ce par ordre d'importance
et il existe des dieux de rang inférieur, divinité d'ordre secondaire (un peu comme le dieu personnel du roi qui est
une sorte d'ange gardien, qui peut lui servir d'intercesseur auprès des divinités suprêmes et peut alors le toucher ,
le conduire par la main auprès du dieu supérieur

Mais dans le monde mésopotamien, nous ne verrons jamais un dieu supérieur toucher le roi comme s'il était son égal


Et d'ailleurs dans la proportion des tailles, la divinité est toujours plus grande en Mésopotamie, et l'humain plus
petit \ (cf photo vu en début du cours avec le Roi Shamash énorme assis sous son dais, et le roi en tout petit, \ alors
que cela aurait dû être l'inverse puisque le roi est debout et donc en toute logique aurait dû être plus grand)


En Egypte au contraire, on observe des faces à faces entre le dieu et le pharaon et ils ont la même taille et cela est
fondamental

Le roi et la divinité ont la même taille, la même couleur , donc la même essence, ils se touchent et c'est intéressant
au niveau de la gestuel, 

On utilise donc l'image du roi régnant pour montrer l'image d'un dieu antropomorphe contemporain

et à la même époque de Thoutmosis III , on va jouer sur des similitudes au niveau des des coiffures, car le toi aura à
partir de Thoutmosis III des coiffures de plus en plus exubérantes, avec de plus en plus de plumes, qui ressemblent
beaucoup à la coiffure divine


Le lien \ profond qui existe entre le roi et les dieux vient du fait que les dieux sont là en permanence pour lui
insuffler la vie, et ce souffle de vie, il ne le reçoit pas uniquement pour lui, mais pour toute l'humanité qu'il
représente, et nous verrons bien que c'est tout à fait caractéristique de la mise en image égyptienne, de
l'iconographie égyptienne, on ne représente pas le dieu en train de faire bénéficier ce souffle de vie à \ la
population d'Egypte. L'humain de référence , l'humain de contact est pharaon, 

Il y a donc une hiérarchie et le roi recevant ce souffle sera garant de la MAAT


L'ordre doit régner sur terre pour que le monde divin soit stable et si l'ordre ne règne pas sur terre cela met en péril
le monde divin


\textit{exemple Hathor accueille Séthi 1er (Louvre)}


\includegraphics{\DirImg FM1_18.jpg} 


Et on voit que leurs mains se tiennent et que quelque chose est donné

et encore une fois, cette thématique ne peut être envisagée dans le monde mésopotamien ou si le roi lui donne quelque
chose , la divinité pourra lui donner également quelque chose, \ mais jamais la divinité ne lui donnerait quelque chose
sans avoir reçu quelque chose avant, sauf le jour du couronnement où le roi reçoit ses insigne


IL faut aussi évoquer le fait que pharaon n'est pas uniquement le fils du dieu sur terre, mais un dieu sur terre \ et
cela s'exprimera intensément durant la période aramarnienne

exemple tombe de Meryre

Cette période est d'ailleurs assez intéressante, elle ne dure que 14 ans sur les 3000 ans de l'histoire égyptienne, mais
elle va changer et chambouler beaucoup de chose


pendentif de Shed Louvre \ C'est à cette époque qu'apparait l'iconographie du dieu SHED, tueur de bête sauvage, et c'est
un enfant royal, il est représenté avec la tresse de l'enfant, 

c'est à la fois un enfant humain et un enfant divin, on voit donc que les choses sont floues, et ils tue les animaux du
désert en tuant les crocodiles

on voit donc que la frontière entre le monde humain royal et le monde divin est ténu


\textit{stèle de l'enfant Horus et les Crocodiles}


\includegraphics{\DirImg FM1_19.jpg} 


un roi en enfant solaire, alors qu'il est adulte

On est là à une relecture durant la période de Ramssès de Shed qui est une divinité tout à fait solaire, et le passage à
partir de la 3ème période intermédiaire , l'image de SHED va disparaître, car elle est liée à un principe de
souveraineté, \ et il y a le développement des stèles d'Horus sur les crocodiles et Horus est représenté en enfant

Et c'est intéressant de voir comment une image royale, mise en place à un certain moment pour dire quelque chose va
évoluer et ne sera conservée pour être acceptable que sous la forme d'une image divine ; car on est revenu à une
orthodoxie religieuse un peu calmée par rapport à la période amarnienne.


\textbf{MAAT et le principe de souveraineté} :


Cela signifie que le roi d'Egypte est garant de MAAT et c'est fondamental, car MAAT c'est la volonté des dieux, le roi
est leur intermédiaire et il se charge de la faire respecter, les hommes doivent lui obéir

Et les textes disent que


RE se nourrit chaque jour de la MAAT, cela veut dire que si on ne la respecte pas , c'est le créateur lui même qui est
en danger


Le roi garant de MAAT, il doit donc lui faire des offrandes chaque jour dans le temple, mais cela va plus loin , car
être garant de MAAT c'est être garant de tout et ce tout va de la construction du temple ,à son fonctionnement, son
organisation intérieure, veiller au bon fonctionnement du culte, et ce jusqu'au fonctionnement du monde paysan, pour
que la boucle soit complète


\textit{Linteau de Medamoud (Louvre) XII dynastie}


\includegraphics{\DirImg FM1_20.jpg} 


On peut voit que Sésostris III est représenté deux fois

et elle correspond à la plus ancienne scène de culte journalier que nous connaissons : Sésostris III fait offrande de
pain blanc et de gâteau


On reverra d'autres scènes d'offrandes dans les temples et tout ceci insiste sur la garantie de la MAAT et que l'Isfet
ne revienne pas et cela se résume notamment à la représentation du roi à genou présentant les vases globulaires. On
sait très bien que cette image apparaît à la VI dynastie \ : le roi offrant , cela va au delà du roi à genou présentant
des vases (globulaires, à vins ..) c'est le roi en offrande et cette représentation existe à toutes les époques


\textit{Siamon roi en Sphinx (Louvre-}

\includegraphics{\DirImg FM1_21.jpg}


On voit les offrandes faites au dieu


il y a donc des éléments de décor aussi sur le mobilier du temple, et également en relief sur les murs


\textbf{Le roi a une destinée post mortem}


Si en Egypte tout le monde a une destinée après la mort, pour le roi c'est un peu spécifique, il a une vie éternelle,
mais il rejoint le monde des dieux, et ce dès les textes des pyramides


Dans les tombes du Nouvel Empire, on voit des scènes où le roi est face au dieu (on reste dans la logique de la
thématique de \ l'offrande), mais il est dans la phase de transition où il va vers cet autre monde et où Am Douat lui
ouvre les portes et l'iconographie est spécifique, on nous représente le voyage des 12 heures de la nuit du soleil car
on associe la destinée post mortem du roi à ce que vit le soleil chaque nuit

Le soleil meurt le soir et renaît le matin, et ce que le soleil de la nuit vit chaque nuit, le roi vit cela après sa
mort

On sait aussi que c'est une évocation à mettre en parallèle avec les rites de l'embaumement


Cet aspect divin de la personne du roi conduit à une imagerie funéraire tout à fait spécifique en Egypte, et très
importante et on construit en pierre (en Mésopotamie, c'est en brique), 

Il y a non seulement le décor du temple, mais il y a également le décor de la tombe, avec cette destinée funéraire
spécifique, alors que le roi mésopotamien , nous avons forcément un corpus d'images moins importants, partant d'une
architecture de briques, qui certes \ a reçu un décor peint, \ (il est évident que dès le III millénaire en Mésopotamie
il y a eu des décors peints, mais ils sont perdus car ils ont été peints à même un enduit et la présence des deux
fleuves ne facilite pas non plus la conservation des choses)


Mais il ne faudrait surtout pas penser que seuls les assyriens ont fait des décors muraux avec leur dalle de gypses
sculptées (en réalité on sait qu'ils ont repris une ancienne tradition de peinture


Durant tout ce cours, nous aurons forcément un corpus d'images moindre pour la Mésopotamie que pour l'Egypte


autre photo stèle des vautours


\includegraphics{\DirImg FM1_22.jpg} 
\includegraphics{\DirImg FM1_23.jpg} 


\end{document}
