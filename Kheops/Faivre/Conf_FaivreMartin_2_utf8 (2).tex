% This file was converted to LaTeX by Writer2LaTeX ver. 1.2
% see http://writer2latex.sourceforge.net for more info
\documentclass[a4paper,10pt]{article}
\usepackage{nefertiyi}
\usepackage{siunitx}

\sisetup{locale=FR, exponent-product=\cdot}
\DeclareSIUnit{\an}{an}

\newcommand{\DirImg}{../img/FaivreMartin/}

%\usepackage[utf8]{inputenc}
%\usepackage[T1]{fontenc}
%\usepackage[french,english]{babel}
%\usepackage{amsmath}
%\usepackage{amssymb,amsfonts,textcomp}
%\usepackage{color}
%\usepackage{array}
%\usepackage{hhline}
%\usepackage{hyperref}
%\hypersetup{pdftex, colorlinks=true, linkcolor=blue, citecolor=blue, filecolor=blue, urlcolor=blue, pdftitle=, pdfauthor=Florence, pdfsubject=, pdfkeywords=}
%\usepackage[pdftex]{graphicx}
%% Page layout (geometry)
%\setlength\voffset{-1in}
%\setlength\hoffset{-1in}
%\setlength\topmargin{1.249cm}
%\setlength\oddsidemargin{2.499cm}
%\setlength\textheight{23.551cm}
%\setlength\textwidth{16.002998cm}
%\setlength\footskip{0.0cm}
%\setlength\headheight{1.251cm}
%\setlength\headsep{1.15cm}
%% Footnote rule
%\setlength{\skip\footins}{0.119cm}
%\renewcommand\footnoterule{\vspace*{-0.018cm}\setlength\leftskip{0pt}\setlength\rightskip{0pt plus 1fil}\noindent\textcolor{black}{\rule{0.25\columnwidth}{0.018cm}}\vspace*{0.101cm}}
%% Pages styles
%\makeatletter
%\newcommand\ps@Standard{
%  \renewcommand\@oddhead{\thepage{}}
%  \renewcommand\@evenhead{\thepage{}}
%  \renewcommand\@oddfoot{}
%  \renewcommand\@evenfoot{}
%  \renewcommand\thepage{\arabic{page}}
%}
%\makeatother
%\pagestyle{Standard}
\title{}
\author{Florence}
\date{2012-07-11}


\usepackage{graphicx}
\begin{document}
%\clearpage\setcounter{page}{1}\pagestyle{Standard}

\textbf{Conférence N° 2}

\begin{figure}
  \centering
  \includegraphics{\DirImg FM2_01.jpg}
  \caption{}
  \label{}
\end{figure}
  
\begin{figure}
  \centering
  \includegraphics{\DirImg FM2_02.jpg}
  \caption{}
  \label{}
\end{figure}

\begin{figure}
  \centering
  \includegraphics{\DirImg FM2_03.jpg}
  \caption{}
  \label{}
\end{figure}

\begin{figure}
  \centering
  \includegraphics{\DirImg FM2_04.jpg}
  \caption{}
  \label{}
\end{figure}

\begin{figure}
  \centering
  \includegraphics{\DirImg FM2_05.jpg}
  \caption{}
  \label{}
\end{figure}

\begin{figure}
  \centering
  \includegraphics{\DirImg FM2_06.jpg}
  \caption{}
  \label{}
\end{figure}

\begin{figure}
  \centering
  \includegraphics{\DirImg FM2_07.jpg}
  \caption{}
  \label{}
\end{figure}

\begin{figure}
  \centering
  \includegraphics{\DirImg FM2_08.jpg}
  \caption{}
  \label{}
\end{figure}

\begin{figure}
  \centering
  \includegraphics{\DirImg FM2_09.png}
  \caption{}
  \label{}
\end{figure}

\begin{figure}
  \centering
  \includegraphics{\DirImg FM2_10.jpg}
  \caption{}
  \label{}
\end{figure}

\begin{figure}
  \centering
  \includegraphics{\DirImg FM2_11.jpg}
  \caption{}
  \label{}
\end{figure}

\begin{figure}
  \centering
  \includegraphics{\DirImg FM2_12.jpg}
  \caption{}
  \label{}
\end{figure}

\begin{figure}
  \centering
  \includegraphics{\DirImg FM2_13.png}
  \caption{}
  \label{}
\end{figure}

\section{Mise en place de l'iconographie royale au IV\ieme millénaire}

\subsection{Mésopotamie}

Aujourd'hui, nous allons glisser sur des \oe{}ufs entre la période d'Uruk 
et celle de Nagada sachant que nous sommes sûrs de la chronologie pour la 
période d'Uruk et pour la période de Nagada

Mais le problème principal viendra non pas de l'\kmt, où nous connaissons 
bien le passage du IV\ieme au III\ieme millénaire, (sauf pour les rois de 
la dynastie~0), mais de la Mésopotamie.

Nous avons en effet un problème pour les objets qui viennent du Sud 
de la Mésopotamie, car beaucoup d'objets qui sont dans les musées 
proviennent du marché de l'art en général et cela pose un problème, 
car ce sont des datations qui sont proposées par une approche stylistique, 
mais au Proche Orient, il n'y a pas d'historien d'art, on est archéologue 
et l'objet n'est utilisé que pour le texte qu'il porte, ou pour faire une 
présentation, présentation par rapport à la civilisation sociologique\dots

Donc, avoir un discours de l'art sur la discipline du Proche Orient, 
est un peu difficile, car les professionnels de cette région n'ont pas 
cette approche, et donc on a parfois des publications assez incroyables, 
les archéologues s'intéressent essentiellement à la couche de 
stratification de l'objet trouvé.

Et il n'y a pas d'article ou de publication récente ou approfondie 
sur l'étude d'un objet en le comparant à d'autres et cela est une 
particularité de cette discipline.

Les objets que nous verrons proviennent du Sud de l'Iraq et de la \HE, 
il y a donc une zone de désert que sépare ces deux régions. Pourtant 
certaines images venant de ces deux pays devront être mises en parallèle.

\subsubsection{Chronologie période Uruk}

\begin{tabular}{ll}
  Uruk Ancien & 4300-3800 \\
  Uruk Moyen & 3800-3400 \\
  Uruk récent & 3400-3100 \\
  Uruk final (ou Djemdet Nasr) & 3100-2900  \\
\end{tabular}

\espace
On constate une chose, qui est importante dans le monde mésopotamien, 
dans la deuxième période de néolithique, entre le VI\ieme et le IV\ieme 
millénaire, c'est l'apparition des chefferies et c'est cette organisation 
en chefferies qui petit à petit va nous conduire à la royauté.

Le IV\ieme millénaire a reçu le nom de période d'Uruk, à partir donc 
d'un site archéologique. En effet, Uruk est le site de référence car 
c'est le seul site à ce jour qui donne le passage entre la culture 
précédente dire Obeid, (néolithique Sud de l'Iraq) et la période de 
l'apparition de l'urbanisation que l'on appelle proto-urbaine et c'est 
cette époque proto-urbaine qui correspond au IV\ieme millénaire et est 
contemporaine de Nagada en \kmt et a reçu le nom d'Uruk.

Cette période est donc divisée nous venons de le voir en séquence 
chronologique.

Attention, si on se réfère à des ouvrages anciens, le découpage n'est 
pas le même, on le faisait démarrer un peu plus tard et on le divisait 
en trois phases. Mais maintenant on a affiné les choses et l'Uruk ancien 
et l'Uruk moyen correspondent à deux périodes dont on ne connaît pas 
grande chose.

Mais au milieu du IV\ieme millénaire, nous avons un peu plus d'information 
et on fait maintenant la distinction entre Uruk récent et Uruk final.

En effet jusqu'à une époque récente on disait Uruk récent = 3300-2900, 
mais l'étude de la stratification des couches nous permet de montrer 
qu'il y a eu une évolution durant cette période, aux alentours de 3000 
et on a décidé de distinguer un Uruk récent 3400-3100 et un Uruk Final 
3100-2900, également appelé \emph{Djemdbet Nsar}.

C'est effectivement grâce à des fouilles allemandes, scientifiques, 
dans les années 1930, qu' il a été établi une stratification en dix-huit 
niveaux archéologiques et si on part de publications d'après la seconde 
guerre mondiale, on voit un affinage des choses et que si Uruk ancien et 
Uruk moyen restent toujours mal connus, on a pu dans les années 80 
distinguer pour la dernière période un Uruk récent et un Uruk final.

On voit donc le découpage de la période à partir des dix-huit couches 
archéologiques relevées par les archéologues et c'est ainsi que l'on 
arrive à ce découpage, la couche la plus profonde -- Uruk~18 -- 
correspondant à la période la plus ancienne.
Et les couches~6 à~4 correspondent à Uruk récent (3500-3100), 
les couches~3 à~1 à Uruk final (3100-2900).

Nous allons donc nous intéresser à ce qui caractérise cet Uruk récent, 
niveaux~6 à~4, car c'est durant cette période que deux choses apparaissent : 
\emph{l'écriture et la royauté}.

C'est une période globalement extrêmement inventive : on a non seulement 
l'attestation de l'urbanisme, mais c'est une période où l'on voit apparaître 
beaucoup d'inventions (roue, tour de potier\dots) et tout ceci révèle une 
société hiérarchisée, et des villes qui sont dirigées par une élite sociale.

On peut aussi établir l'extension de cette culture d'Uruk, partie du sud 
de l'Iraq, vers l'est (Suse), le nord l'est et éventuellement une partie 
du Delta.

Ce sont des choses dont on ne pouvait pas avoir conscience il y a une 
cinquantaine d'années. A la fin des années~30, on fouille dans la région 
d'Uruk et on trouve un même matériel archéologique et on en déduit que 
cette culture est remontée vers le nord-est.

Par contre il existe un zone entre les deux fleuves, où on n'avait jamais 
travaillé avant les années 80 et donc il y a eu une diffusion dans des 
zones nord et sa diffusion en Turquie a été découverte récemment (et on 
attend beaucoup de ces recherches).

En effet ces recherches dans le nord pourront peut être compléter notre 
corpus d'images du simple site d'Uruk.

Donc lorsque l'on parle d'Uruk récent cela correspond aux niveaux~6, 
5 et~4. Pour les nivaux~6 et~5 nous avons peu de choses, c'est donc 
surtout du niveau~4 que nous avons des choses et voit bien qu'au niveau~4 
l'agglomération s'est considérablement agrandie et que le nombre d'habitants 
a été multiplié par dix et finalement que cette agglomération est un centre 
de pouvoir avec des petits villages autour et une économie agricole qui 
en dépend.

Au départ, il semblerait que ce soit deux chefferies qui ont été réunies 
pour constituer une seule et même ville et sur le plan religieux cette 
information est importante, elle est loin d'être anodine, car c'est un 
site archéologique qui aura toujours deux espaces cultuels fondamentaux : 
le grand temple d'Inanna et le grande temple d'Anu.

Il s'agit donc de deux villages, devenus deux bourgs, qui ont été réunis 
pour ne former qu'une seule et même ville.

Les architectures sont importantes car on voit apparaître le plan en 
trois parties et que les ateliers des lapidaires sont dans ce que nous 
appelons le temple et donc dans un espace urbanisé où il n'y a pas 
d'architecture identifiable comme étant liée au palais. Nous sommes donc 
dans un moment où un grand bâtiment, considéré sans doute comme la maison 
terrestre d'une divinité, est le lieu de résidence de son représentant 
sur terre et de sa famille, et c'est également le lieu de stockage, 
d'organisation et de fabrication.

Le temple correspond donc à toute une partie de la ville ;

\textbf{C'est durant cette période d'Uruk~V que vont apparaître les bulles}.

\begin{figure}
  \centering
  \includegraphics{\DirImg FM2_01.jpg}
  \caption{}
  \label{}
\end{figure}

Les jetons sont plus anciens, il est possible que les jetons remontent 
à la période d'Uruk ancien, et même avant.

Le fait d'enfermer les jetons dans une bulle est attesté par des 
morceaux casés, peut être à la fin de l'Uruk moyen, mais cela se 
généralise à L'Uruk récent et on peut constater une multiplication 
énorme durant la période d'Uruk~V.

La bulle contient des jetons et sert à mémoriser des choses. 
Traditionnellement, on a toujours considéré que c'était le premier 
pas vers l'invention de l'écriture, mais peut être pas. Mais on peut 
remarquer que les jetons ont des tailles différentes, ils codifient 
donc des choses (mais on ne sait pas quoi). Ils sont dans des bulles 
d'argile fermées par un cachet plat. Mais sur la partie inférieure 
ou supérieure de la bulle, au moment où la ferme on met des encoches, 
et on a pu constater que le nombre d'encoches correspond toujours au 
nombre de jetons contenus à l'intérieur. Mais ces encoches n'ont jamais 
la forme des jetons (on a des jetons en forme de bulle, mais on ne va 
jamais dessiner de bulle) et cela indique qu'il y avait une codification.

Et finalement le fait d'inventer l'écriture ne résulte pas du fait 
d'utiliser les jetons, mais le fait de tracer une mémorisation de ces 
jetons par un système de code à l'extérieur des bulles.

L'écriture apparaît dans la phase d'Uruk~IV.

Un des objets caractéristique de cette période est \textbf{les 
écuelles grossières}.

\begin{figure}
  \centering
  \includegraphics{\DirImg FM2_02.jpg}
  \caption{\'Ecuelles grossières à bord biseauté avec marque du pouce 
           à la fabrication, Suse~II - Uruk (3500-3100 BC) - Musée du 
           Louvre}
  \label{}
\end{figure}

On retrouve ces écuelles en milliers d'exemplaires, elles sont faites 
par emboutissage au point (on tape au point l'argile pour lui donner 
cette forme d'écuelle), ce sont des objets fabriqués en série, trouvés 
en milliers d'exemplaires dans les tombes, dans les fossés longeant des 
bâtiments identifiés comme des temples et dans des décharges.

Peu importe la fonction de ces écuelles, mais si on trouve ces écuelles 
on peut en déduire que nous sommes dans une couche stratification d'Uruk, 
elles nous servent donc de marqueur, et nous les utilisons pour voir 
l'expansion de la culture d'Uruk dans l'ensemble du proche orient.

Et un certain nombre de documents écrits durant la période archaïque ont 
été retrouvé dans ces coupes.

Le site d'Uruk a livré près de \si{3600} tablettes, sur les niveaux~4 et~3 
et cela veut dire que l'on voit bien que l'invention de l'écriture se fait 
à la fin d'Uruk récent et que le développement de l'écriture se fait au 
début de l'Uruk final. En effet, Uruk~IV correspond à la fin de l'Uruk 
récent et Uruk~III signifie que nous sommes dans l'Uruk final.

\begin{figure}
  \centering
  \includegraphics[scale=0.8]{\DirImg FM2_03.jpg}
  \caption{Tablette de comptabilité : enregistrement d'une livraison de 
           produits céréaliers pour une fête au temple d'Inanna - Uruk 
           récent~III (3200-3000) - Musée Pergam}
  \label{}
\end{figure}

Les tablettes nous montrent la différence entre les pictogrammes extrêmement 
archaïques, qui pour beaucoup ne peuvent être lus, par rapport aux autres 
tablettes qui ont déjà une disposition d'un tracé de case et dans chaque cas 
il y a une information et un pictogramme, et un élément chiffré qui permet 
de lire chaque case et de savoir que c'est un enregistrement de produit de 
céréales pour la déesse Inanna.

Donc en très peu de temps, ce qui relève de l'Uruk récent, on ne le lit pas, 
et ce qui relève de l'Uruk final, cela ressemble à ce que nous aurons par la 
suite et on peut le lire.



%%%%%%%%%%%%%%%%%%%%%%%%%%%%%%%%%%%%%%%%%%%%%%%%%%%%%%%%%%%%%%%%%%%%%%%%%%%%%%%%%%%%%%%%%%%%%%%%%%%%%%%%%%%%%%%%%%%%%%%%%%%%%%%%%%%%%%%%%%%%%%%%%%%%%%%%%%%%%%%%%%%



\textbf{Et c'est effectivement dans cette époque
d'Uruk IV que nous allons voir apparaître
l'iconographie du roi} Mais en réalité nous sommes un
peu mal à l'aise car nous manquons de matériaux,
raison pour laquelle il ne faut jamais oublier dans cette discipline
que certaines données peuvent être balayées du jour au lendemain du
fait de nouvelles découvertes archéologiques

En effet les images du corpus pour l'Uruk récent se
comptent sur les doigts d'une main et ce sont des
objets qui proviennent soit de fouilles très anciennes et donc pas
documentées, soit du marché de l'art et donc sans
documentation associée à cet objet et c'est notamment
le cas des deux statuettes de roi du Louvre, faites en calcaire ,
achetées en 1930 sans aucune information sur leur provenance réelle

A partir de là effectivement, on peut , mais c'est un
autre travail, regarder ce que l'on a comme fouille
archéologique à cette époque là et finalement arriver à une provenance
supposée, mais qui ne pourra jamais figurer dans un catalogue, où dans
un inventaire de musée.

Dans les années 30, tous les archéologues travaillaient dans le sur de
l'Iraq jusqu'au moment où les
français iront dans le monde syrien. Donc nul doute que ces deux objets
, d'une trentaine de cm viennent de cette région. Ils
ont d'ailleurs été analysés et on sait
qu'ils sont faits dans un calcaire grossier identique

\textbf{Statue des deux rois, Louvre}

  [Warning: Image ignored] % Unhandled or unsupported graphics:
%\includegraphics[width=6.174cm,height=8.89cm]{ConfFaivreMartin-img/ConfFaivreMartin-img27.jpg}
 

En les analysant on s'est aperçu que ces deux objets ne
portent aucune trace de peinture et représentent deux personnages
debout, nus , les jambes l'une contre
l'autre et se caractérisent par le même geste,  poings
fermés et rapprochés l'un de l'autre;


Ces personnages ont été identifiés comme étant des rois prêtres ,
partant du principe que certes ils sont dépourvus de source écrite et
on ne connaît pas le contexte de leur provenance, mais
qu'ils portent deux attributs sur la tête, que plus
tard nous retrouverons sur une image où il y aura écrit le mot roi , à
côté du personnage qui lui aussi à ces deux attributs : 

ces deux éléments sont le \textbf{bandeau ou boudin}, pour certains
d'ailleurs c'est
d'ailleurs déjà le bonnet que nous reverrons plus tard
avec Gudéa et la \textbf{barb}e, c'est une barbe ronde
qui a une forme caractéristique puisqu'elle passe sous
le menton et elle n'est pas associée à une moustache

Malheureusement ces deux objets ont toujours été présentés de face
(jamais de profil ou avec un miroir ce qui permettrait de voir leur
dos) . De dos on peut s'apercevoir
qu'il n'y a aucun travail de
sculpture, il n'y a aucun galbe pour marquer les
fesses, 

C'est donc une ronde bosse qui techniquement est
travaillée comme un haut relief. C'est un bloc de
pierre détaché et on voit donc que ces objets étaient faits pour être
vus de face , et ils ont été travaillés dans ce sens (ce qui explique
qu'il n'y ait eu aucun travail pour
le dos)

Quand on passe à la période du IV millénaire, on verra que le souverain
mésopotamien n'a pas systématiquement des insignes
régaliens portés sur la tête, il peut être représenté de la même taille
que les hommes qui l'entourent, avec le même type de
coiffure , même type de vêtements, et c'est simplement
la présence de l'inscription , toujours associée à sa
tête, au dessus, à côté, qui indique Lugal, ou En, et qui nous indique
que le personnage est le roi

On a effectivement un cas, où effectivement celui identifié par les
textes comme un roi, porte ce bandeau et cette barbe,
c'est le prêtre roi de Bagdad

A partir de là, on prend cet élément et on remonte dans le temps et nous
sommes dans la période où l'écriture est en train de
se mettre en place, ou les objets ne portent pas de texte , car pour
eux c'était évident que le personnage représenté était
le roi. Et si on se place à ce niveau l à, on a donc deux objets
uniques au monde, il n'existe aucun article publié, le
seul élément de comparaison est donc le roi de Bagdad, mais qui est
brisé au niveau de la taille et qui présente un niveau de sculpture
plus avancé.

\textbf{roi Prêtre de Bagdad}

  [Warning: Image ignored] % Unhandled or unsupported graphics:
%\includegraphics[width=6.662cm,height=8.952cm]{ConfFaivreMartin-img/ConfFaivreMartin-img28.jpg}
 

C'est un roi, et il aurait vraiment fallu le trouver en
contexte, dans son temple ou dans son lieu d'origine,
ce qui nous aurait permis d'affirmer que nous sommes
bien dans la période d'Uruk IV, car en plus il y a
deux choses qui doivent attirer notre attention, c'est
que toutes les autres images de roi que nous verrons dans ce cours
datant d'Uruk, il sera toujours habillé  et donc cela
rajoute une problématique à notre sujet. Mais on peut exclure le fait
qu'il s'agirait d'un
faux pour cette époque.

\textbf{La nudité}, dans la culture sumérienne (donc plus tard) est
associée à certain moment du culte , et c'est
notamment le cas pour le rituel de libation que
l'officiant qu'il soit souverain ou
prêtre, est représenté nu face à la divinité

ce ne sont pas des gens qui représente la nudité chez
l'adulte de façon volontaire, car ce sont des gens
comme les égyptiens et d'autres civilisations, où le
costume est au contraire utilisé comme valeur sociale, et au contraire
on utilise l'image de la nudité pour signifier
l'humiliation, et donc l'anéanti et
avec tout ce que cela peut avoir comme signification, le vaincu et plus
largement la domination de l'ordre sur le chaos.

Ces images de nudité sont donc assez troublantes pour des images aussi
anciennes, ou peut être que l'on se trompe avec cette
interprétation basée sur des sources du début du III millénaire ; car
pourquoi ne pas imaginer l'inverse et
qu'à cette époque là, le seul officiant étant le roi,
il est absolument normal de le représenter nu, car
c'est l'évocation de son rôle premier
d'être le représentant du dieu sur la terre et le seul
qui officie.

en effet on ne sait pas s'il existait une classe
sacerdotale à cette époque

Mais en tout car c'est remarquable, dans la
civilisation sumérienne pour être mis en évidence

deuxième point : \textbf{la position des mains} . Mais  peut être est ce
une perte de temps

On peut se dire  tout simplement, que l'artiste
n'était pas doué. Ce sont des gens qui pendant trois
millénaires vont être représentés devant la divinité les mains jointes
ou les mains en l'air dans le geste de la prière, et
le sculpteur se serait simplifié la vie en évoquant juste le geste de
la prière , en fermant les poings et en les mettant côte à côte

Cette série de statue (avec celle du Louvre) constituent des objets
fondamentaux

traditionnellement, dans les manuels , nous verrons cette statue de
Bagdad sans provenance, et daté de l'Uruk récent.

mais pour Madame FAIVRE MARTIN, si on regarde attentivement cette
oeuvre, on peut remarquer immédiatement qu'elle
n'a pas le même niveau de sculpture que les rois
prêtres du Louvre, elle est plus aboutie et travaillée. Si on regarde
la musculature du personnages, qui est présentée, même si elle est
simpliste, et mis en parallèle avec les statues du Louvre elle donne à
celles ci d'être deux sculptures à peine dégrossies,  

dans cette statue de Bagdad on a l'impression
qu'il devait tenir quelque chose, et cela est
intéressant car plus tard dans les années - 2600, on aura des
représentations de gens tenant une petite boîte qui contient
l'offrande que nous apportons à la divinité. Est-ce
déjà cela ? et du coup ce serait également cela pour les deux statues
du Louvre.

On peut remarquer également que le travail est plus abouti pour le
bandeau, il y a une démarcation entre la calotte et le bandeau, et
derrière la tête on peut voir une masse , comme si on avait une sorte
de \textbf{chignon }(chevelure ramenée en chignon),  La barbe, par
contre est identique , elle entoure le petit menton, et il
n'y a pas de moustache et on a le même type de bouche

Si on regarde ces oeuvres de profil, on constate que la barbe remonte
bien jusqu'à la base du bandeau, comme si
c'était un seul morceau et du coup ce ne serait plus
une barbe, mais un élément de parure qui aurait été raccordé et cela
expliquerait le fait qu'elle passe bien en arrière du
menton, comme si c'était une sorte de mentonnière,

On peut voir qu'elle est également incisée de lignes
horizontales

Le sourcil est fait en une seule ligne continue, les yeux étaient
incrustées et la forme du visage est la même que ceux du Louvre

On peut aussi se poser la question, s'agit-il
d'une oeuvre inachevée ?, on peut en effet se poser la
question pour les deux statues du Louvre, et on pourrait imaginer
qu'elles proviennent d'un atelier .
Nous n'aurons jamais la réponse à cette question, mais
on peut se la poser

En résumé, les deux statues du Louvre datent-elles de la même période
que cette de Bagdad, et donc de l'Uruk récent, ou
avons nous un décalage dans le temps, ou une provenance différente ?

Mais en l'état de nos connaissances, notre corpus
s'arrête là pour les statues en trois dimensions,  et
donc leur datation d'Uruk récent peut être remise en
cause 

Selon Madame FAIVRE MARTIN, on peut se demander du fait de
l'aboutissement du travail qui est différent si ce ne
sont pas des statues d'époque différentes

\textbf{Stèle de la chasse vers 3300}

  [Warning: Image ignored] % Unhandled or unsupported graphics:
%\includegraphics[width=7.615cm,height=10.964cm]{ConfFaivreMartin-img/ConfFaivreMartin-img29.jpg}
 

Elle fait 80 cm , est en balzate

elle date Uruk niveau IV, et c'est très important car
c'est le seul objet qui soit bien daté et cela va nous
donner des informations

C'est la plus ancienne représentation connue dans la
culture d'Uruk du thème du souverain chasseur, ici
c'est la chasse du lion qui est représentée, thème qui
traversera les millénaires et on aura ensuite les fameuses chasses
assyrienne de Ninive., même les perses reprendront ce thème 

On peut observer sur cette stèle deux registres et nous avons deux fois
le personnages , ressemblant au roi prêtre, nous avons bien le bandeau,
la barbe, et l'intérêt de ce bas relief est que nous
bien l'élément qui est sculpté sur la statuette de
Bagdad, à savoir le chignon (attention ce sont des mots, nous
l'appelons chignon, mais en réalité on ne sait pas
exactement ce que c'est)

\textbf{Un bandeau, une barbe, un chignon, c'est donc
un souverain qui est représenté,} on voit qu'il est
vêtu d'une \textbf{jupe lisse} qui
s'arrête au genou, avec juste
l'indication de la ceinture et
l'homme est pied nu

On voit donc que l'homme s'attaque à
des lions. Est ce deux moments de chasse qui est représenté ? 

dans la partie supérieure , nous avons un roi prêtre face à un lion
qu'il transperce de sa lance, et en bas un roi prêtre
et son arc (et des gens qui ont travaillé sur les flèches de cette
période en Mésopotamie ont remarqué que sur cette stèle la flèche
représentée n'est pas une flèche très performante pour
s'attaquer à un lion, à une époque où
l'on sait que les mésopotamiens savaient réaliser des
flèches bien plus performante pour chasser le lion)

Cela nous pose donc un problème  s'agit il
d'une chasse réelle ou d'une chasse
symbolique ?

on peut se poser la question car s'ils avaient voulu
représenter une chasse réelle, on ne voit pas pourquoi ils
n'auraient pas représenter une flèche adéquate

Il est possible (cet objet a fait couler beaucoup
d'encre et les interrogations sont
d'ailleurs partis dans tous les  sens) , de se
demander également s'il s'agit du
même bonhomme sur la stèle.

Madame FAIVRE MARTIN remarque qu'il
n'y a aucune ligne de sol , et dans la composition
tout est décalé, il n'y a aucun alignement. Mais
surtout, on voit bien que dans les proportions de ce qui est
représenté, l'arme est énorme, et
c'est un élément que nous reverrons sur
d'autres objets,  (c'est aussi une
caractéristique que nous retrouverons en Egypte)

En effet pour montrer la supériorité du souverain sur les autres hommes,
et finalement une force qui tient du monde surnaturel, il aura toujours
des armes énormes, par rapport à sa taille 

en effet, si on regarde la taille du bonhomme et celle de
l'arme, il lui est impossible de la tenir et de
s'en servir et cela  nous le reverrons donc. Or il est
évident qu'il ne s'agit pas
d'une erreur du sculpteur, une erreur dans les
proportions car nous ne sommes qu'au III millénaire,
cela de toute façon nous le reverrons plus tard

Cela signifie qu'il y a des moments où si
l'on veut représenter le souverain
d'une façon particulière, l'arme
qu'il tient en main est énorme, et cela implique donc
qu'il y a déjà une tournure d'esprit
dans ces phases ancienne

Cet objet est daté d'Uruk IV et il sert de marqueur de
datation, 

\textbf{En effet quand on retrouve cette  image du roi, avec son
bandeau, sa barbe ronde, son petit chignon, sa jupe lisse, nous sommes
dans l'Uruk récent}

\textbf{LES SCEAUX CYLINDRES}

ils commencent à apparaître et ils sont liés à
l'apparition de l'écriture à là nous
sommes dans une période charnière

On a vu dans l'introduction que
l'écriture apparaît durant l'Uruk
récent et se développe durant l'Uruk final

Visiblement, (on en est même sur) les premiers sceaux cylindres datent
de la période d'Uruk récent, et même de niveau V,
C'est à dire qu'avant les premières
tablettes nous avons déjà des sceaux cylindres, qui sont faciles à
couler sur la bulle.

mais le problème c'est que pendant longtemps , ils ont
été laissé de côté, et donc s'ils ne proviennent pas
de fouilles allemandes d'Uruk, on ne connaît pas la
couche de stratification dont ils sont issus et donc
c'est par comparaison que nous les daterons (ex
fouilles françaises un peu anarchiques)

C'est ainsi que si l'on veut dater un
sceau d'Uruk IV, il faut que le personnage ait une
barbe ronde, un bandeau, un chignon, une jupe lisse et il faut
également regarder la taille de sa lance, 

ce sont donc les plus anciennes images attestées des personnes et on
peut les regrouper ensemble, 

exemple d'un sceau où  l'on voit des
personnages en haut agenouillé et entravés, dans un autre on a des
personnages entravés (mais en Mésopotamie on les représente totalement
désarticulés , tête en bas, corps de travers,  tête bêche et cela
signifie qu'ils sont des gêneurs et donc entravés)

Or dans les fouilles, nous n'avons trouvé aucune trace
de conflit durant la période d'Uruk, il existe des
traces de conflit visible l'archéologie ou par la
littérature avant les dynasties archaïques, mais vers - 2600.

\newline
Donc nous ne sommes pas dans la commémoration d'un
évènement historique, mais dans l'affirmation 
d'une organisation, et cela si on regarde les images
égyptiennes de l'époque de Nagada, on retrouve
également le souverain, des personnages entravés, sans que
l'on sache à quoi correspondent ces personnages
entravés (mais en Egypte c'est un peu différent car
nous savons qu'à cette période, le Sud de
l'Egypte va absorber le nord

Mais il faut faire attention car avec notre culture nous voulons
rattacher une image à un fait, or les mésopotamiens ne sont absolument
pas dans cette culture, et l'image commémorative
d'un fait pour eux a une valeur prophylactique

les sceaux sont frustrant pour nous, car nous sommes à une époque où ils
ont valeur puisqu'il représente une marque de pouvoir,
richesse pour son propriétaire, mais en même temps il
n'y a aucun signe épigraphique sur cet objet (cela
viendra très tard en fait, en effet pour voir le nom
d'une personne sur une statue ou sur un sceau il
faudra attendre - 2600 environ)

Avant la simple image suffit et le porteur du sceau est un personnage
identifié comme étant important

Dans l'ensemble les sceaux sont dans un état
déplorables, on remarque que c'est une gravure dans le
creux, et c'est donc plus difficile à faire que du bas
relief ou de la ronde bosse, et que finalement il y a beaucoup plus de
détail dans la représentation d'un sceau

les images sont plus nombreuses pour la période d'Uruk
final, cela correspond au niveau III , II et I des couches
d'Uruk et à la période de transition, exactement comme
en Egypte, 3100- 2900, cela veut dire que l'on est
exactement contemporain de Nagada III, dynastie o et de la première
dynastie

Attention dans les ouvrages anciens, la culture de
l'Uruk final n'existe pas car on
parlait d'une culture de \textbf{Djemdet Nasr} et cela
vient du fait qu'en 1928 les archéologues anglais qui
travaillaient à Uruk , ont identifié une culture type Uruk mais
beaucoup plus au Nord et surtout plus aboutie que ceux que les
allemands trouvaient à Uruk et ils ont crée cette culture de Djemdet
Nasr, qui pour eux était contemporaine d'Uruk Récent

Mais plus tard on s'est rendu compte
qu'il s'agissait de la même culture,
et qu'il y avait eu une mauvaise compréhension des
archéologues anglais et à l'heure actuelle on ne garde
cette expression de Djemdet Nasr que pour la région de
l'affluent du Tigre, car effectivement dans les années
60  on a fouillé dans cette région et on a pu
s'apercevoir qu'à la même époque
qu'Uruk c'était une culture un peu
différente et que cela valait donc la peine qu'elle
ait un nom différent

Les sceaux sont faits dans des matières différentes
d'abord en calcaire puis on les fera dans des pierres
de luxe, associé  au développement des élites urbaines et royales

\textbf{Sceau musée de Bagdad  (vers 3000)}

  [Warning: Image ignored] % Unhandled or unsupported graphics:
%\includegraphics[width=13.012cm,height=5.808cm]{ConfFaivreMartin-img/ConfFaivreMartin-img30.jpg}
 

L'iconographie du roi est attesté pour les sceaux de
cette époque  et il a été trouvé dans un temple
d'Uruk, niveau III, et il représente le roi sur un
navire qui avec une image centrale d'un personnage
dans une position proche de ce que nous avons vu
jusqu'à présente, debout, main jointe dans le geste de
la prière, barbe ronde avec une petite pointe au menton,  bandeau
chignon, et une jupe.

Mais là on voit une particularité , il y a un incision au niveau de la
jupe et elle a un quadrillage ce qui n'est pas attesté
avant Uruk III, (et il n'y a pas
d'erreur pour le moment, ce quadrillage nous sert donc
de marqueur de datation

A partir d'Uruk III, il y a une mise en scène de la
personne royale, d'une part avec
d'autres personnages , des représentations
architecturées de la rive (on suppose), d'éléments et
d'animaux

Il faut aussi regarder la représentation du bâtiment, avec ces deux
hampes, motifs très caractéristiques que nous retrouverons

\textbf{Sceau du Louvre  roi prêtre et son acolyte nourrissant le
troupeau sacré}

  [Warning: Image ignored] % Unhandled or unsupported graphics:
%\includegraphics[width=14.041cm,height=5.009cm]{ConfFaivreMartin-img/ConfFaivreMartin-img31.jpg}
 

Là encore nous retrouvons notre petit personnages  : bandeau, chignon,
barbe , jupe avec même un effet de transparence et on voit derrière le
roi un petit personnage qui porte ce que le roi porte, à savoir des
épis

si on reconstitue l'ensemble du sceau,  on peut
supposer que le roi fait face à un troupeau qu'il est
en train de nourrir

De chaque côté de l'image, on voit
l'évocation de deux hampes, que nous avons déjà vu

Pour analyser les choses, il faut se référer à des textes écrits au
XXVII et XXVI siècles avant Jésus Christ, quand nous avons les premiers
textes concernant le rôle du roi par rapport aux dieux et il est
indiqué qu'il doit nourrir le troupe sacré de la
divinité Inanna, 

et on sait également que dans la culture sumérienne, le troupeau
représente non seulement les animaux sacrés qui étaient élevés dans le
temple, mais aussi l'humanité

Peut être que dans le monde mésopotamien, ces troupeaux sont une
évocation des hommes et des femmes, alignés sous la forme
d'un troupeau, car les textes sumériens disent que le
roi est garant de tout l'équilibre, et il doit veiller
à ce qu'il 'y ait pas de famine, que
tout le monde puisse manger.

Monsieur FOREST qui a beaucoup travaillé sur le IV millénaire , était
persuadé que l'on ne se trompait pas en proposant ce
genre de lecture : troupeau sacré, mais aussi évocation de la
population.

\textbf{Sceau du British Muséum}

ll a également été acheté sur le marché de l'art, et on
voit une mise en page différente, car le troupeau est disposé de part
et d'autre des hampes.

Ces hampes c'est l'idéogramme du
roseau, qui sert dans l'écriture cunéiforme à écrire
le nom de la déesse Inanna, déesse maîtresse des troupeaux.

aussi ces hampes que l'on voit sur les sceaux sont là
pour évoquer une architecture difficile à réaliser , et que les
lapidaires ont donc utiliser l'image de ces deux mats
pour évoquer une architecture difficile à réaliser, et que de toute
évidence à l'entrée sur sanctuaire il y avait des mats
, qui permettaient d'identifier le temple

\textbf{le sceau Lapis lazulis d'Uruk à Berlin (vers -
3000)}

  [Warning: Image ignored] % Unhandled or unsupported graphics:
%\includegraphics[width=15.981cm,height=6.491cm]{ConfFaivreMartin-img/ConfFaivreMartin-img32.gif}
 

c'est le sceau le plus abouti, et il est en marbre et
au dessus il était décoré d'un petit bélier en cuivre,
pour être porté en bandoulière

Il a été trouvé dans les environs d'Uruk et appartient
à une série de petits sceaux officiels, trouvé dans un dépôt daté
d'Uruk III, à cause du motif de la jupe à chevron(?)

Ce sceau représente l'image la plus aboutie et est la
plus ancienne représentation, avec un jeu de symétrie parfait , le roi
tient des rinceaux de végétaux où l'on a un motif de
rosettes à huit pétales et ce motif est très précisément dessiné . Il y
a aussi un troupeau représenté en symétrie, les animaux étant dressés
sur leur patte pour tendre la tête vers la nourriture . A
l'arrière des animaux on voit les deux hampes, et
surtout en dessous les deux vases, comme celui retrouvé à Uruk (que
nous verrons tout à l'heure) Mais sur ce vase
d'Uruk on a l'entrée du temple
marquée par deux mats, cela nous permet de dire que dans cette scène du
sceau de Berlin nous sommes à l'intérieur
d'un temple, puisque le roi et les animaux sont
derrière ces hampes

Cette représentation est unique, il n'y a pas un défilé
d'animaux devant le roi qui porte des épis,  mais au
contraire nous avons un foisonnement d'épis (et Mr
FOREST disait que c'était la première représentation
de l'arbre de vie avec une représentation symétrique
des animaux de part et d'autre , Pourquoi pas ?)

d'ailleurs en Mésopotamie, cette thématique va
perdurer,  à savoir les animaux dressés sur leurs pattes.

\textbf{vase d'URUK - Bagad\ \ }

  [Warning: Image ignored] % Unhandled or unsupported graphics:
%\includegraphics[width=15.981cm,height=15.699cm]{ConfFaivreMartin-img/ConfFaivreMartin-img33.jpg}
 

il a été trouvé à Uruk, dans le quartier du grand temple
d'inanna, il n'y a donc aucun doute,
c'est un vase cultuel, dans une couche archéologique
de niveau III, trouvé en 1933, 1934

et c'est important, car nous sommes sur de sa datation
et il peut donc nous servir de marqueur pour dater
d'autres objets similaires où l'on
retrouve la même iconographie ou la même forme de vase

si on observe ce vase, c'est émouvant car on peut
constater qu'il a subi des restaurations  et la
première date de l'antiquité, donc au moment où il
était encore en usage. Cela signifie donc que c'est un
objet qui est resté longtemps dans le temple

et de ce fait c'est une pièce unique et fondamentale

en bas on voit un filet d'eau, grâce à elle les 
plantes vont pousser  et on peut avoir des animaux et ce motif, pendant
trois millénaires nous allons le retrouver

Pour les mésopotamiens, l'eau est un cadeau des dieux,
qui permet aux plantes de pousser, aux animaux de vivre et ensuite aux
humains de vivre également

ensuite au niveau supérieur on voit les porteurs
d'offrandes tout nus, et on voit au niveau supérieur
les deux mats, avec de toute évidence la vaisselle qui se trouvait dans
le temple et un individu qui vient accueillir, et il porte une coiffure
avec des choses dressées, est  ce la première pierre à corne de la
divinité ?

Puisque nous avons les hampes, nous sommes donc dans le temple
d'Inanna, est ce Inana elle même, ou une prêtresse
jouant le rôle d'Inanna. On voit aussi au registre
supérieur un personnage nu et un personnage dont on tient la traîne, ce
qui reste de ce personnage est cassé, mais on peut deviner un motif de
jupe à chevrons

et donc du fait de ce motif de jupes à chevron nous pouvons en déduire
qu'il s'agit du roi qui arrive avec
son offrande

et là , s'il y a bien un rituel né à Uruk, spécifique
d'Uruk, et qui a la fin du troisième millénaire
donnera lieu à un fête à l'ensemble du monde
mésopotamien, c'est le mariage sacré.

le mariage sacré  raconte l'union terrestre du roi à la
déesse Inanna , le jour du printemps et de là le cycle des saisons peut
redémarrer et il n'y aura ainsi pas de famine

Et on sait que le rôle d'Inanna était joué ce jour là
par une prêtresse.

il est donc tentant : cet objet vient d'Uruk, il
représente un être féminin sortant du temple d'Inanna
qui accueille le roi

ce pourrait donc être la plus ancienne représentation  du mariage sacré
qui ne sera plus représenté ainsi ensuite.

et là c'est intéressant car nous sommes encore dans une
époque où ils dessinent , ensuite non, car les sources seront
uniquement textuelles, et il n'y aura plus
d'images

Ce sont des gens qui ont beaucoup de mal à représenter le monde divin,
l'antropomorphisme même 

dans le monde mésopotamien  à toutes les époques,  on sent bien
qu'il y a quelque chose qui fait
qu'ils sont gêner de représenter les  êtres
surnaturels, tellement sublimes, aussi proche d'eux
physiquement et donc la divinité sera évoquée, car ils rechignent à
représenter les divinités fondamentalement importantes

Donc des choses aussi graves que le mariage sacré, dont tout
l'équilibre dépend, ils ne le mettent pas en image

\textbf{ l'EGYPTE}

Les objets proviennent tous de la Haute Egypte, et majoritairement
notamment du site d'Abydos.

depuis les années 1980; on sait que ces objets datent de Nagada II,
phase finale, dite Nagada D, et cela permet de dater le fameux couteau
de Gebel el Arak, acheté par le conservateur du Louvre à un antiquaire
du Caire en 1914

\textbf{COUTEAU DE GEBEL EL ARAK\ \ }

  [Warning: Image ignored] % Unhandled or unsupported graphics:
%\includegraphics[width=8.885cm,height=13.326cm]{ConfFaivreMartin-img/ConfFaivreMartin-img34.jpg}
 

  [Warning: Image ignored] % Unhandled or unsupported graphics:
%\includegraphics[width=15.981cm,height=15.593cm]{ConfFaivreMartin-img/ConfFaivreMartin-img35.jpg}
 

la lame était détachée du manche et le remontage s'est
fait entre les deux guerres

Il y a donc sur l'une des faces de ce manche, une
thématique de la chasse, avec l'image
d'un homme qui peut surprendre car il
n'est pas représenté à l'égyptienne,
sauf dans la règle de la représentation égyptienne de face et de profil
: il a bien le visage de profil, l'oeil et
l'épaule  de face et tout le reste de profil, et cela
a conduit pour certains à dire que ce n'était pas de
l'art égyptien, et de cette petite phrase a coulé
toute une littérature

Ce qui est ennuyeux c'est qu'il est
très rare de rencontrer quelqu'un qui soit compétant à
la fois dans l'histoire de la Mésopotamie et dans
celle de l'Egypte, 

il y a eu tellement d'ineptie , que Madame FAIVRE
MARTIN donne même l'interdiction de lire certains
manuels sur ce sujet !

Il est vrai que cet objet est troublant, certains en ont même déduit que
l'Egypte aurait été conquise par la Mésopotamie, mais
en réalité il ne faut pas oublier que les objets circulent plus
facilement que les hommes, et il y a eu forcément des objets
mésopotamiens qui sont venus en Egypte, à Abydos, et ce sans les
mésopotamiens

de surccroît, cet objet a été trouvé en haute Egypte, et à ce jour on
n'a jamais trouvé la moindre trace
d'une culture d'Uruk

en réalité, nous ne sommes pas capables d'expliquer le
pourquoi du comment, et il faut donc l'étudier de la
façon la plus intelligente possible

Quand on dit que ce coteau donne une représentation typique du Proche
Orient, c'est faux, 

en réalité il faut partir du fait que nous avons une image royale
typique de la culture d'Uruk, 

Ce n'est pas cette image qui vient du Proche Orient et
il faut l'étudier en deux temps

Il y a l'image de l'homme, et là il y
a effectivement une influence d'Uruk, 

mais il y a aussi l'homme et les lions et cela est un
thème

Pour l'image de l'homme,
c'est vrai que nous sommes embêtés, car nul ne doute
que cet objet vienne du cimetière d'Abydos et donc
cette représentation humaine est troublante car elle correspond à
l'iconographie du Proche Orient. IL y a même la jupe
lisse ce qui nous fait penser à un objet d'Uruk récent
(3500 - 3300) et cela colle avec la période de Nagada II 3500- 3200

Mais il ne faut pas oublier un objet qui voyage beaucoup, objet de
prestige et donc de cadeau, le sceau cylindre, le sceau est donné
c'est un beau cadeau, c'est un signe
important de pouvoir et de richesse

Donc un sceau comme tout objet de prestige voyage, (on a bien trouvé en
Bretagne des haches qui provenaient de Russie), et voyage seul, 

en plus il ne faut oublier que nous ne savons pas grand chose sur la
navigation dans le golfe persique à cette époque

Pour Monsieur FAROUD, cet objet serait l'évocation de
l'autre monde et que le sculpteur égyptien aurait su
que cela représentait des gens d'ailleurs,  et cela
aurait donc été une façon de représenter les gens
d'ailleurs, c'est possible.

mais il faut s'intéresser au thème de cette image, ce
n'est pas le thème du roi prêtre, mais le thème du
maître des animaux.

Quand on a un homme qui maîtrise des animaux, ce n'est
pas un prédateur, il les retient mais ne les tue pas et
c'est important. Ce thème apparaît très tôt en
Mésopotamie, dès le Néolithique à Suse, mais là il y a une chose
intéressante et ce sont les anglais qui ont travaillé dessus : 

le Maître des Animaux au Proche Orient, où que l'on
soit, se caractérise par le fait qu'il ne tue pas les
animaux,  mais les maîtrise (lions, caprins , éventuellement des
serpents), mais cet homme est toujours nu, sans attribut aucun, ou
alors avec juste l'évocation de la ceinture 

Donc sur ce couteau nous avons bien une image sumérienne, mais cet
assemblage d'un roi sumérien, dans
l'attitude du maître des animaux , pour le moment ,
est inconnu du répertoire mésopotamien, (du fait qu'il
soit habillé)

Et c'est en cela que ce coteau est surprenant car dans
les images vues précédemment, notre homme roi prêtre, nous
l'avons vu, nourri les animaux mais nos
l'avons pas vu dompter ou retenir les animaux  et
c'est pour cela qu'il faut procéder à
une étude en deux temps, image de l'homme et thème

\textbf{En Egypte, à la période de Nagada II on a
d'ailleurs la peinture de la tombe 100
d'Heriakonpolis}

  [Warning: Image ignored] % Unhandled or unsupported graphics:
%\includegraphics[width=7.615cm,height=5.533cm]{ConfFaivreMartin-img/ConfFaivreMartin-img36.png}
 

et là c'est intéressant car nous avons une image que
jamais en  Mésopotamie on associera à un nom propre, mais simplement le
Maître des Animaux, dont la traduction signifie nu, vêtu
d'une ceinture 

mais peut être que dans la culture égyptienne cela signifie autre chose,
pour représenter la terre , il y avait le lion de
l'est et le lion de l'Ouest,  et
qu'il y aurait là une évocation du nil, et donc chacun
des lions correspondrait à une rive et cela traduirait la domination
animale sur les rives est et Ouest (?)

%%%%%%%%%%%%%%%%%%%%%%%%%%%%%%%%%%%%%%%%%%%%%%%%%%%%%%%%%%%%%%%%%%%%%%%%%%%%%%%%%%%%%%%%%%%%%%%%%%%%%%%%%%%%%%%%%%%%%%%%%%%%%%%%%%%%%%%%%%%%%%%%%%%%%%%%%%%%%%%%%%%
\end{document}

\textbf{Et c'est effectivement dans cette époque d'Uruk IV que nous allons voir apparaître l'iconographie du roi} Mais
en réalité nous sommes un peu mal à l'aise car nous manquons de matériaux, raison pour laquelle il ne faut jamais
oublier dans cette discipline que certaines données peuvent être balayées du jour au lendemain du fait de nouvelles
découvertes archéologiques




En effet les images du corpus pour l'Uruk récent se comptent sur les doigts d'une main et ce sont des objets qui
proviennent soit de fouilles très anciennes et donc pas documentées, soit du marché de l'art et donc sans documentation
associée à cet objet et c'est notamment le cas des deux statuettes de roi du Louvre, faites en calcaire , achetées en
1930 sans aucune information sur leur provenance réelle




A partir de là effectivement, on peut , mais c'est un autre travail, regarder ce que l'on a comme fouille archéologique
à cette époque là et finalement arriver à une provenance supposée, mais qui ne pourra jamais figurer dans un catalogue,
où dans un inventaire de musée.


Dans les années 30, tous les archéologues travaillaient dans le sur de l'Iraq jusqu'au moment où les français iront dans
le monde syrien. Donc nul doute que ces deux objets , d'une trentaine de cm viennent de cette région. Ils ont
d'ailleurs été analysés et on sait qu'ils sont faits dans un calcaire grossier identique




\textbf{Statue des deux rois, Louvre}




% \includegraphics[width=6.174cm,height=8.89cm]{\DirImg FM2_17.jpg} 




En les analysant on s'est aperçu que ces deux objets ne portent aucune trace de peinture et représentent deux
personnages debout, nus , les jambes l'une contre l'autre et se caractérisent par le même geste, \ poings fermés et
rapprochés l'un de l'autre; 


Ces personnages ont été identifiés comme étant des rois prêtres , partant du principe que certes ils sont dépourvus de
source écrite et on ne connaît pas le contexte de leur provenance, mais qu'ils portent deux attributs sur la tête, que
plus tard nous retrouverons sur une image où il y aura écrit le mot roi , à côté du personnage qui lui aussi à ces deux
attributs : 


ces deux éléments sont le \textbf{bandeau ou boudin}, pour certains d'ailleurs c'est d'ailleurs déjà le bonnet que nous
reverrons plus tard avec Gudéa et la \textbf{barb}e, c'est une barbe ronde qui a une forme caractéristique puisqu'elle
passe sous le menton et elle n'est pas associée à une moustache




Malheureusement ces deux objets ont toujours été présentés de face (jamais de profil ou avec un miroir ce qui
permettrait de voir leur dos) . De dos on peut s'apercevoir qu'il n'y a aucun travail de sculpture, il n'y a aucun
galbe pour marquer les fesses, 


C'est donc une ronde bosse qui techniquement est travaillée comme un haut relief. C'est un bloc de pierre détaché et on
voit donc que ces objets étaient faits pour être vus de face , et ils ont été travaillés dans ce sens (ce qui explique
qu'il n'y ait eu aucun travail pour le dos)


\bigskip


Quand on passe à la période du IV millénaire, on verra que le souverain mésopotamien n'a pas systématiquement des
insignes régaliens portés sur la tête, il peut être représenté de la même taille que les hommes qui l'entourent, avec
le même type de coiffure , même type de vêtements, et c'est simplement la présence de l'inscription , toujours associée
à sa tête, au dessus, à côté, qui indique Lugal, ou En, et qui nous indique que le personnage est le roi




On a effectivement un cas, où effectivement celui identifié par les textes comme un roi, porte ce bandeau et cette
barbe, c'est le prêtre roi de Bagdad




A partir de là, on prend cet élément et on remonte dans le temps et nous sommes dans la période où l'écriture est en
train de se mettre en place, ou les objets ne portent pas de texte , car pour eux c'était évident que le personnage
représenté était le roi. Et si on se place à ce niveau l à, on a donc deux objets uniques au monde, il n'existe aucun
article publié, le seul élément de comparaison est donc le roi de Bagdad, mais qui est brisé au niveau de la taille et
qui présente un niveau de sculpture plus avancé.




\textbf{roi Prêtre de Bagdad}




% \includegraphics[width=6.662cm,height=8.952cm]{\DirImg FM2_18.jpg} 





C'est un roi, et il aurait vraiment fallu le trouver en contexte, dans son temple ou dans son lieu d'origine, ce qui
nous aurait permis d'affirmer que nous sommes bien dans la période d'Uruk IV, car en plus il y a deux choses qui
doivent attirer notre attention, c'est que toutes les autres images de roi que nous verrons dans ce cours datant
d'Uruk, il sera toujours habillé \ et donc cela rajoute une problématique à notre sujet. Mais on peut exclure le fait
qu'il s'agirait d'un faux pour cette époque.




\textbf{La nudité}, dans la culture sumérienne (donc plus tard) est associée à certain moment du culte , et c'est
notamment le cas pour le rituel de libation que l'officiant qu'il soit souverain ou prêtre, est représenté nu face à la divinité


ce ne sont pas des gens qui représente la nudité chez l'adulte de façon volontaire, car ce sont des gens comme les
égyptiens et d'autres civilisations, où le costume est au contraire utilisé comme valeur sociale, et au contraire on
utilise l'image de la nudité pour signifier l'humiliation, et donc l'anéanti et avec tout ce que cela peut avoir comme
signification, le vaincu et plus largement la domination de l'ordre sur le chaos.


Ces images de nudité sont donc assez troublantes pour des images aussi anciennes, ou peut être que l'on se trompe avec
cette interprétation basée sur des sources du début du III millénaire ; car pourquoi ne pas imaginer l'inverse et qu'à
cette époque là, le seul officiant étant le roi, il est absolument normal de le représenter nu, car c'est l'évocation
de son rôle premier d'être le représentant du dieu sur la terre et le seul qui officie.


en effet on ne sait pas s'il existait une classe sacerdotale à cette époque


\bigskip


Mais en tout car c'est remarquable, dans la civilisation sumérienne pour être mis en évidence


\bigskip


\bigskip


deuxième point : \textbf{la position des mains} . Mais \ peut être est ce une perte de temps


On peut se dire \ tout simplement, que l'artiste n'était pas doué. Ce sont des gens qui pendant trois millénaires vont
être représentés devant la divinité les mains jointes ou les mains en l'air dans le geste de la prière, et le sculpteur
se serait simplifié la vie en évoquant juste le geste de la prière , en fermant les poings et en les mettant côte à
côte


Cette série de statue (avec celle du Louvre) constituent des objets fondamentaux


\bigskip


traditionnellement, dans les manuels , nous verrons cette statue de Bagdad sans provenance, et daté de l'Uruk récent.


mais pour Madame FAIVRE MARTIN, si on regarde attentivement cette oeuvre, on peut remarquer immédiatement qu'elle n'a
pas le même niveau de sculpture que les rois prêtres du Louvre, elle est plus aboutie et travaillée. Si on regarde la
musculature du personnages, qui est présentée, même si elle est simpliste, et mis en parallèle avec les statues du
Louvre elle donne à celles ci d'être deux sculptures à peine dégrossies, \ 


dans cette statue de Bagdad on a l'impression qu'il devait tenir quelque chose, et cela est intéressant car plus tard
dans les années - 2600, on aura des représentations de gens tenant une petite boîte qui contient l'offrande que nous
apportons à la divinité. Est-ce déjà cela ? et du coup ce serait également cela pour les deux statues du Louvre.


\bigskip


\bigskip


\bigskip


On peut remarquer également que le travail est plus abouti pour le bandeau, il y a une démarcation entre la calotte et
le bandeau, et derrière la tête on peut voir une masse , comme si on avait une sorte de \textbf{chignon }(chevelure
ramenée en chignon), \ La barbe, par contre est identique , elle entoure le petit menton, et il n'y a pas de moustache
et on a le même type de bouche


\bigskip


Si on regarde ces oeuvres de profil, on constate que la barbe remonte bien jusqu'à la base du bandeau, comme si c'était
un seul morceau et du coup ce ne serait plus une barbe, mais un élément de parure qui aurait été raccordé et cela
expliquerait le fait qu'elle passe bien en arrière du menton, comme si c'était une sorte de mentonnière,


On peut voir qu'elle est également incisée de lignes horizontales


Le sourcil est fait en une seule ligne continue, les yeux étaient incrustées et la forme du visage est la même que ceux
du Louvre


\bigskip


On peut aussi se poser la question, s'agit-il d'une oeuvre inachevée ?, on peut en effet se poser la question pour les
deux statues du Louvre, et on pourrait imaginer qu'elles proviennent d'un atelier . Nous n'aurons jamais la réponse à
cette question, mais on peut se la poser


\bigskip


En résumé, les deux statues du Louvre datent-elles de la même période que cette de Bagdad, et donc de l'Uruk récent, ou
avons nous un décalage dans le temps, ou une provenance différente ?


Mais en l'état de nos connaissances, notre corpus s'arrête là pour les statues en trois dimensions, \ et donc leur
datation d'Uruk récent peut être remise en cause 


Selon Madame FAIVRE MARTIN, on peut se demander du fait de l'aboutissement du travail qui est différent si ce ne sont
pas des statues d'époque différentes



\textbf{Stèle de la chasse vers 3300}




% \includegraphics[width=7.615cm,height=10.964cm]{\DirImg FM2_19.jpg} 




Elle fait 80 cm , est en balzate


elle date Uruk niveau IV, et c'est très important car c'est le seul objet qui soit bien daté et cela va nous donner des
informations




C'est la plus ancienne représentation connue dans la culture d'Uruk du thème du souverain chasseur, ici c'est la chasse
du lion qui est représentée, thème qui traversera les millénaires et on aura ensuite les fameuses chasses assyrienne de
Ninive., même les perses reprendront ce thème 




On peut observer sur cette stèle deux registres et nous avons deux fois le personnages , ressemblant au roi prêtre, nous
avons bien le bandeau, la barbe, et l'intérêt de ce bas relief est que nous bien l'élément qui est sculpté sur la
statuette de Bagdad, à savoir le chignon (attention ce sont des mots, nous l'appelons chignon, mais en réalité on ne
sait pas exactement ce que c'est)


\textbf{Un bandeau, une barbe, un chignon, c'est donc un souverain qui est représenté,} on voit qu'il est vêtu d'une
\textbf{jupe lisse} qui s'arrête au genou, avec juste l'indication de la ceinture et l'homme est pied nu




On voit donc que l'homme s'attaque à des lions. Est ce deux moments de chasse qui est représenté ?


dans la partie supérieure , nous avons un roi prêtre face à un lion qu'il transperce de sa lance, et en bas un roi
prêtre et son arc (et des gens qui ont travaillé sur les flèches de cette période en Mésopotamie ont remarqué que sur
cette stèle la flèche représentée n'est pas une flèche très performante pour s'attaquer à un lion, à une époque où l'on
sait que les mésopotamiens savaient réaliser des flèches bien plus performante pour chasser le lion)




Cela nous pose donc un problème \ s'agit il d'une chasse réelle ou d'une chasse symbolique ?


on peut se poser la question car s'ils avaient voulu représenter une chasse réelle, on ne voit pas pourquoi ils
n'auraient pas représenter une flèche adéquate


Il est possible (cet objet a fait couler beaucoup d'encre et les interrogations sont d'ailleurs partis dans tous les
\ sens) , de se demander également s'il s'agit du même bonhomme sur la stèle.


Madame FAIVRE MARTIN remarque qu'il n'y a aucune ligne de sol , et dans la composition tout est décalé, il n'y a aucun
alignement. Mais surtout, on voit bien que dans les proportions de ce qui est représenté, l'arme est énorme, et c'est
un élément que nous reverrons sur d'autres objets, \ (c'est aussi une caractéristique que nous retrouverons en Egypte)


En effet pour montrer la supériorité du souverain sur les autres hommes, et finalement une force qui tient du monde
surnaturel, il aura toujours des armes énormes, par rapport à sa taille 


en effet, si on regarde la taille du bonhomme et celle de l'arme, il lui est impossible de la tenir et de s'en servir et
cela \ nous le reverrons donc. Or il est évident qu'il ne s'agit pas d'une erreur du sculpteur, une erreur dans les
proportions car nous ne sommes qu'au III millénaire, cela de toute façon nous le reverrons plus tard


Cela signifie qu'il y a des moments où si l'on veut représenter le souverain d'une façon particulière, l'arme qu'il
tient en main est énorme, et cela implique donc qu'il y a déjà une tournure d'esprit dans ces phases ancienne




Cet objet est daté d'Uruk IV et il sert de marqueur de datation, 


\textbf{En effet quand on retrouve cette \ image du roi, avec son bandeau, sa barbe ronde, son petit chignon, sa jupe
lisse, nous sommes dans l'Uruk récent}



\textbf{LES SCEAUX CYLINDRES}




ils commencent à apparaître et ils sont liés à l'apparition de l'écriture à là nous sommes dans une période charnière


On a vu dans l'introduction que l'écriture apparaît durant l'Uruk récent et se développe durant l'Uruk final




Visiblement, (on en est même sur) les premiers sceaux cylindres datent de la période d'Uruk récent, et même de niveau V,
C'est à dire qu'avant les premières tablettes nous avons déjà des sceaux cylindres, qui sont faciles à couler sur la
bulle.




mais le problème c'est que pendant longtemps , ils ont été laissé de côté, et donc s'ils ne proviennent pas de fouilles
allemandes d'Uruk, on ne connaît pas la couche de stratification dont ils sont issus et donc c'est par comparaison que
nous les daterons (ex fouilles françaises un peu anarchiques)




C'est ainsi que si l'on veut dater un sceau d'Uruk IV, il faut que le personnage ait une barbe ronde, un bandeau, un
chignon, une jupe lisse et il faut également regarder la taille de sa lance, 


ce sont donc les plus anciennes images attestées des personnes et on peut les regrouper ensemble, 




exemple d'un sceau où \ l'on voit des personnages en haut agenouillé et entravés, dans un autre on a des personnages
entravés (mais en Mésopotamie on les représente totalement désarticulés , tête en bas, corps de travers, \ tête bêche
et cela signifie qu'ils sont des gêneurs et donc entravés)




Or dans les fouilles, nous n'avons trouvé aucune trace de conflit durant la période d'Uruk, il existe des traces de
conflit visible l'archéologie ou par la littérature avant les dynasties archaïques, mais vers - 2600.


%\newline
Donc nous ne sommes pas dans la commémoration d'un évènement historique, mais dans l'affirmation \ d'une organisation,
et cela si on regarde les images égyptiennes de l'époque de Nagada, on retrouve également le souverain, des personnages
entravés, sans que l'on sache à quoi correspondent ces personnages entravés (mais en Egypte c'est un peu différent car
nous savons qu'à cette période, le Sud de l'Egypte va absorber le nord




Mais il faut faire attention car avec notre culture nous voulons rattacher une image à un fait, or les mésopotamiens ne
sont absolument pas dans cette culture, et l'image commémorative d'un fait pour eux a une valeur prophylactique




les sceaux sont frustrant pour nous, car nous sommes à une époque où ils ont valeur puisqu'il représente une marque de
pouvoir, richesse pour son propriétaire, mais en même temps il n'y a aucun signe épigraphique sur cet objet (cela
viendra très tard en fait, en effet pour voir le nom d'une personne sur une statue ou sur un sceau il faudra attendre -
2600 environ)




Avant la simple image suffit et le porteur du sceau est un personnage identifié comme étant important




Dans l'ensemble les sceaux sont dans un état déplorables, on remarque que c'est une gravure dans le creux, et c'est donc
plus difficile à faire que du bas relief ou de la ronde bosse, et que finalement il y a beaucoup plus de détail dans la
représentation d'un sceau




les images sont plus nombreuses pour la période d'Uruk final, cela correspond au niveau III , II et I des couches d'Uruk
et à la période de transition, exactement comme en Egypte, 3100- 2900, cela veut dire que l'on est exactement
contemporain de Nagada III, dynastie o et de la première dynastie




Attention dans les ouvrages anciens, la culture de l'Uruk final n'existe pas car on parlait d'une culture de
\textbf{Djemdet Nasr} et cela vient du fait qu'en 1928 les archéologues anglais qui travaillaient à Uruk , ont
identifié une culture type Uruk mais beaucoup plus au Nord et surtout plus aboutie que ceux que les allemands
trouvaient à Uruk et ils ont crée cette culture de Djemdet Nasr, qui pour eux était contemporaine d'Uruk Récent


Mais plus tard on s'est rendu compte qu'il s'agissait de la même culture, et qu'il y avait eu une mauvaise compréhension
des archéologues anglais et à l'heure actuelle on ne garde cette expression de Djemdet Nasr que pour la région de
l'affluent du Tigre, car effectivement dans les années 60 \ on a fouillé dans cette région et on a pu s'apercevoir qu'à
la même époque qu'Uruk c'était une culture un peu différente et que cela valait donc la peine qu'elle ait un nom
différent




Les sceaux sont faits dans des matières différentes d'abord en calcaire puis on les fera dans des pierres de luxe,
associé \ au développement des élites urbaines et royales




\textbf{Sceau musée de Bagdad \ (vers 3000)}



% \includegraphics[width=13.012cm,height=5.808cm]{\DirImg FM2_20.jpg} 



L'iconographie du roi est attesté pour les sceaux de cette époque \ et il a été trouvé dans un temple d'Uruk, niveau
III, et il représente le roi sur un navire qui avec une image centrale d'un personnage dans une position proche de ce
que nous avons vu jusqu'à présente, debout, main jointe dans le geste de la prière, barbe ronde avec une petite pointe
au menton, \ bandeau chignon, et une jupe.


Mais là on voit une particularité , il y a un incision au niveau de la jupe et elle a un quadrillage ce qui n'est pas
attesté avant Uruk III, (et il n'y a pas d'erreur pour le moment, ce quadrillage nous sert donc de marqueur de
datation




A partir d'Uruk III, il y a une mise en scène de la personne royale, d'une part avec d'autres personnages , des
représentations architecturées de la rive (on suppose), d'éléments et d'animaux




Il faut aussi regarder la représentation du bâtiment, avec ces deux hampes, motifs très caractéristiques que nous
retrouverons



\textbf{Sceau du Louvre \ roi prêtre et son acolyte nourrissant le troupeau sacré}


% \includegraphics[width=14.041cm,height=5.009cm]{\DirImg FM2_21.jpg} 



Là encore nous retrouvons notre petit personnages \ : bandeau, chignon, barbe , jupe avec même un effet de transparence
et on voit derrière le roi un petit personnage qui porte ce que le roi porte, à savoir des épis


si on reconstitue l'ensemble du sceau, \ on peut supposer que le roi fait face à un troupeau qu'il est en train de
nourrir



De chaque côté de l'image, on voit l'évocation de deux hampes, que nous avons déjà vu




Pour analyser les choses, il faut se référer à des textes écrits au XXVII et XXVI siècles avant Jésus Christ, quand nous
avons les premiers textes concernant le rôle du roi par rapport aux dieux et il est indiqué qu'il doit nourrir le
troupe sacré de la divinité Inanna, 


et on sait également que dans la culture sumérienne, le troupeau représente non seulement les animaux sacrés qui étaient
élevés dans le temple, mais aussi l'humanité




Peut être que dans le monde mésopotamien, ces troupeaux sont une évocation des hommes et des femmes, alignés sous la
forme d'un troupeau, car les textes sumériens disent que le roi est garant de tout l'équilibre, et il doit veiller à ce
qu'il 'y ait pas de famine, que tout le monde puisse manger.



Monsieur FOREST qui a beaucoup travaillé sur le IV millénaire , était persuadé que l'on ne se trompait pas en proposant
ce genre de lecture : troupeau sacré, mais aussi évocation de la population.


\textbf{Sceau du British Muséum}



ll a également été acheté sur le marché de l'art, et on voit une mise en page différente, car le troupeau est disposé de
part et d'autre des hampes.


Ces hampes c'est l'idéogramme du roseau, qui sert dans l'écriture cunéiforme à écrire le nom de la déesse Inanna, déesse
maîtresse des troupeaux.


aussi ces hampes que l'on voit sur les sceaux sont là pour évoquer une architecture difficile à réaliser , et que les
lapidaires ont donc utiliser l'image de ces deux mats pour évoquer une architecture difficile à réaliser, et que de
toute évidence à l'entrée sur sanctuaire il y avait des mats , qui permettaient d'identifier le temple


\textbf{le sceau Lapis lazulis d'Uruk à Berlin (vers - 3000)}


\bigskip

% \includegraphics[width=15.981cm,height=6.491cm]{\DirImg FM2_22.png} 




c'est le sceau le plus abouti, et il est en marbre et au dessus il était décoré d'un petit bélier en cuivre, pour être
porté en bandoulière


Il a été trouvé dans les environs d'Uruk et appartient à une série de petits sceaux officiels, trouvé dans un dépôt daté
d'Uruk III, à cause du motif de la jupe à chevron(?)




Ce sceau représente l'image la plus aboutie et est la plus ancienne représentation, avec un jeu de symétrie parfait , le
roi tient des rinceaux de végétaux où l'on a un motif de rosettes à huit pétales et ce motif est très précisément
dessiné . Il y a aussi un troupeau représenté en symétrie, les animaux étant dressés sur leur patte pour tendre la tête
vers la nourriture . A l'arrière des animaux on voit les deux hampes, et surtout en dessous les deux vases, comme celui
retrouvé à Uruk (que nous verrons tout à l'heure) Mais sur ce vase d'Uruk on a l'entrée du temple marquée par deux
mats, cela nous permet de dire que dans cette scène du sceau de Berlin nous sommes à l'intérieur d'un temple, puisque
le roi et les animaux sont derrière ces hampes




Cette représentation est unique, il n'y a pas un défilé d'animaux devant le roi qui porte des épis, \ \ mais au
contraire nous avons un foisonnement d'épis (et Mr FOREST disait que c'était la première représentation de l'arbre de
vie avec une représentation symétrique des animaux de part et d'autre , Pourquoi pas ?)


d'ailleurs en Mésopotamie, cette thématique va perdurer, \ à savoir les animaux dressés sur leurs pattes.




\textbf{vase d'URUK - Bagad\ \ }




% \includegraphics[width=15.981cm,height=15.699cm]{\DirImg FM2_23.jpg} 





il a été trouvé à Uruk, dans le quartier du grand temple d'inanna, il n'y a donc aucun doute, c'est un vase cultuel,
dans une couche archéologique de niveau III, trouvé en 1933, 1934




et c'est important, car nous sommes sur de sa datation et il peut donc nous servir de marqueur pour dater d'autres
objets similaires où l'on retrouve la même iconographie ou la même forme de vase




si on observe ce vase, c'est émouvant car on peut constater qu'il a subi des restaurations \ et la première date de
l'antiquité, donc au moment où il était encore en usage. Cela signifie donc que c'est un objet qui est resté longtemps
dans le temple


et de ce fait c'est une pièce unique et fondamentale




en bas on voit un filet d'eau, grâce à elle les \ plantes vont pousser \ et on peut avoir des animaux et ce motif,
pendant trois millénaires nous allons le retrouver


Pour les mésopotamiens, l'eau est un cadeau des dieux, qui permet aux plantes de pousser, aux animaux de vivre et
ensuite aux humains de vivre également


\bigskip


ensuite au niveau supérieur on voit les porteurs d'offrandes tout nus, et on voit au niveau supérieur les deux mats,
avec de toute évidence la vaisselle qui se trouvait dans le temple et un individu qui vient accueillir, et il porte une
coiffure avec des choses dressées, est \ ce la première pierre à corne de la divinité ?


Puisque nous avons les hampes, nous sommes donc dans le temple d'Inanna, est ce Inana elle même, ou une prêtresse jouant
le rôle d'Inanna. On voit aussi au registre supérieur un personnage nu et un personnage dont on tient la traîne, ce qui
reste de ce personnage est cassé, mais on peut deviner un motif de jupe à chevrons


et donc du fait de ce motif de jupes à chevron nous pouvons en déduire qu'il s'agit du roi qui arrive avec son offrande


\bigskip


et là , s'il y a bien un rituel né à Uruk, spécifique d'Uruk, et qui a la fin du troisième millénaire donnera lieu à un
fête à l'ensemble du monde mésopotamien, c'est le mariage sacré.


\bigskip


le mariage sacré \ raconte l'union terrestre du roi à la déesse Inanna , le jour du printemps et de là le cycle des
saisons peut redémarrer et il n'y aura ainsi pas de famine


Et on sait que le rôle d'Inanna était joué ce jour là par une prêtresse.


il est donc tentant : cet objet vient d'Uruk, il représente un être féminin sortant du temple d'Inanna qui accueille le roi


ce pourrait donc être la plus ancienne représentation \ du mariage sacré qui ne sera plus représenté ainsi ensuite.


et là c'est intéressant car nous sommes encore dans une époque où ils dessinent , ensuite non, car les sources seront
uniquement textuelles, et il n'y aura plus d'images


Ce sont des gens qui ont beaucoup de mal à représenter le monde divin, l'antropomorphisme même 





dans le monde mésopotamien \ à toutes les époques, \ on sent bien qu'il y a quelque chose qui fait qu'ils sont gêner de
représenter les \ êtres surnaturels, tellement sublimes, aussi proche d'eux physiquement et donc la divinité sera
évoquée, car ils rechignent à représenter les divinités fondamentalement importantes


Donc des choses aussi graves que le mariage sacré, dont tout l'équilibre dépend, ils ne le mettent pas en image





\textbf{l'EGYPTE}




Les objets proviennent tous de la Haute Egypte, et majoritairement notamment du site d'Abydos.


depuis les années 1980; on sait que ces objets datent de Nagada II, phase finale, dite Nagada D, et cela permet de dater
le fameux couteau de Gebel el Arak, acheté par le conservateur du Louvre à un antiquaire du Caire en 1914





\textbf{COUTEAU DE GEBEL EL ARAK\ \ }




% \includegraphics[width=8.885cm,height=13.326cm]{\DirImg FM2_24.jpg} 



% \includegraphics[width=15.981cm,height=15.593cm]{\DirImg FM2_25.jpg} 





la lame était détachée du manche et le remontage s'est fait entre les deux guerres




Il y a donc sur l'une des faces de ce manche, une thématique de la chasse, avec l'image d'un homme qui peut surprendre
car il n'est pas représenté à l'égyptienne, sauf dans la règle de la représentation égyptienne de face et de profil :
il a bien le visage de profil, l'oeil et l'épaule \ de face et tout le reste de profil, et cela a conduit pour certains
à dire que ce n'était pas de l'art égyptien, et de cette petite phrase a coulé toute une littérature




Ce qui est ennuyeux c'est qu'il est très rare de rencontrer quelqu'un qui soit compétant à la fois dans l'histoire de la
Mésopotamie et dans celle de l'Egypte, 


il y a eu tellement d'ineptie , que Madame FAIVRE MARTIN donne même l'interdiction de lire certains manuels sur ce sujet
!




Il est vrai que cet objet est troublant, certains en ont même déduit que l'Egypte aurait été conquise par la
Mésopotamie, mais en réalité il ne faut pas oublier que les objets circulent plus facilement que les hommes, et il y a
eu forcément des objets mésopotamiens qui sont venus en Egypte, à Abydos, et ce sans les mésopotamiens


de surccroît, cet objet a été trouvé en haute Egypte, et à ce jour on n'a jamais trouvé la moindre trace d'une culture
d'Uruk




en réalité, nous ne sommes pas capables d'expliquer le pourquoi du comment, et il faut donc l'étudier de la façon la
plus intelligente possible


Quand on dit que ce coteau donne une représentation typique du Proche Orient, c'est faux, 





en réalité il faut partir du fait que nous avons une image royale typique de la culture d'Uruk, 


Ce n'est pas cette image qui vient du Proche Orient et il faut l'étudier en deux temps


Il y a l'image de l'homme, et là il y a effectivement une influence d'Uruk, 


mais il y a aussi l'homme et les lions et cela est un thème





Pour l'image de l'homme, c'est vrai que nous sommes embêtés, car nul ne doute que cet objet vienne du cimetière d'Abydos
et donc cette représentation humaine est troublante car elle correspond à l'iconographie du Proche Orient. IL y a même
la jupe lisse ce qui nous fait penser à un objet d'Uruk récent (3500 - 3300) et cela colle avec la période de Nagada II
3500- 3200







Mais il ne faut pas oublier un objet qui voyage beaucoup, objet de prestige et donc de cadeau, le sceau cylindre, le
sceau est donné c'est un beau cadeau, c'est un signe important de pouvoir et de richesse





Donc un sceau comme tout objet de prestige voyage, (on a bien trouvé en Bretagne des haches qui provenaient de Russie),
et voyage seul, 


en plus il ne faut oublier que nous ne savons pas grand chose sur la navigation dans le golfe persique à cette époque





Pour Monsieur FAROUD, cet objet serait l'évocation de l'autre monde et que le sculpteur égyptien aurait su que cela
représentait des gens d'ailleurs, \ et cela aurait donc été une façon de représenter les gens d'ailleurs, c'est
possible.




mais il faut s'intéresser au thème de cette image, ce n'est pas le thème du roi prêtre, mais le thème du maître des
animaux.


Quand on a un homme qui maîtrise des animaux, ce n'est pas un prédateur, il les retient mais ne les tue pas et c'est
important. Ce thème apparaît très tôt en Mésopotamie, dès le Néolithique à Suse, mais là il y a une chose intéressante
et ce sont les anglais qui ont travaillé dessus : 


le Maître des Animaux au Proche Orient, où que l'on soit, se caractérise par le fait qu'il ne tue pas les animaux,
\ mais les maîtrise (lions, caprins , éventuellement des serpents), mais cet homme est toujours nu, sans attribut
aucun, ou alors avec juste l'évocation de la ceinture




Donc sur ce couteau nous avons bien une image sumérienne, mais cet assemblage d'un roi sumérien, dans l'attitude du
maître des animaux , pour le moment , est inconnu du répertoire mésopotamien, (du fait qu'il soit habillé)


Et c'est en cela que ce coteau est surprenant car dans les images vues précédemment, notre homme roi prêtre, nous
l'avons vu, nourri les animaux mais nos l'avons pas vu dompter ou retenir les animaux \ et c'est pour cela qu'il faut
procéder à une étude en deux temps, image de l'homme et thème




\textbf{En Egypte, à la période de Nagada II on a d'ailleurs la peinture de la tombe 100 d'Heriakonpolis}



% \includegraphics[width=7.615cm,height=5.533cm]{\DirImg FM2_26.png} 





et là c'est intéressant car nous avons une image que jamais en \ Mésopotamie on associera à un nom propre, mais
simplement le Maître des Animaux, dont la traduction signifie nu, vêtu d'une ceinture 




mais peut être que dans la culture égyptienne cela signifie autre chose, pour représenter la terre , il y avait le lion
de l'est et le lion de l'Ouest, \ et qu'il y aurait là une évocation du nil, et donc chacun des lions correspondrait à
une rive et cela traduirait la domination animale sur les rives est et Ouest (?)
