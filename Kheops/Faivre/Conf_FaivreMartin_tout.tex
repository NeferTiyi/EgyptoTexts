% This file was converted to LaTeX by Writer2LaTeX ver. 1.0.2
% see http://writer2latex.sourceforge.net for more info
\documentclass{article}
\usepackage[utf8]{inputenc}
\usepackage[T3,T1]{fontenc}
\usepackage[french]{babel}
\usepackage[noenc]{tipa}
\usepackage{tipx}
\usepackage[geometry,weather,misc,clock]{ifsym}
\usepackage{pifont}
\usepackage{eurosym}
\usepackage{amsmath}
\usepackage{wasysym}
\usepackage{amssymb,amsfonts,textcomp}
\usepackage{array}
\usepackage{supertabular}
\usepackage{hhline}
\usepackage{graphicx}
\setlength\tabcolsep{1mm}
\renewcommand\arraystretch{1.3}
\title{}
\begin{document}
\ \ \ \ ATTRIBUTS EN EGYPTE ET EN MESOPOTAMIE\ \ 

\begin{center}
 [Warning: Image ignored] % Unhandled or unsupported graphics:
%\includegraphics[width=13.964cm,height=3.698cm]{}

\end{center}
\begin{center}
 [Warning: Image ignored] % Unhandled or unsupported graphics:
%\includegraphics[width=3.456cm,height=6.49cm]{}

\end{center}
\begin{center}
 [Warning: Image ignored] % Unhandled or unsupported graphics:
%\includegraphics[width=10.759cm,height=10.829cm]{}

\end{center}
\begin{center}
 [Warning: Image ignored] % Unhandled or unsupported graphics:
%\includegraphics[width=13.964cm,height=20.141cm]{}

\end{center}
\begin{center}
 [Warning: Image ignored] % Unhandled or unsupported graphics:
%\includegraphics[width=6.808cm,height=10.088cm]{}

\end{center}
\begin{center}
 [Warning: Image ignored] % Unhandled or unsupported graphics:
%\includegraphics[width=7.76cm,height=6.032cm]{}

\end{center}
\begin{center}
 [Warning: Image ignored] % Unhandled or unsupported graphics:
%\includegraphics[width=9.201cm,height=7.511cm]{}

\end{center}
\begin{center}
 [Warning: Image ignored] % Unhandled or unsupported graphics:
%\includegraphics[width=7.76cm,height=10.335cm]{}

\end{center}
\begin{center}
 [Warning: Image ignored] % Unhandled or unsupported graphics:
%\includegraphics[width=6.843cm,height=7.548cm]{}

\end{center}
\begin{center}
 [Warning: Image ignored] % Unhandled or unsupported graphics:
%\includegraphics[width=8.249cm,height=12.791cm]{}

\end{center}
\begin{center}
 [Warning: Image ignored] % Unhandled or unsupported graphics:
%\includegraphics[width=15.98cm,height=19.966cm]{}

\end{center}
\begin{center}
 [Warning: Image ignored] % Unhandled or unsupported graphics:
%\includegraphics[width=15.98cm,height=11.394cm]{}

\end{center}
\begin{center}
 [Warning: Image ignored] % Unhandled or unsupported graphics:
%\includegraphics[width=5.82cm,height=6.667cm]{}

\end{center}
\begin{center}
 [Warning: Image ignored] % Unhandled or unsupported graphics:
%\includegraphics[width=0.528cm,height=0.387cm]{}

\end{center}
\begin{center}
 [Warning: Image ignored] % Unhandled or unsupported graphics:
%\includegraphics[width=15.98cm,height=23.882cm]{}

\end{center}
\begin{center}
 [Warning: Image ignored] % Unhandled or unsupported graphics:
%\includegraphics[width=15.98cm,height=11.993cm]{}

\end{center}
\begin{center}
 [Warning: Image ignored] % Unhandled or unsupported graphics:
%\includegraphics[width=15.98cm,height=16.373cm]{}

\end{center}
\begin{center}
 [Warning: Image ignored] % Unhandled or unsupported graphics:
%\includegraphics[width=8.43cm,height=18.061cm]{}

\end{center}
\begin{center}
 [Warning: Image ignored] % Unhandled or unsupported graphics:
%\includegraphics[width=9.201cm,height=16.484cm]{}

\end{center}
\begin{center}
 [Warning: Image ignored] % Unhandled or unsupported graphics:
%\includegraphics[width=15.98cm,height=7.125cm]{}

\end{center}
\begin{center}
 [Warning: Image ignored] % Unhandled or unsupported graphics:
%\includegraphics[width=15.98cm,height=15.557cm]{}

\end{center}
\begin{center}
 [Warning: Image ignored] % Unhandled or unsupported graphics:
%\includegraphics[width=15.98cm,height=8.395cm]{}

\end{center}
\begin{center}
 [Warning: Image ignored] % Unhandled or unsupported graphics:
%\includegraphics[width=15.98cm,height=9.877cm]{}

\end{center}
CONFERENCE N°1  

Aujourd{\textquotesingle}hui, nous allons poser les choses,
c{\textquotesingle}est à dire voir comment à partir
d{\textquotesingle}une certaine idéologie royale, on voit apparaître un
type d{\textquotesingle}images. Sans approfondir,  il faut quand même
marquer les grandes caractéristiques de la royauté mésopotamienne, et
les grandes caractéristiques de la royauté égyptienne.

A partir du deuxième cours, nous nous intéresserons à
l{\textquotesingle}apparition de l{\textquotesingle}iconographie royale
et nous ne traiterons que de la fin du IV millénaire pour croiser
certains objets, les regarder en chronologie, la plus juste possible.
Entre Uruk et Nagada, on peut s{\textquotesingle}aligner et on pourra
ainsi constater que des points communs sont ainsi à remarquer, mais
très tôt également que des caractéristiques spécifiques à chacune de
ces deux civilisations s{\textquotesingle}affirment et finalement dès
les premières images qui sont parfois extrêmement succinctes.

On s{\textquotesingle}intéressera ensuite , durant les trois séances
suivantes,  à la thématique liées à la relation privilégiée du
souverain avec son dieu, et là encore il faut toujours
s{\textquotesingle}interroger sur sa nature. Ce lien privilégié qui
fait de ce roi un constructeur, et nous verrons donc la mise en image
de ce roi constructeur.

Nous verrons également comment il est administrateur des biens du dieu,
et de ce fait un défenseur de ces biens,  Et cela nous conduira  à
envisager l{\textquotesingle}image du face à face entre le roi et le
dieu , l{\textquotesingle}image de l{\textquotesingle}action en faveur
du dieu, et l{\textquotesingle}image du conflit au nom du dieu.

Et tout cela fait donc à peu près quatre millénaires
d{\textquotesingle}images, sur ces civilisations assez vaste; aussi il
faudra se limiter et ne prendre que quelques exemples fondamentaux et
faire une lecture plus posée de certaines images absolument
fondamentales. 

Des  images du roi en deux ou trois dimensions, et la littérature qui
elle aussi servira de support , 

Nous avons donc deux territoires vastes, qui se sont développés de
manière parallèle, et qui pour le moment ne semblent pas avoir été
tellement en contact les uns avec les autres. De ces contacts, on ne
sait trop comment cela se passait et on a donc toujours considéré que
le lien était le monde palestinien, mais ce n{\textquotesingle}est pas
si évident car finalement on a une connaissance assez floue de
l{\textquotesingle}archéologie de la péninsule arabe et de la
possibilité de la circulation par bateau via le golfe persique

On a certes des idées et l{\textquotesingle}on sait
qu{\textquotesingle}il y a eu des points de contact possible avec le
monde phénicien, à partir du moment où ces deux civilisations
s{\textquotesingle}y sont tournés. Le monde mésopotamien vers
l{\textquotesingle}est pour y trouver du bois et des marchandises (il
n{\textquotesingle}ont rien) et l{\textquotesingle}Egypte pour avoir
certaines matières dont l{\textquotesingle}étain.

Seulement les recherches récentes dans le monde mésopotamien montrent
que ces gens se tournaient également de l{\textquotesingle}autre côté

On va donc partir du principe d{\textquotesingle}un développement par
l{\textquotesingle}est

Nous sommes donc dans deux civilisations où l{\textquotesingle}on voit
apparaître les deux plus anciens systèmes d{\textquotesingle}écriture
de l{\textquotesingle}histoire de l{\textquotesingle}humanité et on
constate dans ces deux civilisations  qu{\textquotesingle}il y a
apparition de l{\textquotesingle}écriture et apparition de la royauté

Effectivement en Egypte, nous aurons une unification partant de la Haute
Egypte et se diffusant vers le Nord, avec toute cette inconnue que
représente l{\textquotesingle}annexion de la Basse Egypte, et ce au
tournant du IV , III millénaire. Et on arrive finalement à un royaume
unifié, organisé autour du Nil, et dirigé par un seul roi (sauf durant
les périodes troublées que l{\textquotesingle}on appelle périodes
intermédiaires) Nous n{\textquotesingle}étudierons pas ces périodes, on
se limite aux périodes où l{\textquotesingle}Egypte
n{\textquotesingle}a qu{\textquotesingle}un seul roi.

En Mésopotamie : c{\textquotesingle}est un pays qui a deux fleuves
(Tigre et Euphrate) et une géographie très différente. Elle voit
apparaître une royauté au IV millénaire, mais c{\textquotesingle}est un
monde morcelé, et va fonctionner dans un premier temps sur la base des
cités états, ce que l{\textquotesingle}on pourrait appeler des
principautés (une ville autour d{\textquotesingle}un micro royaume). On
peut constater qu{\textquotesingle}avec l{\textquotesingle}apparition
du phénomène de guerre (qui n{\textquotesingle}est pas attesté avant -
2600), ces cités entrent en conflit et forment des royaumes un peu plus
importants en s{\textquotesingle}absorbant et de là on arrivera
finalement au II et I millénaire à ce que le monde mésopotamien
fonctionne en deux royaumes, le centre et au Sud  Babylone, et au nord
le royaume assyrien.

Cela veut dire que même quand nous aurons le grand royaume de Babylone ,
au Nord,  il existe toujours le royaume assyrien, il
n{\textquotesingle}y a jamais eu dans le monde mésopotamien, une seule
unité politique.

Mais ce qui est intéressant c{\textquotesingle}est que ces mésopotamiens
fonctionnent de la même façon, certes il existe des particularités
entre Babylone et l{\textquotesingle}Assyrie, on ne peut pas dire que
le principe de la souveraineté soit le même  entre Babylone et Assur,
car les babyloniens sont supérieurement intellectuels par rapport aux
assyriens, (les babyloniens ont repris à leur compte la tradition
culturelle sumérienne, et c{\textquotesingle}est une lacune que les
assyriens n{\textquotesingle}arriveront jamais à combler.

Dons déjà là nous avons quand même deux principes qui
s{\textquotesingle}affirment très tôt, et on constate que le monde
égyptien , comme le monde mésopotamien, sont des sociétés basées sur
des croyances polythéistes, des religions appelées préhistoriques et
que dans ces deux systèmes  les rois dirigent au nom des dieux

Certes sur quatre millénaires nous aurons des variantes religieuses,
(par exemple dans le monde mésopotamien en fonction des périodes où des
endroits où l{\textquotesingle}on se trouve on ne désigne pas les
choses exactement de la même façon, c{\textquotesingle}est à dire que
les assyriens par exemple considèrent que les rois sont leur dieu ASUR
, et que celui que nous appelons roi est son représentant sur terre. 
Il y a en effet cette ambiguïté dans le monde de la Mésopotamie, car
certains textes parlant d{\textquotesingle}un dieu utilise le mot roi

Chronologie entre ces deux mondes 

Il faut mettre en regard les grandes périodes (elles sont données un peu
à la louche), et il faut se référer à Nagada qui est contemporain
d{\textquotesingle}Uruk , et la période Thinite qui correspond à la
période d{\textquotesingle}Uruk final et le début des dynasties
archaïques, dynasties qui seront finalement contemporaines à
l{\textquotesingle}Ancien Empire :

{}-  l{\textquotesingle}empire d{\textquotesingle}Akad est finalement
contemporain de la fin de la V et VI dynastie

{}- la période néo sumérienne est contemporaine de la période
intermédiaire

{}- le Moyen Empire est contemporain à la première dynastie de Babylone 
et à cette époque les assyriens forment un royaume paysan renfermé sur
lui même

et il y a un moment où il n{\textquotesingle}y a rien dans le monde
mésopotamien, alors que l{\textquotesingle}Egypte est à
l{\textquotesingle}apogée de sa civilisation, c{\textquotesingle}est à
dire une période où le Pharaon est fort et période de grandes
production d{\textquotesingle}images

Cette période en Mésopotamie correspond à ce que les historiens
appelaient autrefois en référence au monde grec,  les siècles obscurs
du monde mésopotamien, pour bien monter qu{\textquotesingle}à
l{\textquotesingle}époque où la Mésopotamie est dominée par les
Kassites, paradoxalement c{\textquotesingle}est une époque importante
au niveau culturel, copie de textes, archivage des textes,
transcription de documents, et c{\textquotesingle}est donc une époque
fondamentale pour les linguistes, mais c{\textquotesingle}est une
royauté fermée sur elle même, qui ne produit aucune image, aucun objet

On peut utiliser le terme de bronze récent pour cette époque et cela
correspond à l{\textquotesingle}époque de la richesse de toutes les
cités états de la  Syrie Palestine

Et contrairement au moment où le monde mésopotamien va reprendre de
l{\textquotesingle}ampleur avec les grands empires conquérants néo
sumériens et néo babyloniens, cela correspond à une période
intermédiaire en Egypte de la basse époque, où il y a un répertoire
artistique intéressant mais qui évolue de façon particulière

Finalement, nous aurons vraiment un développement parallèle entre ces
deux mondes jusqu{\textquotesingle}à la fin du bronze moyen au XVI
siècle avant JC

Bronze Ancien, Bronze moyen, nous avons les deux civilisations qui se
développent un peu à la même vitesse, et ensuite on a ce balancier.

\ \ \ \ LE MONDE MESOPOTAMIEN

C{\textquotesingle}est un monde qui d{\textquotesingle}une part
correspond à une archéologie récente et très vaste. Mais attention ce
monde ne correspond pas uniquement à l{\textquotesingle}iraq et il faut
regarder sur une carte qui présente la géographie actuelle et on voit
que la partie haute Euphrate qui est un territoire syrien actuellement,
et se référer aux cours des deux fleuves, où les mésopotamiens se sont
installés  et il y a des sites anciens dans l{\textquotesingle}actuel
Turquie. Pour l{\textquotesingle}Iran, il n{\textquotesingle}y  a pas
grand chose, car il y avait la limite naturelle formée par la montagne 
ZAGROS. (de toute façon l{\textquotesingle}Iran était une zone paria
dans le monde antique (à vérifier)

On peut dire aussi que les sites mésopotamiens sont vraiment liés aux
zones des fleuves

Attention, dans l{\textquotesingle}antiquité la côte entrait davantage à
l{\textquotesingle}intérieur des terres ,les deux fleuves se jetaient
indépendamment dans le golfe persique,  et donc UR était un ensemble
portuaire important et qu{\textquotesingle}au IV et début du III
millénaire le sud a été fondamental dans le développement de cette
région. 

C{\textquotesingle}est un monde sédentaire et un monde de nomades. Les
historiens ont toujours voulu opposer deux aspects culturels et
retrouver d{\textquotesingle}ailleurs dans certains royaumes ou dans
certaines caractéristiques des spécificités liées aux deux mondes. On
développe encore l{\textquotesingle}idée du sumérien plus intellectuel
et le fait que si dans le monde sumérien des conflits se développèrent
vers - 2600, c{\textquotesingle}est au moment où beaucoup de
populations nomades , sémites, venaient de s{\textquotesingle}installer
, 

Dans le monde mésopotamien, que nous dit-on  à propos de la royauté ?
Les textes nous donnent des informations avec d{\textquotesingle}une
part un texte que l{\textquotesingle}on appelle : la liste royale
sumérienne, dont on a différentes copies dans différents musées du
monde, (copie sur tablette, sur de l{\textquotesingle}argile, qui
pouvait être aussi un objet de fondation)

La liste royale sumérienne est un texte qui démarre , mis en place dans
le courant du III millénaire, et raconte l{\textquotesingle}histoire de
la partie sud du monde mésopotamien, depuis les origines de
l{\textquotesingle}humanité, sachant que pour le monde mésopotamien,
dans le monde de la création, il y avait un premier temps, qui était le
temps parfait, le temps des dieux. A ce moment les dieux étaient
divisés en deux groupes : les dieux suprêmes et les dieux de rang
inférieur qui serviront les grands dieux jusqu{\textquotesingle}au jour
où ils se révoltent et refusent de continuer à servir les grands dieux.
Là il y a différentes versions selon les endroits et les époques et ces
versions deviennent sanglantes dans les sources babyloniennes du II
millénaire, avec le meneur qui sera mis à mort.

Dans les sources du III millénaire , c{\textquotesingle}est un peu
différent et c{\textquotesingle}est là que les grands dieux demandent à
la déesse mère de prendre de l{\textquotesingle}argile et de fabriquer
sur son tour de potier une créature à leur image, mais une créature sur
laquelle on aura une prise, cette créature c{\textquotesingle}est
l{\textquotesingle}homme dans le sens humanité (il n{\textquotesingle}y
a pas de premier homme ou de première femme ) c{\textquotesingle}est
l{\textquotesingle}humain qui est ainsi créé et le récit dit que cet
humain, on lui perce le bras et le sang coule, contrairement à une
divinité mais surtout les dieux garderont pour eux
l{\textquotesingle}immortalité et laisseront la mort aux humains

Et c{\textquotesingle}est dans ce contexte de récit que
l{\textquotesingle}on apprend qu{\textquotesingle}il y aura un
évènement fondamental, dans cette histoire de
l{\textquotesingle}humanité c{\textquotesingle}est la descente de la
royauté sur terre, que l{\textquotesingle}on attribue au dieu ENLIL,
(seigneur du vent)

En en sumérien veut dire maitre seigneur, et IL veut dire vent, air, 

ENLIL est donc un dieu associé à l{\textquotesingle}air,
l{\textquotesingle}espace, au vent, et dans le panthéon mésopotamien il
est le numéro 2, après le dieu du ciel

on attribue à ENLIL une descendance, mais surtout c{\textquotesingle}est
lui qui gère les affaires de la terre

Donc ENLIL est le créateur de la royauté, car il constata que cette
humanité était absolument incontrable et qu{\textquotesingle}il fallait
qu{\textquotesingle}elle soit dominée et il inventa la royauté
qu{\textquotesingle}il lança du ciel sur la tête des hommes, afin
d{\textquotesingle}organiser cette création. 

L{\textquotesingle}humanité a toujours fondamentalement agacée ENLIL, 
et il va même les punir

Dans le récit de ENLIL dans l{\textquotesingle}EKUR à Nippur qui est son
grand temple :

  [Warning: Image ignored] % Unhandled or unsupported graphics:
%\includegraphics[width=13.965cm,height=3.699cm]{ConfFaivreMartintout-img/ConfFaivreMartintout-img1.jpg}
 

C{\textquotesingle}est la résidence terrestre de ce dieu, où il avait un
grand temple dans la ville de Nippur, et on a ce passage qui nous dit :

{\textquotedbl}Sans ENLIL de la grande montagne, aucune ville
n{\textquotesingle}aurait été construite, aucun habitat
n{\textquotesingle}aurait été érigé, aucun enclos habitable,
n{\textquotesingle}aurait été construit, aucune bergerie
n{\textquotesingle}aurait été établie, aucun roi
n{\textquotesingle}aurait été élevé, aucun seigneur ne serait  né aucun
grand prêtre, aucune grande prêtresse n{\textquotesingle}accomplirait
le culte {\textquotedbl} 

On voit ainsi une référence rurale permanente, ils commencent à parler
du bétail avant les hommes et cela traduit un ordre de pensée. Il est
évident que dans un texte égyptien, on ne classerait pas les choses
dans cet ordre là.

Les mésopotamiens sont des gens très pragmatique et ils sont toujours
très proches des contingences  terre à terre, liées à leur économie,
qui est une économie agricole.

Donc sans ENLIL, aucun roi n{\textquotesingle}aurait été élevé

Cette humanité braillarde agace ENLIL, qui depuis que
l{\textquotesingle}homme est sur terre l{\textquotesingle}empêche de
dormir , il ne trouve plus le sommeil à cause du bruit de
l{\textquotesingle}humanité; alors il décide d{\textquotesingle}un
geste d{\textquotesingle}agacement d{\textquotesingle}envoyer le déluge
et nous aurons ainsi le deuxième temps

Et dans la liste royale sumérienne, on nous parle des rois
{\textquotedbl} ceux d{\textquotesingle}avant le déluge{\textquotedbl}
et il y ensuite ceux {\textquotedbl} d{\textquotesingle}après le déluge
{\textquotedbl}

et cela veut dire qu{\textquotesingle}après le déluge
c{\textquotesingle}est le moment d{\textquotesingle}entrer dans les
temps historiques, et on nous dit qu{\textquotesingle}ENLIL ré envoie
la royauté sur terre, c{\textquotesingle}est à dire que nous avons une
deuxième descente, nouvelle descente de la royauté sur terre

\begin{flushleft}
\tablehead{}
\begin{supertabular}{m{20.119999cm}}
\\
  [Warning: Image ignored] % Unhandled or unsupported graphics:
%\includegraphics[width=3.457cm,height=6.491cm]{ConfFaivreMartintout-img/ConfFaivreMartintout-img2.gif}
  liste sumérienne 

\\
\end{supertabular}
\end{flushleft}
Cette idée est assez intéressante et il faut garder à
l{\textquotesingle}esprit cette division : rois des temps anciens, et
rois de l{\textquotesingle}époque historique

On peut même se demander si cela ne pourrait par être lié aux rois avant
l{\textquotesingle}écriture et aux rois postérieurs à
l{\textquotesingle}écriture

dans l{\textquotesingle}esprit de la culture sumérienne ce
qu{\textquotesingle}il présente comme étant mythologie, ce texte du
déluge, ait un fondement finalement par rapport aux rois des temps
anciens, que l{\textquotesingle}on a oublié, dont les lignées se sont
éteintes, par rapport à de nouvelles royautés qui se sont mises en
place dès le III millénaire

La royauté est un cadeau des dieux aux hommes,  C{\textquotesingle}est
donc à la fin du IV millénaire que l{\textquotesingle}on va voir
apparaître les premières représentations de roi  et les rois seront
présents dans l{\textquotesingle}art jusqu{\textquotesingle}à la fin de
la civilisation mésopotamienne

En général on considère la fin du monde mésopotamien en - 539, car il
est complètement chamboulé par l{\textquotesingle}invasion perse, et
les perses ont une mentalité tout à fait différentes, un rapport au
divin tout à fait différent et donc un rapport à la souveraineté tout
aussi différent

Donc on n{\textquotesingle}intègre pas les perses dans
l{\textquotesingle}histoire de la Mésopotamie ancienne et on
s{\textquotesingle}arrête donc à la prise de Babylone en - 539..

Ce qui est notable, c{\textquotesingle}est la représentation par
allusion, qui est extrêmement fréquente, comme nous pourrons le
constater : comment magnifier le souverain sans avoir à la représenter
? : il suffit de développer certains thèmes et on sait très bien que
tout cela  a été initié par le roi. 

Après, il faut voir à quoi servent ces images

Epoque d{\textquotesingle}URUK : 

C{\textquotesingle}est la période de formation, et correspond grosso
modo à la période de Nagada en Egypte

Les dynasties archaïques , ce sont ces fameux rois après le déluge,
c{\textquotesingle}est à dire le moment où nous avons
l{\textquotesingle}écriture, donc le moment où nous avons des textes,
et où dans les textes nous avons un mot qui désigne le roi et des
images associées à ce mot et cela signifie que pour la première fois un
homme, la représentation d{\textquotesingle}un humain, qui aurait à
côté de lui le mot que l{\textquotesingle}on sait être le titre
correspondant au mot roi et nous  pourrons alors avec certitude
qu{\textquotesingle}il est souverain, car malheureusement la plupart
des sources des listes royales que nous avons sont des copies tardives
. C{\textquotesingle}est le problème général avec les annales, plus
c{\textquotesingle}est ancien, plus c{\textquotesingle}est oublié et
c{\textquotesingle}est farfelu (on peut voir des noms qui ne sont que
des jeux de mots).

Les dynasties archaïques sont importantes car elles posent un principe
de société, qui est le dieu , le roi son vicaire humain et les hommes.
Déjà c{\textquotesingle}est quelque chose d{\textquotesingle}essentiel,
et cela signifie un fonctionnement par un intermédiaire, il
n{\textquotesingle}y a pas de lien direct entre les hommes et leurs
dieux.  Dans l{\textquotesingle}ensemble, il y a même plutôt un lien de
crainte. A partir du moment où  l{\textquotesingle}on sait
qu{\textquotesingle}ENLIL ne supporte pas les hommes, il a déjà détruit
une fois l{\textquotesingle}humanité par le déluge, il y a toujours la
crainte d{\textquotesingle}une nouvelle punition.

Les dynasties archaïques posent finalement toutes les bases qui seront
admirées et se perpéturont à travers les millénaires

Et on voit bien qu{\textquotesingle}une première dynastie de Babylone,
militaire, du début du deuxième millénaire s{\textquotesingle}appuie
complètement sur ce vieux fond culturel sumérien.
C{\textquotesingle}est donc fondamental, de bien voir comment les
choses se mettent en place.

L{\textquotesingle}empire d{\textquotesingle}AKAD 

C{\textquotesingle}est  la première fois que les hommes vont avoir la
volonté de créer non pas un royaume, mais un empire,
c{\textquotesingle}est à dire qu{\textquotesingle}ils vont aller
guerroyer au delà d{\textquotesingle}où on pouvait aller avant, et
aller vers le Nord, le Sud, l{\textquotesingle}Est et dominer
complètement le  monde des deux fleuves

Mais AKAD, c{\textquotesingle}est une utopie car ils étaient incapable
de gouverner ces territoires auxquels ils aspiraient, car il
n{\textquotesingle}existait pas de structures d{\textquotesingle}état
qui en étaient capable. Ils n{\textquotesingle}avaient pas
l{\textquotesingle}organisation que les égyptiens avaient à la même
époque, car tout simplement c{\textquotesingle}étaient des gens qui
fonctionnaient sur la base de micro royaume dans le sud.

Mais AKKAD est fondamental, même si c{\textquotesingle}est une période
pour laquelle les sources sont maigres. Mais c{\textquotesingle}est la
première fois qu{\textquotesingle}il est attesté qu{\textquotesingle}un
roi ait été divinisé de son vivant, et cela pour la royauté
mésopotamienne, c{\textquotesingle}est quelque chose
d{\textquotesingle}absolument ponctuel

Le roi est un homme, il a une nature humaine et rien de plus. 
C{\textquotesingle}est un humain à la tête d{\textquotesingle}autres
humains et il est le vicaire des dieux sur terre. Il
n{\textquotesingle}a aucun élément surnaturel en lui, le couronnement
ne lui donne aucun pouvoir surnaturel. C{\textquotesingle}est
d{\textquotesingle}autant plus frustrant que le quatrième roi de la
dynastie d{\textquotesingle}Akkad NARAM Sîn (petit fils de Sargon) se
fait diviniser car devant son nom on met le déterminatif de
l{\textquotesingle}étoile, et cela implique une divinisation. Ces gens
régnaient à partir d{\textquotesingle}une ville qui est la seule grande
capitale du monde mésopotamien, que l{\textquotesingle}on ne sait pas
situer et tant que l{\textquotesingle}on n{\textquotesingle}aura pas
les archives royales de l{\textquotesingle}époque, on ne pourra pas
réellement étudier cette période

Après AKKAD on notera également quelques divinisations de roi, de son
vivant, notamment sous la 3ème dynastie d{\textquotesingle}UR, et
quelques unes dans la dynastie babylonienne et après plus jamais;

On sait donc que cela a existé, mais on ne sait pas concrètement comment
cela se faisait, existe t il un temple où l{\textquotesingle}on rendait
une culte du roi ?

Akkad après avoir duré  150 ans s{\textquotesingle}effondre, mais là
encore on ne sait pas comment.

Il y aura un moment de renaissance sumérienne, mais dans une logique
différente qui sera de renaître à partir d{\textquotesingle}un royaume
unifié démembré. C{\textquotesingle}est à dire que ce ne sont pas les
minuscules cités états qui revoient le jour. On voit bien que la
troisième dynastie d{\textquotesingle}UR correspond en gros au Sud
c{\textquotesingle}est  à dire que cela englobe les régions de Sumer et
de Babylone

A partir du deuxième millénaire, les deux royaumes qui seront rivaux
jusqu{\textquotesingle}à la fin  : Babylone au centre et au sud et
Akkad au nord ; sauf que durant tout le deuxième millénaire on
n{\textquotesingle}entend plus parler des assyriens, ils sont dominés
par d{\textquotesingle}autres, et il ne se réveillent que vers -1200
avant JC, dans le système boule de neige du passage des peuples de la
mer et durant le premier millénaire il y aura un conflit permanent
entre le Nord et le Sud pour l{\textquotesingle}Hégémonie.

En Mésopotamie, ce sont donc des royaumes, Zone de Sumer, zone de
Babylone, Akkad les  assyriens sont vraiment  au Nord et autour du
Tigre. C{\textquotesingle}est ainsi que nous n{\textquotesingle}avons
aucun texte assyrien avant le 9ème siècle qui nous parle de
l{\textquotesingle}Euphrate, ils ne partaient pas aussi loin, et ils
allaient plutôt vers le Nord

Les premières sources dans les listes royales sumériennes qui parlent du
roi, leur donnent le titre de EN, en langue sumérienne cela signifie
Maître, Seigneur, mais il semblerait que ce titre puisse correspondre à
prêtre et c{\textquotesingle}est la raison pour laquelle les historiens
dans le temps parlaient du roi prêtre, car cela correspondait à une
formule française correspondant à la traduction d{\textquotesingle}un
mon sumérien 

Nous avons un deuxième titre  LUGAL (?) qui en sumérien veut dire Homme,
et gal en sumérien veut dire rateau et quand on a la place de bien
écrire, le rateau est dessiné au dessus de la tête de
l{\textquotesingle}homme.

On a beaucoup discuté pour tenter de comprendre à quoi correspondaient
tous ces titres et finalement nous sommes incapables
d{\textquotesingle}en faire quelque chose de cohérent, Quand on lit
tout ce qui a été dit sur le sujet, on s{\textquotesingle}aperçoit que
pour le moment on ne comprend pas la logique. On voit que dans certains
petits royaumes sumériens (ex à l{\textquotesingle}époque des cités
éclatées) du Sud on va appeler le roi En, et à trente km de là  le roi
s{\textquotesingle}appellera LUGAL. Mais on ne sait pas pourquoi, car
nous n{\textquotesingle}avons pas compris la logique de la chose.

Et il y a un troisième titre, qui est ENSI, qui lui aussi est ambigu,
car plus tard il pourra avoir la notation de gouverneur, donc le ENSI
serait dépendant d{\textquotesingle}un Lugal ?  On voit bien en tout
cas que les premiers titres associés au roi , sont assez ambigus, et
montrent bien le lien entre le pouvoir et le dieu. Et
d{\textquotesingle}ailleurs, il y a quelque chose qui peut être
rattaché à cela  : c{\textquotesingle}est qu{\textquotesingle}au niveau
de l{\textquotesingle}archéologie, on constate que dans les phases
anciennes, le palais comme élément architecturé
n{\textquotesingle}existe pas. Les grands bâtiments des cités états
sont des temples. Il semble bien donc que celui que nous appelons roi,
vive dans le temple et que les entrepôts du royaume soient également
dans le temple.

C{\textquotesingle}est à cette époque, autour de 2600 AV JC, que
l{\textquotesingle}on voit apparaître la filiation, la filiation
humaine, c{\textquotesingle}est à dire que le roi donne son nom en le
faisant précéder du nom suivi de la mention {\textquotedbl} fils de ..
{\textquotedbl} et le nom qui suit est celui de son père

Et ceci par les sources écrites prouvent de façon certaine
qu{\textquotesingle}au moins à partir de - 2600, la royauté est
héréditaire

est elle héréditaire à partir de ce moment là ? (dynastique archaïque
III) , on  ne le sait pas. On pense qu{\textquotesingle}elle devient
progressivement héréditaire à partir du III millénaire ; avant on
devait avoir le système du choix parmi les anciens

Nous avons donc établi un premier point : la fonction royale est
descendue sur terre, cadeau de ENLIL et la première caractéristique  :
l{\textquotesingle}hérédité : je deviens roi car mon père
l{\textquotesingle}était. L{\textquotesingle}hérédité, la filiation est
quelque chose qui se met en  place et qui va durer pendant 3,000 ans
(en effet la plupart des rois sont fils de ... )

Mais il y a une deuxième raison pour devenir roi, ce sont les grandes
qualités : c{\textquotesingle}est à dire que les textes développent
l{\textquotesingle}idée que certains humains étaient destinés à devenir
roi à partir de leur haute qualité 

Le premier à s{\textquotesingle}en référer est Sargon (premier fondateur
de l{\textquotesingle}empire d{\textquotesingle}Akkad) et
n{\textquotesingle}a cessé de vanter les mérites , Il a renversé le
souverain, en réalité c{\textquotesingle}est un usurpateur, et
d{\textquotesingle}ailleurs on ne connait pas son nom de naissance,
mais il se fait couronner sous le nom de SARGON SHARRUKIN et Sharrum
veut dire roi dans la langue d{\textquotesingle}Akkad et Kin est un
adjectif qui veut dire vrai, légitime,  C{\textquotesingle}est donc un
usurpateur qui se fait couronner sous le nom de roi légitime.

Tête de Ninive, qui pourrait être celle de Sargon

  [Warning: Image ignored] % Unhandled or unsupported graphics:
%\includegraphics[width=10.76cm,height=10.83cm]{ConfFaivreMartintout-img/ConfFaivreMartintout-img3.jpg}
 

Après sa mort on a fait de Sargon un personnage merveilleux, mais avec
deux siècles de retard, car tous les textes relatifs à Sargon datent de
la première dynastie de Babylone car cette première dynastie de
Babylone , sont des gens qui n{\textquotesingle}avaient pas de lien
royal avant et se sont appuyés sur cette fausse filiation . Alors on
montre SARGON comme un personnage merveilleux et on se place comme
étant ses descendants , dotés des mêmes qualités, (et
qu{\textquotesingle}ils sont ainsi devenus rois du fait de ces mêmes
qualités)

LE ROI EN MESOPOTAMIE EST :

{}- Vicaire du dieu sur terre,

{}- gestionnaire et défenseur des biens du dieu 

{}- c{\textquotesingle}est un homme, un humain tout simplement

C{\textquotesingle}est donc un mortel qui au delà de la mort
n{\textquotesingle}aura rien de plus que les autres, et cet élément est
très important pour notre sujet

le Roi, vicaire du dieu sur terre :  

Pendant 3,000 ans, les images royales le représenteront debout, assis,
les mains jointes dans le signe de la prière, car dans
l{\textquotesingle}état actuel des choses toutes les statues que nous
connaissons proviennent des temples et sont donc systématiquement
associées à des divinités

Nous n{\textquotesingle}avons en effet à ce jour aucun exemple connu de
statue identifiée comme ayant pu être mise à l{\textquotesingle}entrée
d{\textquotesingle}un bâtiment officiel, sur une place visible des
gens, ou associée à la sépulture d{\textquotesingle}un roi

(mais cela devait arranger les sculpteurs qui ne disposaient pas
d{\textquotesingle}un matériel extraordinaire et qu{\textquotesingle}il
était très difficile ainsi de travailler dans la dorrite)

Statue ESHNUNNA (1600) 

  [Warning: Image ignored] % Unhandled or unsupported graphics:
%\includegraphics[width=13.965cm,height=20.142cm]{ConfFaivreMartintout-img/ConfFaivreMartintout-img4.jpg}
 

On voit bien les mains jointes et dans une attitude de prière.

Le roi étant le vicaire du dieu, il a  être représenté dans un face à
face avec des dieux. Ce personnage, nous ne le connaissons jamais par
son nom de naissance,  En effet quand on regarde ce que veut dire le
nom de ces rois, ce n{\textquotesingle}est pas un prénom, le nom du roi
est en effet lié au contexte d{\textquotesingle}une époque , à sa
demande de protection à l{\textquotesingle}une des divinités, ce sont
donc des noms de couronnement.

Et si on reprenait tous les noms des rois depuis 3000 ans, ce que
l{\textquotesingle}on ne pourrait faire, on verrait peut être quelques
prénoms, mais ce sont surtout des noms de couronnement.

On peut également s{\textquotesingle}interroger pour savoir
s{\textquotesingle}il existe un couronnement, un rituel de sacre, et
dans ce cas à quand il remonte, quel est le plus ancien exemple.  En
réalité, les sources ne sont pas très anciennes, la plus ancienne
attestation connue est la stèle de UR NAMMU, (Philadelphie)

  [Warning: Image ignored] % Unhandled or unsupported graphics:
%\includegraphics[width=6.809cm,height=10.089cm]{ConfFaivreMartintout-img/ConfFaivreMartintout-img5.jpg}
   [Warning: Image ignored] % Unhandled or unsupported graphics:
%\includegraphics[width=7.761cm,height=6.033cm]{ConfFaivreMartintout-img/ConfFaivreMartintout-img6.jpg}
 

stèle UR NAMMU et détail

UR NAMMU est le fondateur de la troisième dynastie d{\textquotesingle}Ur
au 22 siècle avant JC, et c{\textquotesingle}est à cette époque que les
textes parlent d{\textquotesingle}un cérémonie de sacre qui se passait
dans le sanctuaire d{\textquotesingle}ENLIL, (ce qui est normal puisque
ENLIL donne la royauté aux hommes) dans son grand sanctuaire de Nippur
qui s{\textquotesingle}appelle, nous l{\textquotesingle}avons vu , EKUR

On ne sait pas très bien comment cela se passait, mais on sait que le
roi recevait des choses, même si nous n{\textquotesingle}avons aucun
récit exhaustif)

Stèle de Shamash : Bristish Muséum

  [Warning: Image ignored] % Unhandled or unsupported graphics:
%\includegraphics[width=9.202cm,height=7.512cm]{ConfFaivreMartintout-img/ConfFaivreMartintout-img7.jpg}
 

Le Roi Nabû Appla Iddina est introduit par deux divinités protectrices
auprès du dieu Shamash, assis sur son trône

cette stèle date du 9ème siècle avant JC, période où les rois de
Babylone ont beaucoup de mal à se débarrasser des assyriens qui les
dominent, les désignent dans la plupart des cas, et où en réalité il
s{\textquotesingle}agit de pseudo royautés autonomes, sous le contrôle
des assyriens

Cette stèle est extrêmement importante au niveau iconographique car elle
nous montre à droite sous un dais, et sur un trône qui peut être
identifié car il y a en gros l{\textquotesingle}idéogramme de son nom,
Shamas, dieu soleil, et dans les textes de la troisième dynastie
d{\textquotesingle}Ur, nous savons que la cérémonie du sacre se passait
dans le temple d{\textquotesingle}ENLIL . C{\textquotesingle}est donc
le dieu soleil SHAMASH qui remet au roi deux objets : une corde
enroulée sur elle même et un bâton à mesurer, qui sont le symbole de sa
fonction, être garant des normes.

Avant on pensait qu{\textquotesingle}il s{\textquotesingle}agissait
d{\textquotesingle}une corde et d{\textquotesingle}un piquet de
fondation et on y voyait l{\textquotesingle}image du roi arpenteur
prenant les mesures du futur temple; On sait maintenant
qu{\textquotesingle}il s{\textquotesingle}agit d{\textquotesingle}un
bâton à mesurer et en aucun cas  d{\textquotesingle}un spectre.

En réalité il s{\textquotesingle}agit d{\textquotesingle}objet très
concret : la règle sert à mesure

On voit le roi arriver près du dieu , tel un dieu secondaire dans une
tenue dont nous reparlerons  : en effet son costume est différent de
celui des deux autres, qui ont des tuniques, alors que le roi  a
toujours l{\textquotesingle}épaule droite dénudée. On sait
qu{\textquotesingle}en Mésopotamie, la bonne main
d{\textquotesingle}usage est la main droite, la main gauche est celle
avec laquelle on fait les choses sales (on écrit , mon mange, on salue,
on prie de la  main droite). Le salut suprême se fait main droite levée
devant la bouche, c{\textquotesingle}est un signe de respect. Et
d{\textquotesingle}ailleurs ce sont des traditions qui vont se
perpétuer et on en trouve même référence dans la Bible

aussi, cette stèle peut très bien représenter une scène de couronnement

Les textes nous parlent également d{\textquotesingle}Isthar  du jour
(attention il existe Isthar du jour et Isthar de la Nuit , la
différence est qu{\textquotesingle}Isthar de la nuit est représentée
avec des ailes et des chouettes)

  [Warning: Image ignored] % Unhandled or unsupported graphics:
%\includegraphics[width=7.761cm,height=10.336cm]{ConfFaivreMartintout-img/ConfFaivreMartintout-img8.jpg}
 

Isthar de la nuit ( vers - 1800)\ \ 

Isthar est importante à la fin du III et début du II millénaire,
tellement importante d{\textquotesingle}ailleurs que son nom devient un
nom commun pour signifier déesse

Isthar donne au roi son trône, sa thiarre et son spectre 

Et nous avons ainsi la liste de cinq objets que le roi recevait lors de
son couronnement et cela se  perpétuera sans doute après des mises en
scènes différentes en fonction des époques et des royaumes
jusqu{\textquotesingle}à la chute de Babylone

Ceci va conduire l{\textquotesingle}iconographie mésopotamienne à
montrer le lien entre le roi et son dieu par une proximité, mais on
s{\textquotesingle}arrête à cette proximité, il
n{\textquotesingle}existe aucune familiarité entre le roi et son dieu,
car le roi mésopotamien est un homme.

Le roi affirme une action , et la légitime par la volonté du dieu et
c{\textquotesingle}est ce qui se passera durant la période assyrienne,
où on utilise un motif , et là il faut repartir sur la notion du disque
solaire ailé égyptien d{\textquotesingle}Edifou

On sait que ce motif, qui est au départ le disque solaire égyptien, a
voyagé par le monde palestinien, le monde hittite, le monde
mésopotamien et même chez les perses et au premier millénaire, les
assyriens (on ne sait pas comment ils en ont eu
l{\textquotesingle}idée, représentent leur dieu national Assur comme un
buste sortant de l{\textquotesingle}astre solaire. Les babyloniens
représenteront leur dieu Marduk de la même façon et les perses ensuite
quand ils auront vaincu la Mésopotamie se mettront également à
représenter leur dieu suprême de la même façon

\newline
Cela permet pour les assyriens une composition très pratique en image
pour montrer le niveau céleste de la décision et son application 
terrestre par le roi et cette image a du succès

Nous constaterons que lié à la personne royale, les femmes de sa famille
(mère, épouse, fille) sont absentes de l{\textquotesingle}iconographie.
En Mésopotamie, le roi est toujours seul et c{\textquotesingle}est de
façon tout à fait exceptionnelle que l{\textquotesingle}on verra la 
présence de ces femmes.

Et c{\textquotesingle}est intéressant car si on se réfère aux sources
littéraires, on sait que les femmes en Mésopotamie avait un pouvoir
important. Au moins à partir du II millénaire, le palais est entre
leurs mains, elles ont la clés de l{\textquotesingle}entrepôt et même
du palais (à certaines époques , à certains endroits, le roi
n{\textquotesingle}a pas les clés de son palais et doit les demander
aux femmes

Donc leur absence dans l{\textquotesingle}iconographie ne signifie pas
qu{\textquotesingle}elles ont aucun rôle,  Mais on voit que par la
définition même de la royauté en Mésopotamie, elles
n{\textquotesingle}ont aucun rôle à jouer

Y a -t-il eu des femmes rois ?  à part la fameuse Kubada (?),  et ce à
la période dynastique à Ur au moment où la succession par hérédité se
met en place, elle a été roi (on le sait par un texte au British
Muséum), et mais il n{\textquotesingle}y a aucune autre femme roi
attestée

C{\textquotesingle}est pourquoi la stèle d{\textquotesingle}ASSARHADDON
,  avec sa maman (Naqi{\textquotesingle}a) du musée du louvre 

  [Warning: Image ignored] % Unhandled or unsupported graphics:
%\includegraphics[width=6.842cm,height=7.549cm]{ConfFaivreMartintout-img/ConfFaivreMartintout-img9.jpg}
 

est particulièrement intéressante. Sa maman était une déportée
babylonienne, et elle va convaincre son fils Assarhddon (qui est un roi
assyrien qui va conquérir l{\textquotesingle}Egypte)
d{\textquotesingle}avoir une politique favorable aux ennemis
babyloniens. On peut voir qu{\textquotesingle}elle est représentée
comme un ennuque,  Ce petit morceau a été acheté par le Louvre dans les
années 60, car particulièrement intéressant, et très rare, mais peu
intéressant d{\textquotesingle}un point de vue de
l{\textquotesingle}art

Le Roi mésopotamien est le gestionnaire et le défenseur des biens de son
dieu

on voit bien que cela marche dans les deux sens

En effet, et nous le verrons toutes les  premières  images
l{\textquotesingle}associe à ces deux mondes le temple et son troupeau
et son étable,  et on voit bien que étable, temple, troupeau, sac de
grains, tout cela va ensemble et c{\textquotesingle}est dans le même
bâtiment  et que toute l{\textquotesingle}économie passe dans ce fameux
lieu, qui nous appelons temple.

Ce sont des territoires agricoles et ils ont deux obsessions  : que les
champs soient fertiles et que les récoltes soient bonnes, et le
troupeau fécond, ce qui permet de nourrir tout le monde

Donc , on part du principe que c{\textquotesingle}est maintenir la
création des dieux, telle qu{\textquotesingle}elle a été donnée aux
hommes C{\textquotesingle}est à dire que c{\textquotesingle}est partir
du principe que le monde terreste a été fait par les dieux pour que
l{\textquotesingle}humanité y vive et que les hommes doivent la
respecter, et la mettre en valeur, ce qui est une façon de rendre
hommage aux deux

Mais le territoire agricole appartient aux divinités

Prenons un exemple ancien, du IV millénaire, le sceau de Berlin, (que
nous verrons plus tard) , et au XII siècle avant JC :

Kuduri Kassite du Roi Méli Shipak II  Louvre

  [Warning: Image ignored] % Unhandled or unsupported graphics:
%\includegraphics[width=8.25cm,height=12.79cm]{ConfFaivreMartintout-img/ConfFaivreMartintout-img10.jpg}
 

On voit donc le roi devant une divinité qui est peut être shamash et il
salue le dieu. Le texte devant est effacé , mais derrière il existe
encore et il ne concerne que l{\textquotesingle}arpentage, mesure de
terrains, (tel champs qui appartient au domaine agricole du dieu, et
qui mesure ... , a produit ceci ...; et est géré comme cela ...)

Et à partir de là, la destination de la guerre en découle,  en effet à
partir du moment où le roi s{\textquotesingle}affirme comme étant le
vicaire de son dieu, il défend le territoire de son dieu. Et si on
prend la plus ancienne image sculptée de la guerre, stèle des vautours,
Louvre, date vers 2500 avant JC et dynastie archaïque III

stèle des vautours : 

  [Warning: Image ignored] % Unhandled or unsupported graphics:
%\includegraphics[width=15.981cm,height=19.967cm]{ConfFaivreMartintout-img/ConfFaivreMartintout-img11.jpg}
 

Commentaire  (hors conférence, site du Louvre)

Partiellement reconstituée à partir de plusieurs fragments trouvés dans
les vestiges de la cité sumérienne de Girsu, cette stèle de victoire
constitue le plus ancien document historiographique connu. Une longue
inscription en langue sumérienne fait le récit du conflit récurrent qui
opposait les cités-États voisines de Lagash et Umma, puis de la
victoire d{\textquotesingle}Eannatum, roi de Lagash. Son triomphe est
illustré avec un luxe de détails par le remarquable décor en bas-relief
qui couvre les deux faces.

Un document historique exceptionnel

Malgré sa conservation lacunaire, cette stèle de grande taille, sculptée
et inscrite sur ses deux faces, est un monument d{\textquotesingle}une
valeur incomparable puisqu{\textquotesingle}il s{\textquotesingle}agit
du plus ancien document historiographique connu. Les fouilles du site
de Tello permirent d{\textquotesingle}en retrouver plusieurs fragments
disséminés parmi les vestiges de l{\textquotesingle}ancienne cité
sumérienne de Girsu. Cette stèle commémore, par le texte et
l{\textquotesingle}image, une importante victoire remportée par le roi
de Lagash, Eannatum, sur la cité voisine d{\textquotesingle}Umma. Les
deux villes entretenaient en effet un état de guerre récurrent à propos
de la délimitation de leur frontière commune, à
l{\textquotesingle}image de ce que pouvaient être les relations entre
cités-États à l{\textquotesingle}époque des dynasties
archaïques.[2028?]Petit-fils d{\textquotesingle}Ur-Nanshe et fondateur
de la Ière dynastie de Lagash, Eannatum régna vers 2450 av. J.-C. et
conduisit sa cité-État à l{\textquotesingle}apogée de sa puissance.
L{\textquotesingle}inscription gravée sur La Stèle des vautours,
d{\textquotesingle}une ampleur remarquable bien qu{\textquotesingle}il
n{\textquotesingle}en subsiste qu{\textquotesingle}une petite moitié,
exalte les triomphes d{\textquotesingle}un souverain placé dès sa
naissance sous la protection divine. Nourri au lait de la déesse
Ninhursag et tenant son nom de la déesse Inanna, c{\textquotesingle}est
du dieu Ningirsu lui-même qu{\textquotesingle}il reçut la royauté de
Lagash. Assuré du soutien des divinités par un songe prophétique,
Eannatum va s{\textquotesingle}engager avec fermeté dans la lutte
contre Umma afin d{\textquotesingle}imposer son contrôle sur le
Gu-edina, territoire frontalier enjeu de la rivalité entre les deux
cités.[2028?]{\textquotedbl}Moi Eannatum, le puissant,
l{\textquotesingle}appelé de Ningirsu, au pays [ennemi], avec colère,
ce [qui fut] de tout temps, je le proclame ! Le prince
d{\textquotesingle}Umma, chaque fois qu{\textquotesingle}avec ses
troupes il aura mangé le Gu-edina, le domaine bien-aimé de Ningirsu,
que [celui-ci] l{\textquotesingle}abatte !{\textquotedbl}.

La face {\textquotedbl}historique{\textquotedbl}

La narration de la campagne militaire contre Umma est illustrée de
manière spectaculaire par des représentations figurées, sculptées dans
le champ de la stèle selon une disposition traditionnelle en registres.
Elles offrent ici la particularité d{\textquotesingle}être réparties
sur chacune des deux faces en fonction de leur perspective symbolique.
L{\textquotesingle}une des faces est ainsi consacrée à la dimension
{\textquotedbl}historique{\textquotedbl} et l{\textquotesingle}autre à
la dimension {\textquotedbl}mythologique{\textquotedbl}, la première
rendant compte de l{\textquotesingle}action des hommes et la seconde de
l{\textquotesingle}intervention des dieux. Détermination humaine et
protection divine se conjuguent ainsi pour conduire à la
victoire.[2028?]La face dite {\textquotedbl}historique{\textquotedbl}
montre, au registre supérieur, le souverain de Lagash marchant à la
tête de son armée. Eannatum est vêtu de la jupe à mèches laineuses
appelée kaunakès, recouverte partiellement par une tunique en laine
passant sur l{\textquotesingle}épaule gauche. Il porte le casque à
chignon, apanage des hauts personnages. Les soldats, casqués eux aussi
et armés de longues piques, s{\textquotesingle}avancent en formation
serrée, se protégeant mutuellement derrière de hauts boucliers
rectangulaires. L{\textquotesingle}armée de Lagash triomphante piétine
les cadavres des ennemis qu{\textquotesingle}une nuée de vautours a
commencé à déchiqueter, scène dont la stèle tire son nom.
L{\textquotesingle}inscription proclame :[2028?]{\textquotedbl}Eannatum
frappa Umma. Il eut vite dénombré 3 600 cadavres [...]. Moi Eannatum,
comme un mauvais vent d{\textquotesingle}orage, je déchaînai la tempête
!{\textquotedbl}.[2028?]Au deuxième registre est représenté ce qui
semble constituer le défilé de la victoire. Les soldats marchent
alignés sur deux colonnes derrière leur souverain monté sur un char.
Ils tiennent leur pique relevée et la hache de guerre à
l{\textquotesingle}épaule. Eannatum brandit lui aussi une longue pique
ainsi qu{\textquotesingle}une harpé à lame courbe, une arme
d{\textquotesingle}apparat. Il se tient debout sur un char à quatre
roues pourvu d{\textquotesingle}un haut tablier frontal duquel émergent
des javelots rangés dans un carquois.

Le troisième registre, très fragmentaire, illustre les cérémonies
funéraires qui viennent clôturer l{\textquotesingle}engagement
militaire. Pour ensevelir les cadavres amoncelés de leurs camarades,
les soldats de Lagash gravissent une échelle en portant sur la tête un
panier rempli de terre. Des animaux, dont un taureau couché sur le dos
et ligoté, sont prêts à être immolés tandis que l{\textquotesingle}on
accomplit une libation au-dessus de grands vases porteurs de rameaux
végétaux.

La face {\textquotedbl}mythologique{\textquotedbl}

La face dite {\textquotedbl}mythologique{\textquotedbl} illustre
l{\textquotesingle}intervention divine qui offre la victoire à
Eannatum. Elle est dominée par la figure imposante du dieu Ningirsu,
protecteur de la cité-État de Lagash. Celui-ci tient les troupes
ennemies emprisonnées pêle-mêle dans un gigantesque filet et les frappe
de sa masse d{\textquotesingle}armes. Instrument de combat par
excellence du dieu, le filet est tenu fermé par
l{\textquotesingle}emblème d{\textquotesingle}Imdugud,
l{\textquotesingle}aigle à tête de lion, attribut de Ningirsu, qui est
représenté les ailes déployées et agrippant deux lions dans ses serres.

Le reste de la face {\textquotedbl}mythologique{\textquotedbl}, très
lacunaire, semble évoquer la présence aux côtés du dieu triomphant
d{\textquotesingle}une déesse, sans doute Nanshe,
l{\textquotesingle}épouse de Ningirsu, également associée à
l{\textquotesingle}aigle léontocéphale. Le registre inférieur laisse
entrevoir le dieu sur un char, en compagnie de la même
déesse.[2028?]L{\textquotesingle}inscription, après avoir glorifié
l{\textquotesingle}action victorieuse d{\textquotesingle}Eannatum, fait
une large place aux serments prêtés par les deux souverains devant les
grandes divinités du panthéon. Ayant réintégré le Gu-edina au sein du
territoire de Lagash, Eannatum délimite avec Umma la frontière, sur
laquelle est érigée une stèle. Mais la réussite du projet humain ne
peut s{\textquotesingle}accomplir que par faveur divine ;
c{\textquotesingle}est donc elle qui est invoquée afin de garantir la
pérennité du nouvel ordre des choses : {\textquotedbl}Que jamais
l{\textquotesingle}homme d{\textquotesingle}Umma ne franchisse la
frontière de Ningirsu ! Qu{\textquotesingle}il n{\textquotesingle}en
altère pas le talus et le fossé ! Qu{\textquotesingle}il
n{\textquotesingle}en déplace pas la stèle ! S{\textquotesingle}il
franchissait la frontière, que le grand filet d{\textquotesingle}Enlil,
le roi du ciel et de la terre, par lequel il a prêté serment,
s{\textquotesingle}abatte sur Umma !{\textquotedbl}.

On voit donc que c{\textquotesingle}est très intéressant ,
d{\textquotesingle}une part cette stèle comporte le plus long texte en
langue sumérien archaïque, et dit, que si le roi représenté ici est
parti en guerre sur son char, contre le petit royaume voisin, 
c{\textquotesingle}set car Umma n{\textquotesingle}a pas respecté la
frontière, c{\textquotesingle}est à dire un canal
d{\textquotesingle}irrigation Umma et sa troupe ont franchi ce canal,
et sont donc venus sur un territoire qui n{\textquotesingle}était pas
le leur, territoire qui appartenait au dieu Ennatum

L{\textquotesingle}idée est donc que l{\textquotesingle}on ne peut
tolérer l{\textquotesingle}idée que l{\textquotesingle}homme
d{\textquotesingle}Umma (façon d{\textquotesingle}appeler le roi
ennemi) vienne revendiquer un terrain 

Au départ c{\textquotesingle}est une guerre céleste,
c{\textquotesingle}est le dieu SHAMAH dans le ciel qui revendique un
terrain qui appartient à un autre dieu, mais cette guerre sera vécue à
un niveau humain. et ceci permet de légitimer tout ce que
l{\textquotesingle}on veut, car on considère que la guerre est une
volonté de la divinité (on le reverra) et cela permettra de mettre en
place tout un discours qui révèle que si l{\textquotesingle}ennemi a
perdu, et cela va même au delà car le texte dit que celui qui gagne est
béni du dieu, et donc le béni du dieu est l{\textquotesingle}homme pur
victorieux et l{\textquotesingle}homme qui a perdu la guerre est un
impie, et donc on le met tout nu on lui saute dessus, on lui fait en
réalité ce que l{\textquotesingle}on veut. 

Et ce même langage appartient dans les anales assyriennes. En effet
quelque soit le conflit, les assyriens entre le IX et le VII siècle
parlent des autres comme cela. Un roi assyrien fait la guerre au nom de
son dieu, donc d{\textquotesingle}Assur, et ils ont toujours ce côté
dépréciatif les uns par rapport aux autres, (ce lâche qui fuit devant
nous) et ils ont la certitude d{\textquotesingle}être redoutables, car
ils font la guerre au nom d{\textquotesingle}Assur et donc ils ne
peuvent qu{\textquotesingle}être victorieux , et ils peuvent massacrer
leurs ennemis

Le roi est un mortel sans destinée particulière : 

Cette conception est embêtante pour nous, car cela veut dire
qu{\textquotesingle}il n{\textquotesingle}y a pas
d{\textquotesingle}art funéraire.

C{\textquotesingle}est un mortel, donc on a un homme à la tête des
autres hommes, même si après et parfois il y aura des mises en scène
pour le magnifier,  

Mais les dieux ne le récompense pas pour ses hauts faits en lui donnant
quelque chose de mieux que les autres; il va comme
n{\textquotesingle}importe quel autre humain après sa mort dans le
monde des enfers, et en Mésopotamie la vision de l{\textquotesingle}au
delà n{\textquotesingle}est pas très drôle

Aussi dans cette logique, il est normal de ne pas rechercher de
magnifiques tombes royales mésopotamiennes

Il est évident qu{\textquotesingle}ils ont dû emmener des choses avec
eux,  des objets précieux, qu{\textquotesingle}ils avaient un belle
tombe, mais qu{\textquotesingle}il n{\textquotesingle}existait pas un
art architecturé de la tombe, il n{\textquotesingle}y a aucune
construction élaborée

En 3000 ans,  il n{\textquotesingle}y a  à ce jour
qu{\textquotesingle}un seul exemple , ce sont les tombes royales
d{\textquotesingle}Ur. C{\textquotesingle}est le seul moment et
seulement huit tombes identifiées comme étant des tombes royales, où
l{\textquotesingle}on voit qu{\textquotesingle}il y avait deux grandes
structures à plusieurs pièces, avec des sacrifices humains et une mise
en scène d{\textquotesingle}un aménagement funéraire

 L{\textquotesingle} EGYPTE

Nous avons là une unité géographique et une unification qui
s{\textquotesingle}est faite entre le IV et le III millénaire, avec des
récits de la création de la royauté qui elle aussi
s{\textquotesingle}appuie sur un monde divin

Et même s{\textquotesingle}il est évident que des traditions plus
anciennes ont existé, la plus ancienne source résulte des Textes des
Pyramides, car dans ces textes, le roi est décrit comme arrivant du
ciel, (intéressant, car en Mésopotamie, la royauté est tombé du ciel).
Mais le roi égyptien arrive sur terre , après avoir mis MAAT à la place
de l{\textquotesingle}Isefet, dans l{\textquotesingle}ile des flammes
(et dans les textes des pyramides, lorsque l{\textquotesingle}on parle
de l{\textquotesingle}ile des flammes, c{\textquotesingle}est la terre
dans son état rudimentaire, sauvage, rustique

Et  à partir du moment où le roi a mis MAAT à la place de
l{\textquotesingle}Isefet, ce geste fait que la terre devient habitable

la principal source qui rattache le roi d{\textquotesingle}Egypte à son
créateur, le soleil, c{\textquotesingle}est l{\textquotesingle}hymne
qui évoque l{\textquotesingle}adoration matinale du soleil, cet hymne
constitué de 44 vers, et qui découle de cette pensée de
l{\textquotesingle}ancien empire , exprimée dans les textes des
pyramides, qu{\textquotesingle}il doit y avoir une origine
rédactionnelle au Moyen Empire , mais que l{\textquotesingle}on connaît
par des copies, notamment celle de deir el bahari

Rê a installé le roi sur la terre des vivants,  jamais et à toute
éternité de sorte qu{\textquotesingle}il juge les hommes, et satisfasse
les dieux, qu{\textquotesingle}il réalise MAAT et anéantisse
l{\textquotesingle}Isefet, donne des sacrifices aux dieux et des
offrandes funéraires aux morts immortalisés

Donc là le point commun , c{\textquotesingle}est que la royauté est un
cadeau aux hommes : le créateur donne aux hommes le roi pour la terre
des vivants, pour l{\textquotesingle}éternité

Le roi d{\textquotesingle}Egypte doit donc juger les hommes et
satisfaire les dieux; On voit donc dans le fait que le roi juge les
hommes, il y a cette notion d{\textquotesingle}ordre qui existe
également en Mésopotamie, l{\textquotesingle}obligation de satisfaire
les dieux,  il a par contre un lien privilégié avec les dieux et une
obligation qui est une spécificité égyptienne :
l{\textquotesingle}opposition entre Maât et l{\textquotesingle}Isefet

Effectivement, en égypte, on sait que l{\textquotesingle}on part là
encore sur le principe d{\textquotesingle}une royauté héréditaire (par
les male de la famille), et il y a la possibilité pour une personne qui
n{\textquotesingle}est pas de la lignée royale de prendre le pouvoir, 
en effet un homme de mérite peut devenir roi, cela est attesté avec bon
nombres de rois, qui ont été appelés usurpateurs, et finalement leur
légitimité est accordée par l{\textquotesingle}acte de couronnement

Mais, la différence fondamentale avec le roi mésopotamien,
c{\textquotesingle}est que l{\textquotesingle}homme qui subit le rituel
du couronnement en Egypte, devient un netchernéfer,
c{\textquotesingle}est à dire un  être qui contient en lui une parcelle
divine et c{\textquotesingle}est le grand bénéficie du couronnement.

Il va recevoir aussi des insignes régaliens, mais à partir de ce moment
là, il peut toucher le monde des dieux, et avoir en lui une partie
d{\textquotesingle}essence surnaturelle.

Exemple THOT qui inscrit le nom du roi KALABASHA

  [Warning: Image ignored] % Unhandled or unsupported graphics:
%\includegraphics[width=15.981cm,height=11.395cm]{ConfFaivreMartintout-img/ConfFaivreMartintout-img12.jpg}
 

(rite de purification pendant le couronnement et Thot écrit non pas le
nom du pharaon mais le nom  per aa

A partir du couronnement, peut s{\textquotesingle}exprimer cet aspect
surnaturel de la personne royale, qui s{\textquotesingle}exprimera de
façon différente en fonction des époques. 

On sent très bien la magie des insignes régaliens donnés également au
roi lors de son couronnement

A l{\textquotesingle}ancien empire, nous verrons des représentations où
le roi est en contact direct avec les dieux, 

mais contrairement au monde mésopotamien, où l{\textquotesingle}on voit
le roi humain agir pour la divinité, et éventuellement que la divinité
lui donne quelque chose, il y a beaucoup plus d{\textquotesingle}aller
et retour dans le monde égyptien.

On a en effet des scènes de deux divinités qui entourent le roi (et on
peut imaginer deux autres divinités, de l{\textquotesingle}autre côte
et on fait la boucle avec les quatre points cardinaux) et on est dans
la logique égyptienne, des rituels de protection, pour purifier le roi.
Cela est inenvisageable en Mésopotamie, par le fait que le roi
n{\textquotesingle}a rien de divin

La notion de royauté liée  l{\textquotesingle}hérédité,
c{\textquotesingle}est par les sources écrites que nous la connaissons,
(Pierre de Palerme) et c{\textquotesingle}est la légitimation par le
simple fait que l{\textquotesingle}on est fils de 

Le roi en Egypte est mis en image dès le IV millénaire, et il y a la
mise en place d{\textquotesingle}une iconographie royale, donc avec
ceux d{\textquotesingle}avant l{\textquotesingle}écriture, ceux sans
nom

En egypte  il n{\textquotesingle}y a qu{\textquotesingle}un seul roi, et
un seul royaume, (sauf durant les périodes intermédiaires), il y a donc
eu l{\textquotesingle}unification politique et il en découle un seul
roi

l{\textquotesingle}Egypte aussi est un pays protégé naturellement par le
désert, alors que la Mésopotamie est une plaine qui de tout temps a été
traversée par des populations différentes

EN EGYPTE : LE ROI EST :

{}- fils de dieu sur terre,

{}-garant de l{\textquotesingle}ordre de MAAT

{}- un dieu parfait de son vivant, et un dieu après sa mort

fils de dieu sur terre : 

Il est intéressant de voir que très tôt dans le temps, nous allons avoir
des faces à faces entre Pharaon et les dieux, il y a vraiment un lien
charnel, physique qui va être affirmé.

  [Warning: Image ignored] % Unhandled or unsupported graphics:
%\includegraphics[width=5.821cm,height=6.668cm]{ConfFaivreMartintout-img/ConfFaivreMartintout-img13.jpg}
 

  [Warning: Image ignored] % Unhandled or unsupported graphics:
%\includegraphics[width=0.529cm,height=0.388cm]{ConfFaivreMartintout-img/ConfFaivreMartintout-img14.png}
 

Niouserrê recevant la vie du dieu Anubis - Relevé du temple funéraire du
roi à Abousir

  [Warning: Image ignored] % Unhandled or unsupported graphics:
%\includegraphics[width=15.981cm,height=23.883cm]{ConfFaivreMartintout-img/ConfFaivreMartintout-img15.jpg}
 

Et on voit  sur le relief du roi Niouserré à Berlin, une déesse lionne
(on discute encore pour connaître son nom) et elle tient son sein pour
nourrir le roi, comme si elle était sa mère terrestre. Et  à ce jour
c{\textquotesingle}est la plus ancienne scène
d{\textquotesingle}allaitement d{\textquotesingle}un roi humain par une
divinité que l{\textquotesingle}on connaisse. Niouserrê étant un roi de
la V dynastie

Cette image est toujours présente dans la XVIII dynastie, où la
littérature égyptienne développe le fait qu{\textquotesingle}à partir
de la 18ème dynastie l{\textquotesingle}essence divine
s{\textquotesingle}affirme , en disant {\textquotedbl} il est le fils
du dieu sur terre, il est l{\textquotesingle}enfant terrestre des dieux
et  à partir de ce moment là on a la triade divine céleste, (père mère
enfant),  et le dieu et la déesse ont donc un enfant sur terre : le
pharaon régnant

Et sur la rive Ouest de Thèbes consacrée à Hathor,  on le met en image,
roi debout devant Hathor :  [Warning: Image ignored]
% Unhandled or unsupported graphics:
%\includegraphics[width=15.981cm,height=11.994cm]{ConfFaivreMartintout-img/ConfFaivreMartintout-img16.jpg}
 

 et cette image sera reprise

On peut même voir le souverain téter le pie de la vache

et cela va s{\textquotesingle}exprimer d{\textquotesingle}une autre
façon car nous aurons la représentation qui existe dès la 18ème
dynastie du souverain encadré par son père et sa mère céleste,
c{\textquotesingle}est à dire que là le prend véritablement le rôle de
l{\textquotesingle}enfant divin et au lieu d{\textquotesingle}avoir le
dieu fils de Ptah et Sekmet on a carrément Ramsès II entre Ptah et
Seckmet

  [Warning: Image ignored] % Unhandled or unsupported graphics:
%\includegraphics[width=15.981cm,height=16.374cm]{ConfFaivreMartintout-img/ConfFaivreMartintout-img17.jpg}
 

Et ce lien conduit dans l{\textquotesingle}art dès
l{\textquotesingle}ancien empire, à des représentations où le dieu
touche le roi et là effectivement ceci est inenvisageable dans le monde
mésopotamien, (on peut voir parfois le roi être touché par un dieu,
mais il s{\textquotesingle}agit d{\textquotesingle}un dieu inférieur),
on le sait car les sumériens ont établi des listes de dieux , et ce par
ordre d{\textquotesingle}importance et il existe des dieux de rang
inférieur, divinité d{\textquotesingle}ordre secondaire (un peu comme
le dieu personnel du roi qui est une sorte d{\textquotesingle}ange
gardien, qui peut lui servir d{\textquotesingle}intercesseur auprès des
divinités suprêmes et peut alors le toucher , le conduire par la main
auprès du dieu supérieur

Mais dans le monde mésopotamien, nous ne verrons jamais un dieu
supérieur toucher le roi comme s{\textquotesingle}il était son égal

Et d{\textquotesingle}ailleurs dans la proportion des tailles, la
divinité est toujours plus grande en Mésopotamie, et
l{\textquotesingle}humain plus petit  (cf photo vu en début du cours
avec le Roi Shamash énorme assis sous son dais, et le roi en tout
petit,  alors que cela aurait dû être l{\textquotesingle}inverse
puisque le roi est debout et donc en toute logique aurait dû être plus
grand)

En Egypte au contraire, on observe des faces à faces entre le dieu et le
pharaon et ils ont la même taille et cela est fondamental 

Le roi et la divinité ont la même taille, la même couleur , donc la même
essence, ils se touchent et c{\textquotesingle}est intéressant au
niveau de la gestuel, 

On utilise donc l{\textquotesingle}image du roi régnant pour montrer
l{\textquotesingle}image d{\textquotesingle}un dieu antropomorphe
contemporain

et à la même époque de Thoutmosis III , on va jouer sur des similitudes
au niveau des des coiffures, car le toi aura à partir de Thoutmosis III
des coiffures de plus en plus exubérantes, avec de plus en plus de
plumes, qui ressemblent beaucoup à la coiffure divine

Le lien  profond qui existe entre le roi et les dieux vient du fait que
les dieux sont là en permanence pour lui insuffler la vie, et ce
souffle de vie, il ne le reçoit pas uniquement pour lui, mais pour
toute l{\textquotesingle}humanité qu{\textquotesingle}il représente, et
nous verrons bien que c{\textquotesingle}est tout à fait
caractéristique de la mise en image égyptienne, de
l{\textquotesingle}iconographie égyptienne, on ne représente pas le
dieu en train de faire bénéficier ce souffle de vie à  la population
d{\textquotesingle}Egypte. L{\textquotesingle}humain de référence ,
l{\textquotesingle}humain de contact est pharaon, 

Il y a donc une hiérarchie et le roi recevant ce souffle sera garant de
la MAAT

L{\textquotesingle}ordre doit régner sur terre pour que le monde divin
soit stable et si l{\textquotesingle}ordre ne règne pas sur terre cela
met en péril le monde divin

exemple Hathor accueille Séthi 1er (Louvre)

  [Warning: Image ignored] % Unhandled or unsupported graphics:
%\includegraphics[width=8.431cm,height=18.062cm]{ConfFaivreMartintout-img/ConfFaivreMartintout-img18.jpg}
 

Et on voit que leurs mains se tiennent et que quelque chose est donné

et encore une fois, cette thématique ne peut être envisagée dans le
monde mésopotamien ou si le roi lui donne quelque chose , la divinité
pourra lui donner également quelque chose,  mais jamais la divinité ne
lui donnerait quelque chose sans avoir reçu quelque chose avant, sauf
le jour du couronnement où le roi reçoit ses insigne

IL faut aussi évoquer le fait que pharaon n{\textquotesingle}est pas
uniquement le fils du dieu sur terre, mais un dieu sur terre  et cela
s{\textquotesingle}exprimera intensément durant la période aramarnienne


exemple tombe de Meryre

Cette période est d{\textquotesingle}ailleurs assez intéressante, elle
ne dure que 14 ans sur les 3000 ans de l{\textquotesingle}histoire
égyptienne, mais elle va changer et chambouler beaucoup de chose

pendentif de Shed Louvre  C{\textquotesingle}est à cette époque
qu{\textquotesingle}apparait l{\textquotesingle}iconographie du dieu
SHED, tueur de bête sauvage, et c{\textquotesingle}est un enfant royal,
il est représenté avec la tresse de l{\textquotesingle}enfant, 

c{\textquotesingle}est à la fois un enfant humain et un enfant divin, on
voit donc que les choses sont floues, et ils tue les animaux du désert
en tuant les crocodiles

on voit donc que la frontière entre le monde humain royal et le monde
divin est ténu

stèle de l{\textquotesingle}enfant Horus et les Crocodiles

  [Warning: Image ignored] % Unhandled or unsupported graphics:
%\includegraphics[width=9.202cm,height=16.485cm]{ConfFaivreMartintout-img/ConfFaivreMartintout-img19.jpg}
 

un roi en enfant solaire, alors qu{\textquotesingle}il est adulte

On est là à une relecture durant la période de Ramssès de Shed qui est
une divinité tout à fait solaire, et le passage à partir de la 3ème
période intermédiaire , l{\textquotesingle}image de SHED va
disparaître, car elle est liée à un principe de souveraineté,  et il y
a le développement des stèles d{\textquotesingle}Horus sur les
crocodiles et Horus est représenté en enfant

Et c{\textquotesingle}est intéressant de voir comment une image royale,
mise en place à un certain moment pour dire quelque chose va évoluer et
ne sera conservée pour être acceptable que sous la forme
d{\textquotesingle}une image divine ; car on est revenu à une
orthodoxie religieuse un peu calmée par rapport à la période
amarnienne.

MAAT et le principe de souveraineté :

Cela signifie que le roi d{\textquotesingle}Egypte est garant de MAAT et
c{\textquotesingle}est fondamental, car MAAT c{\textquotesingle}est la
volonté des dieux, le roi est leur intermédiaire et il se charge de la
faire respecter, les hommes doivent lui obéir

Et les textes disent que 

RE se nourrit chaque jour de la MAAT, cela veut dire que si on ne la
respecte pas , c{\textquotesingle}est le créateur lui même qui est en
danger

Le roi garant de MAAT, il doit donc lui faire des offrandes chaque jour
dans le temple, mais cela va plus loin , car être garant de MAAT
c{\textquotesingle}est être garant de tout et ce tout va de la
construction du temple ,à son fonctionnement, son organisation
intérieure, veiller au bon fonctionnement du culte, et ce
jusqu{\textquotesingle}au fonctionnement du monde paysan, pour que la
boucle soit complète

Linteau de Medamoud (Louvre) XII dynastie

  [Warning: Image ignored] % Unhandled or unsupported graphics:
%\includegraphics[width=15.981cm,height=7.126cm]{ConfFaivreMartintout-img/ConfFaivreMartintout-img20.jpg}
 

On peut voit que Sésostris III est représenté deux fois 

et elle correspond à la plus ancienne scène de culte journalier que nous
connaissons : Sésostris III fait offrande de pain blanc et de gâteau

On reverra d{\textquotesingle}autres scènes d{\textquotesingle}offrandes
dans les temples et tout ceci insiste sur la garantie de la MAAT et que
l{\textquotesingle}Isfet ne revienne pas et cela se résume notamment à
la représentation du roi à genou présentant les vases globulaires. On
sait très bien que cette image apparaît à la VI dynastie  : le roi
offrant , cela va au delà du roi à genou présentant des vases
(globulaires, à vins ..) c{\textquotesingle}est le roi en offrande et
cette représentation existe à toutes les époques

Siamon roi en Sphinx (Louvre-

   [Warning: Image ignored] % Unhandled or unsupported graphics:
%\includegraphics[width=15.981cm,height=15.558cm]{ConfFaivreMartintout-img/ConfFaivreMartintout-img21.jpg}
 

On voit les offrandes faites au dieu

il y a donc des éléments de décor aussi sur le mobilier du temple, et
également en relief sur les murs

Le roi a une destinée post mortem

Si en Egypte tout le monde a une destinée après la mort, pour le roi
c{\textquotesingle}est un peu spécifique, il a une vie éternelle, mais
il rejoint le monde des dieux, et ce dès les textes des pyramides

Dans les tombes du Nouvel Empire, on voit des scènes où le roi est face
au dieu (on reste dans la logique de la thématique de 
l{\textquotesingle}offrande), mais il est dans la phase de transition
où il va vers cet autre monde et où Am Douat lui ouvre les portes et
l{\textquotesingle}iconographie est spécifique, on nous représente le
voyage des 12 heures de la nuit du soleil car on associe la destinée
post mortem du roi à ce que vit le soleil chaque nuit

Le soleil meurt le soir et renaît le matin, et ce que le soleil de la
nuit vit chaque nuit, le roi vit cela après sa mort

On sait aussi que c{\textquotesingle}est une évocation à mettre en
parallèle avec les rites de l{\textquotesingle}embaumement

Cet aspect divin de la personne du roi conduit à une imagerie funéraire
tout à fait spécifique en Egypte, et très importante et on construit en
pierre (en Mésopotamie, c{\textquotesingle}est en brique), 

Il y a non seulement le décor du temple, mais il y a également le décor
de la tombe, avec cette destinée funéraire spécifique, alors que le roi
mésopotamien , nous avons forcément un corpus d{\textquotesingle}images
moins importants, partant d{\textquotesingle}une architecture de
briques, qui certes  a reçu un décor peint,  (il est évident que dès le
III millénaire en Mésopotamie il y a eu des décors peints, mais ils
sont perdus car ils ont été peints à même un enduit et la présence des
deux fleuves ne facilite pas non plus la conservation des choses)

Mais il ne faudrait surtout pas penser que seuls les assyriens ont fait
des décors muraux avec leur dalle de gypses sculptées (en réalité on
sait qu{\textquotesingle}ils ont repris une ancienne tradition de
peinture

Durant tout ce cours, nous aurons forcément un corpus
d{\textquotesingle}images moindre pour la Mésopotamie que pour
l{\textquotesingle}Egypte

autre photo stèle des vautours

  [Warning: Image ignored] % Unhandled or unsupported graphics:
%\includegraphics[width=15.981cm,height=8.396cm]{ConfFaivreMartintout-img/ConfFaivreMartintout-img22.jpg}
   [Warning: Image ignored] % Unhandled or unsupported graphics:
%\includegraphics[width=15.981cm,height=9.878cm]{ConfFaivreMartintout-img/ConfFaivreMartintout-img23.jpg}
 

Conférence N° 2

MISE EN PLACE DE l{\textquotesingle}ICONOGRAPHIE ROYALE AU IV MILLENAIRE

MESOPOTAMIE

aujourd{\textquotesingle}hui, nous allons glisser sur des oeufs entre la
période d{\textquotesingle}Uruk et celle de Nagada sachant que nous
sommes sur de la chronologie pour la période d{\textquotesingle}Uruk et
pour la période de Nagada

Mais le problème principal viendra non pas de l{\textquotesingle}Egypte,
où nous connaissons bien le passage du IV au III millénaire, (sauf pour
les rois de la dynastie 0), mais de la Mésopotamie.

Nous avons en effet un problème pour les objets qui viennent du Sud de
la Mésopotamie, car beaucoup d{\textquotesingle}objets qui sont dans
les musées proviennent du marché de l{\textquotesingle}art en général
et cela pose un problème, car ce sont des datations qui sont proposées
par une approche stylistique , mais au Proche Orient, il
n{\textquotesingle}y a pas d{\textquotesingle}historien
d{\textquotesingle}art, on est archéologue et l{\textquotesingle}objet
n{\textquotesingle}est utilisé que pour le texte qu{\textquotesingle}il
porte, ou pour faire une présentation, présentation par rapport à la
civilisation sociologique ...

Donc, avoir un discours de l{\textquotesingle}art sur la discipline du
Proche Orient, est un peu difficile, car les professionnels de cette
région n{\textquotesingle}ont pas cette approche, et donc on a parfois
des publications assez incroyables , les archéologues
s{\textquotesingle}intéressent essentiellement à la couche de
stratification de l{\textquotesingle}objet trouvé

Et il n{\textquotesingle}y a pas d{\textquotesingle}article ou de
publication récente ou approfondie sur l{\textquotesingle}étude d
{\textquotesingle}un objet en le comparant à d{\textquotesingle}autres
et cela est une particularité de cette discipline

Les objets que nous verrons proviennent du Sud de
l{\textquotesingle}Iraq et de la Haute Egypte, il y a donc une zone de
désert que sépare ces deux régions. Pourtant certaines images venant de
ces deux pays devront être mises en parallèle

Chronologie période Uruk : 

Uruk Ancien 4300 - 3800

Uruk Moyen 3800 - 3400

Uruk récent  3400 - 3100

Uruk final  3100 - 2900 (ou Djemdet Nasr)

On constate une chose, qui est importante dans le monde mésopotamien,
dans la deuxième période de néolithique, entre le VI et le IV
millénaire, c{\textquotesingle}est l{\textquotesingle}apparition des
chefferies et c{\textquotesingle}est cette organisation en chefferies
qui petit à petit va nous conduire à la royauté.

le IV millénaire a reçu le nom de période d{\textquotesingle}Uruk,  à
partir donc d{\textquotesingle}un site archéologique. En effet, Uruk
est le site de référence car c{\textquotesingle}est le seul site à ce
jour qui donne le passage entre la culture précédente dire Obeid ,
(néolithique Sud de l{\textquotesingle}Iraq) et la période de
l{\textquotesingle}apparition de l{\textquotesingle}urbanisation que
l{\textquotesingle}on appelle Proto  urbaine et c{\textquotesingle}est
cette époque proto urbaine qui correspond au IV millénaire et est
contemporaine de Nagada en Egypte et a reçu le nom
d{\textquotesingle}Uruk.

Cette période est donc divisée nous venons de le voir en séquence
chronologique

Attention, si on se réfère à des ouvrages anciens , le découpage
n{\textquotesingle}est pas le même, on le faisait démarrer un peu plus
tard et on le divisait en trois phases . Mais maintenant on a affiné
les choses et l{\textquotesingle}Uruk ancien et l{\textquotesingle}Uruk
 moyen correspondent à deux périodes dont on ne connaît pas grande
chose.

mais au milieu du IV millénaire , nous avons un peu plus
d{\textquotesingle}information et on fait maintenant la distinction
entre Uruk récent et Uruk final,

en effet jusqu{\textquotesingle}à une époque récente on disait Uruk
récent : 3300- 2900, mais l{\textquotesingle}étude de la stratification
des couches nous permet de montrer qu{\textquotesingle}il y a eu une
évolution durant cette période , aux alentours de 3000 et on a décidé
de distinguer un Uruk récent 3400 -3100 et un Uruk Final 3100 - 2900 ,
également appelé Djemdbet Nsar, 

C{\textquotesingle}est effectivement grâce à des fouilles allemandes,
scientifiques, dans les années 1930 , qu{\textquotesingle} il a été
établi une stratification en 18 nivaux archéologiques et si on part de
publications d{\textquotesingle}après la seconde guerre mondiale, on
voit un affinage des choses et que si Uruk ancien et Uruk moyen restent
toujours mal connus, on a pu dans les années 80 distinguer pour la
dernière période un Uruk récent et un Uruk final

On voit donc le découpage de la période à partir des 18 couches
archéologiques relevées par les archéologues et c{\textquotesingle}est 
ainsi que l{\textquotesingle}on arrive à ce découpage, la couche la
plus profonde  Uruk 18 correspondant à la période la plus ancienne.

Et les couches 6 à 4 correspondent à Uruk récent  3500 - 3100

couches 3 à 1 correspondant à Uruk final 3100 - 2900 

Nous allons donc nous intéresser à ce qui caractérise cet Uruk récent ,
niveau 6 à 4, car c{\textquotesingle}est durant cette période que deux
choses apparaissent :  l{\textquotesingle}écriture et
l{\textquotesingle}apparition de la royauté.

c{\textquotesingle}est un période , globalement extrêmement inventive,
on a non seulement l{\textquotesingle}attestation de
l{\textquotesingle}urbanisme, mais c{\textquotesingle}est une période
où l{\textquotesingle}on voit apparaître beaucoup
d{\textquotesingle}inventions (roue, tour de potier ...) et tout ceci
révèle une société hiérarchisée, et des villes qui sont dirigées par
une élite sociale.

On peut aussi établir l{\textquotesingle}extension de cette culture
d{\textquotesingle}Uruk, partie du sud de l{\textquotesingle}Iraq, vers
l{\textquotesingle}est (Suse), le nord l{\textquotesingle}est et
éventuellement une partie du delta

Ce sont des choses dont on ne pouvait pas avoir conscience il y a une
cinquante d{\textquotesingle}années. A la fin des années 30, on fouille
dans la région d{\textquotesingle}Uruk et on trouve un même matériel
archéologique et on en déduit que cette culture est remontée vers le
nord est. 

Par contre il existe un zone entre les deux fleuves, où on
n{\textquotesingle}avait jamais travaillé avant les années 80 et donc
il y a eu une diffusion dans des zones nord et sa diffusion en Turquie
a été découverte récemment., (et on attend beaucoup de ces recherches).

En effet ces recherches dans le nord pourront peut être compléter notre
corpus d{\textquotesingle}images du simple site d{\textquotesingle}Uruk

Donc lorsque l{\textquotesingle}on parle d{\textquotesingle}Uruk récent
cela correspond au niveau 6 , 5 et 4. pour les nivaux 6 et 5 nous avons
peu de chose, c{\textquotesingle}est donc surtout du niveau IV que nous
avons des choses et voit bien qu{\textquotesingle}au niveau IV
l{\textquotesingle}agglomération s{\textquotesingle}est
considérablement agrandie et que le nombre d{\textquotesingle}habitants
a été multiplié par 10 et finalement que cette agglomération est un
centre de pouvoir avec des petits villages autour et une économie
agricole qui en dépend

Au départ, il semblerait que ce soit deux chefferies qui ont été réunies
pour constituer une seule et même ville et sur le plan religieux cette
information est importante , elle est loin d{\textquotesingle}être
anodine, car c{\textquotesingle}est un site archéologique qui aura
toujours deux espaces cultuels fondamentaux : le grand temple
d{\textquotesingle}Inanna  et le grande temple d{\textquotesingle}Anu

Il s{\textquotesingle}agit donc de deux villages, devenus deux bourgs,
qui ont été réunis pour ne former qu{\textquotesingle}une seule et même
ville.

Les architectures sont importantes car on voit apparaître le plan en
trois parties et que les ateliers des lapidaires sont dans ce que nous
appelons le temple et donc dans un espace urbanisé où il
n{\textquotesingle}y a pas d{\textquotesingle}architecture 
identifiable comme étant liée au palais,  Nous sommes donc dans un
moment où un grand bâtiment , considéré sans doute comme la maison
terrestre d{\textquotesingle}une divinité est le lieu de résidence de
son représentant sur terre et de sa famille, et c{\textquotesingle}est
également le lieu de stockage , d{\textquotesingle}organisation et de
fabrication.

Le temple correspond donc à toute une partie de la ville;

C{\textquotesingle}est  cette période d{\textquotesingle}Uruk V que vont
apparaître les bulles, 

  [Warning: Image ignored] % Unhandled or unsupported graphics:
%\includegraphics[width=6.98cm,height=12.642cm]{ConfFaivreMartintout-img/ConfFaivreMartintout-img24.jpg}
 

les jetons sont plus anciens, il est possible que les jetons remontent à
la période d{\textquotesingle}Uruk ancien, et même avant.

le fait d{\textquotesingle}enfermer les jetons dans une bulle est
attesté par des morceaux casés , peut être à la fin de
l{\textquotesingle}Uruk moyen, mais cela se généralise à
L{\textquotesingle}Uruk récent et on peut constater une multiplication
énorme durant la période d{\textquotesingle}Uruk V

la bulle contient des jetons et sert à mémoriser des choses.
Traditionnellement, on a toujours considéré que
c{\textquotesingle}était le premier pas vers
l{\textquotesingle}invention de l{\textquotesingle}écriture, mais peut
être pas. Mais on peut remarquer que les jetons ont des tailles
différentes, ils codifient donc des choses, (mais on ne sait pas quoi).
ils sont dans des bulles d{\textquotesingle}argile fermées par un
cachet plat. Mais sur la partie inférieure ou supérieur de la bulle, au
moment où la ferme on met des encoches, et on a pu constater que le
nombre d{\textquotesingle}encoches correspond toujours au nombre de
jetons contenus à l{\textquotesingle}intérieur Mais ces encoches
n{\textquotesingle}ont jamais la forme des jetons  (on a des jetons en
forme de bulle, mais on ne va jamais dessiner de bulle ) et cela
indique qu{\textquotesingle}il y avait une codification

Et finalement le fait d{\textquotesingle}inventer
l{\textquotesingle}écriture ne résulte pas du fait
d{\textquotesingle}utiliser les jetons, mais le fait de tracer une
mémorisation de ces jetons par un système de code à
l{\textquotesingle}extérieur des bulles

L{\textquotesingle}écriture apparaît dans la phase
d{\textquotesingle}Uruk IV

Un des objets caractéristique de cette période est les écuelles
grossières

\newline


\begin{center}
 [Warning: Image ignored] % Unhandled or unsupported graphics:
%\includegraphics[width=4.725cm,height=5.29cm]{ConfFaivreMartintout-img/ConfFaivreMartintout-img25.jpg}

\end{center}
Suse II : époque d{\textquotesingle}Uruk (3500 - 3100 avant J.-C.) 

Ecuelles grossières à bord biseauté avec marque du pouce à la
fabrication  (Louvre)

on retrouve ces écuelles en milliers d{\textquotesingle}exemplaires ,
elles sont faites par emboutissage au point,  (on tape au point
l{\textquotesingle}argile pour lui donner cette forme
d{\textquotesingle}écuelle) , ce sont des objets fabriqués en série,
trouvés en milliers d{\textquotesingle}exemplaires dans les tombes,
dans les fossés longeant des bâtiments identifiés comme des temples et
dans des décharges;

Peu importe la fonction de ces écuelles, mais si on trouve ces écuelles
on peut en déduire que nous sommes dans une couche stratification
d{\textquotesingle}Uruk, elles nous servent donc de marqueur , et nous
les utilisons pour voir l{\textquotesingle}expansion de la culture
d{\textquotesingle}Uruk dans l{\textquotesingle}ensemble du proche
orient, 

Et un certain nombre de documents écrits durant la période archaïque ont
été retrouvé dans ces coupes, 

Le site d{\textquotesingle}Uruk a livré près de 3600 tablettes , sur les
 niveaux 4 et 3 et cela veut dire que l{\textquotesingle}on voit bien
que l{\textquotesingle}invention de l{\textquotesingle}écriture se fait
à la fin d{\textquotesingle}Uruk récent et que le développement de
l{\textquotesingle}écriture se fait au début de l{\textquotesingle}Uruk
final;

en effet Uruk IV correspond à la fin de l{\textquotesingle}Uruk récent
et Uruk III signifie que nous sommes dans l{\textquotesingle}Uruk final

  [Warning: Image ignored] % Unhandled or unsupported graphics:
%\includegraphics[width=6.35cm,height=8.361cm]{ConfFaivreMartintout-img/ConfFaivreMartintout-img26.jpg}
 

(tablette de comptabilité Uruk récent III, 3200 - 3000

enregistrement d{\textquotesingle}une livraison de produits céréaliers
pour une fête au temple d{\textquotesingle}Inanna (musée Pergam)

Les tablettes nous montrent la différence entre les pictogrammes
extrêmement archaïques, qui pour beaucoup ne peuvent être lus , par
rapport aux autres tablettes qui ont déjà une disposition
d{\textquotesingle}un tracé de case et dans chaque cas il y a une
information et un pictogramme,  et un élément chiffré qui  permet de
lire chaque case et de savoir que c{\textquotesingle}est un
enregistrement de produit de céréales pour la déesse Inanna.

Donc en très peu de temps, ce qui relève d{\textquotesingle}
l{\textquotesingle}Uruk récent, on ne le lit pas, et ce qui relève de
l{\textquotesingle}Uruk final, cela ressemble à ce que nous aurons par
la suite et on peut le lire

Et c{\textquotesingle}est effectivement dans cette époque
d{\textquotesingle}Uruk IV que nous allons voir apparaître
l{\textquotesingle}iconographie du roi Mais en réalité nous sommes un
peu mal à l{\textquotesingle}aise car nous manquons de matériaux,
raison pour laquelle il ne faut jamais oublier dans cette discipline
que certaines données peuvent être balayées du jour au lendemain du
fait de nouvelles découvertes archéologiques

En effet les images du corpus pour l{\textquotesingle}Uruk récent se
comptent sur les doigts d{\textquotesingle}une main et ce sont des
objets qui proviennent soit de fouilles très anciennes et donc pas
documentées, soit du marché de l{\textquotesingle}art et donc sans
documentation associée à cet objet et c{\textquotesingle}est notamment
le cas des deux statuettes de roi du Louvre, faites en calcaire ,
achetées en 1930 sans aucune information sur leur provenance réelle

A partir de là effectivement, on peut , mais c{\textquotesingle}est un
autre travail, regarder ce que l{\textquotesingle}on a comme fouille
archéologique à cette époque là et finalement arriver à une provenance
supposée, mais qui ne pourra jamais figurer dans un catalogue, où dans
un inventaire de musée.

Dans les années 30, tous les archéologues travaillaient dans le sur de
l{\textquotesingle}Iraq jusqu{\textquotesingle}au moment où les
français iront dans le monde syrien. Donc nul doute que ces deux objets
, d{\textquotesingle}une trentaine de cm viennent de cette région. Ils
ont d{\textquotesingle}ailleurs été analysés et on sait
qu{\textquotesingle}ils sont faits dans un calcaire grossier identique

Statue des deux rois, Louvre

  [Warning: Image ignored] % Unhandled or unsupported graphics:
%\includegraphics[width=6.174cm,height=8.89cm]{ConfFaivreMartintout-img/ConfFaivreMartintout-img27.jpg}
 

En les analysant on s{\textquotesingle}est aperçu que ces deux objets ne
portent aucune trace de peinture et représentent deux personnages
debout, nus , les jambes l{\textquotesingle}une contre
l{\textquotesingle}autre et se caractérisent par le même geste,  poings
fermés et rapprochés l{\textquotesingle}un de l{\textquotesingle}autre;


Ces personnages ont été identifiés comme étant des rois prêtres ,
partant du principe que certes ils sont dépourvus de source écrite et
on ne connaît pas le contexte de leur provenance, mais
qu{\textquotesingle}ils portent deux attributs sur la tête, que plus
tard nous retrouverons sur une image où il y aura écrit le mot roi , à
côté du personnage qui lui aussi à ces deux attributs : 

ces deux éléments sont le bandeau ou boudin, pour certains
d{\textquotesingle}ailleurs c{\textquotesingle}est
d{\textquotesingle}ailleurs déjà le bonnet que nous reverrons plus tard
avec Gudéa et la barbe, c{\textquotesingle}est une barbe ronde qui a
une forme caractéristique puisqu{\textquotesingle}elle passe sous le
menton et elle n{\textquotesingle}est pas associée à une moustache

Malheureusement ces deux objets ont toujours été présentés de face
(jamais de profil ou avec un miroir ce qui permettrait de voir leur
dos) . De dos on peut s{\textquotesingle}apercevoir
qu{\textquotesingle}il n{\textquotesingle}y a aucun travail de
sculpture, il n{\textquotesingle}y a aucun galbe pour marquer les
fesses, 

C{\textquotesingle}est donc une ronde bosse qui techniquement est
travaillée comme un haut relief. C{\textquotesingle}est un bloc de
pierre détaché et on voit donc que ces objets étaient faits pour être
vus de face , et ils ont été travaillés dans ce sens (ce qui explique
qu{\textquotesingle}il n{\textquotesingle}y ait eu aucun travail pour
le dos)

Quand on passe à la période du IV millénaire, on verra que le souverain
mésopotamien n{\textquotesingle}a pas systématiquement des insignes
régaliens portés sur la tête, il peut être représenté de la même taille
que les hommes qui l{\textquotesingle}entourent, avec le même type de
coiffure , même type de vêtements, et c{\textquotesingle}est simplement
la présence de l{\textquotesingle}inscription , toujours associée à sa
tête, au dessus, à côté, qui indique Lugal, ou En, et qui nous indique
que le personnage est le roi

On a effectivement un cas, où effectivement celui identifié par les
textes comme un roi, porte ce bandeau et cette barbe,
c{\textquotesingle}est le prêtre roi de Bagdad

A partir de là, on prend cet élément et on remonte dans le temps et nous
sommes dans la période où l{\textquotesingle}écriture est en train de
se mettre en place, ou les objets ne portent pas de texte , car pour
eux c{\textquotesingle}était évident que le personnage représenté était
le roi. Et si on se place à ce niveau l à, on a donc deux objets
uniques au monde, il n{\textquotesingle}existe aucun article publié, le
seul élément de comparaison est donc le roi de Bagdad, mais qui est
brisé au niveau de la taille et qui présente un niveau de sculpture
plus avancé.

roi Prêtre de Bagdad

  [Warning: Image ignored] % Unhandled or unsupported graphics:
%\includegraphics[width=6.662cm,height=8.952cm]{ConfFaivreMartintout-img/ConfFaivreMartintout-img28.jpg}
 

C{\textquotesingle}est un roi, et il aurait vraiment fallu le trouver en
contexte, dans son temple ou dans son lieu d{\textquotesingle}origine,
ce qui nous aurait permis d{\textquotesingle}affirmer que nous sommes
bien dans la période d{\textquotesingle}Uruk IV, car en plus il y a
deux choses qui doivent attirer notre attention, c{\textquotesingle}est
que toutes les autres images de roi que nous verrons dans ce cours
datant d{\textquotesingle}Uruk, il sera toujours habillé  et donc cela
rajoute une problématique à notre sujet. Mais on peut exclure le fait
qu{\textquotesingle}il s{\textquotesingle}agirait d{\textquotesingle}un
faux pour cette époque.

La nudité, dans la culture sumérienne (donc plus tard) est associée à
certain moment du culte , et c{\textquotesingle}est notamment le cas
pour le rituel de libation que l{\textquotesingle}officiant
qu{\textquotesingle}il soit souverain ou prêtre, est représenté nu face
à la divinité

ce ne sont pas des gens qui représente la nudité chez
l{\textquotesingle}adulte de façon volontaire, car ce sont des gens
comme les égyptiens et d{\textquotesingle}autres civilisations, où le
costume est au contraire utilisé comme valeur sociale, et au contraire
on utilise l{\textquotesingle}image de la nudité pour signifier
l{\textquotesingle}humiliation, et donc l{\textquotesingle}anéanti et
avec tout ce que cela peut avoir comme signification, le vaincu et plus
largement la domination de l{\textquotesingle}ordre sur le chaos.

Ces images de nudité sont donc assez troublantes pour des images aussi
anciennes, ou peut être que l{\textquotesingle}on se trompe avec cette
interprétation basée sur des sources du début du III millénaire ; car
pourquoi ne pas imaginer l{\textquotesingle}inverse et
qu{\textquotesingle}à cette époque là, le seul officiant étant le roi,
il est absolument normal de le représenter nu, car
c{\textquotesingle}est l{\textquotesingle}évocation de son rôle premier
d{\textquotesingle}être le représentant du dieu sur la terre et le seul
qui officie.

en effet on ne sait pas s{\textquotesingle}il existait une classe
sacerdotale à cette époque

Mais en tout car c{\textquotesingle}est remarquable, dans la
civilisation sumérienne pour être mis en évidence

deuxième point : la position des mains . Mais  peut être est ce une
perte de temps

On peut se dire  tout simplement, que l{\textquotesingle}artiste
n{\textquotesingle}était pas doué. Ce sont des gens qui pendant trois
millénaires vont être représentés devant la divinité les mains jointes
ou les mains en l{\textquotesingle}air dans le geste de la prière, et
le sculpteur se serait simplifié la vie en évoquant juste le geste de
la prière , en fermant les poings et en les mettant côte à côte

Cette série de statue (avec celle du Louvre) constituent des objets
fondamentaux

traditionnellement, dans les manuels , nous verrons cette statue de
Bagdad sans provenance, et daté de l{\textquotesingle}Uruk récent.

mais pour Madame FAIVRE MARTIN, si on regarde attentivement cette
oeuvre, on peut remarquer immédiatement qu{\textquotesingle}elle
n{\textquotesingle}a pas le même niveau de sculpture que les rois
prêtres du Louvre, elle est plus aboutie et travaillée. Si on regarde
la musculature du personnages, qui est présentée, même si elle est
simpliste, et mis en parallèle avec les statues du Louvre elle donne à
celles ci d{\textquotesingle}être deux sculptures à peine dégrossies,  

dans cette statue de Bagdad on a l{\textquotesingle}impression
qu{\textquotesingle}il devait tenir quelque chose, et cela est
intéressant car plus tard dans les années - 2600, on aura des
représentations de gens tenant une petite boîte qui contient
l{\textquotesingle}offrande que nous apportons à la divinité. Est-ce
déjà cela ? et du coup ce serait également cela pour les deux statues
du Louvre.

On peut remarquer également que le travail est plus abouti pour le
bandeau, il y a une démarcation entre la calotte et le bandeau, et
derrière la tête on peut voir une masse , comme si on avait une sorte
de chignon (chevelure ramenée en chignon),  La barbe, par contre est
identique , elle entoure le petit menton, et il n{\textquotesingle}y a
pas de moustache et on a le même type de bouche

Si on regarde ces oeuvres de profil, on constate que la barbe remonte
bien jusqu{\textquotesingle}à la base du bandeau, comme si
c{\textquotesingle}était un seul morceau et du coup ce ne serait plus
une barbe, mais un élément de parure qui aurait été raccordé et cela
expliquerait le fait qu{\textquotesingle}elle passe bien en arrière du
menton, comme si c{\textquotesingle}était une sorte de mentonnière,

On peut voir qu{\textquotesingle}elle est également incisée de lignes
horizontales

Le sourcil est fait en une seule ligne continue, les yeux étaient
incrustées et la forme du visage est la même que ceux du Louvre

On peut aussi se poser la question, s{\textquotesingle}agit-il
d{\textquotesingle}une oeuvre inachevée ?, on peut en effet se poser la
question pour les deux statues du Louvre, et on pourrait imaginer
qu{\textquotesingle}elles proviennent d{\textquotesingle}un atelier .
Nous n{\textquotesingle}aurons jamais la réponse à cette question, mais
on peut se la poser

En résumé, les deux statues du Louvre datent-elles de la même période
que cette de Bagdad, et donc de l{\textquotesingle}Uruk récent, ou
avons nous un décalage dans le temps, ou une provenance différente ?

Mais en l{\textquotesingle}état de nos connaissances, notre corpus
s{\textquotesingle}arrête là pour les statues en trois dimensions,  et
donc leur datation d{\textquotesingle}Uruk récent peut être remise en
cause 

Selon Madame FAIVRE MARTIN, on peut se demander du fait de
l{\textquotesingle}aboutissement du travail qui est différent si ce ne
sont pas des statues d{\textquotesingle}époque différentes

Stèle de la chasse vers 3300

  [Warning: Image ignored] % Unhandled or unsupported graphics:
%\includegraphics[width=7.615cm,height=10.964cm]{ConfFaivreMartintout-img/ConfFaivreMartintout-img29.jpg}
 

Elle fait 80 cm , est en balzate

elle date Uruk niveau IV, et c{\textquotesingle}est très important car
c{\textquotesingle}est le seul objet qui soit bien daté et cela va nous
donner des informations

C{\textquotesingle}est la plus ancienne représentation connue dans la
culture d{\textquotesingle}Uruk du thème du souverain chasseur, ici
c{\textquotesingle}est la chasse du lion qui est représentée, thème qui
traversera les millénaires et on aura ensuite les fameuses chasses
assyrienne de Ninive., même les perses reprendront ce thème 

On peut observer sur cette stèle deux registres et nous avons deux fois
le personnages , ressemblant au roi prêtre, nous avons bien le bandeau,
la barbe, et l{\textquotesingle}intérêt de ce bas relief est que nous
bien l{\textquotesingle}élément qui est sculpté sur la statuette de
Bagdad, à savoir le chignon (attention ce sont des mots, nous
l{\textquotesingle}appelons chignon, mais en réalité on ne sait pas
exactement ce que c{\textquotesingle}est)

Un bandeau, une barbe, un chignon, c{\textquotesingle}est donc un
souverain qui est représenté, on voit qu{\textquotesingle}il est vêtu
d{\textquotesingle}une jupe lisse qui s{\textquotesingle}arrête au
genou, avec juste l{\textquotesingle}indication de la ceinture et
l{\textquotesingle}homme est pied nu

On voit donc que l{\textquotesingle}homme s{\textquotesingle}attaque à
des lions. Est ce deux moments de chasse qui est représenté ? 

dans la partie supérieure , nous avons un roi prêtre face à un lion
qu{\textquotesingle}il transperce de sa lance, et en bas un roi prêtre
et son arc (et des gens qui ont travaillé sur les flèches de cette
période en Mésopotamie ont remarqué que sur cette stèle la flèche
représentée n{\textquotesingle}est pas une flèche très performante pour
s{\textquotesingle}attaquer à un lion, à une époque où
l{\textquotesingle}on sait que les mésopotamiens savaient réaliser des
flèches bien plus performante pour chasser le lion)

Cela nous pose donc un problème  s{\textquotesingle}agit il
d{\textquotesingle}une chasse réelle ou d{\textquotesingle}une chasse
symbolique ?

on peut se poser la question car s{\textquotesingle}ils avaient voulu
représenter une chasse réelle, on ne voit pas pourquoi ils
n{\textquotesingle}auraient pas représenter une flèche adéquate

Il est possible (cet objet a fait couler beaucoup
d{\textquotesingle}encre et les interrogations sont
d{\textquotesingle}ailleurs partis dans tous les  sens) , de se
demander également s{\textquotesingle}il s{\textquotesingle}agit du
même bonhomme sur la stèle.

Madame FAIVRE MARTIN remarque qu{\textquotesingle}il
n{\textquotesingle}y a aucune ligne de sol , et dans la composition
tout est décalé, il n{\textquotesingle}y a aucun alignement. Mais
surtout, on voit bien que dans les proportions de ce qui est
représenté, l{\textquotesingle}arme est énorme, et
c{\textquotesingle}est un élément que nous reverrons sur
d{\textquotesingle}autres objets,  (c{\textquotesingle}est aussi une
caractéristique que nous retrouverons en Egypte)

En effet pour montrer la supériorité du souverain sur les autres hommes,
et finalement une force qui tient du monde surnaturel, il aura toujours
des armes énormes, par rapport à sa taille 

en effet, si on regarde la taille du bonhomme et celle de
l{\textquotesingle}arme, il lui est impossible de la tenir et de
s{\textquotesingle}en servir et cela  nous le reverrons donc. Or il est
évident qu{\textquotesingle}il ne s{\textquotesingle}agit pas
d{\textquotesingle}une erreur du sculpteur, une erreur dans les
proportions car nous ne sommes qu{\textquotesingle}au III millénaire,
cela de toute façon nous le reverrons plus tard

Cela signifie qu{\textquotesingle}il y a des moments où si
l{\textquotesingle}on veut représenter le souverain
d{\textquotesingle}une façon particulière, l{\textquotesingle}arme
qu{\textquotesingle}il tient en main est énorme, et cela implique donc
qu{\textquotesingle}il y a déjà une tournure d{\textquotesingle}esprit
dans ces phases ancienne

Cet objet est daté d{\textquotesingle}Uruk IV et il sert de marqueur de
datation, 

En effet quand on retrouve cette  image du roi, avec son bandeau, sa
barbe ronde, son petit chignon, sa jupe lisse, nous sommes dans
l{\textquotesingle}Uruk récent

LES SCEAUX CYLINDRES

ils commencent à apparaître et ils sont liés à
l{\textquotesingle}apparition de l{\textquotesingle}écriture à là nous
sommes dans une période charnière

On a vu dans l{\textquotesingle}introduction que
l{\textquotesingle}écriture apparaît durant l{\textquotesingle}Uruk
récent et se développe durant l{\textquotesingle}Uruk final

Visiblement, (on en est même sur) les premiers sceaux cylindres datent
de la période d{\textquotesingle}Uruk récent, et même de niveau V,
C{\textquotesingle}est à dire qu{\textquotesingle}avant les premières
tablettes nous avons déjà des sceaux cylindres, qui sont faciles à
couler sur la bulle.

mais le problème c{\textquotesingle}est que pendant longtemps , ils ont
été laissé de côté, et donc s{\textquotesingle}ils ne proviennent pas
de fouilles allemandes d{\textquotesingle}Uruk, on ne connaît pas la
couche de stratification dont ils sont issus et donc
c{\textquotesingle}est par comparaison que nous les daterons (ex
fouilles françaises un peu anarchiques)

C{\textquotesingle}est ainsi que si l{\textquotesingle}on veut dater un
sceau d{\textquotesingle}Uruk IV, il faut que le personnage ait une
barbe ronde, un bandeau, un chignon, une jupe lisse et il faut
également regarder la taille de sa lance, 

ce sont donc les plus anciennes images attestées des personnes et on
peut les regrouper ensemble, 

exemple d{\textquotesingle}un sceau où  l{\textquotesingle}on voit des
personnages en haut agenouillé et entravés, dans un autre on a des
personnages entravés (mais en Mésopotamie on les représente totalement
désarticulés , tête en bas, corps de travers,  tête bêche et cela
signifie qu{\textquotesingle}ils sont des gêneurs et donc entravés)

Or dans les fouilles, nous n{\textquotesingle}avons trouvé aucune trace
de conflit durant la période d{\textquotesingle}Uruk, il existe des
traces de conflit visible l{\textquotesingle}archéologie ou par la
littérature avant les dynasties archaïques, mais vers - 2600.

\newline
Donc nous ne sommes pas dans la commémoration d{\textquotesingle}un
évènement historique, mais dans l{\textquotesingle}affirmation 
d{\textquotesingle}une organisation, et cela si on regarde les images
égyptiennes de l{\textquotesingle}époque de Nagada, on retrouve
également le souverain, des personnages entravés, sans que
l{\textquotesingle}on sache à quoi correspondent ces personnages
entravés (mais en Egypte c{\textquotesingle}est un peu différent car
nous savons qu{\textquotesingle}à cette période, le Sud de
l{\textquotesingle}Egypte va absorber le nord

Mais il faut faire attention car avec notre culture nous voulons
rattacher une image à un fait, or les mésopotamiens ne sont absolument
pas dans cette culture, et l{\textquotesingle}image commémorative
d{\textquotesingle}un fait pour eux a une valeur prophylactique

les sceaux sont frustrant pour nous, car nous sommes à une époque où ils
ont valeur puisqu{\textquotesingle}il représente une marque de pouvoir,
richesse pour son propriétaire, mais en même temps il
n{\textquotesingle}y a aucun signe épigraphique sur cet objet (cela
viendra très tard en fait, en effet pour voir le nom
d{\textquotesingle}une personne sur une statue ou sur un sceau il
faudra attendre - 2600 environ)

Avant la simple image suffit et le porteur du sceau est un personnage
identifié comme étant important

Dans l{\textquotesingle}ensemble les sceaux sont dans un état
déplorables, on remarque que c{\textquotesingle}est une gravure dans le
creux, et c{\textquotesingle}est donc plus difficile à faire que du bas
relief ou de la ronde bosse, et que finalement il y a beaucoup plus de
détail dans la représentation d{\textquotesingle}un sceau

les images sont plus nombreuses pour la période d{\textquotesingle}Uruk
final, cela correspond au niveau III , II et I des couches
d{\textquotesingle}Uruk et à la période de transition, exactement comme
en Egypte, 3100- 2900, cela veut dire que l{\textquotesingle}on est
exactement contemporain de Nagada III, dynastie o et de la première
dynastie

Attention dans les ouvrages anciens, la culture de
l{\textquotesingle}Uruk final n{\textquotesingle}existe pas car on
parlait d{\textquotesingle}une culture de Djemdet Nasr et cela vient du
fait qu{\textquotesingle}en 1928 les archéologues anglais qui
travaillaient à Uruk , ont identifié une culture type Uruk mais
beaucoup plus au Nord et surtout plus aboutie que ceux que les
allemands trouvaient à Uruk et ils ont crée cette culture de Djemdet
Nasr, qui pour eux était contemporaine d{\textquotesingle}Uruk Récent

Mais plus tard on s{\textquotesingle}est rendu compte
qu{\textquotesingle}il s{\textquotesingle}agissait de la même culture,
et qu{\textquotesingle}il y avait eu une mauvaise compréhension des
archéologues anglais et à l{\textquotesingle}heure actuelle on ne garde
cette expression de Djemdet Nasr que pour la région de
l{\textquotesingle}affluent du Tigre, car effectivement dans les années
60  on a fouillé dans cette région et on a pu
s{\textquotesingle}apercevoir qu{\textquotesingle}à la même époque
qu{\textquotesingle}Uruk c{\textquotesingle}était une culture un peu
différente et que cela valait donc la peine qu{\textquotesingle}elle
ait un nom différent

Les sceaux sont faits dans des matières différentes
d{\textquotesingle}abord en calcaire puis on les fera dans des pierres
de luxe, associé  au développement des élites urbaines et royales

Sceau musée de Bagdad  (vers 3000)

  [Warning: Image ignored] % Unhandled or unsupported graphics:
%\includegraphics[width=13.012cm,height=5.808cm]{ConfFaivreMartintout-img/ConfFaivreMartintout-img30.jpg}
 

L{\textquotesingle}iconographie du roi est attesté pour les sceaux de
cette époque  et il a été trouvé dans un temple
d{\textquotesingle}Uruk, niveau III, et il représente le roi sur un
navire qui avec une image centrale d{\textquotesingle}un personnage
dans une position proche de ce que nous avons vu
jusqu{\textquotesingle}à présente, debout, main jointe dans le geste de
la prière, barbe ronde avec une petite pointe au menton,  bandeau
chignon, et une jupe.

Mais là on voit une particularité , il y a un incision au niveau de la
jupe et elle a un quadrillage ce qui n{\textquotesingle}est pas attesté
avant Uruk III, (et il n{\textquotesingle}y a pas
d{\textquotesingle}erreur pour le moment, ce quadrillage nous sert donc
de marqueur de datation

A partir d{\textquotesingle}Uruk III, il y a une mise en scène de la
personne royale, d{\textquotesingle}une part avec
d{\textquotesingle}autres personnages , des représentations
architecturées de la rive (on suppose), d{\textquotesingle}éléments et
d{\textquotesingle}animaux

Il faut aussi regarder la représentation du bâtiment, avec ces deux
hampes, motifs très caractéristiques que nous retrouverons

Sceau du Louvre  roi prêtre et son acolyte nourrissant le troupeau sacré

  [Warning: Image ignored] % Unhandled or unsupported graphics:
%\includegraphics[width=14.041cm,height=5.009cm]{ConfFaivreMartintout-img/ConfFaivreMartintout-img31.jpg}
 

Là encore nous retrouvons notre petit personnages  : bandeau, chignon,
barbe , jupe avec même un effet de transparence et on voit derrière le
roi un petit personnage qui porte ce que le roi porte, à savoir des
épis

si on reconstitue l{\textquotesingle}ensemble du sceau,  on peut
supposer que le roi fait face à un troupeau qu{\textquotesingle}il est
en train de nourrir

De chaque côté de l{\textquotesingle}image, on voit
l{\textquotesingle}évocation de deux hampes, que nous avons déjà vu

Pour analyser les choses, il faut se référer à des textes écrits au
XXVII et XXVI siècles avant Jésus Christ, quand nous avons les premiers
textes concernant le rôle du roi par rapport aux dieux et il est
indiqué qu{\textquotesingle}il doit nourrir le troupe sacré de la
divinité Inanna, 

et on sait également que dans la culture sumérienne, le troupeau
représente non seulement les animaux sacrés qui étaient élevés dans le
temple, mais aussi l{\textquotesingle}humanité

Peut être que dans le monde mésopotamien, ces troupeaux sont une
évocation des hommes et des femmes, alignés sous la forme
d{\textquotesingle}un troupeau, car les textes sumériens disent que le
roi est garant de tout l{\textquotesingle}équilibre, et il doit veiller
à ce qu{\textquotesingle}il {\textquotesingle}y ait pas de famine, que
tout le monde puisse manger.

Monsieur FOREST qui a beaucoup travaillé sur le IV millénaire , était
persuadé que l{\textquotesingle}on ne se trompait pas en proposant ce
genre de lecture : troupeau sacré, mais aussi évocation de la
population.

Sceau du British Muséum

ll a également été acheté sur le marché de l{\textquotesingle}art, et on
voit une mise en page différente, car le troupeau est disposé de part
et d{\textquotesingle}autre des hampes.

Ces hampes c{\textquotesingle}est l{\textquotesingle}idéogramme du
roseau, qui sert dans l{\textquotesingle}écriture cunéiforme à écrire
le nom de la déesse Inanna, déesse maîtresse des troupeaux.

aussi ces hampes que l{\textquotesingle}on voit sur les sceaux sont là
pour évoquer une architecture difficile à réaliser , et que les
lapidaires ont donc utiliser l{\textquotesingle}image de ces deux mats
pour évoquer une architecture difficile à réaliser, et que de toute
évidence à l{\textquotesingle}entrée sur sanctuaire il y avait des mats
, qui permettaient d{\textquotesingle}identifier le temple

le sceau Lapis lazulis d{\textquotesingle}Uruk à Berlin (vers - 3000)

  [Warning: Image ignored] % Unhandled or unsupported graphics:
%\includegraphics[width=15.981cm,height=6.491cm]{ConfFaivreMartintout-img/ConfFaivreMartintout-img32.gif}
 

c{\textquotesingle}est le sceau le plus abouti, et il est en marbre et
au dessus il était décoré d{\textquotesingle}un petit bélier en cuivre,
pour être porté en bandoulière

Il a été trouvé dans les environs d{\textquotesingle}Uruk et appartient
à une série de petits sceaux officiels, trouvé dans un dépôt daté
d{\textquotesingle}Uruk III, à cause du motif de la jupe à chevron(?)

Ce sceau représente l{\textquotesingle}image la plus aboutie et est la
plus ancienne représentation, avec un jeu de symétrie parfait , le roi
tient des rinceaux de végétaux où l{\textquotesingle}on a un motif de
rosettes à huit pétales et ce motif est très précisément dessiné . Il y
a aussi un troupeau représenté en symétrie, les animaux étant dressés
sur leur patte pour tendre la tête vers la nourriture . A
l{\textquotesingle}arrière des animaux on voit les deux hampes, et
surtout en dessous les deux vases, comme celui retrouvé à Uruk (que
nous verrons tout à l{\textquotesingle}heure) Mais sur ce vase
d{\textquotesingle}Uruk on a l{\textquotesingle}entrée du temple
marquée par deux mats, cela nous permet de dire que dans cette scène du
sceau de Berlin nous sommes à l{\textquotesingle}intérieur
d{\textquotesingle}un temple, puisque le roi et les animaux sont
derrière ces hampes

Cette représentation est unique, il n{\textquotesingle}y a pas un défilé
d{\textquotesingle}animaux devant le roi qui porte des épis,  mais au
contraire nous avons un foisonnement d{\textquotesingle}épis (et Mr
FOREST disait que c{\textquotesingle}était la première représentation
de l{\textquotesingle}arbre de vie avec une représentation symétrique
des animaux de part et d{\textquotesingle}autre , Pourquoi pas ?)

d{\textquotesingle}ailleurs en Mésopotamie, cette thématique va
perdurer,  à savoir les animaux dressés sur leurs pattes.

vase d{\textquotesingle}URUK - Bagad\ \ 

  [Warning: Image ignored] % Unhandled or unsupported graphics:
%\includegraphics[width=15.981cm,height=15.699cm]{ConfFaivreMartintout-img/ConfFaivreMartintout-img33.jpg}
 

il a été trouvé à Uruk, dans le quartier du grand temple
d{\textquotesingle}inanna, il n{\textquotesingle}y a donc aucun doute,
c{\textquotesingle}est un vase cultuel, dans une couche archéologique
de niveau III, trouvé en 1933, 1934

et c{\textquotesingle}est important, car nous sommes sur de sa datation
et il peut donc nous servir de marqueur pour dater
d{\textquotesingle}autres objets similaires où l{\textquotesingle}on
retrouve la même iconographie ou la même forme de vase

si on observe ce vase, c{\textquotesingle}est émouvant car on peut
constater qu{\textquotesingle}il a subi des restaurations  et la
première date de l{\textquotesingle}antiquité, donc au moment où il
était encore en usage. Cela signifie donc que c{\textquotesingle}est un
objet qui est resté longtemps dans le temple

et de ce fait c{\textquotesingle}est une pièce unique et fondamentale

en bas on voit un filet d{\textquotesingle}eau, grâce à elle les 
plantes vont pousser  et on peut avoir des animaux et ce motif, pendant
trois millénaires nous allons le retrouver

Pour les mésopotamiens, l{\textquotesingle}eau est un cadeau des dieux,
qui permet aux plantes de pousser, aux animaux de vivre et ensuite aux
humains de vivre également

ensuite au niveau supérieur on voit les porteurs
d{\textquotesingle}offrandes tout nus, et on voit au niveau supérieur
les deux mats, avec de toute évidence la vaisselle qui se trouvait dans
le temple et un individu qui vient accueillir, et il porte une coiffure
avec des choses dressées, est  ce la première pierre à corne de la
divinité ?

Puisque nous avons les hampes, nous sommes donc dans le temple
d{\textquotesingle}Inanna, est ce Inana elle même, ou une prêtresse
jouant le rôle d{\textquotesingle}Inanna. On voit aussi au registre
supérieur un personnage nu et un personnage dont on tient la traîne, ce
qui reste de ce personnage est cassé, mais on peut deviner un motif de
jupe à chevrons

et donc du fait de ce motif de jupes à chevron nous pouvons en déduire
qu{\textquotesingle}il s{\textquotesingle}agit du roi qui arrive avec
son offrande

et là , s{\textquotesingle}il y a bien un rituel né à Uruk, spécifique
d{\textquotesingle}Uruk, et qui a la fin du troisième millénaire
donnera lieu à un fête à l{\textquotesingle}ensemble du monde
mésopotamien, c{\textquotesingle}est le mariage sacré.

le mariage sacré  raconte l{\textquotesingle}union terrestre du roi à la
déesse Inanna , le jour du printemps et de là le cycle des saisons peut
redémarrer et il n{\textquotesingle}y aura ainsi pas de famine

Et on sait que le rôle d{\textquotesingle}Inanna était joué ce jour là
par une prêtresse.

il est donc tentant : cet objet vient d{\textquotesingle}Uruk, il
représente un être féminin sortant du temple d{\textquotesingle}Inanna
qui accueille le roi

ce pourrait donc être la plus ancienne représentation  du mariage sacré
qui ne sera plus représenté ainsi ensuite.

et là c{\textquotesingle}est intéressant car nous sommes encore dans une
époque où ils dessinent , ensuite non, car les sources seront
uniquement textuelles, et il n{\textquotesingle}y aura plus
d{\textquotesingle}images

Ce sont des gens qui ont beaucoup de mal à représenter le monde divin,
l{\textquotesingle}antropomorphisme même 

dans le monde mésopotamien  à toutes les époques,  on sent bien
qu{\textquotesingle}il y a quelque chose qui fait
qu{\textquotesingle}ils sont gêner de représenter les  êtres
surnaturels, tellement sublimes, aussi proche d{\textquotesingle}eux
physiquement et donc la divinité sera évoquée, car ils rechignent à
représenter les divinités fondamentalement importantes

Donc des choses aussi graves que le mariage sacré, dont tout
l{\textquotesingle}équilibre dépend, ils ne le mettent pas en image

 l{\textquotesingle}EGYPTE

Les objets proviennent tous de la Haute Egypte, et majoritairement
notamment du site d{\textquotesingle}Abydos.

depuis les années 1980; on sait que ces objets datent de Nagada II,
phase finale, dite Nagada D, et cela permet de dater le fameux couteau
de Gebel el Arak, acheté par le conservateur du Louvre à un antiquaire
du Caire en 1914

COUTEAU DE GEBEL EL ARAK\ \ 

  [Warning: Image ignored] % Unhandled or unsupported graphics:
%\includegraphics[width=8.885cm,height=13.326cm]{ConfFaivreMartintout-img/ConfFaivreMartintout-img34.jpg}
 

  [Warning: Image ignored] % Unhandled or unsupported graphics:
%\includegraphics[width=15.981cm,height=15.593cm]{ConfFaivreMartintout-img/ConfFaivreMartintout-img35.jpg}
 

la lame était détachée du manche et le remontage s{\textquotesingle}est
fait entre les deux guerres

Il y a donc sur l{\textquotesingle}une des faces de ce manche, une
thématique de la chasse, avec l{\textquotesingle}image
d{\textquotesingle}un homme qui peut surprendre car il
n{\textquotesingle}est pas représenté à l{\textquotesingle}égyptienne,
sauf dans la règle de la représentation égyptienne de face et de profil
: il a bien le visage de profil, l{\textquotesingle}oeil et
l{\textquotesingle}épaule  de face et tout le reste de profil, et cela
a conduit pour certains à dire que ce n{\textquotesingle}était pas de
l{\textquotesingle}art égyptien, et de cette petite phrase a coulé
toute une littérature

Ce qui est ennuyeux c{\textquotesingle}est qu{\textquotesingle}il est
très rare de rencontrer quelqu{\textquotesingle}un qui soit compétant à
la fois dans l{\textquotesingle}histoire de la Mésopotamie et dans
celle de l{\textquotesingle}Egypte, 

il y a eu tellement d{\textquotesingle}ineptie , que Madame FAIVRE
MARTIN donne même l{\textquotesingle}interdiction de lire certains
manuels sur ce sujet !

Il est vrai que cet objet est troublant, certains en ont même déduit que
l{\textquotesingle}Egypte aurait été conquise par la Mésopotamie, mais
en réalité il ne faut pas oublier que les objets circulent plus
facilement que les hommes, et il y a eu forcément des objets
mésopotamiens qui sont venus en Egypte, à Abydos, et ce sans les
mésopotamiens

de surccroît, cet objet a été trouvé en haute Egypte, et à ce jour on
n{\textquotesingle}a jamais trouvé la moindre trace
d{\textquotesingle}une culture d{\textquotesingle}Uruk

en réalité, nous ne sommes pas capables d{\textquotesingle}expliquer le
pourquoi du comment, et il faut donc l{\textquotesingle}étudier de la
façon la plus intelligente possible

Quand on dit que ce coteau donne une représentation typique du Proche
Orient, c{\textquotesingle}est faux, 

en réalité il faut partir du fait que nous avons une image royale
typique de la culture d{\textquotesingle}Uruk, 

Ce n{\textquotesingle}est pas cette image qui vient du Proche Orient et
il faut l{\textquotesingle}étudier en deux temps

Il y a l{\textquotesingle}image de l{\textquotesingle}homme, et là il y
a effectivement une influence d{\textquotesingle}Uruk, 

mais il y a aussi l{\textquotesingle}homme et les lions et cela est un
thème

Pour l{\textquotesingle}image de l{\textquotesingle}homme,
c{\textquotesingle}est vrai que nous sommes embêtés, car nul ne doute
que cet objet vienne du cimetière d{\textquotesingle}Abydos et donc
cette représentation humaine est troublante car elle correspond à
l{\textquotesingle}iconographie du Proche Orient. IL y a même la jupe
lisse ce qui nous fait penser à un objet d{\textquotesingle}Uruk récent
(3500 - 3300) et cela colle avec la période de Nagada II 3500- 3200

Mais il ne faut pas oublier un objet qui voyage beaucoup, objet de
prestige et donc de cadeau, le sceau cylindre, le sceau est donné
c{\textquotesingle}est un beau cadeau, c{\textquotesingle}est un signe
important de pouvoir et de richesse

Donc un sceau comme tout objet de prestige voyage, (on a bien trouvé en
Bretagne des haches qui provenaient de Russie), et voyage seul, 

en plus il ne faut oublier que nous ne savons pas grand chose sur la
navigation dans le golfe persique à cette époque

Pour Monsieur FAROUD, cet objet serait l{\textquotesingle}évocation de
l{\textquotesingle}autre monde et que le sculpteur égyptien aurait su
que cela représentait des gens d{\textquotesingle}ailleurs,  et cela
aurait donc été une façon de représenter les gens
d{\textquotesingle}ailleurs, c{\textquotesingle}est possible.

mais il faut s{\textquotesingle}intéresser au thème de cette image, ce
n{\textquotesingle}est pas le thème du roi prêtre, mais le thème du
maître des animaux.

Quand on a un homme qui maîtrise des animaux, ce n{\textquotesingle}est
pas un prédateur, il les retient mais ne les tue pas et
c{\textquotesingle}est important. Ce thème apparaît très tôt en
Mésopotamie, dès le Néolithique à Suse, mais là il y a une chose
intéressante et ce sont les anglais qui ont travaillé dessus : 

le Maître des Animaux au Proche Orient, où que l{\textquotesingle}on
soit, se caractérise par le fait qu{\textquotesingle}il ne tue pas les
animaux,  mais les maîtrise (lions, caprins , éventuellement des
serpents), mais cet homme est toujours nu, sans attribut aucun, ou
alors avec juste l{\textquotesingle}évocation de la ceinture 

Donc sur ce couteau nous avons bien une image sumérienne, mais cet
assemblage d{\textquotesingle}un roi sumérien, dans
l{\textquotesingle}attitude du maître des animaux , pour le moment ,
est inconnu du répertoire mésopotamien, (du fait qu{\textquotesingle}il
soit habillé)

Et c{\textquotesingle}est en cela que ce coteau est surprenant car dans
les images vues précédemment, notre homme roi prêtre, nous
l{\textquotesingle}avons vu, nourri les animaux mais nos
l{\textquotesingle}avons pas vu dompter ou retenir les animaux  et
c{\textquotesingle}est pour cela qu{\textquotesingle}il faut procéder à
une étude en deux temps, image de l{\textquotesingle}homme et thème

En Egypte, à la période de Nagada II on a d{\textquotesingle}ailleurs la
peinture de la tombe 100 d{\textquotesingle}Heriakonpolis

  [Warning: Image ignored] % Unhandled or unsupported graphics:
%\includegraphics[width=7.615cm,height=5.533cm]{ConfFaivreMartintout-img/ConfFaivreMartintout-img36.png}
 

et là c{\textquotesingle}est intéressant car nous avons une image que
jamais en  Mésopotamie on associera à un nom propre, mais simplement le
Maître des Animaux, dont la traduction signifie nu, vêtu
d{\textquotesingle}une ceinture 

mais peut être que dans la culture égyptienne cela signifie autre chose,
pour représenter la terre , il y avait le lion de
l{\textquotesingle}est et le lion de l{\textquotesingle}Ouest,  et
qu{\textquotesingle}il y aurait là une évocation du nil, et donc chacun
des lions correspondrait à une rive et cela traduirait la domination
animale sur les rives est et Ouest (?)
\end{document}
