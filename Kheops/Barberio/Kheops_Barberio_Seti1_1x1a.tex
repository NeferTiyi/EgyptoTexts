% This file was converted to LaTeX by Writer2LaTeX ver. 1.0.2
% see http://writer2latex.sourceforge.net for more info
\documentclass{article}
\usepackage[utf8]{inputenc}
\usepackage[T3,T1]{fontenc}
\usepackage[french]{babel}
\usepackage[noenc]{tipa}
\usepackage{tipx}
\usepackage[geometry,weather,misc,clock]{ifsym}
\usepackage{pifont}
\usepackage{eurosym}
\usepackage{amsmath}
\usepackage{wasysym}
\usepackage{amssymb,amsfonts,textcomp}
\usepackage{array}
\usepackage{supertabular}
\usepackage{hhline}
\usepackage{graphicx}
\setlength\tabcolsep{1mm}
\renewcommand\arraystretch{1.3}
\title{}
\begin{document}
LA TOMBE de SETHI I  (KV17)

Topologie vallée : 

  [Warning: Image ignored] % Unhandled or unsupported graphics:
%\includegraphics[width=17.956cm,height=12.268cm]{KheopsBarberioSeti11xa-img1}
 

FB a mis sur l’écran le plan de la KV comme il apparaît sur les planches
de KW avec l’orientation vers le nord qui est tout à fait normal et à
côté le plan inversé car lorsqu’on est en visiteur dans la KV on a une
image inversée avec le nord qui se retrouve derrière nous et on se
dirige vers le sud.  Notre chemin, après l’entrée des guichets, conduit
au centre de la vallée où on abandonne Toutankhamon. On part sur la
gauche et c’est ici que se trouve 3 tombes juxtaposées.

La tombe de Ramsès X (KV18) déjà connue et ouverte à l’époque de
Belzoni. C’est une tombe tardive dont l’entrée était visible et décorée
et donc connue même si elle était ensablée. Les 2 autres tombes, Sethi
I (KV 17) au centre et Ramsès I (KV16) à côté ont été découvertes par
Belzoni.

L’entrée de la tombe de Séthi I°  était entièrement camouflée dans le
sol par les déblais qui s’étaient accumulées mais aussi parce qu’on
avait cherché à la dissimuler. Dans toute l’histoire des tombes royales
la tombe de Séthi I° est la dernière tombe à se rapprocher des tombes
de la XVIII° D qui étaient conçues pour être totalement cachées.
L’entrée avait été complètement dissimulée et c’est au terme de 2 jours
de fouilles que Belzoni a pu atteindre le monument.

\begin{flushleft}
\tablehead{}
\begin{supertabular}{|l|l|}
\hline
  [Warning: Image ignored] % Unhandled or unsupported graphics:
%\includegraphics[width=3.981cm,height=6.105cm]{KheopsBarberioSeti11xa-img2}
  &   [Warning: Image ignored] % Unhandled or unsupported graphics:
%\includegraphics[width=5.674cm,height=7.639cm]{KheopsBarberioSeti11xa-img3}
 \\\hline
\end{supertabular}
\end{flushleft}
Belzoni était un homme avisé et qui avait un flair incroyable. Il avait
repéré un élément important qui lui a permis de trouver les tombes de
Ramsès I et de Séthi I. Il avait observé qu’il y avait des ravinements
dans la vallée, des traces qui indiquaient un écoulement d’eau et que
si on suivait cet écoulement il devait y avoir une cavité et donc une
tombe. Cela l’a conduit à cet endroit.

1° plan pour voir les différentes désignations des salles. Suivant les
auteurs les salles sont désignées différemment. 

2° : voir le site de Ken Weeks.

  [Warning: Image ignored] % Unhandled or unsupported graphics:
%\includegraphics[width=8.516cm,height=8.602cm]{KheopsBarberioSeti11xa-img4}
 

Belzoni : il a laissé des témoignages importants si on sait les lire et
les interpréter. Voir le n° 1 de la biblio (1820). La découverte est de
1817. Belzoni va rentrer assez vite en Europe et en 1820 il publie ses
aventures égyptiennes dans lesquelles il y a un passage consacré à la
découverte de la tombe de Séthi I. Ce texte est en anglais et a été
traduit en français puis republié en 1979 (cf Biblio :Belzoni -voyages
en Egypte et en Nubie ... traduction française Paris 1979). Attention
car lorsqu’on lit le texte en anglais (ed A Silotti) on
s{\textquotesingle}aperçoit qu’il s’agit d’une traduction littéraire et
qu’il y a des approximations. Le texte orignal de Belzoni pêche car il
présente au début sa découverte en progressant dans les salles depuis
l’entrée jusqu’aux dernières salles et dans les pages suivantes il
présente la décoration en repartant du début jusqu’à la fin avec des
informations sur certaines salles qu’il ne donne pas dans la découverte
mais qu’il reprend dans la seconde partie. Il faut donc mettre les 2
parties en miroir. Il y a des oublis et des retours en arrière. Donc à
lire avec attention.

Dans l’illustration de ce récit ont paru 2 volumes de planches dont
certaines correspondent à la tombe de Séthi I. Ce sont les planches qui
apparaissent dans l’édition de Siliotti qu’il a bien mis en regard du
texte. 

En plus il y  a toute une série de dessins inédits qui sont conservés au
musée de Bristol. On les appelle les dessins de Belzoni mais en réalité
la plus grande partie a été réalisée par son collaborateur Alexandro
Ricci. Belzoni a fait quelques dessins mais ce ne sont pas les plus
réussis. Tous ces dessins donnent une image de la tombe au moment de la
découverte. Il faut néanmoins prendre quelques précautions. Sur le plan
de la tombe assez schématisé on voit, en rouge, les salles pour
lesquelles les dessins sont fiables (scènes et textes copiés fidèlement
même si on ne savait pas les déchiffrer à l’époque). C’était un travail
gigantesque et plus on s’approche de l’extrémité de la tombe moins cela
est fiable. Pour la salle du sarcophage et pour l’annexe de la 1° salle
à piliers les dessins sont incomplets (représentations mais pas de
texte) et peu fiables avec des textes qui ne correspondent pas à la
réalité car copiés d’après d’autres planches. Par exemple pour
accompagner les représentations fidèles de la salle du sarcophage les
textes proviennent de la 1° salle à piliers pour le Livre des Portes.
Il faut être vigilant mais cela reste un témoignage remarquable.

A son retour Belzoni a fait une exposition à Londres. Dans cette
exposition il a été présenté une maquette de la tombe réalisé avec les
dessins inédits de Bristol (réalisés à l’échelle 1/6). 

  [Warning: Image ignored] % Unhandled or unsupported graphics:
%\includegraphics[width=8.784cm,height=6.934cm]{KheopsBarberioSeti11xa-img5}
 

Ces dessins ont été en partie conservés. Il y a eu une reconstitution
grandeur nature de la 1° salle à piliers et de «l’antichambre» à partir
de moulages (moulage à la cire ). Tout à été monté, peint et exposé
avec la maquette. Cette partie de la documentation a été perdue car les
moulages auraient été perdus dans un incendie.  Cette exposition est
aussi venue à Paris en 1822. Il y a un lien très fort avec Champollion
car, coïncidence,  le jour où Champollion fait sa découverte des
chalands remontaient la Seine avec les modèles de la tombe de Séthi I.
La brochure de l{\textquotesingle}exposition a été traduite par
Champollion même si elle n’est pas parue sous son nom pour des raisons
diplomatiques car Belzoni était en contact avec tous les égyptologues y
compris T. Young qui cherchait à traduire les hiéroglyphes. Il n’était
pas possible que le nom de Champollion, rival des anglais, apparaisse
mais c’est bien Champollion qui a traduit le texte en corrigeant un
certain nombre de choses, déjà sur la dénomination du nom de Séthi.
Champollion a toujours eu une bonne entente avec Belzoni et regretté sa
disparition brutale. Belzoni va mourir l’année suivante. Quand
Champollion  sera dans la tombe de Séthi I en 1829 il fera, dans cette
tombe, un banquet à la mémoire de Belzoni. Cela ne
l{\textquotesingle}empêchera pas d’être assez critique sur les dessins
de la publication de Belzoni. En effet les dessins n’étaient pas aussi
fidèles que ceux qu’il fera lui-même.

DÉCOUVERTE DE LA TOMBE  sur les pas de Belzoni  au moment de la
découverte. 

On va essayer de se faire une idée en suivant son parcours, sa
description et ses dessins. Cela va permettre d’imaginer la façon dont
la tombe se présentait au moment de la découverte.

Sur une planche publiée en 1822 on voit en coupe les différentes salles.
A cette échelle les dessins sont grossiers mais ils vont rythmer la
visite. Même sur ces dessins grossiers il y a des éléments
intéressants.

  [Warning: Image ignored] % Unhandled or unsupported graphics:
%\includegraphics[width=15.439cm,height=11.354cm]{KheopsBarberioSeti11xa-img6}
 

On voit que  l’entrée  était encore encombrée de gravats. Dans la
description de la découverte il n’est pas question d’un mur qu’il
aurait fallu briser à l’entrée. Il est à peu près certain que l’entrée
dissimulée de la tombe supposait qu’il y avait eu un murage sinon les
gravats seraient rentrés dans la tombe beaucoup trop tôt. Belzoni parle
de grosses pierres qui obstruaient l’entrée. Ce n’est pas un murage en
place mais probablement des vestiges du murage d’origine. Certains
gravats avaient déjà pénétrés dans la tombe. Sur la planche il y a un
raccourci très sommaire de la scène d’entrée du roi face à Rê-Horakhty
et ensuite un rond qui est la partie centrale de frontispice des
Litanies du soleil. Ce qui est frappant c’est qu’à l’époque le décor
est déjà décoloré ce qui n’est pas le cas dans les autres salles. Plus
on avance et plus les couleurs sont conservées.  Il est vraisemblable
que ce soit les gravats et l’humidité qui avaient déjà attaqué la
peinture. 

\begin{flushleft}
\tablehead{}
\begin{supertabular}{|l|l|}
\hline
  [Warning: Image ignored] % Unhandled or unsupported graphics:
%\includegraphics[width=7.034cm,height=7.967cm]{KheopsBarberioSeti11xa-img7}
  &   [Warning: Image ignored] % Unhandled or unsupported graphics:
%\includegraphics[width=5.81cm,height=9.137cm]{KheopsBarberioSeti11xa-img8}
 \\\hline
\end{supertabular}
\end{flushleft}
Sur les «bons» dessins de Belzoni et Ricci ils sont très pâles.

A l{\textquotesingle}extrémité du 1° couloir, lorsqu’on s’avance
jusqu’au 2° couloir qui est en fait un escalier la tombe présente la
particularité d’avoir une décoration qui n’est pas achevée. Des parties
de la paroi sont en dessins préparatoires et la paroi du fond est
seulement peinte. Cela permet de voir l’ordre dans lequel ces salles
ont été décorées. 

A l’extrémité une «triple scène» de Maât puisque sur le linteau on a une
Maât horizontale et 2 Maât ailées qui entourent les cartouches. C’est
exactement la scène qui est reproduite par les dessins de Belzoni et
Ricci dans le volume de planches publié. 

\begin{flushleft}
\tablehead{}
\begin{supertabular}{|l|l|}
\hline
  [Warning: Image ignored] % Unhandled or unsupported graphics:
%\includegraphics[width=7.38cm,height=6.373cm]{KheopsBarberioSeti11xa-img9}
  &   [Warning: Image ignored] % Unhandled or unsupported graphics:
%\includegraphics[width=10.28cm,height=7.153cm]{KheopsBarberioSeti11xa-img10}
 \\\hline
\end{supertabular}
\end{flushleft}
Ce motif des 2 Maât qui encadre les cartouches à été abimé car il
n’était que peint. Contrairement à d’autres endroits de la tombe où, si
la décoration souffre, on perd la peinture mais on garde le décor par
les reliefs. Ici ce sont des parties entières qui ont disparu.

Ce linteau à l’extrémité du 2° couloir et les montants avec les
cartouches ont été repris dans la fiche de l’exposition. Belzoni
voulait présenter l’entrée mais dans cette tombe l’entrée n’était pas
décorée et pour rentre la tombe plus attractive il a fait un montage du
fond du couloir qui correspond à l’encadrement et dans cet encadrement
fictif il a placé le détail qui correspond à la scène d’entrée et les
vautours ailés du plafond qui sont caractéristiques du 1° couloir. 
Cette représentation est parfois présentée comme un exemple de
l’absence de fiabilité des dessins de Belzoni. Ce n’est pas vrai
puisque dans ce cas précis il s’agit d’un montage pour présenter
l’exposition.

  [Warning: Image ignored] % Unhandled or unsupported graphics:
%\includegraphics[width=17.979cm,height=8.308cm]{KheopsBarberioSeti11xa-img11}
 

On aborde le 3° couloir avec le déploiement à gauche de la 5° heure de
Livre de l’Amdouat qui est reproduite par les dessins de bristol. C’est
une planche de 1,5 m. Ces dessins montrent qu’il y avait déjà à
l’époque de Belzoni des lacunes importantes dans les parois. Par contre
sur les bordures des dessins de Belzoni les manques sont dus à la
mauvaise conservation de ces documents.

Dans le récit de Belzoni la progression est arrêtée par le puits qui
s’ouvre sous les pieds des visiteurs. 

  [Warning: Image ignored] % Unhandled or unsupported graphics:
%\includegraphics[width=4.336cm,height=6.849cm]{KheopsBarberioSeti11xa-img12}
 

La présence de ce puits est intéressante puisqu’il est décrit assez en
détail par Belzoni et cela apporte des informations utiles. Il dit
qu’au-delà de ce puits la tombe se poursuit et qu’il y a une paroi du
fond face à lui. Il indique qu’il voit une brèche qui est de 2 pieds 6
pouces de hauteur et de 2 pieds de largeur (75x60 cm) ce qui est petit.
La paroi du fond était murée ce qui était un système de leurre faisant
croire que la tombe s’arrête. Sans cette brèche il était impossible de
savoir que la tombe se poursuivait. Pour progresser il faut casser le
mur et actuellement l’ouverture fait presque 3 m de large. On a perdu
un décor original qui devait être réalisé de manière plus grossière et
seulement peint.

Belzoni dit qu’il y avait 2 cordes qui pendaient dans le puits, une à
l’extrémité du 3° couloir et l’autre sur la paroi en face. Ces cordes
pendaient pratiquement jusqu’au fond et c’était donc un dispositif qui
avait permis à quelqu’un de descendre et de remonter dans le puits. La
1° corde est tombée en poussière au moment où Belzoni a voulu s’en
saisir. La 2° corde était bien conservée dans sa partie supérieure. Il
en reste environ 3 m conservés au BM. Cette corde a été soumise au C14
qui a montré que sa datation était environ 950 +- 60 AVJC, soit époque
de la TPI.

La présence de cette corde évoquait l’existence de pillards mais cela
n’est pas documenté. Pour la TPI on sait qu’il y a eu beaucoup de
mouvements dans la KV puisque dans les tombes qui avaient déjà été
pillées on a cherché, de manière officielle,  à rassembler et à retirer
les valeurs qui se trouvaient dans les tombes ainsi que les momies. Les
pharaons ont ensuite été ré-inhumés dans les 2 cachettes, celle sur
place dans la tombe d’Amenophis II et la cachette de Deir el Bahari où
on a retrouvé la momie de Séthi I. Cordes en relation avec les pillards
ou avec la préservation de la momie de Séthi I ?

Belzoni signale que la 1° corde qui est tombée en ruine avait été abimée
par le ruissellement des eaux de pluie. Belzoni a touché du doigt la
principale fonction matérielle de ce puits qui était
d{\textquotesingle}arrêter aux eaux de pluies torrentielles. Il a ainsi
relevé 2 fonctions matérielles principales du puits : constituer un
obstacle et servir de réceptacle aux eaux de pluie.

Cela ne présage pas du fait qu’il peut y avoir aussi une fonction
religieuse.

On franchit actuellement le puits au moyen d’une passerelle et on voit
la partie du fond ouverte avec le reste du décor cassé. 

Le puits va être comblé dès l’époque de Belzoni pour sortir le
sarcophage de la tombe. Il l’a probablement comblé avec les blocs qui
étaient à l’entrée et peut-être avec les débris du mur qui séparait le
puits de la salle à piliers. On a retrouvé des blocs qui ont un fond de
couleur bleu-gris et qui sont simplement peints. Il y a les restes de 2
scènes affrontées avec Dd md.w. Le Dd mdw de droite est en bas-relief
peint et celui de gauche est seulement peint. Il faut imaginer que les
3 m de largeur de ce passage muré qui était décoré de manière à former
un leurre étaient seulement peints. Après que le puits comblé ait été
déblayé, et tout ce qui était à l’intérieur sorti, on a trouvé des
blocs en grès à l’entrée de la tombe. Le grès, dans les tombes royales,
est généralement le matériau qu’on emploie pour des interventions
secondaires. Quand on restaure une paroi c’est en grès. Dans la tombe
de Merenptah par exemple on a du cassé les montants pour introduire un
sarcophage très large et à l’époque pharaonique on les a restaurés en
grès. On ne sait pas pourquoi mais on constate que les restaurations
dans les tombes royales qui sont creusées dans le calcaire se font en
grès. Il ne serait donc pas étonnant que ces blocs de grès avec un fond
bleu puissent provenir de ce murage. 

La petite brèche représenterait l’ouverture ancienne et tout le reste a
été cassé pour ouvrir le passage.

Après avoir franchi la salle du puits on pénètre dans la 1° salle à
piliers qui est noté E ou F suivant la nomenclature. Cette salle est
appelé, dans le récit de Belzoni, «l’antichambre» alors qu’aujourd’hui
on appelle antichambre la salle I qui est juste avant la salle du
sarcophage. C’est cette salle à piliers qui a été réalisée en
fac-simile dans le cadre de l’exposition de Belzoni. Pour cela il a
fait des empreintes à la cire qui ont laissé des traces.  La cire
d’abeille était trop molle avec la chaleur et Belzoni a du mêler de la
terre et la cire est devenue grise ce qui a laissé ces traces qui ont
été en partie nettoyées. Cette technique a laissé des marques mais n’a
pas abimé la couleur.

Cette salle à piliers se prolongeait par une annexe que Belzoni a appelé
«salle à dessins» puisque le décor de cette annexe était restée en
dessins préparatoires. Cela est utile aujourd’hui pour détailler la
technique de mise en place du décor. L’esquisse est en rouge et on
repasse les contours en noir.

  [Warning: Image ignored] % Unhandled or unsupported graphics:
%\includegraphics[width=8.139cm,height=5.581cm]{KheopsBarberioSeti11xa-img13}
 

Retour dans la salle à piliers et, décalé sur la gauche il y a un
escalier qui conduit à la partie inférieure de l’hypogée.

A noter que sur aucun des plans cet escalier n’a été numéroté. Vue de
l’escalier en direction de la tombe : il conduit dans le couloir. Au
bas de ce couloir, si on lève la tête et qu’on regarde la paroi droite
on voit qu’il y a une brèche dans la paroi qui relie le commencement de
ce 4° couloir, couloir G, et le sol de l’annexe de la 1° salle à
piliers, l’annexe F.  Cette brèche est dans le haut de la paroi du
couloir et au niveau du sol dans la salle à piliers. 

  [Warning: Image ignored] % Unhandled or unsupported graphics:
%\includegraphics[width=5.144cm,height=7.911cm]{KheopsBarberioSeti11xa-img14}
 

La présence des brèches et le dessin de Belzoni montrent que cet
escalier état bouché. Belzoni (cf photocopie dans dossier laissé à
Khéops) n’évoque pas le fait que le passage est bouché dans le début du
récit mais il en parle plus loin lorsqu’il évoque le tunnel bouché  sur
lequel reposait le sarcophage dans la salle du sarcophage. 

Il y avait donc un murage et au niveau du sol on ne devait pas voir
qu’il y avait une suite à la tombe. 

  [Warning: Image ignored] % Unhandled or unsupported graphics:
%\includegraphics[width=17.429cm,height=6.796cm]{KheopsBarberioSeti11xa-img15}
 

Après le 3° couloir on était interrompu par le puits avec un système de
leurre et si on franchissait malgré tout ce puits on se retrouvait dans
cette salle qui semblait être l’avant dernière salle avant la salle
annexe. Certains le savaient malgré tout puisqu’il y avait une brèche
et qu’on pouvait contourner le passage par la salle annexe. Il y avait
des systèmes de bouchage à la XVIII° D qui n’existent plus après.

Juste après, dans le passage escalier et début du couloir il y avait des
montants de porte qui se trouvent au Louvre et à Florence. Sur les
dessins, au pied de l’escalier, dos fond de la tombe, on voit les 2
reliefs.

  [Warning: Image ignored] % Unhandled or unsupported graphics:
%\includegraphics[width=8.599cm,height=9.479cm]{KheopsBarberioSeti11xa-img16}
 

\end{document}
