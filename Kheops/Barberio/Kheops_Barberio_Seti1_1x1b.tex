% This file was converted to LaTeX by Writer2LaTeX ver. 1.0.2
% see http://writer2latex.sourceforge.net for more info
\documentclass{article}
\usepackage[utf8]{inputenc}
\usepackage[T3,T1]{fontenc}
\usepackage[french]{babel}
\usepackage[noenc]{tipa}
\usepackage{tipx}
\usepackage[geometry,weather,misc,clock]{ifsym}
\usepackage{pifont}
\usepackage{eurosym}
\usepackage{amsmath}
\usepackage{wasysym}
\usepackage{amssymb,amsfonts,textcomp}
\usepackage{array}
\usepackage{supertabular}
\usepackage{hhline}
\usepackage{graphicx}
\setlength\tabcolsep{1mm}
\renewcommand\arraystretch{1.3}
\title{}
\begin{document}
On continue à progresser vers le fond de la tombe, avec derrière le
couloir G et devant le couloir H puis l’antichambre I. 

A partir de cet endroit, même si cela est très abimé les couleurs du
décor sont particulièrement vives. C’est à cause de cela, couleurs des
hiéroglyphes et représentations très belles et parce qu’on était loin
de l’entrée, qu’au cours du XIX° siècle on a enlevé énormément de
parties de reliefs qu’on retrouve actuellement dans de nombreux musées.

\begin{flushleft}
\tablehead{}
\begin{supertabular}{|l|l|}
\hline
\multicolumn{1}{|c|}{  [Warning: Image ignored]
% Unhandled or unsupported graphics:
%\includegraphics[width=4.314cm,height=6.401cm]{KheopsBarberioSeti11x1b-img1}
 } & \multicolumn{1}{c|}{  [Warning: Image ignored]
% Unhandled or unsupported graphics:
%\includegraphics[width=7.892cm,height=5.526cm]{KheopsBarberioSeti11x1b-img2}
 }\\\hline
\end{supertabular}
\end{flushleft}
Puis on aborde antichambre I que Belzoni appelait la «salle des
beautés». Dans cette salle le décor a été en partie détruit et la
peinture disparue du fait d’empreintes faites au papier mouillé.
C’était une superbe salle et on voit aujourd’hui, à travers les
reliefs, que la qualité était remarquable. Malheureusement il en manque
une grande partie.

\begin{flushleft}
\tablehead{}
\begin{supertabular}{|l|l|}
\hline
  [Warning: Image ignored] % Unhandled or unsupported graphics:
%\includegraphics[width=8.534cm,height=5.505cm]{KheopsBarberioSeti11x1b-img3}
  & \multicolumn{1}{c|}{  [Warning: Image ignored]
% Unhandled or unsupported graphics:
%\includegraphics[width=7.749cm,height=5.403cm]{KheopsBarberioSeti11x1b-img4}
 }\\\hline
\end{supertabular}
\end{flushleft}
Ensuite on arrive à la salle du sarcophage avec ses annexes. 

\begin{flushleft}
\tablehead{}
\begin{supertabular}{|l|l|}
\hline
\multicolumn{1}{|c|}{  [Warning: Image ignored]
% Unhandled or unsupported graphics:
%\includegraphics[width=5.131cm,height=7.865cm]{KheopsBarberioSeti11x1b-img5}
 } & \multicolumn{1}{c|}{  [Warning: Image ignored]
% Unhandled or unsupported graphics:
%\includegraphics[width=7.041cm,height=7.856cm]{KheopsBarberioSeti11x1b-img6}
 }\\\hline
\end{supertabular}
\end{flushleft}
Cette salle a la particularité d’être composée de 2 parties. Une partie
haute dont le plafond est plat, soutenu par 6 piliers et une partie
basse atteint par quelques marches et dont le plafond est vouté. Cette
salle, du point de vu du fonctionnement, forme un tout mais en raison
de ces 2 niveaux certains (PM) décompose cette salle et nomme J la
partie Haute et K la partie basse. Dans la nouvelle numérotation qui
est celle de K Weeks  du TMP et on dit salle J partie haute et partie
basse. Parfois on est gêné par ces désignations.

Des 6 piliers que Belzoni a vu en place il y en a un qui est totalement
manquant. 

De part et d’autre de la salle du sarcophage, partie haute et partie
basse il y avait des annexes. A droite, dans la salle Je ou M il y
avait le Livre de la vache du ciel avec la représentation bien connue
de la vache. C’en en raison de cette vache que Belzoni associe à Isis
qu’il a appelé cette annexe la chambre d’Isis.

  [Warning: Image ignored] % Unhandled or unsupported graphics:
%\includegraphics[width=8.121cm,height=6.272cm]{KheopsBarberioSeti11x1b-img7}
 

Aujourd’hui le plafond nécessite d’être soutenu car la salle du
sarcophage est particulièrement fragile.

On arrive à la partie basse : là encore le plafond est étayé et il y a
de nombreuses fissures. On note des parties en  briques qui sont des
consolidations faites par Carter au début du XX° siècle (voir article
de Carter). On voit la voute peinte avec son plafond astronomique.

C’est dans cette partie basse que reposait le sarcophage.

Cette salle est particulièrement abimée du fait de son emplacement dans
la géologie de la Vallée des rois. La partie inférieure de la salle du
sarcophage est creusée à un niveau où se rencontre 2 couches
géologiques, la couche supérieure en calcaire et la couche inférieure
en schiste. voir schéma de la géologie. Les carriers avaient eu de la
chance jusque-là puisqu’ils avaient creusé la tombe dans une roche de
bonne qualité. Simplement au niveau du puits on avait atteint le niveau
du schiste. A noté que le puits a une forme particulière avec 2 petites
cavités latérales vides. 

Lorsqu’on est au niveau de l’antichambre et de la partie haute de la
salle du sarcophage on est simplement quelques centimètres au-dessus du
schiste. Cela pose un problème de stabilité. La partie basse a été
creusée dans le schiste. C’est pour cela que tout se les parties
inférieures des parois ne sont plus décorées. Aujourd’hui on voit de la
brique dû aux restaurations de Carter qui ont permis de consolider la
salle. 

L’annexe Jd ou O n’est pas décorée et avant 2005 cette annexe était
encombrée en particulier par des fragments décorés qui provenaient des
parties abimées de la tombe. 

La mission de Bâle a eu l’autorisation de les étudier et donc de vider
cette salle. Il y avait également des blocs anépigraphes. Lorsqu’on a
procédé au dégagement de cette salle on a sorti ce qui était décoré et
ne sont restés sur place que ces dalles en grès qui ont un lien avec le
sarcophage.  

On voit bien la limite entre la partie blanche en calcaire et la partie
grise en schiste. En regardant de plus près on voit les marques des
ciseaux des carriers qui ont excavé la salle et dessous on voit le
schiste très abimé.

Toute la partie basse de la salle du sarcophage et ses annexes dont la
salle décorée Jb avec la banquette sont à ce même niveau. 

  [Warning: Image ignored] % Unhandled or unsupported graphics:
%\includegraphics[width=12.046cm,height=8.098cm]{KheopsBarberioSeti11x1b-img8}
 

Les carriers ont ensuite placé un enduit sur cette paroi et ont décoré.
Dans la salle à la banquette ils ont même posé un plaquage en calcaire.
Dans la salle du sarcophage on a camouflé mais cela n’a pas résisté.
C’est sans surprise que l’on voit sur les dessins de Belzoni la marque
brune au bas des parois car cela était déjà abimé.

A l’arrière de la salle du sarcophage il y a une autre annexe Jc non
décorée. Située au même plan elle était déjà abimée. Son plafond était
soutenu par 4 piliers placés en longueur, la partie haute taillée dans
le calcaire et la partie dans le schiste. Déjà du temps de Belzoni il y
en avait un pilier cassé et depuis cela n’a fait que s’aggraver et il
n’y en a plus qu’un debout mais étayé. Les autres se sont écroulés et
une partie du plafond est tombé. 

C’est dans cette salle que Belzoni a trouvé le plus d’objets. De manière
curieuse et on ne sait toujours pas pourquoi il a trouvé une «carcasse
de taureau embaumée avec de l’asphalte». C’était un taureau entier
apparemment embaumé mais ce n’en est pas de l’asphalte. Il s’agit d’une
matière noire appelée parfois bitume qui n’est pas du bitume, matière
minérale, mais probablement  une matière végétale. Ce taureau : reste
de banquet funéraire ? mais s’il est embaumé, lien avec des croyances
funéraires ? On ne sait pas mais c’est la présence de ce taureau qui a
conduit Belzoni à nommer cette annexe salle du taureau ou salle de
l’Apis. Parmi le matériel il y avait des statuettes en bois qui sont au
BM mais elles sont présentées avec des statuettes retrouvées dans la
tombe de Ramsès I et dans la tombe de Horemheb. Entre Ramsès I et Séthi
I on ne sait plus exactement à qui elles appartenaient car il y a eu un
mélange. Il semble qu’une seule appartienne de façon certaine à Séthi
I. Ce sont des statuettes de génies gardiens. Celle de Séthi I a une
tête humaine.

Il y avait plusieurs centaines d’oushebtis (voir article de JL Bovot -
EAO 11). On ne sait pas exactement le nombre exact des oushebtis et de
plus ils étaient de qualité variable (voir Louvre : N472- E31 880). Les
plus beaux étaient en faïence bleue mais il y en avait en bois grossier
recouvert de cette résine végétale noire.  On n’en connait pas la
signification mais le noir évoque la couleur d’Osiris même si on ne
sait pas réellement. Le bois était une matière que l’on pouvait bruler,
surtout le bois recouvert par cette résine. On sait que certains
voyageurs ont cuit leurs repas en utilisant des statuettes retrouvées
dans cette salle. Certains s’en servaient aussi comme torches pour
s’éclairer. Beaucoup de ces oushebtis sont partie en fumée et il est
impossible d’en établir le nombre exact. Certains de ces oushebtis sont
tellement grossiers qu’il était facile de faire des faux en mettant le
cartouche de Séthi I. 

  [Warning: Image ignored] % Unhandled or unsupported graphics:
%\includegraphics[width=4.267cm,height=10.012cm]{KheopsBarberioSeti11x1b-img9}
 

Il y a actuellement environ 380 oushebtis recensés dans le monde dont 60
qui sont douteux. On tourne autour de 300. On ne sait pas s’il y en
avait beaucoup plus. 

Ces oushebtis et le sarcophage prouvent bien qu’il y a eu une inhumation
réelle de Séthi I dans la salle du sarcophage. De plus on a sa momie.

La plus belle découverte est le sarcophage en calcite qui se trouve à
Londres.  La cuve du sarcophage est au musée John Soane
(malheureusement !) car le BM n’a pas voulu l’acheter. Belzoni n’a pas
trouvé le couvercle mais des fragments de couvercle ont été retrouvés à
l’extérieur de la tombe. 

\begin{flushleft}
\tablehead{}
\begin{supertabular}{|l|l|}
\hline
  [Warning: Image ignored] % Unhandled or unsupported graphics:
%\includegraphics[width=9.419cm,height=8.67cm]{KheopsBarberioSeti11x1b-img10}
  &   [Warning: Image ignored] % Unhandled or unsupported graphics:
%\includegraphics[width=4.897cm,height=9.91cm]{KheopsBarberioSeti11x1b-img11}
 \\\hline
\end{supertabular}
\end{flushleft}
Le couvercle avait été cassé et emporté à l’extérieur. Dans les
dernières fouilles réalisées à l’entrée de la tombe de Sethy et de la
tombe de Ramsès I des très petits fragments ont été retrouvés. Il y a
actuellement une étude en cours pour essayer de reconstituer d’avantage
de ce couvercle. 

  [Warning: Image ignored] % Unhandled or unsupported graphics:
%\includegraphics[width=17.981cm,height=16.32cm]{KheopsBarberioSeti11x1b-img12}
 

Il y a des morceaux au musée J Soane et au BM (BM29948). La
reconstitution n’est pas trop difficile puisqu’on sait qu’il s’agit du
LdP à l’intérieur et à l’extérieur de la cuve. Le début et la fin sont
sur la cuve et les parties manquantes (6°, 7°h et 8°h) étaient sur le
couvercle. 

Le sarcophage a été retrouvé au centre de la salle et on comprend qu’il
était au-dessus du tunnel mais cette ouverture n’était pas béante. On
avait la même situation que celle décrite au niveau de l’escalier qui
s’enfonçait de la 1° salle à pilier dans le couloir G. L’entrée du
tunnel était comblée et il y avait des blocs de pierres qui
reconstituaient une sorte de plancher. C’était nivelé au niveau du sol
de la salle du sarcophage. Il y avait des blocs de pierre sur lesquels
reposait le sarcophage.

  [Warning: Image ignored] % Unhandled or unsupported graphics:
%\includegraphics[width=16.759cm,height=9.116cm]{KheopsBarberioSeti11x1b-img13}
 

Il est à peu près certain aujourd’hui que les dalles de grès
anépigraphes de l’annexe Jd soient les dalles qui servaient de socle au
sarcophage et cachaient l’entrée du tunnel.

Le sarcophage est momiforme et il n’est pas très grand. C’est la
première fois que l’on trouve dans la KV un sarcophage en calcite.
Auparavant (Horemheb, Ramsès I) c’était des cuves  en granit. C’est
aussi la première fois qu’on a quelque chose de momiforme et non
rectangulaire. Si on se projette par la suite on a l’exemple de Ramsès
II où Christian Leblanc (cf Memnonia) a retrouvé des fragments de
calcite également avec le LdP peut être disposé différemment mais c’est
le même type de sarcophage. Chez Merenptah on retrouve des restes de
calcite avec le LdP (1 fragment au BM) mais il a en plus l’emboitement
de 3 sarcophages de granit. Il avait donc 4 sarcophages dont le 1° en
calcite. 

Pendant longtemps FB a pensé que cela devait être la même chose chez
Séthi I. Pour Ramsès II, Christian Leblanc dit très clairement qu’il
n’y a pas de trace de sarcophage en granit. Chez Sethy I on n’a jamais
retrouvé de trace de granit. Si on pense à la brèche il n’y a pas la
place pour qu’à aucun moment on ait pu sortir un sarcophage de granit.
On doit en conclure que Séthi I ne reposait que dans ce sarcophage posé
à même le sol. On sait que les petites chambres de la KV 5 (tombe des
fils de Ramsès II) où étaient censées être le sarcophage sont trop
petites pour placer une cuve. Cela est à peu près contemporain. Il y
avait peut-être l’idée que seul un sarcophage suffisait. Si on regarde
dans l’univers minéral de S Aufrère sur la symbolique de l’albâtre il
évoque le fait qu’un texte du papyrus Jumilhac évoque l’albâtre en
association avec une formule peu claire en rapport avec Osiris.
L’albâtre c’est «ce qui rend beau Osiris» (snfr) et cela a été souvent
traduit «ce qui régénère Osiris». Manifestement on n’avait pas un
dispositif d’emboitement mais la matière exceptionnelle de ce
sarcophage qui s’apparente à un cercueil anthropoïde pouvait peut-être
lui conférer qualité particulière.

L’entrée du tunnel était camouflée mais actuellement elle est dégagée.
On voit que le tunnel a été creusé dans le schiste.

Ce tunnel a été exploré dans les années 1960 par Abdel Rassoul. On a des
traces d’un escalier à rampe dans le tunnel et donc on avait l’idée
qu’il y avait une vraie salle de sarcophage plus bas ce qui était quand
même contredit par la présence du sarcophage en calcite et des
oushebtis. 

  [Warning: Image ignored] % Unhandled or unsupported graphics:
%\includegraphics[width=9.121cm,height=6.939cm]{KheopsBarberioSeti11x1b-img14}
 

L’idée a été de procéder à un nouvel examen de ce tunnel sous la
direction de Zahi Hawass. Tarek El Awadi  a exploré ce tunnel.
Préalablement, en 2007, pour protéger la salle du sarcophage en vue de
la fouille future toutes les parois avaient été tendues de toiles et le
sol recouvert de nattes. 

  [Warning: Image ignored] % Unhandled or unsupported graphics:
%\includegraphics[width=8.356cm,height=6.528cm]{KheopsBarberioSeti11x1b-img15}
 

La coupe du tunnel que nous avons est celle des dessins des TMP. Il faut
savoir que les architectes qui ont fait ce plan sont partis de
l’existant et que parfois il y a des éléments enregistrés sur les plans
qui ne correspondent pas à une architecture originale. Par exemple les
escaliers qui se trouvent au début du tunnel sont des escaliers faits
par Carter. Un peu plus bas il y a des vestiges d’un escalier à rampe
qui sont les vestiges que Belzoni a vu et dont il parle. Aujourd’hui le
tunnel creusé dans le schiste est étayé de tous les côtés. A certains
endroits il y a des marches en calcaire (calcaire rapporté ou veine de
calcaire) avec la rampe centrale et les marches sur le côté. Les
montants sont perdus mais on peut voir que la largeur du tunnel est
identique à la largeur des premiers couloirs de la tombe. On a donc
voulu faire quelque chose de construit et de pensé. Le tunnel avait été
exploré par Abdel Rassoul qui s’est arrêté car il n’aboutissait plus et
était toujours dans le schiste. Le tunnel était encombré et en avançant
A R s’est demandé s’il ne creusait pas son propre tunnel dans le
schiste. C’est en fait ce qui s’est passé.

Les fouilles récentes ont montré qu’en creusant plus bas on retrouvait
une couche de calcaire et un nouveau couloir avec un escalier à rampe
et de la même largeur que plus haut. 

En 2010 Tarek El Awadi a trouvé ces premières marches du tunnel. FB n’a
pas pu prendre de photos.

  [Warning: Image ignored] % Unhandled or unsupported graphics:
%\includegraphics[width=10.035cm,height=6.777cm]{KheopsBarberioSeti11x1b-img16}
 

Au total le tunnel se prolonge dans cette veine de calcaire. On retrouve
un couloir d’une dizaine de mètres de long avec cette rampe, un nouveau
passage et l’amorce d’un 2° couloir et là cela
s{\textquotesingle}interrompt. Aujourd’hui on a atteint au fond du
tunnel. On sait qu’il a atteint 174 m depuis l’entrée du tunnel. Cela
s’ajoute au 137 m de la tombe et donc au total la tombe fait plus de
300 m. Le tunnel n’a jamais abouti nul part mais Tarek El Awadi a
trouvé une «fausse-porte» et des dessins préparatoires. FB pense avoir
compris qu’à un moment donné dans le 1° ou le 2° couloir il y le dessin
préparatoire d’une porte en ligne ocre, sur le côté , avec une mention
en hiératique «une porte à creuser». FB pense que le terme fausse-porte
indique simplement la présence de cette indication d’une ouverture à
creuser.

Au total la fonction du tunnel après ces fouilles n’est pas élucidée.
L’hypothèse de J Lipinska (cf biblio) était que ce tunnel, d’après sa
pente, se rapprochait du niveau de la nappe phréatique. Cela est encore
plus vrai aujourd’hui puisqu’on voit qu’il va encore plus loin que ce
qu’on imaginait.  On peut éventuellement supposer que ce tunnel mettait
la salle du sarcophage et le sarcophage posé au-dessus symboliquement
au niveau de la nappe phréatique, au niveau des eaux primordiales. On
pense à la scène finale du LdP qui se trouve sur la paroi de la cuve du
sarcophage, cette scène où le Noun élève la barque solaire. Les eaux,
lieu de naissance faisant référence à l’eau primordiale de la création.
Spontanément c’est ce symbolisme qui semble s’imposer. Il est vrai que
les fouilles ont bien montré que ce tunnel avait été pensé mais du fait
que le tunnel s{\textquotesingle}interrompt on ne saura jamais ce
qu’était réellement ce tunnel. Il n’y a pas d’autre exemple. 

L’idée que peut -être il y a un tunnel chez Ramsès II a été avancée et
cela a donné à S Hawass l’idée de le chercher. Cela est loin d’être
certain. De plus cette tombe, pour des raisons géologiques, est
délabrée. Il est plus probable qu’on ait un début de salle annexe
plutôt qu’un tunnel. 

L’état de la tombe de Séthi I s’est très vite dégradé du fait
d’interventions malheureuses ou vandales. En 1818, quelques mois après
la découverte et le comblement du puits pour sortir le sarcophage il y
a eu, par malchance, un épisode de pluies torrentielles. Du fait que le
puits ait été comblé l’eau n’a pu être arrêtée totalement et a ruisselé
dans la salle du sarcophage. En raison de la géologie la roche a gonflé
avec l’humidité puis s’est rétractée et cela a fait éclater des parois.
Les fissures et le début du délitement des piliers datent de cette
époque. C’est d’autant plus dommage que Belzoni avait bien analysé la
fonction du puits dans la KV et que c’est bien lui qui a comblé le
puits et l’a rendu non opérationnel. FB a été surprise de constater que
ce puits comblé par Belzoni a dû le rester longtemps. Sur les photos de
Burton (1921-1928) on voit que le puits est comblé et qu’il n’y pas
encore le dispositif de la passerelle. FB ne sait pas s’il s’agit du
comblement d’origine ou bien si cela a été déblayé puis de nouveau
comblé. Si cela est resté dans l’état du premier comblement cela
signifie que pendant tout le 19° siècle toutes les pluies qui sont
tombées, et on sait qu’il y a eu plusieurs épisodes, ont continué à se
déverser. 

\end{document}
