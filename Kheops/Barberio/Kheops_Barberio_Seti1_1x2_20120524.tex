% This file was converted to LaTeX by Writer2LaTeX ver. 1.0.2
% see http://writer2latex.sourceforge.net for more info
\documentclass{article}
\usepackage[utf8]{inputenc}
\usepackage[T3,T1]{fontenc}
\usepackage[french]{babel}
\usepackage[noenc]{tipa}
\usepackage{tipx}
\usepackage[geometry,weather,misc,clock]{ifsym}
\usepackage{pifont}
\usepackage{eurosym}
\usepackage{amsmath}
\usepackage{wasysym}
\usepackage{amssymb,amsfonts,textcomp}
\usepackage{array}
\usepackage{supertabular}
\usepackage{hhline}
\usepackage{graphicx}
\setlength\tabcolsep{1mm}
\renewcommand\arraystretch{1.3}
\title{}
\begin{document}
\ \ \ \ \ \ \ \ \ \ \ \ \ \ \ \ \ \ \ \ \ \ \ \   24/05/12

SUITE de L’HISTORIQUE

PLANS et dénomination des salles : 

Désignation des salles d’après PM reprises par Hornung et Burton : début
au 1° couloir {}- E 1° salle à piliers et F son annexe {}- G le 4°
couloir et I l’antichambre {}- la salle du sarcophage reçoit 2 lettres
J pour la partie à piliers et K pour la partie voutée. Autour les
annexes numérotées depuis L à gauche avec la 4° heure du LdP, M à
droite avec la vache du ciel, N à gauche pour la salle à la banquette,
O à droite la salle non décorée et P la salle du fond et le tunnel
aurait la lettre Q.

Désignation des salles d’après K.Weeks (TMP) : début dés l’escalier
extérieur non décoré  A {}-  ensuite c’est décalé et la salle à piliers
devient F {}- pour les salles annexes il donne une grande lettre pour
la salle dont elle dépend et une petite lettre ainsi l’annexe de la
salle à piliers F devient la salle Fa {}- l’escalier suivant n’est pas
numéroté et donc on retrouve la numérotation de PM {}- J pour la salle
du sarcophage et les salles annexes sont numérotées en partant à gauche
mais dans le sens des aiguilles (Ja .. Je) et K pour le tunnel.

Attention Reeves n’emploie pas la même numérotation pour les annexes
même s’il utilise Ja, Jb ...



\begin{figure}[htp]
\centering
 [Warning: Image ignored] % Unhandled or unsupported graphics:
%\includegraphics[width=12.065cm,height=8.311cm]{KheopsBarberioSeti11x220120524-img1.jpg}

\end{figure}
Numérotation des piliers : PM : on commence par la rangée de gauche avec
A {}- B {}- C puis on revient à D {}- E- F {}- on numérote les faces
des piliers en partant de la face A qui se trouve devant quand on
rentre {}-



\begin{figure}[htp]
\centering
 [Warning: Image ignored] % Unhandled or unsupported graphics:
%\includegraphics[width=5.084cm,height=4.833cm]{KheopsBarberioSeti11x220120524-img2.jpg}

\end{figure}
  [Warning: Image ignored] % Unhandled or unsupported graphics:
%\includegraphics[width=18.284cm,height=5.752cm]{KheopsBarberioSeti11x220120524-img3}
 

LES DEGATS après la DECOUVERTE :

Après la découverte par Belzoni les ennuis commencent: en comblant le
puits pour extraire le sarcophage Belzoni va rendre la tombe
extrêmement vulnérable. Pour la 1° fois depuis l’antiquité, les salles
au delà du puits ne seront plus protégées par le puits. L’eau des
inondations qui surviennent rapidement après le comblement vont pouvoir
atteindre la partie inférieure de la tombe qui était la partie la plus
fragile. 

A partir de là, avec les alternances de sécheresse et d’humidité, cela
va craquer un peu partout. 

Belzoni va revenir et il signale que dés 1818 des parties du décor sont
tombés et que ce qui a le plus souffert ce sont les passages de portes
et les piliers. De fait, dans la 1° salle à piliers et dans la salle du
sarcophage on a des dégâts importants. Le pilier E qui a été dessiné
par Belzoni est déjà ruiné 11 ans plus tard quand Champollion vient
dans la tombe. Sa destruction se fera au début du XX° siècle.

En plus de ces dégâts «naturels» la tombe va souffrir de la main de
l’homme. Il s’agit de dégâts «involontaires» et des dégâts volontaires
dus au vandalisme.

\begin{itemize}
\item les empreintes prises à de nombreux endroits pour saisir le décor
en relief. Ce sont des empreintes à la cire qui sont surtout le fait de
Belzoni, des empreintes faites avec un mélange de gypse et d’eau qui
sont très abrasives et enlèvent la couche picturale et des empreintes
au carton mouillé qui auront également des conséquences désastreuses. 
\item  les découpes.
\end{itemize}
Retour à l’annexe Jb : 

Belzoni indique que le passage entre la salle du sarcophage et l’annexe
Jb était bouché et qu’il a fallu enlever un murage pour pénétrer dans
cette annexe. Il ne précise pas l’état dans lequel était ce murage,
intact ou non. 

Belzoni appelait cette salle la «salle au buffet» en raison de la
banquette qui court tout autour de la salle à mi-hauteur. 

Dans l’angle on voit un exemple avec des traces d’empreintes.

 Il s’agit de la 6° heure de l’Amdouat avec ces personnages comme assis
dans le vide. 

Autour de tous les personnages du registre supérieur, pour la partie
supérieure ou pour la totalité du personnage,  il y a des sortes
d’auréoles brunes qui sont des marques de l’empreinte de cire. Cela n’a
pas abimé la couleur même si cela a altéré l’aspect général. Au dire
des restaurateurs la pellicule de cire appliquée a même protégé la
couleur. On peut imaginer d’enlever la pellicule et de retrouver la
couleur intacte. Cela a conduit quand même a des sortes de
ruissellements  Photo  sous le personnage central de la photo on a la
cabine du dieu solaire et si on agrandit cette partie de la décoration
on voit ces ruissellements. On voit bien ces résidus sur la paroi mais
cela est réversible. 

 Il y a d’autres exemples qui correspondent à des empreintes qui ont
totalement enlevé la couleur comme l’empreinte qui a été prise du motif
du dieu Thot à tête de babouin qui tient dans sa main le hiéroglyphe de
son nom avec l’oiseau ibis. Cette empreinte a été faite avec le mélange
de gypse et d’eau et cela a provoqué des ruissellements en dessous qui
ont emporté sur les trainées les couleurs des hiéroglyphes situés en
dessous. 

Ces empruntes ont été faites par Wilkinson en 1826 qui ne savait pas
qu’en décrochant les empreintes il enlèverait également la couleur.
C’est à lui que l’on doit la numérotation des tombes de la KV. Ces
empruntes sont conservées au BM. FB ne les a pas vues mais il est dit
que les pigments sont au dos de ces empreintes.

  [Warning: Image ignored] % Unhandled or unsupported graphics:
%\includegraphics[width=10.116cm,height=6.715cm]{KheopsBarberioSeti11x220120524-img4}
 

Cette pratique d’empreintes était relativement commune. Beaucoup de
personnes ont fait des empreintes. L’usage du carton mouillé n’avait
pas toujours des conséquences aussi désastreuses. Les empreintes
conservées sont concentrées  en Angleterre. A Oxford, au Griffith
Institute,  il y a des empreintes faites dans la tombe de Séthy I par
un ecclésiastique anglais Henry Stobert. Photos correspondant 2 aux
marques vues sur la paroi dont la représentation de Thot. On remarque
que l’empreinte est plus petite que l’empreinte initiale.  On peut
imaginer qu’à cet endroit Wilkinson a fait une empreinte qui a enlevé
la couleur et qu’ensuite on a fait une empreinte plus petite qui ne
prenait pas la totalité du personnage. Il est probable qu’au dos de
cette empreinte il ne devait plus y avoir que quelques résidus de
couleur. 

Toutes ces empreintes ont laissé des marques irréversibles et de très
nombreuses scènes sont maintenant dépourvues de couleur.

  [Warning: Image ignored] % Unhandled or unsupported graphics:
%\includegraphics[width=9.446cm,height=4.293cm]{KheopsBarberioSeti11x220120524-img5}
 

On a pu évaluer qu’il y avait plus de 800 traces d’empreintes dont 650 à
la cire (essentiellement dues à Belzoni), plus de 100 empreintes
abrasives et irréversibles et 130 empreintes au carton mouillé. Ces
chiffres viennent d’un article de 2003 de Michael Jones.

Les autres types d’agression ce sont les découpes sauvages.  

On les voit déjà sur des photos anciennes comme celles de Burton en noir
et blanc ou des photos de 70 quand cela n’a pas encore été rebouché par
le service des antiquités. On voit mieux les ronds creusés. On
sélectionne les beaux motifs (hiéroglyphe ou petit personnage) et on
creuse autour pour l’extraire. Cela fait des dommages considérables. Ce
sont des découpes à «petite échelle» mais cela défigure la paroi. Il y
a eu des découpes très tôt et parfois pour de plus grands tableaux. 

Voir la citation d’un extrait du journal de Linant de Bellefonds.
C’était plutôt un géographe de formation mais il a accompagné un
anglais William Banks dans ses voyages en Egypte dans les années
1818/1820. Il s’est lié avec Belzoni et en Juillet 1821 il s’arrête à
Louxor et va à KV . Il écrit « je trouvais le tombeau gâté et un
tableau tout à fait taillé que j’ordonnais à Yanni d’enlever». Il ne
parle pas spécifiquement du tombeau de Séthy I mais dans le contexte
c’est celui là. Yanni est un personnage d’origine grecque qui a vécu
très longuement à Louxor, qui a travaillé pour le consul anglais Salt
et qui ensuite a travaillé dans la vente d’antiquités. On ne sait pas
ce qu’était ce tableau ni ce qu’il est devenu. On peut penser que s’il
est passé entre les mains de Yanni il a été vendu. Cela prouve que dés
1821 il y a déjà un témoignage d’une découpe. 

Il y a également un témoignage écrit de Robert Hay qui a fait des
séjours en Egypte 1824-1834). Dans ses manuscrits, sur la tombe de
Séthy I, il évoque une vignette de la Vache du Ciel représentant une
divinité  masculine et une divinité féminine, les incarnations du temps
nHH et D.t  soutenant les piliers du ciel. Il dit qu’on a tenté
d’enlever cette vignette dans la tombe. En regardant la vignette on
voit que la partie centrale est décolorée et donc que c’est après le
passage de Wilkinson de 1826. 

Moins de 10 ans après sa découverte la tombe a subi des dommages :
inondations de 1818, les empreintes de Belzoni et de Wilkinson et les
premières découpes sauvages dont une attestée dés 1821.

Au moment où l’expédition franco-toscane arrive sur les lieux en
1828-1829 il est évident, pour FB, que la tombe n’était plus intacte et
cela sans vouloir disculper qui que ce soit, car avoir enlevé des
reliefs n’est clairement pas une bonne chose. Champollion est
généralement présenté dans l’historique de la tombe comme étant le
premier a avoir abimé la tombe mais FB pense que la tombe n’était plus
intacte au moment où il l’a voit. FB suppose que si elle avait été
intacte peut-être qu’il n’aurait pas enlevé les reliefs.  Il y avait,
dans le cadre de l’expédition, Alessandro  Ricci qui avait
consciencieusement dessiné tous les reliefs pour Belzoni plus de 10 ans
avant qui était à même de mesurer les dégâts. Cela a ôté à Champollion
ses derniers scrupules pour enlever les 2 reliefs (Louvre et Florence).

En 1829 un autre relief a été découpé. Il s’agit du relief d’un pilier
de l’annexe Jb qui a été découpé à l’intention du BM. Au moment où
Champollion était sur place il y avait également Joseph Bonami qui est
connu ultérieurement pour avoir travaillé au BM et fait la publication
du sarcophage de Séthy I en 1860. Bonami va essayer de persuader
Champollion de ne pas découper ces reliefs, pour ne pas attenter à ce
monument et de plus, que la tombe ayant été découverte par Belzoni pour
le compte du consul anglais Salt c’est un monument qui doit revenir aux
anglais. Finalement il se rend aux arguments de Champollion comme quoi
cette tombe va être abimée et qu’il vaut mieux retirer ces reliefs pour
les présenter dans un musée. Champollion lui propose de profiter des
ouvriers sur place  pour emporter un autre fragment pour le BM.  On
voit bien sur le pilier les traces de la découpe. Il a fallu attaquer
le plafond pour pouvoir dégager la place pour passer la scie. Au milieu
ce relief s’est cassé. Par les dessins de Belzoni et Ricci on sait quel
était le décor. C’est une représentation unique, tout à fait étonnante
dans le contexte d’une tombe royale, le roi faisant une course à la
rame. 

La mission de Bâle a vidé la petite salle Jd qui contenait, au-dessus
des dalles de grès, de nombreux fragments. On a retrouvé 2 fragments
d’angle jointifs qui ont pu être identifiés comme appartenant à ce
relief. On voit bien le reste de l’aviron et le bas, en rouge,  du
signe d’ l’Occident. Il reste peut-être encore d’autres fragments non
identifiés dans des collections ou sur place. L’annexe arrière contient
elle aussi des fragments non exploités. Ce sont les dessins de Belzoni
et surtout Ricci qui permettent de savoir à quoi ressemblait ce décor
perdu.

 1844 : expédition de Lepsius qui retire 2 fragments : un de la face A
du pilier B de la salle du sarcophage et un autre  fragment dans un
angle de ?

 .

\begin{flushleft}
\tablehead{}
\begin{supertabular}{|m{9.262cm}|m{8.729cm}|}
\hline
  [Warning: Image ignored] % Unhandled or unsupported graphics:
%\includegraphics[width=6.828cm,height=15.24cm]{KheopsBarberioSeti11x220120524-img6}
  &
Le 1° fragment se trouve à Berlin. Il semblerait qu’il ait souffert au
moment de la seconde guerre mondiale. Cela se voit en comparant des
photos antérieures et postérieures. Ce fragment représente le roi
devant Osiris qui a une iconographie particulière avec le némès, la
couronne blanche et les plumes.

  [Warning: Image ignored] % Unhandled or unsupported graphics:
%\includegraphics[width=6.269cm,height=11.289cm]{KheopsBarberioSeti11x220120524-img7}
 

L’emplacement a été restauré par Carter d’où la présence de briques.

\\\hline
\end{supertabular}
\end{flushleft}
Un peu plus bas du côté de la paroi on a d’autres briques et c’est là
d’où provenait l’autre fragment d’angle. On a enlevé de ce fragment des
éléments de restauration faite après la 2° guerre mondiale. Il parait
donc en moins bon était. Cela correspond à une scène où l’on voit les
Egyptiens avec les autres peuples de l’humanité. Là un Libyen. Cette
représentation a déjà une version dans la 1° salle à piliers.

  [Warning: Image ignored] % Unhandled or unsupported graphics:
%\includegraphics[width=7.038cm,height=12.331cm]{KheopsBarberioSeti11x220120524-img8}
 

Un article de Lefébure sur Lepsius : Lepsius a cristallisé toute la
rancoeur de beaucoup de gens qui considéraient qu’il était responsable
de toutes les dégradations dans la tombe de Séthy I et dans d’autres
tombes thébaines. Lefébure en 1880 environ, en pleine période
revancharde,  a eu le courage de réhabiliter la mémoire de Lepsius qui
a emporté ces morceaux comme beaucoup de monde. Il y a eu sorte de
légende qui a poursuivi le nom de Lepsius de manière exagérée car il
n’a pas fait pire que les autres.

Tout au long du XIX° siècle on a énormément prélevé d’éléments dans la
tombe de Séthy I°.  Le plus souvent il s’agissait de ramasser des
fragments à terre mais il y a en plus tous les reliefs volontairement
détachés des parois. Citation d’Edouard Naville qui a été longtemps
dans la KV et dans la tombe de Séthy I° en 1869  «le tombeau de Séthy
I°  est l’une des carrières les plus fructueuses où les arabes viennent
se pourvoir de sculptures qu’ils vendent aux étrangers». Il y avait un
gros trafique illicite. Jusqu’à la fin du XIX° où le service des
Antiquités a posé une porte cadenassée la nuit il y a eu de nombreux
prélèvements. 

Ce ne sont plus de grosses découpes mais de petites découpes rondes. 
Les motifs les plus prisés sont les cartouches de telle sorte qu’à de
nombreux endroits de la tombe il y a ces formes ovales allongées,
verticales ou horizontales selon la position du cartouche d’origine. 

  [Warning: Image ignored] % Unhandled or unsupported graphics:
%\includegraphics[width=8.045cm,height=7.244cm]{KheopsBarberioSeti11x220120524-img9}
 

Dans le Livre de la Vache du Ciel, dans la seconde vignette en dehors de
la vache proprement dite, la vignette du roi soutenant le ciel, le
cartouche a été emporté. Dans toute la décoration de la tombe où les
cartouches étaient accessibles on constate qu’ils sont enlevés.

  [Warning: Image ignored] % Unhandled or unsupported graphics:
%\includegraphics[width=8.881cm,height=5.74cm]{KheopsBarberioSeti11x220120524-img10}
 

Il y a 2 textes en particulier qui contenaient des cartouches, les
Litanies du Soleil dans le 1° couloir et le rituel de l’ouverture de la
bouche dans les 4° et 5° couloir. Les litanies du soleil étaient trop
près de l’entrée et cela n’était assez discret de les prélever.
Certains cartouches ont quand même été ôtés (voir photos de Burton).
Dans les scènes d’ouverture de la bouche il n’y a pas que les
cartouches du texte qui ont été retirés. Dans ces passages, dans les
vignettes, il y des représentations de petites statues du roi
accompagnées du cartouche. Ce sont des statues sur lesquelles sont
effectué le rituel (fumigation d’encens, ouverture de la bouche avec
l’herminette .... Comme cela était plus profond dans la tombe et que
les couleurs étaient particulièrement belles les parois ont été
dévastées. On voit également qu’à certains endroits on a essayé de
retirer des fragments sans y parvenir. La plupart de ces petites
représentations ont été enlevées. Il y a des exemples dans les musées
de cartouches provenant de ces découpes : Musée de Brunswick un
cartouche de Séthy I° entré dans les collections en 1869. C’est l’année
où Naville était sur place et témoigne de cet état. L’ensemble de la
découpe mesure 25 cm et cela ne peut pas être le cartouche d’un texte.
C’est le cartouche qui accompagne ces petites représentations. A côté
du cartouche il y a une ligne jaune bordée de rouge qui est le bâton.
Il y a plusieurs possibilités mais le plus probable est que cela
provienne du commencement du 4° couloir de la tombe. 

En plus des cartouches des hiéroglyphes ont été retirés. On retrouve des
découpes en rond. Voir un fragment provenant de Copenhague. C’est un
déterminatif divin à fond rouge ce qui n’est pas très fréquent dans la
tombe. C’est plus particulièrement dans le rituel de l’ouverture de la
bouche. Il y a le reste d’un serpent et c’est probablement une graphie
de jt, le père. On voit bien que sur les dessins de Belzoni cela figure
en rouge.  Mais attention car parfois les couleurs des dessins de
Belzoni ne sont pas toujours exactes. Du fait du rebouchage  il n’y a
plus les empreintes en creux qui auraient pu faciliter le repérage des
morceaux retirés.

Dans la tombe il y aussi un certain nombre d’exemples d’éléments de
tentatives de découpes. Ce sont souvent des oiseaux, de petits
personnages ... Dans la salle du sarcophage on voit des petits
personnages et des signes qui ont échappé à la découpe. 

Au cours du XX° siècle cette activité de vandalisme est contenue grâce à
la surveillance accrue mais on est en butte à de nouveaux problèmes
matériels.

Des mouvements de terrain vont affecter la vallée et avoir un impact sur
la salle du sarcophage à 2 niveaux. 

Dans la partie haute le pilier E, vu ruinée par Champollion, s’écroule
définitivement emportant une partie du plafond étoilé.

Dans la partie basse on a des fissures sur toutes les parois et pour la
première fois le plafond, intact jusque-là, va souffrir et des morceaux
vont s’écrouler.

Actuellement on voit encore toutes les fissures sur la paroi qui datent
du début du XX° siècle. 

Cet écroulement du pilier et du plafond conduit à envisager la
restauration de la tombe par Carter en 1903, 1904.  Carter publie son
rapport d’activité dans les ASAE 6  de 1905. La restauration consiste a
étayer toutes les parties abimées au moyen de briques. Toutes les
briques sont des restaurations de Carter qui ont tenu jusqu’à ce jour.

Carter indique qu’il a replacé ce qu’il a pu et qu’il a remis les débris
qu’il n’a pas pu replacer à l’entrée du tunnel.

Tout le matériel provenant des dégradations de la tombe, tout ce qui
traine est regroupé derrière la grille qui bouche l’entrée du tunnel.

Ensuite on ne sait ce qui se passe mais en 1960, 1961 il y a les
fouilles égyptiennes du tunnel de Sheik Ali el Rassoul On peut supposer
qu’à ce moment là les débris placés à l’entrée du tunnel par Carter ont
été sortis avec les déblais du tunnel et mis à l’extérieur. 

De 1998 à 2008 la mission MISR de l’Université de Bâle a pratiqué des
fouilles à l’extérieur, à proximité de la tombe de Séthy I. La
concession s’étendait devant l’entrée de la tombe voisine de Ramsès X
jusqu’à la limite de la tombe de Séthy I° d’un côté et en direction de
la montagne de l’autre côté. Au cours de cette mission il a été
retrouvé un ensemble très important de fragments. Cela doit en partie
correspondre aux fragments mis de côté par Carter. Ces fragments sont
souvent à fond jaune, provenant de piliers. Ils viennent probablement
de la salle du sarcophage même si on trouve ailleurs d’autres scènes à
fond jaune comme les reliefs du Louvre et de Florence.

En 1988 et en 1991 il y eu de nouveau des mouvements de terrain qui ont
provoqué par 2 fois des chutes de morceaux du plafond astronomique dans
la salle du sarcophage. Le service des Antiquités a décidé alors de
fermer définitivement la tombe aux visiteurs.

Si on met en parallèle des photos de différentes périodes on constate : 

1970 : collection de Bâle : les dégradations sont à peu près celles
qu’on voyait sur les photos de Burton

Début 1980 : la partie centrale du plafond astronomique avec les textes
correspondant aux légendes des 2 moitiés est déjà sérieusement
endommagée.

Début 1990 : la dégradation s’est étendue. Certains fragments ont été
replacés mais pas toujours au bon endroit. On ne sait pas où sont les
autres fragments.

Suite à la fermeture de la tombe il y a eu un «audit» de la tombe
commandé par le Service des Antiquités. Michael Jones (ARCE) a fait une
étude très complète, étude géologique, étude des parois, études sur les
empreintes. Dans la tombe on a procédé à des tests de restauration qui
montrent que là où les pigments sont conservés la restauration est
possible. 

Les conclusions de ce rapport :

 2350 m2 de surface décorée y compris les plafonds {}- Le travail serait
estimé au minimum à 10 ans {}- c’est à peu près le temps qu’il a fallu
pour la réalisation de la tombe. 

Rien n’a été fait de plus dans la tombe si ce n’est les fouilles dans le
tunnel et des études ponctuelles sur le décor à l’occasion d’autres
travaux dont ceux de la mission de Bâle sur les fragments retrouvés.

LES INNOVATIONS de la KV 17 :

C’est la première tombe entièrement décorée. 

Avant on ne décorait que la salle du puits, l’antichambre et la salle du
sarcophage dans les tombes achevées. On voit cela depuis Thoutmosis III
et jusqu’à Horemheb.

C’est la première tombe dont la salle du sarcophage présente un plafond
vouté dans la partie inférieure. 

Ces 2 innovations majeures rendent cette tombe célèbre. Ce sont 2
innovations qui n’étaient probablement pas prévues à l’origine.

Le décor a t’il été pensé dans sa totalité dés le début ? Pour répondre
à cette question il faut faire un détour par les cartouches de Séthy I.


On connaît le nom de Séthy qui est formé sur le nom du dieu Seth :

[Warning: Draw object ignored]  ou [Warning: Draw object ignored]  

\begin{figure}[htp]
\centering
 [Warning: Image ignored] % Unhandled or unsupported graphics:
%\includegraphics[width=2.399cm,height=0.741cm]{KheopsBarberioSeti11x220120524-img11.png}

\end{figure}
\begin{figure}[htp]
\centering
 [Warning: Image ignored] % Unhandled or unsupported graphics:
%\includegraphics[width=2.787cm,height=0.741cm]{KheopsBarberioSeti11x220120524-img12.png}

\end{figure}
stXy-mr(w)-n ptH {}- c’est écrit avec le nom du dieu Seth et soit le
signe de la houe soit le signe du canal. 

 son nom de couronnement  [Warning: Draw object ignored]  ou  [Warning:
Draw object ignored]  mn-m3at-ra

\begin{figure}[htp]
\centering
 [Warning: Image ignored] % Unhandled or unsupported graphics:
%\includegraphics[width=1.799cm,height=0.741cm]{KheopsBarberioSeti11x220120524-img13.png}

\end{figure}
\begin{figure}[htp]
\centering
 [Warning: Image ignored] % Unhandled or unsupported graphics:
%\includegraphics[width=1.799cm,height=0.741cm]{KheopsBarberioSeti11x220120524-img14.png}

\end{figure}
On constate que dans la tombe de Séthy on n’a pas noté le nom  de Séthy
avec l{\textquotesingle}idéogramme du dieu Seth mais on l’a remplacé
par l’idéogramme d’Osiris  : [Warning: Draw object ignored]

\begin{figure}[htp]
\centering
 [Warning: Image ignored] % Unhandled or unsupported graphics:
%\includegraphics[width=2.893cm,height=0.741cm]{KheopsBarberioSeti11x220120524-img15.png}

\end{figure}
C’est ce qui a fait que Champollion a lu ce nom comme étant Ousirei. 

Il ne semble pas qu’il ait fait le rapprochement avec Séthy.  C’était
déjà un progrès par rapport à Young qui l’avait nommé Ptany (?). 

On sait qu’en Abydos le signe de Seth est proscrit.  Cela parait logique
que dans un contexte funéraire ou du moins dans un contexte Osirien,
dans une tombe ou dans le temple d’Abydos, il était peu religieusement
correct de faire apparaitre de manière si visible le nom du meurtrier
d’Osiris.

La graphie d’Abydos comme dans ce cartouche [Warning: Draw object
ignored]  ou  [Warning: Draw object ignored]  provient d’un article de
El Sawi dans un supplément aux ASEE 70 de 1987. Il étudie les noms de
Séthy I particulièrement à Abydos mais il indique aussi la graphie de
la tombe. Il propose une lecture qui n’est pas exacte. Il fait le
rapprochement entre le 1° cartouche et le 2°. Lorsqu’on part de la
graphie normale on a le nom de Séthy avec le dieu Seth, les 2 yod puis
mr et ptH parfois associé au déterminatif du dieu Ptah. En faisant le
rapprochement entre les 2 cartouches il les  décompose de la même
manière : avant le double yod on a le signe de la mouvance osirienne,
le noeud tit et on aurait tjty-mr-n  et un signe qui n’est pas Ptah
mais clairement Osiris. Il considère qu’Osiris est à prendre comme un
idéogramme : tjty mr(w)-n-pth-wsjr : Titi aimé de Ptah et d’Osiris. Ce
n’est pas cela mais néanmoins il consigne le fait que dans les graphies
d’Abydos on a des «variantes où on trouve le signe d’Osiris et le noeud
d’Isis accolés  devant le double yod.

\begin{figure}[htp]
\centering
 [Warning: Image ignored] % Unhandled or unsupported graphics:
%\includegraphics[width=2.787cm,height=0.741cm]{KheopsBarberioSeti11x220120524-img16.png}

\end{figure}
\begin{figure}[htp]
\centering
 [Warning: Image ignored] % Unhandled or unsupported graphics:
%\includegraphics[width=2.293cm,height=0.741cm]{KheopsBarberioSeti11x220120524-img17.png}

\end{figure}
En fait c’est cela qui doit faire autorité car en écriture
cryptographique le signe d’Osiris peut valoir pour la valeur S, le
signe du tit pour la valeur T et on a donc S T JJ  = Séthy. Cette
combinaison de signes en se basant sur des éléments osiriens, ce qui
n’est pas un hasard, conduit bien à  donner une lecture S+T+Y = SETY.
On «prononce» le nom de Seth sans avoir à passer par la graphie du
dieu. 

Si on reprend l’exemple commenté par El Sawi, il faut comprendre que
c’est une disposition graphique qui fait que le signe d’Osiris est
placé après Ptah mais  en réalité il fonctionne avec le t et c’est
l’ensemble qui fait STY {}-mr-{}-ptH.

C’est la graphie qui est en vigueur à Abydos et même si on comprend bien
que le nom de Séthy figure à travers cette disposition graphique il
n’en est pas moins vrai que l’on a supprimé le hiéroglyphe de Seth dans
la graphie.

  [Warning: Image ignored] % Unhandled or unsupported graphics:
%\includegraphics[width=11.959cm,height=12.547cm]{KheopsBarberioSeti11x220120524-img18}
 

Dans la tombe de Séthy I  on n’a pas employé cette graphie mais on a
supprimé le hiéroglyphe de Seth en le remplaçant par Osiris. Là on ne
peut pas parler de jeu graphique. C’est simplement le remplacement du
signe de Seth par le signe d’Osiris. Il y a une impossibilité de faire
figurer le signe de Seth dans la tombe.

C’est le même chose sur le sarcophage. Les planches de Bonami peuvent
enduire en erreur. Sur les dernières planches, on voit les cartouches
avec Seth. En fait Bonami a représenté ce cartouche normal en
illustration d’un petit chapitre consacré au nom de Séthy. C’est sur
les planches mais cela ne correspond pas au sarcophage de Séthy.

Au total sur le sarcophage et la tombe on emploie cette graphie 
[Warning: Draw object ignored].  de manière général.

\begin{figure}[htp]
\centering
 [Warning: Image ignored] % Unhandled or unsupported graphics:
%\includegraphics[width=2.893cm,height=0.741cm]{KheopsBarberioSeti11x220120524-img19.png}

\end{figure}
Cependant on constate qu’il y a des graphies particulières dans la salle
du puits. En effet on a ce cartouche [Warning: Draw object ignored]  où
le dieu Seth est remplacé par le simple dieu assis A40. Cette graphie
est caractéristique du puits mais en regardant de près on voit qu’en
réalité, dans un cas, le première hiéroglyphe gravé en relief était
celui de Seth. La peinture a camouflé cette gravure. Cela se voit bien
en regardant avec des éclairages différents. Cela apparait au {}-dessus
du roi dans la 1° scène du puits à gauche. Les décors commencent
généralement sur le côté  gauche. C’est le cas dans les tombes. Ici
c’est la représentation d’Horus fils d’Isis conduisant le roi devant
Isis. Tout le thème du décor du puits est osirien et ce cartouche avec
le nom de Seth faisait désordre.

\begin{figure}[htp]
\centering
 [Warning: Image ignored] % Unhandled or unsupported graphics:
%\includegraphics[width=2.469cm,height=0.741cm]{KheopsBarberioSeti11x220120524-img20.png}

\end{figure}
Ce qui est curieux c’est que c’est dans une phase ultime qu’il a été
modifié après avoir été tracé en rouge, puis en noir, puis sculpté
avant d’être corrigé à la peinture.

Ceci a été vu par Champollion et par Lepsius. A l’époque le puits était
comblé et il était possible de s’approcher de la paroi. (Voir document)

Lepsius indique aussi qu’il y avait d’abord le cartouche avec Seth et
qu’ensuite cela avait été repassé à la peinture  avec le signe du dieu.
De manière assez curieuse, dans la planches qu’il consacre à cette
scène, il donne le hiéroglyphe de Séthy sans la correction. C’est la
raison pour laquelle certains auteurs s’appuyant sur Lepsius évoque ce
cartouche sans dire qu’il a été corrigé.

Le fait que le nom de Seth a été remplacé par A40 est la 1°
particularité de la graphie du puits mais il y a également pour le nom
de couronnement une graphie particulière de mn-m3at-ra  : 

[Warning: Draw object ignored]  au lieu  de  [Warning: Draw object
ignored]

\begin{figure}[htp]
\centering
 [Warning: Image ignored] % Unhandled or unsupported graphics:
%\includegraphics[width=1.799cm,height=0.741cm]{KheopsBarberioSeti11x220120524-img21.png}

\end{figure}
\begin{figure}[htp]
\centering
 [Warning: Image ignored] % Unhandled or unsupported graphics:
%\includegraphics[width=1.799cm,height=0.741cm]{KheopsBarberioSeti11x220120524-img22.png}

\end{figure}
Pour les spécialistes des monuments de Séthy I, en dehors de la tombe,
cette graphie est caractéristique de la première année du règne de
Séthy (voir P. Brand- téléchargeable sur internet). 

La salle du puits, pour un visiteur, est la première salle que l’on
atteint après les trois premiers couloirs. Ce n’est pas la 1° salle
creusée mais on constate que c’est dans cette salle qu’on a les
graphies les plus anciennes alors que dans les salles antérieures et
notamment dans le 2° couloir on a des graphies les plus récentes. Il
semble bien que le puits ait été la première salle décorée. C’est là où
on a fait le décor avec hésitation sur le cartouche royal du nom de
Séthy et qu’on emploie la forme ancienne pour le nom de couronnement.

Si on a l’esprit que la salle du puits a été la première salle décorée
on comprend mieux que le 2° couloir, situé avant le puits, soit
inachevé. Si on avait commencé le décor par le début de la tombe le 2°
couloir ne serait pas achevé.

Vraisemblablement on a commencé le décor par la salle du puits ce qui
n’est pas surprenant puisqu’on suivait le schéma des tombes
antérieures. On ne compte pas la tombe de Ramsès I qui est petite et
aménagée dans l’urgence. 

Dans la tombe d’Horembeb la première salle décorée est la salle du
puits. L’escalier et les 3 premiers couloirs bien équarris sont restés
blancs. 

  [Warning: Image ignored] % Unhandled or unsupported graphics:
%\includegraphics[width=12.596cm,height=9.539cm]{KheopsBarberioSeti11x220120524-img23}
 

A noter que le puits est décoré avec une couleur bleu caractéristique.
(Il a aussi une couleur bleue dans la chambre du puits d’Amenophis
III).

Séthy I fait un puits à l’identique  avec un programme décoratif
semblable et lié au cycle Osirien.

La 2° salle décorée chez Horemheb est directement l’antichambre qui est
sur un fond bleu. 

Les 2 parois du fond (abimées car passage) présentent :

d’un côté, le roi face à Ptah associé au pilier djed et de l’autre côté,
le roi face à Nefertoum associé au noeud d’Isis.

Dans l’antichambre de Séthy on constate un même type de décor mais sur
fond blanc et avec un traitement iconographique un peu différent. Bien
que ce soit cassé on retrouve Ptah associé au pilier Djed et  Nefertoum
associé au noeud d’Isis. On voit les restaurations de Carter. Cela fait
partie des passages dont Belzoni disait qu’ils avaient été
particulièrement abimés. 

On constate que le programme décoratif est voisin dans les 2 tombes.

  [Warning: Image ignored] % Unhandled or unsupported graphics:
%\includegraphics[width=12.472cm,height=11.211cm]{KheopsBarberioSeti11x220120524-img24}
 

Si on regarde de près on s’aperçoit que, sous la couleur blanche, il y
avait à l’origine une couleur bleue. Il semble donc que l’antichambre
chez Séthy ait été, dans un premier temps, peinte en bleue comme chez
Horemheb. 

Il semble que chez Séthy I on a pensé, au début, réaliser un décor comme
chez ses prédécesseurs en commentant par la salle du puits avec un
décor de scènes divines sur un fond bleu. Puis, après la salle du
puits, on va décorer l’antichambre sur un fond bleu pour ensuite
changer de programme et passer à la couleur blanche

Il faut noter que ce fond bleu sous une couche blanche se voit aussi
dans la 1° salle à piliers dans les parties abimées. Ex à droite, dans
la représentation de la 6° heure LdP avec au registre inférieur les
suivant d’Osiris allongés sur une couche qui est un corps de serpent.
On voit des traces de couleur bleue. Cela a té noté dans les rapports
de M. Jones sur les décorations des tombes royales. Il constate qu’il y
a un fond bleu gris dans 2 salles, la 1° salle à piliers et dans
l’antichambre. Quand on met en parallèle ces salles et la salle du
puits on a l’impression que c’est le même bleu et le blanc ressort un
peu comme le blanc des cartouches qui ont été repeints sur le fond
bleu.

L’idée du déroulement pour Françoise Barbério 

Chez Séthy I on a conçu une décoration, dans un premier temps, dans la
mouvance traditionnelle. 

On décore la 1° salle qui doit être décorée en premier dans une tombe
royale et qui est la salle du puits. Elle est décorée en bleu et avec
la forme ancienne des cartouches qui prouve que c’est la première salle
décorée.

Ensuite on passe à la salle suivante, l’antichambre, avec la couleur
initiale en bleu.

Puis on utilise le bleu dans la salle à piliers.

Puis on généralise le décor à toute la tombe. 

Il reste quelques salles qui n’ont pas pu être décorées en totalité et
restent en dessins préparatoires : le 2° couloir et
l{\textquotesingle}annexe de la 1° salle à piliers. Cette annexe a
probablement été conçue tardivement car on se rend compte que pour
faire son ouverture on a du casser le décor préexistant. Dans les
montants du passage donnant dans cette annexe on a des éléments du LdP
qui auraient du être dans la salle à piliers. Ils sont sur les montants
et en dessins préparatoires. C’est sans doute une réalisation tardive
qui explique son inachèvement.

Cela donne un regard différent sur ce programme décoratif qui sans doute
n’a pas été conçu globalement dès le départ. Cela n’est pas surprenant,
on les voit aussi dans les temples, les réalisations sont souvent
progressives. Cela n’empêche pas que quand on a conçu le décor à
l’échelle de toute la tombe on avait un plan en tête. On a choisi le
décor pour des raisons particulières.

On a toutes les raisons de croire que le plafond vouté ne l’était pas à
l’origine mais qu’il était plat. Ce plafond s’est écroulé et c’est de
ce fait qu’on a conçu la voute. 

Au cours des fouilles de la mission de Bâle on a retrouvé, parmi les
fragments qui proviennent sans doute des choses amassées par Carter et
sorties au moment des fouilles du tunnel, des fragments qui
manifestement appartiennent à un décor qui vient en supplément du décor
que l’on connaît. Ce sont des fragments qui ont un aspect différent
avec des cassures plus nettes et plus blanches. Ils ont des décors avec
des couleurs, quand elles sont conservées, extrêmement vives. On se
retrouve avec des fragments de décor astronomique qu’on ne peut pas
replacer dans le contexte de la tombe car ils font double emploi. Il y
a des restes de fragments qui ont des motifs qui sont encore visibles
aujourd’hui au plafond de la tombe. On constate en plus qu’ils devaient
former un ensemble plat et non avec une courbe.

\end{document}
