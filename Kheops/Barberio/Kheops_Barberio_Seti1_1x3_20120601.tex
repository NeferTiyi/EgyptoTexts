% This file was converted to LaTeX by Writer2LaTeX ver. 1.0.2
% see http://writer2latex.sourceforge.net for more info
\documentclass{article}
\usepackage[utf8]{inputenc}
\usepackage[T3,T1]{fontenc}
\usepackage[french]{babel}
\usepackage[noenc]{tipa}
\usepackage{tipx}
\usepackage[geometry,weather,misc,clock]{ifsym}
\usepackage{pifont}
\usepackage{eurosym}
\usepackage{amsmath}
\usepackage{wasysym}
\usepackage{amssymb,amsfonts,textcomp}
\usepackage{array}
\usepackage{supertabular}
\usepackage{hhline}
\usepackage{graphicx}
\makeatletter
\newcommand\arraybslash{\let\\\@arraycr}
\makeatother
\setlength\tabcolsep{1mm}
\renewcommand\arraystretch{1.3}
\title{}
\begin{document}
 \ \ \ \ \ \ \ \ \ \ \ \ \ \ \ \ \ \ \ \ \ \   \ \   1/06/12

LES RÉSULTATS DE LA MISSION MISR de BÂLE

Les fouilles de la mission MISR de Bâle ont eu lieu sur un programme de
10 ans de 1998 à 2008 et ont été initiées,  juste après le départ du Pr
E. Hornung à la retraite, par certains de ses étudiants. Elles ont été
poursuivies par ses successeurs, Antony P puis par Suzanne Bickel
aujourd{\textquotesingle}hui. 

Il y a un site internet :
http://aegyptologie.unibas.ch/forscung/projekte/misr-mission-siptah-ramses-x/aktuell

Dans ce projet il y avait 2 objectifs parallèles, la tombe de Ramsès X
peu intéressante en soi mais avec tout l’environnement qui a donné des
fouilles intéressantes et la tombe de Siptah (KV 47) qui vise à un
publication nouvelle de cette tombe. Cela devrait bientôt sortir dans
la collection Aegytiaca helvetica -

Le terme «aktuell» n’est plus vraiment d’actualité puisque la mission de
Bâle est passé à un projet nouveau qui a conduit à la découverte de la
KV 64.

 La tombe de Ramsès X (KV18) est la seule tombe représentée sur ce plan
en totalité alors que pour les autres tombe il n’y a que l’entrée de
reproduite. La tombe de Ramsès X est inachevée. Il y a juste l’entrée,
un 1° couloir et l’amorce d’un 2° couloir. Les fouilles ont confirmé
qu’il n’y a rien lié à l{\textquotesingle}inhumation de Ramsès X. Il a
commencé une tombe dans la KV mais ne l’a jamais utilisée et n’a jamais
été enterré dans la KV. 

\begin{figure}[htp]
\centering
 [Warning: Image ignored] % Unhandled or unsupported graphics:
%\includegraphics[width=6.212cm,height=4.304cm]{KheopsBarberioSeti11x320120601-img1.png}

\end{figure}
\begin{figure}[htp]
\centering
 [Warning: Image ignored] % Unhandled or unsupported graphics:
%\includegraphics[width=10.478cm,height=8.137cm]{KheopsBarberioSeti11x320120601-img2.png}

\end{figure}
Le but était de dégager cette tombe et d’en être sûr. 

Le secteur exploré était au niveau des fouilles de part et d’autre de la
tombe de Ramsès X. La tombe s’ouvre en direction du sud de même que
globalement celle de Séthy I (S/SO). L’ouest de la tombe de Ramsès X a
été fouillé et on est tout près de Séthy I. On a trouvé des fragments
en grande quantité. Sur le côté Est, un petit peu partout dans les
déblais,  il y avait des fragments.  

Sous ces déblais, la fouille étant arrivée à son terme, on a découvert
un ensemble de huttes d’ouvriers datant du règne de Ramsès IV. Elles
ont servi de camp pour les ouvriers qui travaillaient à la tombe de
Ramsès IV, tombe qui avait un plan gigantesque à l’origine. C’est
l’époque du doublement des équipes de chantiers de Deir el Medineh.
Andreas Dorn s’est chargé de l’étudier notamment dans le cadre de sa
thèse  qui est publiée dans Aegytiaca helvetica n=° 23. La concession
s{\textquotesingle}arrêtait au chemin (vers Thoutmosis IV) et l’autre
côté n’a pas pu être fouillé. Il est certain que d’autres tombes se
trouvent de l’autre côté.  

  [Warning: Image ignored] % Unhandled or unsupported graphics:
%\includegraphics[width=3.981cm,height=6.105cm]{KheopsBarberioSeti11x320120601-img3}
 

A l’ouest de la tombe de Ramsès X il y avait la plus grande
concentration de fragments ce qui est normal puisque la  

KV 17 de Séthy I est située juste à côté.

En 2004 les américains avaient conçu une sorte de drainage en ciment
pour détourner les eaux de pluie en cas de nouvelles intempéries.
C’était une bonne idée mais ratée du point de vue esthétique. Cela a
été modifié depuis. 

A l’époque de la présence de ce muret, autour de 2004 on voit les
fouilles de cette partie de la concession de Bâle. C’est Elina
Paulin-Grothe qui dirigeait et dirige encore ces travaux.

Aujourd’hui l’aspect de l’extérieur est modifié.

Les fouilles ont conduit à la découverte de fragments en très grande
quantité représentant une masse très  impressionnante. Ils ont été
inventoriés essentiellement entre 2003 et fin 2007 avec encore la
poursuite du travail jusqu’à fin 2010. Le travail a été confié à FB
aidée des étudiants de Bâle.

70\% des fragments font moins de 10 cm. Il y a aussi de très grands
fragments. Parfois les petits fragments se raccordent les uns aux
autres. 

Ils sont stockés à l’intérieur de la tombe de Ramsès X. 

  [Warning: Image ignored] % Unhandled or unsupported graphics:
%\includegraphics[width=10.158cm,height=6.491cm]{KheopsBarberioSeti11x320120601-img4}
 

A l’intérieur, il y a une saillie à l’extrémité du 1° et derrière
commençait le 2° couloir. On a gardé une butte témoin, qui monte haut,
des différentes strates d’alluvions qui avaient encombré la tombeau au
cours du temps. Il a pu être démontré qu’il y avait eu plus d’une
douzaine d’épisodes d’inondation. Dans cette salle il y a une partie
des boites de fragments. La photo montre les boites de fragments de
parois et les boites de fragments de plafond sont ailleurs. 

On a inventorié environ 5000 fragments. 

En plus il y a d’innombrables caisses de fragments non étudiés car n’ont
été inventoriés que les fragments décorés. Il y a 40 caisses de
fragments à surface monochrome avec des restes de blanc ou de jaune.

Il y a d’autres missions qui ont travaillé ponctuellement dont la
mission américaine qui avait mis cet entourage en ciment autour de
l’entrée de la tombe. Ces missions ont trouvé des fragments en moindre
quantité. Certains ont été confiés à la mission MISR. Cela représente
quelques dizaines jusqu’à 100 fragments

Les fragments inventoriés pour la mission MISR :

2605 fragments de parois en bas-relief peint

2280 fragments de plafond seulement peints dont 1274 à décor
astronomique et 1006 à décor étoilé.

Certains éléments provenant de cet ensemble donnent des indications
intéressantes sur la réalisation de la tombe. Au cours de ce travail
d{\textquotesingle}inventaire il est apparu à FB qu’il y a dans cette
documentation 2 types de fragments : 

 des fragments provenant des destructions modernes de la tombe ce qui
était attendu.

 des fragments provenant de remaniements réalisés à l’époque du
chantier.

Les premiers appartiennent aux parties abimées du décor de la tombe. Si
on identifie le décor, grâce aux sources anciennes donnant une idée
précise du décor (Belzoni, Champollion, Lepsius ...) on peut le
replacer, au moins virtuellement, dans la tombe. Avec les restaurations
modernes les trous sont bouchés et il n’y a plus la place pour les
positionner et les fixer. 

Les fragments qui proviennent, selon FB, des remaniements de la tombe à
l’époque antique n’appartiennent pas au décor actuel de la tombe mais à
une version antérieure de ce décor. C’est une version qui a été
détruite au cours du chantier. Manifestement on a conçu quelque chose
pour un plafond ou une paroi avec un décor particulier. Pour une raison
particulière, volontaire ou involontaire, ce décor s’est trouvé détruit
et les fragments ont été évacués à l’extérieur de la tombe dans le
périmètre du chantier de l’époque.  C’est pour cela qu’on les  retrouve
à l’extérieur. Dans les 2 cas identifiés ces fragments font double
emploi avec le décor : c’est un élément de décor non abimé dans la
tombe et on a trouvé un autre fragment avec une autre version. On peut
les identifier mais pas les replacer. 

Ces fragments provenant des remaniements antiques proviennent d’un état
antérieur de la salle du sarcophage. Il y en a 2 types. D’une part des
fragments appartenant à la 1° version du décor astronomique sur un
plafond plat et d’autre part des fragments appartenant à la 1° version
du Livre de l’Amdouat dans la salle du sarcophage au niveau de la paroi
du fond.

LE PLAFOND ACTUEL : 

Il est assez peu abimé. Il était intact à l’époque de Belzoni et de
Lepsius et il est resté assez longtemps en bon état. 

On a vu que les lacunes dataient pour une part du début du XX° siècle et
pour une autre part des années 1980, 1990. 

Il est difficile d’avoir des photos de la totalité de l’ensemble du
plafond. On peut voir qu’en dehors de lacunes circonscrites et de
fissures il est assez bien conservé.

Le plafond astronomique est fait de 2 moitiés qui se répondent en
miroir. 

Du côté du fond on a les constellations du sud qui sont faciles à
reconnaître car plus petites. Il y a un groupe de divinités plus
serrées, un groupe plus important de petit format. 

  [Warning: Image ignored] % Unhandled or unsupported graphics:
%\includegraphics[width=10.726cm,height=8.057cm]{KheopsBarberioSeti11x320120601-img5}
 

Les constellations du sud se reconnaissent à la présence d’Orion qui est
dans sa barque et qui se tourne vers Isis. 

  [Warning: Image ignored] % Unhandled or unsupported graphics:
%\includegraphics[width=15.012cm,height=5.006cm]{KheopsBarberioSeti11x320120601-img6}
 

Les légendes se lisent aussi avec ce qui est dans le petit compartiment
attaché à l’image : on voit que Isis   [Warning: Image ignored]
% Unhandled or unsupported graphics:
%\includegraphics[width=0.686cm,height=0.6cm]{KheopsBarberioSeti11x320120601-img7.png}
   est assimilée à Sothis   [Warning: Image ignored]
% Unhandled or unsupported graphics:
%\includegraphics[width=0.686cm,height=0.6cm]{KheopsBarberioSeti11x320120601-img8.png}
 . Orion   [Warning: Image ignored] % Unhandled or unsupported graphics:
%\includegraphics[width=0.494cm,height=0.691cm]{KheopsBarberioSeti11x320120601-img9.png}
  n’est pas appelé Osiris ici mais on sait qu’Orion est associé à Osiris
d’une manière très étroite. Ce sont les constellations qui représentent
le ciel du sud.  

Les constellations du nord sont plus grandes. On a des divinités
beaucoup plus grandes qui ont des disques sur la tête de part et
d’autre d’un tableau central. Au coeur du tableau central on a la
grande ourse Meskhétyou   [Warning: Image ignored]
% Unhandled or unsupported graphics:
%\includegraphics[width=1.87cm,height=0.603cm]{KheopsBarberioSeti11x320120601-img10.png}
  . Elle est représentée sous forme d’un taureau dans la version de
Séthy I. Dans d’autres versions elle est représentée sous la forme
d’une patte de taureau    [Warning: Image ignored]
% Unhandled or unsupported graphics:
%\includegraphics[width=1.446cm,height=0.543cm]{KheopsBarberioSeti11x320120601-img11.png}
 . La Grande Ourse est reliée à un piquet qui est tenu par une divinité
à tête d’hippopotame qui porte un crocodile sur son dos. 

  [Warning: Image ignored] % Unhandled or unsupported graphics:
%\includegraphics[width=11.786cm,height=8.888cm]{KheopsBarberioSeti11x320120601-img12}
 

En mettant en commun ce qu’on connait des autres versions et des textes
astronomiques on sait que la Grande Ourse était comprise comme devant
être fixée fermement par une divinité qui a ces traits d’hippopotame
ici mais sous lesquels il faudrait reconnaitre Isis. Isis maintient
fermement le piquet d’ancrage de la grande Ourse. La grande Ourse qui
peut être représentée par la cuisse elle-même, symbole de l’ennemi
abattu, peut être comprise comme étant une incarnation de Seth. Seth
n’a qu’une envie, s’échapper et aller menacer Osiris-Orion qui est de
l’autre côté dans les constellations du sud. Isis et un dieu à tête de
faucon qui semble être un Horus sont là pour le maintenir. La Grande
Ourse qui est le point fixe du ciel nocturne à besoin d’être maintenue
sinon c’est le chaos incarné sous cette forme de Seth. C’est quelque
chose d’assez ambivalent.

Ce tableau qui peut avoir des variantes est au centre des constellations
du nord.

On voit la partie basse de la salle du sarcophage avec la partie voutée
: on peut constater que la partie qui correspond aux constellations du
sud avec Isis et Orion est placée au fond de la tombe ce qui correspond
au sud géographique. Il y a une organisation de ce plafond en fonction
de données d’orientation.

\begin{figure}[htp]
\centering
 [Warning: Image ignored] % Unhandled or unsupported graphics:
%\includegraphics[width=7.657cm,height=20.357cm]{KheopsBarberioSeti11x320120601-img13.jpg}

\end{figure}
Aujourd’hui le plafond est globalement plus abimé dans la moitié sud que
dans la moitié nord. Dans la moitié sud on a des lacunes au niveau des
personnages et au niveau du texte. Dans la moitié nord il y a des
lacunes en miroir au niveau du texte mais aucun des personnages n’est
abimé.

QUELQUES FRAGMENTS identifiés appartenant aux lacunes du plafond actuel.


\begin{itemize}
\item un fragment avec de petits points rouges sur fond jaune et une
double ligne rouge, contrastant avec un fond sombre. C’est un élément
de la bande de sable qui  entoure, dans les tombes royales, les
plafonds astronomiques. Elle sert de base aux pieds des personnages. A
différents endroits la bande de sable est abimée. Ce fragment pourrait
être un petit morceau entre les pieds de 2 divinités.
\end{itemize}
\begin{itemize}
\item Un autre fragment avec un motif qui se trouve près d’une ligne
verticale. C’est le haut de la tête d’un personnage avec son oreille.
Cela ne peut donc être qu’un personnage à tête humaine. Il n’y a que
dans le ciel du sud qu’il y a des personnages abimés.  Dans le groupe
avec l’oiseau Benou il n’y a qu’un personnage à tête humaine. Il est
placé derrière le personnage à tête de cynocéphale qui se trouve
derrière le personnage à tête de canidé. 
\end{itemize}
\begin{itemize}
\item   [Warning: Image ignored] % Unhandled or unsupported graphics:
%\includegraphics[width=9.142cm,height=5.697cm]{KheopsBarberioSeti11x320120601-img14}
 
\end{itemize}
Cette oreille ne peut appartenir qu’à ce personnage. Son nom est inscrit
au-dessus de lui    [Warning: Image ignored]
% Unhandled or unsupported graphics:
%\includegraphics[width=0.572cm,height=0.706cm]{KheopsBarberioSeti11x320120601-img15.png}
   imsti : Amset. Cela correspond sans doute à des représentations de
fils d’Horus mais qui ont des noms de constellations. De plus les noms
sont un peu embrouillés suivant les versions. Il est fort probable que
ce reste de personnage soit la tête de cette représentation humaine. On
pourrait imaginer de le replacer.

Si on regarde de près le plafond actuel (voir livre ZH) on voit bien que
les figures ont été peintes directement sur le calcaire. Quand la
peinture s’écaille on voit directement le blanc de la pierre et le fond
foncé qui est un mélange de bleu et de noir a été peint autour des
figures. Il est autour des figures mais pas sous les figures.

Parmi les 1274 fragments inventoriés seuls 9\% correspondent à des
éléments appartenant au plafond actuel et donc 91\% sont incompatibles
avec ce plafond. Pour FB ils forment une autre version, la première
version du plafond de Séthy I : 

\begin{itemize}
\item un fragment montre que le bleu est plus marqué mais surtout qu’il
ressort sous les figures. Cet élément est le reste de 2 pieds qui se
chevauchent et qui sont orientés vers la gauche. Ces pieds relativement
grands (30 cm environ à l’origine) ont des points rouge au niveau des
chevilles. Ces points rouges sont caractéristiques des figures de la
moitié nord. D’une manière générale quand on regarde toutes les figures
de part et d’autre du tableau central, aucune figure grande ou petite
n’est abimée. Ces points rouges correspondent sans doute aux étoiles
qui font partie des constellations matérialisées. Aucune figure à point
rouge de la tombe n’est abimée alors que l’on a un grand nombre de
fragments qui montre les reste de motifs à points rouges. Ils ne
peuvent pas venir du plafond actuel.
\end{itemize}
\begin{itemize}
\item  un autre élément, très petit mais avec un élément très
reconnaissable : un ovale avec un élément central. Il s’agit d’une
écaille de crocodile. Il y a 2 crocodiles dans le ciel du nord, l’un
sur le dos de l’hippopotame et l’autre sous la constellation du lion. 
Ils ne sont pas nommés et on ne sait pas de qui il s’agit. Leurs
écailles sont intactes. Ce sont des restes très minimes mais d’un
crocodile.
\end{itemize}
\begin{itemize}
\item  dans des cas rarissimes il y a des fragments qui ont gardé des
restes de signes. Sur certains fragments il est très clair que l’on eu
besoin d’égaliser la surface du plafond plus que dans le plafond
actuel. On trouve souvent une fine couche de mortier sur laquelle on a
peint. Dessous on voit les traces du ciseau du sculpteur qui a égalisé
la paroi sur laquelle on va placer l’enduit puis peindre. Aujourd’hui
pour peu que l’enduit ne tienne pas ce qui le cas de la plupart des
fragments, les choses sont perdues. Il y a de très nombreux fragments
qu’on reconnait comme étant des fragments de plafonds (petite trace
bleue) mais non identifiables. Cela est très fragile et part en
écailles. Ex : on reconnait  un   [Warning: Image ignored]
% Unhandled or unsupported graphics:
%\includegraphics[width=0.143cm,height=0.6cm]{KheopsBarberioSeti11x320120601-img16.png}
   et le haut de la chouette. 
\end{itemize}
\begin{flushleft}
\tablehead{}
\begin{supertabular}{|m{8.872cm}|m{4.5080004cm}|}
\hline
Cette combinaison    [Warning: Image ignored]
% Unhandled or unsupported graphics:
%\includegraphics[width=0.67cm,height=0.563cm]{KheopsBarberioSeti11x320120601-img17.png}
   on ne l’a dans tout le plafond qu’à 2 endroits et dans la moitié sud
et dans la moitié nord. C’est la notation de smd qui est le nom d’un
décan. Il y a un smd du nord et un smd du sud. Pour le premier on voit 
 [Warning: Image ignored] % Unhandled or unsupported graphics:
%\includegraphics[width=0.882cm,height=0.579cm]{KheopsBarberioSeti11x320120601-img18.png}
   smd mHt(y) puis un Douamoutef qui doit correspondre à autre chose. 
On le retrouve intact dans la tombe et ce fragment ne vient pas de la
tombe. On retrouve la même combinaison dans la moitié sud : smdt tpya 
est écrit avec le m sous le s mais c’est suivi de smd   [Warning: Image
ignored] % Unhandled or unsupported graphics:
%\includegraphics[width=0.564cm,height=0.658cm]{KheopsBarberioSeti11x320120601-img19.png}
  . Ce fragment pourrait provenir de la notation du nom de ce décan dans
la moitié nord ou la moitié sud mais les 2 sont intacts sur le plafond
actuel. Il n’y a pas d’autre endroit où se retrouve cette combinaison.

 &
  [Warning: Image ignored] % Unhandled or unsupported graphics:
%\includegraphics[width=4.307cm,height=11.945cm]{KheopsBarberioSeti11x320120601-img20}
 \\\hline
\end{supertabular}
\end{flushleft}
\begin{itemize}
\item technique picturale : on voit 2 pieds (10 cm environ)sur la bande
de sable avec le ruban et un petit point rouge dans la bande de sable.
Il s’agit probablement de pieds appartenant à une petite figure (10 cm
environ). Ces petites figures très serrées sont dans la moitié sud.
Cela prouve qu’il y avait déjà une moitié sud et pas seulement une
moitié nord. La couche bleue a été reproduite dans la totalité, ensuite
on a reproduit les figures. Au endroits où la peinture s’écaille on
voit le bleu en dessous.
\end{itemize}
\begin{itemize}
\item  sur le plus grand fragment très écaillé on voit un motif de bande
de sable et un motif étoilé en dessous.
\end{itemize}
On a vu qu’il y a avait une incompatibilité et qu’on ne pouvait pas
replacer ces fragments dans la tombe de Séthy I. 

La question est de savoir si ces fragments pouvaient provenir d’une
autre tombe de la KV. La réponse est négative.

Dans les tombes antérieures à Séthy I il n’y a pas de plafond
astronomique. 

Dans les tombes postérieures à Séthy I et présentant un décor
astronomique il n’y a aucune tombe où se retrouve cette technique
picturale de figures peintes sur un fond bleu posé préalablement. Dans
toutes les autres le fond est  peint  après les figures.

Le seul doute possible pourrait provenir de la tombe de Ramsès II (KV7).
La tombe de Ramsès II est très abimée surtout dans sa moitié inférieure
et il n’y a plus de représentation astronomique sur la voute. On
pourrait envisager un plafond astronomique perdu mais Christian Leblanc
confirme que la voute est très lisse. Les piliers sont écroulés mais la
voute elle-même dans sa globalité est relativement bien conservée. Les
fragments retrouvés représentent une grosse masse car en plus des
fragments inventoriés il y a tous les fragments simplement  peints en
bleu et qui n’entre pas dans le décompte inventorié. Certains de ces
fragments ont une certaine épaisseur et ne peuvent être tombés sans
laisser de marque sur la voute. Il y aurait des trous et non un plafond
lisse. Christian Leblanc dit que cela ne peut pas provenir de la tombe
de Ramsès II. Il a également dit que lorsqu’il avait dégagé la salle du
sarcophage il y avait, de manière assez uniforme, un reste d’enduit un
peu dilué qui recouvrait la presque totalité de la surface de la salle
et qui est parti est poussière.  Il pense que c’est l’enduit peint du
plafond qui serait tombé et se serait désagrégé au sol. On peut
imaginer qu’il y ait eu chez Ramsès II un décor astronomique disparu
mais que c’est la couche picturale qui a disparu. Cela ne peut pas
corresponde aux fragments retrouvés. 

Ces morceaux ne peuvent appartenir qu’à la tombe de Séthy I.  Il y a clé
de compréhension grâce au fragment D137 

vu précédemment. On a remarqué qu’en plus des pieds sur la bande de
sable il y avait des étoiles.  

Sur ce fragment d’abord uniformément bleu on a peint d’un côté un motif
de bande de sable qui est la bordure du plafond astronomique. Juste à
côté on a un motif d’étoiles en rang serré qui évoque plutôt un plafond
étoilé. Si on regarde la photo de la totalité du plafond à aucun moment
il n’y a de groupements d’étoile à côté de la bande de sable. 

Le plafond étoilé serait le plafond soutenu par les piliers de la partie
haute de la salle du sarcophage. Entre la bordure de sable et le ciel
étoilé il y a un mur qui ne permet pas la juxtaposition entre ces
éléments dans le contexte actuel. Dans les autres tombes royales, que
de toute façon la technique de peinture rend incompatibles avec cet
élément, il n’y a pas  non plus cette juxtaposition. 

Cette juxtaposition ne peut s’expliquer pour FB que si on imagine qu’il
y avait un plafond plat à l’origine. 

PLAN : quand on compare le plan en coupe de la tombe d’Horemheb et de la
tombe de Séthy I on constate que la tombe de Séthy a une conception de
la salle du sarcophage très voisine de celle d’Horemheb. Il y a le
passage entre l’antichambre et la salle du sarcophage, l’emplacement
des piliers et la partie basse de la salle du sarcophage. 

Chez Horemheb il y a bien une distinction entre partie haute à piliers
et partie basse mais il y a un trait d’union entre les 2 éléments de la
salle qui est le plafond plat sur toute la superficie de la salle du
sarcophage. Chez Séthy I il y a un même profil en dehors de la présence
de la voute. Pour FB on pourrait imaginer que la tombe de Séthy I était
conçue sur le même modèle que la tombe d’Horemheb avec un plafond
totalement plat. Dans ce cas on peut imaginer que si on avait un
plafond étoilé dans la partie haute à piliers avec, dans le
prolongement, un plafond astronomique 1° version dans la partie basse
le fragment D137 pourrait y trouver sa place. Il faut imaginer que pour
une raison inconnue le plafond s’est écroulé. Cela expliquerait
l’abondance de la documentation. Le plafond s’étant écroulé les
fragments auraient été évacués à l’extérieur de la tombe et récupérés
des siècles plus tard. Cela pourrait être une explication à cette
documentation avec des fragments qui ne peuvent pas appartenir au
plafond de la tombe actuelle de Séthy I ni au plafond d’une autre tombe
de la KV. Ils pourraient appartenir à la 1° version du plafond de Séthy
I.

\begin{flushleft}
\tablehead{}
\begin{supertabular}{|m{7.559cm}|m{10.431cm}|}
\hline
Il y a un précédent connu de plafond plat à décor astronomique. Il
s’agit de la tombe de Senenmout (TT353). La version de Senenmout est,
dans l’état de nos connaissances, la toute première version d’un
plafond astronomique.

 &
  [Warning: Image ignored] % Unhandled or unsupported graphics:
%\includegraphics[width=10.231cm,height=6.241cm]{KheopsBarberioSeti11x320120601-img21}
 \\\hline
\end{supertabular}
\end{flushleft}
On peut noter certaines différences avec Séthy I. Il y a bien les 2
moitiés mais elles ne sont pas en miroir et on n’a pas l’entourage
d’une bande de sable mais des étoiles.  La bande de sable est
caractéristique des Livres funéraires du monde souterrain. Avoir un
entourage d’étoiles parait plus logique pour un plafond astronomique
qu’une bande de sable. On peut se demander si dans les tombes royales
l{\textquotesingle}iconographie des Livres du monde souterrain n’a pas
contaminé ce plafond astronomique.

\begin{figure}[htp]
\centering
 [Warning: Image ignored] % Unhandled or unsupported graphics:
%\includegraphics[width=6.387cm,height=5.75cm]{KheopsBarberioSeti11x320120601-img22.png}

\end{figure}
L’idée d’un décor astronomique sur un plafond plat, dans la tombe de
Séthy I, n’est pas totalement absurde si l’on considère que le seul cas
antérieur de décor astronomique est sur un plafond plat.

\begin{flushleft}
\tablehead{}
\begin{supertabular}{|m{8.919001cm}|m{8.937cm}|}
\hline
Il y a un autre élément de comparaison intéressant dans la tombe de
Siptah (KV47), tombe qui fait l’objet de l’autre étude de la mission.
On a pu constater que tous les couloirs et toutes les salles au-delà de
la 1° salle à piliers avaient un plafond détruit et cela donne en coupe
un aspect un peu bizarre. La tombe de Siptah communiquait, suite à une
collision, avec une petite tombe de la XVIII° D. A l’occasion de pluies
torrentielles la tombe a été abimée par de l’eau qui a coulé depuis la
petite tombe. Elle a été totalement envahie par les eaux à plusieurs
reprises ce qui a conduit à des destructions curieuses dans la partie
inférieure. Ce qu’on voit sur le plan de K. Weeks ce sont les plafonds
des couloirs qui ont éclaté en forme de voute. 

 &
  [Warning: Image ignored] % Unhandled or unsupported graphics:
%\includegraphics[width=5.262cm,height=7.814cm]{KheopsBarberioSeti11x320120601-img23}
 \\\hline
\end{supertabular}
\end{flushleft}
Au total un plafond plat qui s’écroule peut donner une voute. Quand on
voit l’aspect des couloirs  actuellement on a l’impression d’être dans
une  grotte. Aujourd’hui on n’a pas une vision claire de la tombe de
Siptah car le sol est encombré de déblais sur lesquels a été posé le
plancher. Cela rehausse le niveau du sol. De plus les plafonds se sont
cassés en forme de voute et sont rehaussés. Ce que l’on voit
aujourd’hui n’est pas conforme car sols et plafonds sont rehaussés. Il
y a un endroit où il a été possible de déterminer l’angle originel du
plafond.

Cet exemple permet d’imaginer qu’un plafond plat qui s’écroule peut
donner une courbe. Cela permet d’imaginer que chez Séthy I si le
plafond plat s’est écroulé on a pu avoir une voute naturelle qui s’est
formée. On a pu alors transformer cette voute naturelle dans la forme
de la voute actuelle.

  [Warning: Image ignored] % Unhandled or unsupported graphics:
%\includegraphics[width=7.151cm,height=6.553cm]{KheopsBarberioSeti11x320120601-img24}
 

Chez Séthy I c’est la 1° voute que l’on connaisse et elle est placée de
manière décentrée dans la salle du sarcophage au-dessus de la partie
basse, partie basse qui ressemblait à la partie basse de la tombe
d’Horemheb. A partir de Ramsès II , Merenptah et les tombes suivantes
on a fait une voute mais on l’a placée dans la partie centrale. Séthy I
est le 1° exemple mais c’est aussi le seul exemple de ce type de voute.
Peut-être parce que cela n’avait pas été prévu au départ. Ensuite quand
on a conçu une voute on l’a fait de manière différente.

En conclusion pour FB ces vestiges seraient les restes de la toute
première attestation d’un plafond astronomique dans la KV. Il aurait
été réalisé à l’origine sur un plafond plat. Le plafond actuel de Séthy
I serait le 2°. Sur aucun des fragments retrouvés on ne constate de
courbe même si cela n’est pas évident sur de petits fragments.

Toute cette documentation est difficile mais elle a permis de découvrir
des éléments de cette tombe qui n’étaient pas prévus.

Dans la tombe de Séthy I on a un certain nombre
d{\textquotesingle}innovations : une tombe entièrement décorée, une
première voute, un premier plafond astronomique. Tout cela ne s’est pas
fait en une fois. L’histoire de la réalisation de cette tombe se révèle
plus complexe et on peut en partie la réécrire.  

ELEMENTS du PROGRAMME DECORATIF : 

L’orientation de la tombe  réelle et symbolique : 

L’orientation réelle : la tombe regarde en gros vers le S/SO. Cette
orientation a été respectée pour la mise en place du plafond
astronomique puisque la partie du ciel du sud est du côté du fond de la
salle du sarcophage, au sud. 

Pour la 1° fois dans une tombe royale on a des marqueurs d’orientation
symbolique qui sont placés sur l’axe de la tombe. Même si la tombe
n’est pas parfaitement rectiligne les parties haute et basse sont sur
le même alignement.

Ces marqueurs d’orientation étaient placés de part et d’autre de
l’antichambre mais actuellement on ne les voit quasiment plus. C’est
pour cela que l’étude des fragments est intéressante.

\begin{flushleft}
\tablehead{}
\begin{supertabular}{|l|l|}
\hline
  [Warning: Image ignored] % Unhandled or unsupported graphics:
%\includegraphics[width=10.294cm,height=8.154cm]{KheopsBarberioSeti11x320120601-img25}
  &   [Warning: Image ignored] % Unhandled or unsupported graphics:
%\includegraphics[width=4.565cm,height=7.14cm]{KheopsBarberioSeti11x320120601-img26}
 \\\hline
\end{supertabular}
\end{flushleft}
Dans l’antichambre, au sortir du 5° couloir : on avait à l’origine des
représentations de la déesse Maât de part et d’autre, deux en vis à vis
en entrant dans l’antichambre. Maât était associée d’un côté avec la
déesse du sud Nekhbet et de l’autre côté avec la déesse du nord Ouadjet
et avec les plantes héraldiques.

Il y avait le même schéma dans l’autre passage menant de l’antichambre à
la salle du sarcophage.

Au total 4 représentations de Maât et du côté gauche pour les visiteurs
la déesse Nekhbet du sud et du côté droit la déesse Ouadjet du nord. 

Actuellement : il ne reste pas grand chose. 

\begin{itemize}
\item Passage H/I : d’un côté il reste le nom de la déesse Maât. Il y
avait la déesse Nekhbet sur son bouquet de «lys» et il ne reste que le
haut de sa coiffure. De l’autre côté il ne reste rien de la déesse Maât
et uniquement le nom et le haut de la couronne rouge de Ouadjet.
\end{itemize}
  [Warning: Image ignored] % Unhandled or unsupported graphics:
%\includegraphics[width=10.158cm,height=6.491cm]{KheopsBarberioSeti11x320120601-img27}
  Passage H/I actuel.

\begin{itemize}
\item Passage I/J il reste un peu plus de matière. Sur le côté gauche il
reste un morceau de la robe de Maât. Nekhbet est dans toute sa
splendeur et son intégralité : la déesse cobra avec sa coiffure sur un
signe nb et le tout sur les plantes héraldiques, le lys du sud. De
l’autre côté il ne reste plus rien excepté une ombrelle de papyrus très
peu visible.
\end{itemize}
Avec les éléments du nord à droite en entrant et les éléments du sud à
gauche cela oriente symboliquement la tombe vers l’Occident ce qui
prend tout son sens dans un contexte funéraire. Cela est tellement vrai
et important que dans les tombes ultérieures, à partir de  Ramsès II,
et jusqu’à Ramsès III ces marqueurs d’orientation vont être remontés à
l’entrée de la tombe. Les représentations de plantes héraldiques avec
Ouadjet et Nekhbet se trouvent à l’entrée des tombes royales.

Restes de ces éléments. Dans les fouilles on a retrouvé des fragments
correspondant au bouquet de papyrus. Le bouquet de papyrus prend racine
dans une sorte de tertre caractéristique dont on a les restes sur un
fragment jointif avec les tiges. Le 1° fragment a été trouvé dans les
fouilles et l’autre dans l’annexe Jb. C’est pour cela qu’ils ont des
numérotations différentes. Ces 2 éléments sont sans doute la tige
montante du papyrus central avec les 2 petits morceaux du calice.

Heureusement qu’il y avait les planches publiées en 1820 par Belzoni.
Sur une planche il a regroupé les représentations côté droit du 1° et
2° passage. 

De ces éléments  on a d’autres fragments notamment dans les musées : 

\begin{itemize}
\item du passage H/I on a un fragment qui se trouve au BM (884) qui
correspond à un détail de la déesse Maât qui est appelée   [Warning:
Image ignored] % Unhandled or unsupported graphics:
%\includegraphics[width=1.199cm,height=0.6cm]{KheopsBarberioSeti11x320120601-img28.png}
   Hnw.t t3.wy. On retrouvera cette épithète sur un autre fragment en
vis à vis. Ce fragment a en plus la particularité très intéressante
d’avoir une notation en hiératique qui donne le nom de la salle  wsxt 
m3at, la salle de la Maât. Cette antichambre est encadrée par des
représentations de Maât à l’entrée et à la sortie. Cette désignation de
«chambre de Maât» se retrouve dans certains documents relatifs aux
tombes royales et notamment sur le plan de Turin très mité publié par
Sarah ? où on a l’indication wsx.t-m3a.t  pour l’antichambre.
\end{itemize}
Ce fragment est en réalité un fragment d’angle ce qui est normal
puisqu’il s’agit d’un passage. Il y a des restes de cartouches que l’on
voit de face. Cela montre l’emplacement de ce fragment sur le côté
droit de l’antichambre.

  [Warning: Image ignored] % Unhandled or unsupported graphics:
%\includegraphics[width=7.643cm,height=10.058cm]{KheopsBarberioSeti11x320120601-img29}
 

\begin{itemize}
\item Parmi les fouilles d’Otto Schaden il y a des fragments retrouvés
un peu plus bas dans la vallée et qui ont été confiés pour étude. C’est
un fragment d’angle complètement décoloré qui provient d’après FB de la
désignation Hnwt t3wy Hryt tp snyt qui est l’exact pendant de la
désignation de Maât qu’on a sur le fragment du BM. Ce fragment pourrait
appartenir au montant de gauche en entrant dans l’antichambre. 
\end{itemize}
\begin{itemize}
\item un autre petit fragment du BM (1374) pourrait correspondre à cette
Maât en vis à vis. Ce serait le bout de son bras, le reste du  
[Warning: Image ignored] % Unhandled or unsupported graphics:
%\includegraphics[width=0.635cm,height=0.607cm]{KheopsBarberioSeti11x320120601-img30.png}
   et les signes    [Warning: Image ignored]
% Unhandled or unsupported graphics:
%\includegraphics[width=0.529cm,height=0.617cm]{KheopsBarberioSeti11x320120601-img31.png}
   qui forme le groupe ns.t du trône d’Osiris.
\end{itemize}
Il y a quelques maigres vestiges de ces 2 montants qui formaient le
décor de ce 1° passage H/I. Pour le second passage  I/J :

\begin{itemize}
\item \end{itemize}
\begin{flushleft}
\tablehead{}
\begin{supertabular}{|m{8.873cm}|m{8.91cm}|}
\hline
\begin{itemize}
\item il y a le fragment de Florence (2469). C’est un autre fragment
prélevé ou ramassé à l’occasion de l’exposition Franco-Toscane. Il
représente la déesse Maât sur un fond blanc. On voit, sous le blanc,
qu’il y avait cette couche bleue identique à celle du puits. Il serait
à replacer sur le montant gauche du passage dans le représentation de
Maât dont il reste dans la tombe seulement un bout de la
robe.\end{itemize}
 &
\begin{itemize}
\item   [Warning: Image ignored] % Unhandled or unsupported graphics:
%\includegraphics[width=4.754cm,height=6.877cm]{KheopsBarberioSeti11x320120601-img32}
 \end{itemize}
\\\hline
\end{supertabular}
\end{flushleft}
\begin{itemize}
\item pour l’autre côté le Pr Hornung a retrouvé un fragment à Cracovie.

\end{itemize}
  [Warning: Image ignored] % Unhandled or unsupported graphics:
%\includegraphics[width=8.827cm,height=9.881cm]{KheopsBarberioSeti11x320120601-img33}
 

\begin{itemize}
\item Ce fragment a posé problème car il apparait bleu et dans un
premier temps il avait été associé à la salle du puits mais on ne
savait pas où le placer. En fait en regardant de près il y a des traces
de peinture blanche sur le bleu. De plus ce sont des mêmes khekerou
verts caractéristiques du 2° passage. Au 1° passage ils sont bleus. Ce
fragment peut donc se replacer dans la succession des khekerou verts du
2° passage. On peut le replacer virtuellement. On retrouve le même
entourage que pour le fragment de Florence et on peut penser qu’il
s’agit de son pendant de l’autre côté de l’entourage d’une Maât
elle-même disparue. 
\end{itemize}
On n’a pas assez de fragments pour faire une reconstitution des
représentations. Ce sont 2 passages de portes qui sont abimés depuis
Belzoni et il est possible qu’on puisse encore en retrouver. On peut
aussi imaginer que ces montants n{\textquotesingle}étaient pas en
calcaire plein et que peut-être il y avait de l’enduit de bouchage et
dans ce cas le décor se serait dissous rapidement avec les pluies.

Ces fragments sont très importants. On arrive à en replacer et cela a
donné la clé de la compréhension d’une salle. Ce nom de salle de Maât
sur un fragment isolé n’est pas parlant tant qu’on ne l’a pas
positionné.

\end{document}
