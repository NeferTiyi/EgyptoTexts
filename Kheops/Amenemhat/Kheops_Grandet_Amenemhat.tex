\documentclass[%
  ngerman, %
  dvipsnames, %
  svgnames, %
  a4paper, %
  twoside, %
  10pt, %
  openany, %
  article, %
]{memoir}

\usepackage[firstpage=true,angle=0,opacity=0.75]{background}
\usepackage{pdfpages}
\usepackage{nefermemoir}
\usepackage{nefertiyi}
\usepackage{neferhiero}
\usetikzlibrary{%
  backgrounds,
  shadows,
  % calc,
  % patterns,
  % arrows,
  % positioning,
  % decorations.text
}

\settypeblocksize{22cm}{15cm}{*}
\setulmargins{*}{*}{1}
\setlrmargins{*}{*}{1.3}
% \setmarginnotes{17pt}{51pt}{\onelineskip}
\setheadfoot{\onelineskip}{5\onelineskip}
\setheaderspaces{*}{2\onelineskip}{*}
\checkandfixthelayout

\newcommand{\DirUtils}{../../../utils}
\newcommand{\DirImage}{../../../images}

\graphicspath{%
  % {\DirUtils/}%
  {\DirImage/Kheops/Amenemhat/}%
}

\sideparmargin{outer}

% \setlength{\parindent}{0pt}
\setlength\fboxsep{0.5mm}
\addtolength{\columnsep}{15pt}
%\setlength\tabcolsep{0mm}
%\setlength\parskip{1.0\baselineskip}

\frenchbsetup{ReduceListSpacing=false,CompactItemize=false}

\input{\DirUtils/ChapterStyles}
\input{\DirUtils/TitlepageKheops}
\input{\DirUtils/HeadingsKheops}

\renewcommand{\appendixpagename}{Annexes}
\renewcommand{\appendixtocname}{\appendixpagename}
\renewcommand{\sectionrefname}{\S\,}

% \input{Kheops_Grandet_Glossaire}

\chapterstyle{souligne}
% \chapterstyle{daleif3}
% \chapterstyle{monchap}
\pagestyle{ruled}

\addbibresource{\DirUtils/Hiero.bib}
\setlength\bibitemsep{0.5\baselineskip}



\makeatletter
  \AtEndDocument{%
    % .. Quatrième de couv' ..
    \cleardoublepage
    \pagestyle{empty}
    ~\newpage
    ~\BgThispage
  }
\makeatother



%======================================================================
\title{L'enseignement \mbox{d'Amenemhat~I\ier} à son fils}
\short{L'enseignement d'Amenemhat~I\ier à son fils}
% \subtitle{de la Préhistoire au \NK}
\subtitle{}
\orga{Cours de \textsf{l'\IK} par}
\lecturer{\PG}
\author{Sonia~\bsc{Labetoulle}}
\date{janvier -- juin 2015}

\hypersetup{%
  pdftitle  = {L'enseignement d'Amenemhat~Ier à son fils}, 
  pdfauthor = {Sonia Labetoulle}
}
% pdfsubject % pdfcreator % pdfproducer % pdfkeywords

%======================================================================

\backgroundsetup{%
  contents = \includegraphics{Amenemhat_ostracon},
  position = {17,-35},
  scale    = 0.5
}

%%%%%%%%%%%%%%%%%%%%%%%%%%%%%%%%%%%%%%%%%%%%%%%%%%%%%%%%%%%%%%%%%%%%%%%
\begin{document}

\input{\DirUtils/NeferBabel}

\thispagestyle{empty}
\maketitle
%%%%%%%%%%%%%%%%%%%%%%%%%%%%%%%%%%%%%%%%%%%%%%%%%%%%%%%%%%%%%%%%%%%%%%%

\TextHieroglyphs

\frontmatter
\tableofcontents*

\mainmatter


\chapter[Introduction]{\protect\faitle{06/01/2015}Introduction}

Amedeo~\bsc{Peyron}

\foreignlanguage{ngerman}{Der Text der "`Lehre Amenemhets~I. für 
seinen Sohn"'}


\chapter{Traduction}

\section{Paragraphe~I}

Ce premier paragraphe est le titre du texte. Il le caractérise comme 
un testament politique.

\begin{hierobox}
  \tl{\lgn{Ia} HA.t-a m sbAy.t \scansion%
      jr(w).t\col n Hm n(y) (ny)-sw.t bjty %
      \crtch{\%Htp(w)-jb-Ra} \scansion %
     }

  \tg{participe perfectif passif}

  \td{Début de l'enseignement qui a été fait par la Majesté du roi 
      de \HBE Séhétepibrê (\litt celui qui apaise le c{\oe}ur de Rê)}%

  \tcblower

  \tl{sbA} : c'est l'idée de guider, de chemin.
\end{hierobox}

\begin{hierobox}
  \tl{\lgn{Ib} sA Ra \crtch{Jmn-m-HA.t} \scansion mAa-xrw \scansion}

  \tg{}

  \td{le fils de Rê Amenemhat (\litt Amon est en avant), à la voix 
      juste}%

  \tcblower

  Amenemhat est le premier roi à avoir Amon dans son nom. Le dieu 
  apparaît au \MK, mais les rois de la \dyn{xi} ont privilégié Montou, 
  dieu de Médamoud où ils avaient un palais.

  \tl{mAa-xrw} : le roi est défunt, c'est un enseignement 
  d'outre-tombe.
\end{hierobox}

\begin{hierobox}
  \tl{\lgn{Ic} Dd=f m wp.t mAa.t \scansion %
      n sA=f (m) Nd-r-Dr \scansion}

  \tg{prospectif séquentiel consécutif}

  \td{voulant révéler oralement (\litt dire) la manière adéquate 
      de se comporter pour son fils, en tant que \frquote{seigneur 
      universel} (\litt jusqu'à la limite (du monde connu))}%

  \tcblower

  \tl{wp}, c'est l'idée de séparer, départager. Ça va avec l'idée de 
  Maât et les procédures contradictoires. \tl{wp mAa.t}, c'est faire 
  la part de Maât dans un jugement.

  Ajout d'un \tl{(m)} assimilé au \tl{n} initial de \tl{Nb-r-Dr}. 
  L'expression se rapporte alors au roi et non à son fils. Mort, il 
  est assimilé à Rê.

  Le roi d'\kmt, c'est un ornithorynque : il est inclassable, ni tout 
  à fait homme, ni vraiment dieu. C'est un homme avec un attribut 
  divin --~la royauté. Les textes égyptiens disent qu'il est le seul 
  de son espèce.
\end{hierobox}

\begin{hierobox}
  \tl{\lgn{Id} Dd=f (jw=f) xa(=w) m nTr \scansion}

  \tg{prospectif, accompli séquentiel}

  \td{voulant dire, après sa mort (\litt étant apparu comme un dieu),}%
\end{hierobox}

\begin{hierobox}
  \tl{\og sDm n Dd.tj=j n=k \scansion %
      n(y)-swty=k tA \scansion %
      hoAy=k jdb.w\scansion %
      \lgn{Ie} jr=k HA.w Hr nfr \scansion}

  \tg{discours direct à l'impératif, et prospectifs consécutifs}

  \td{\og écoute ce que je vais te dire, et tu seras roi du pays,
      tu dirigeras les rives, et tu ajouteras un excédant aux biens.}%
\end{hierobox}

\begin{hierobox}[breakable]
  \tl{\lgn{IIa} sAq tw r "smd.t rf tm(w).t xpr(w) \scansion 
      \lgn{IIb} tmm(w).t rd(w) jb m sAH Hr.w=s \zero" \scansion}

  \tg{impératif, }

  \td{Garde-toi du précepte ``Comme c'est un groupe de serviteurs 
      qui n'est pas encore advenu, c'est un groupe de personnes des 
      complots de qui on ne s'est jamais préoccupé''}%

  \tcblower

  \begin{description}
    \item [anaphore] le fait qu'un pronom se réfère à quelque chose 
          qui précède ;
    \item [kataphore] le fait qu'un pronom se réfère à quelque chose 
          qui suit.
  \end{description}

  Ce passage est généralement rendu comme \tl{sAq tw r smd.t {r=f} 
  tm(w).t xpr(w)}, \td{Garde-toi d'un groupe de serviteurs qui n'est 
  pas encore advenu}, mais ça ne tient pas compte de 
  \EnPetit{\begin{hieroglyph}{\leavevmode \Cadrat{\CadratLineI{\Aca GD/52/}\CadratLine{\Aca GI/41/}}}\end{hieroglyph}}.

  On a placé \tl{rf} de sorte qu'il puisse être particule enclitique 
  et jouer son rôle de marqueur de protase, rendu dans la traduction 
  par \td{comme}.
\end{hierobox}

\begin{hierobox}
  \tl{\lgn{Ia} \scansion}

  \tg{}

  \td{}%
\end{hierobox}


\appendix
\appendixpage*

\chapter[Extrait de la publication de \bsc{Helck}]
        {Extrait de la publication de \bsc{Helck} \autocite{Helck}}
\label{sec:pdf}

Voir pages suivantes.

% Les \emph{Spruch} sont les numéros de chapitre, les \emph{W} et 
% \emph{T} en marge gauche valent respectivement pour \emph{Ounas} 
% et \emph{Téti}, les numéros à gauche des lignes sont ceux des
% publications officielles de chacune des pyramides, ceux à droite, 
% ainsi que les lettres, correspondent aux paragraphes définis par 
% \bsc{Sethe}.

\includepdf[pages=1, frame, noautoscale, scale=0.85,
            pagecommand={\pagestyle{ruled}}]
           {Helck_Lehre_Amenemhat}

\backmatter
\newpage

\listoffigures

\nocite{EG, Faulkner, GM, Hannig}
\printbibliography[heading=memoir,title=Bibliographie]

% % .. Quatrième de couv' ..
% \cleardoublepage
% \pagestyle{empty}
% ~\newpage
% ~\BgThispage



%%%%%%%%%%%%%%%%%%%%%%%%%%%%%%%%%%%%%%%%%%%%%%%%%%%%%%%%%%%%%%%%%%%%%%%
\end{document}
