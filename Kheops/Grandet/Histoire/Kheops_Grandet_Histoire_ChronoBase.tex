Chronologie sommaire\,\footnote{D'après Beckerath, \emph{Handbuch}, 
et K.A.~Kitchen, \emph{The Third Intermediate Period in Egypt}, 
2\ieme éd., Warminster, Aris \& Philips, 1986.}

\section*{Prédynastique (env. 5500-3000)}

\section*{\'Epoque thinite ou archaïque (env. 3000-2670)}

\subsection*{I\iere~dynastie (env. 3000-2820)}
\subsection*{\dyn{II} (env. 2820-2670)}

\section*{\OK (env. 2670-2195)}

\subsection*{\dyn{III} (env. 2670-2600)}
\subsection*{\dyn{IV} (env. 2600-2475)}
  \begin{itemize}
    \item Snéfrou
    \item Khéops
    \item Djédefrê
    \item Khéphren
    \item Mykérinos
    \item Chepseskaf
  \end{itemize}
\subsection*{\dyn{V} (2475-2345)}
\subsection*{\dyn{VI} (env. 2345-2195)}

\section*{\PPI (env. 2195-2065)}

\subsection*{\dyn{VII} \emph{fictive}}
\subsection*{\dyn{VIII} (env. 2195-2065)}
\subsection*{IX\ieme-\dyn{X}s (env. 2160-2140)}
\subsection*{\dyn{XI}, \emph{début} (env. 2160-2065)}
  \begin{itemize}
    \item Montouhotep\,I\ier
    \item Antef\,I\ier
    \item Antef\,II (2123-2073)
    \item Antef\,III (2073-2065)
  \end{itemize}

\section*{\MK (2065-1781)}

\subsection*{\dyn{XI}, \emph{fin} (env. 2065-1994)}
  \begin{itemize}
    \item Montouhotep\,II (2065-2014)
    \item Montouhotep\,III (2014-2001)
    \item Montouhotep\,IV (2001-1994)
  \end{itemize}
\subsection*{\dyn{XII} (1994-1781)}
  \begin{itemize}
    \item Amenemhat\,I\ier (1994-1964)
    \item Sésostris\,I\ier (1974-1929)
    \item Amenemhat\,II (1932-1898)
    \item Sésostris\,II (1900-1881)
    \item Sésostris\,III (1881-1842)
    \item Amenemhat\,III (1842-1794)
    \item Amenemhat\,IV (1798-1785)
    \item Néfrousobek (\emph{reine}, 1785-1781)
  \end{itemize}

\section*{\DPI (1781-1550)}

\subsection*{\dyn{XIII} (1781-env. 1650)
\subsection*{\dyn{XIV} (env. 1710-1650)
\subsection*{\dyn{XV} (rois Hyksôs, env. 1650-1540)
\subsection*{\dyn{XVI} (rois Hyksôs, env. 1650-1550)
\subsection*{\dyn{XVII} (env. 1650-1550)

\section*{\NK} (1550-1069)}

\subsection*{\dyn{XVIII} (1550-1291)}
  \begin{itemize}
    \item Ahmosis (1550-1525)
    \item Amenhotep\,I\ier (1525-1504)
    \item Thoutmosis\,I\ier (1504-1492)
    \item Thoutmosis\,II (1492-1479)
    \item Hatchepsout (\emph{reine}, 1479-1458)
    \item Thoutmosis\,III (1479-1425)
    \item Amenhotep\,II (1425-1397)
    \item Thoutmosis\,IV (1397-1387)
    \item Amenhotep\,III (1387-1350)
    \item Amenhotep\,IV-Akhénaton (1350-1333)
    \item Smenkhkarê
    \item Toutânkhamon (1333-1323)
    \item Ay (1323-1319)
    \item Horemheb (1319-1291)
  \end{itemize}

\section*{Période ramesside (1291-1069)}

\subsection*{\dyn{XIX} (1291-1185)}
  \begin{itemize}
    \item Ramsès\,I\ier (1291-1185)
    \item Séthy\,I\ier (1289-1279)
    \item Ramsès\,II (1279-1212)
    \item Mérenptah (1212-1202)
    \item Amenmessé (1202-1199)
    \item Séthy\,II (1199-1193)
    \item Siptah (1193-1187)
    \item Taousert(\emph{reine}, 1193-1185)
  \end{itemize}
\subsection*{\dyn{XX} (1185-1069)}
  \begin{itemize}
    \item Sethnakht (1185-1184)
    \item Ramsès\,III (1184-1153)
    \item Ramsès\,IV (1154-1148)
    \item Ramsès\,V (1148-1144)
    \item Ramsès\,VI (1144-1136)
    \item Ramsès\,VII (1136-1128)
    \item Ramsès\,VIII (1128-1125)
    \item Ramsès\,IX (1125-1107)
    \item Ramsès\,X (1107-1098)
    \item Ramsès\,XI (1098-1069)
  \end{itemize}

\section*{\TPI (1069-664)}

\subsection*{\dyn{XXI} (1069-945)}
  Dynastie parallèle des \og rois-prêtres \fg de Thèbes, 
  env. 1080-945.
\subsection*{\dyn{XXII} (945-718)}
\subsection*{\dyn{XXIII} (820-718)}
\subsection*{\dyn{XXIV} (730-712)}
\subsection*{\dyn{XXV} (rois kouchites, env. 775-663)}

\section*{\LP (664-332)}

\subsection*{\dyn{XXVI} (664-525)}
\subsection*{\dyn{XXVII} (1\iere Domination perse, 525-401)}
\subsection*{\dyn{XXVIII} (404-399)}
\subsection*{\dyn{XXIX} (399-380)}
\subsection*{\dyn{XXX} (380-342)}
\subsection*{Seconde domination perse (342-332)}

\section*{\'Epoque grecque ou \ptol (332-30)}

\section*{\'Epoque romaine (30\,BC-395\,AD)}

\section*{\'Epoque copte ou byzantine (395-641\,AD)}

\section*{\emph{641\,AD : conquête arabe de l'\kmt}