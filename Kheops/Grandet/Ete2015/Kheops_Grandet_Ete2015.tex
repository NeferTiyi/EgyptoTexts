\documentclass[%
  % draft, %
  hiero, %
  background, %
  dvipsnames, %
  svgnames, %
  a4paper, %
  twoside, %
  openany, %
  10pt, %
  article, %
  oldfontcommands %
]{nefermemoir}

\usepackage{pgf,tikz}
\usetikzlibrary{%
  backgrounds,
  calc,
  patterns,
  shapes.multipart,
  arrows,
  arrows.meta,
  positioning,
  shadows,
  decorations.text
}
\usepackage{xparse}
\usepackage{pdfpages}
\usepackage{multicol}
% \usepackage{minitoc}
\usepackage{rotating}
% \usepackage[version=3]{mhchem}
\usepackage{neferkheops}
% \usepackage{arabtex}

\addtolength{\columnsep}{15pt}

\newcommand{\DirUtils}{../../../../utils}
\newcommand{\DirImage}{../../../../images}

\graphicspath{%
  {\DirImage/Kheops/Grandet/Ete2015/}%
}

% \addtolength{\intextsep}{-0.5\baselineskip}
% % \addtolength{\intextsep}{-\baselineskip}

\sideparmargin{outer}

\setlength\fboxsep{0.5mm}
%\setlength\tabcolsep{0mm}
%\setlength\parskip{1.0\baselineskip}


\DeclareCiteCommand{\citeauthor}{%
  % \defcounter{maxnames}{99}%
  % \defcounter{minnames}{99}%
  % \defcounter{uniquename}{2}%
  \boolfalse{citetracker}%
  \boolfalse{pagetracker}%
  \usebibmacro{prenote}%
}{%
  \ifciteindex{\indexnames{labelname}}{}%
  \printnames[first-last]{labelname}%
}%
{\multicitedelim}
{\usebibmacro{postnote}}

\DeclareCiteCommand{\citetitle}{%
  \boolfalse{citetracker}%
  \boolfalse{pagetracker}%
  \usebibmacro{prenote}%
}{%
  \ifciteindex{\indexfield{indextitle}}{}%
  \printtext[bibhyperref]{\printfield[citetitle]{labeltitle}}%
}%
{\multicitedelim}
{\usebibmacro{postnote}}

\newlength{\imgwidth}
\newlength{\offset}

% \newcommand{\lacune}{[\dots]\xspace}
\newcommand{\lacune}[1][\dots]{%
  \ensuremath{%
    \left[\text{#1}\right]%
  }\xspace%
}
\newcommand{\trop}[1]{%
  \ensuremath{%
    \left\{\text{#1}\right\}%
  }\xspace%
}
\newcommand{\arc}[1]{%
  \ensuremath{%
    \wideparen{\text{#1}}%
  }\xspace%
}

\newcommand{\lin}[3][]{%
  \lgn[#1]{#2\texttimes#3}%
}

\newcommand{\insertimg}[2]{%
  \pgfmathsetmacro{\scale}{0.18}
  \settowidth{\imgwidth}{%
    \lin{#1}{#2} \includegraphics[scale=\scale]{Mose_#1_#2}%
  }%
  \setlength{\offset}{(\textwidth - \imgwidth) / 2}%
  \noindent%
  \hspace*{\offset}%
  \lin[enlarge by=0pt, on line]{#1}{#2} 
  \raisebox{-0.5\totalheight}{%
    \includegraphics[scale=\scale]{Mose_#1_#2}%
  }%
}

\NewDocumentEnvironment{bloc}{O{6.5} m m}{%
  \par\nobreak\vfil\penalty0\vfilneg\vtop\bgroup
  \insertimg{#2}{#3}

  % \begin{hierobox}%
}{%
  % \end{hierobox}%
  \vspace*{#1\baselineskip}%
  \par\xdef\tpd{\the\prevdepth}\egroup\prevdepth=\tpd
}


\renewenvironment{vocab}[1][\qquad\qquad]%
{%
  \begin{list}{}
    {\renewcommand\makelabel[1]{\hfill\tl{##1}}%
     \settowidth\labelwidth{\makelabel{#1}}%
     \setlength\leftmargin{\labelwidth + \labelsep}%
    }%
}%
{\unskip\end{list}}


%======================================================================
\title{Le procès de Mosé}
\short{Le procès de Mosé}
\subtitle{}
\lecturer{\PG}
\orga{Cours de \textsf{l'\IK} par}
\author{\SL}
\date{6-10 juillet 2015}

\hypersetup{%
  pdftitle  = {Le procès de Mosé, cours de \PG à l'\IK}, 
  pdfauthor = {Sonia Labetoulle}
}
% pdfsubject % pdfcreator % pdfproducer % pdfkeywords

%======================================================================

\addbibresource{\DirUtils/Hiero.bib}

% \backgroundsetup{%
%   opacity   = 0.18,
%   contents  = \includegraphics{DeB_Hathor},
%   % position  = current page.center,
%   scale     = 0.5
% }

%%%%%%%%%%%%%%%%%%%%%%%%%%%%%%%%%%%%%%%%%%%%%%%%%%%%%%%%%%%%%%%%%%%%%%%
\begin{document}

\thispagestyle{empty}
\maketitle
%%%%%%%%%%%%%%%%%%%%%%%%%%%%%%%%%%%%%%%%%%%%%%%%%%%%%%%%%%%%%%%%%%%%%%%

\frontmatter
\tableofcontents*

\mainmatter
\CaptionNormal

\chapter{Introduction}

\section{Les gens}

\subsection{Rois}

\begin{itemize}
  \item Âhmosis~I\ier (\dyn{18}, \datesregne[]{1550}{1525})
        \begin{itemize}
          \item \tl{ny-sw.t bjty Nb-pH.ty-Ra}, 
                \td{le roi de \HBE Nebpéhtyrê}
        \end{itemize}
  \item Horemheb (\dyn{18}, \datesregne[]{1319}{1291})
        \begin{itemize}
          \item \tl{ny-sw.t bjty +sr(=w)-xpr.w-Ra-stp\col n-Ra}, 
                \td{le roi de \HBE Djéseroukhépérourê Sétepenrê}
          \item \tl{sA Ra @r-m-Hb-mry-Jmn}, 
                \td{le fils de Rê Horemheb aimé d'Amon}
        \end{itemize}
  \item Ramsès~II (\dyn{19}, \datesregne[]{1279}{1212})
        \begin{itemize}
          \item \tl{sA Ra Ra-ms(w)-sw-mry-Jmn}, 
                \td{le fils de Rê Ramsès aimé d'Amon}
        \end{itemize}
\end{itemize}


\begin{multicols}{2}[\subsection{Particuliers}]
\begin{description}
  \item [\begin{hieroglyph}{\leavevmode \Cadrat{\CadratLineI{\Aca GG/70/}\CadratLine{\Aca GD/52/}}\HinterSignsSpace
\Cadrat{\CadratLineI{\Aca GN/66/}\CadratLine{\Aca GZ/33/}}\HinterSignsSpace
\Cadrat{\CadratLineI{\Aca GD/52/}\CadratLine{\Aca GZ/32/}}\HinterSignsSpace
\loneSign{\Aca GB/32/}}\end{hieroglyph}] 
        \tl{Wr\arc{nr}}, \td{Ourel} \\
        \autocite[83.2]{RK}
  \item [\begin{hieroglyph}{\leavevmode \Cadrat{\CadratLineI{\Aca GN/66/}\CadratLine{\Aca GN/69/}}\HinterSignsSpace
\loneSign{\Aca GM/48/}\HinterSignsSpace
\loneSign{\Aca GF/67/}\HinterSignsSpace
\loneSign{\Aca GA/32/}}\end{hieroglyph}] 
        \tl{NSj}, \td{Néchi} \\
        \autocite[213.8]{RK}
  \item [\begin{hieroglyph}{\leavevmode \loneSign{\Aca GM/48/}\HinterSignsSpace
\loneSign{\Aca GA/33/}\HinterSignsSpace
\Cadrat{\CadratLineI{\Aca GN/66/}\CadratLine{\Aca GZ/37/}}\HinterSignsSpace
\loneSign{\Aca GM/48/}\HinterSignsSpace
\loneSign{\Aca GM/48/}\HinterSignsSpace
\loneSign{\Aca GA/32/}}\end{hieroglyph}] 
        \tl{Jny}, \td{Iny} \\
        \autocite[33.16]{RK}
  \item [\begin{hieroglyph}{\leavevmode \loneSign{\Aca GF/62/}\HinterSignsSpace
\loneSign{\Aca GS/63/}\HinterSignsSpace
\loneSign{\Aca GU/64/}\HinterSignsSpace
\loneSign{\Aca GY/40/}\HinterSignsSpace
\loneSign{\Aca GA/32/}}\end{hieroglyph}] 
        \tl{Ms-mn}, \td{} \\
        \autocite[164.24]{RK}
  \begin{comment}
  \item [\begin{hieroglyph}{\leavevmode \loneSign{\Aca GS/68/}\HinterSignsSpace
\Cadrat{\CadratLineI{\Aca GN/66/}\CadratLine{\Aca GAa/32/}}\HinterSignsSpace
\Cadrat{\CadratLineI{\Aca GN/66/}\CadratLine{\Aca GX/32/}}\HinterSignsSpace
\Cadrat{\CadratLineI{\Aca GO/80/}\CadratLine{\Aca GX/32/\hfill\Aca GZ/32/}}\HinterSignsSpace
\loneSign{\Aca GB/32/}}\end{hieroglyph}] 
          \tl{anx-Njw.t}, \td{Ânkhnioutp\
          \autocite[68.16]{RK}
  \end{comment}
  \item [\begin{hieroglyph}{\leavevmode \loneSign{\Aca GX/32/}\HinterSignsSpace
\loneSign{\Aca GG/32/}\HinterSignsSpace
\loneSign{\Aca GM/43/}\HinterSignsSpace
\loneSign{\Aca GG/32/}\HinterSignsSpace
\Cadrat{\CadratLineI{\Aca GE/55/}\CadratLine{\Aca GZ/32/}}\HinterSignsSpace
\loneSign{\Aca GT/49/}\HinterSignsSpace
\loneSign{\Aca GB/32/}}\end{hieroglyph}] 
        \tl{\&A-xAr.t}, \td{Takharet} \\
        \autocite[367.3]{RK}
  \item [] 
        \tl{\%A-mw.t}, \td{Samout} \\
        \autocite[282.3]{RK}
  \item [] 
        \tl{\Smaj rj.t-Ra}, \td{Chéritrê} \\
        \autocite[329.15]{RK}
  \item [\begin{hieroglyph}{\leavevmode \Cadrat{\CadratLineI{\Aca GF/49/}\CadratLine{\Aca GY/32/}}\HinterSignsSpace
\loneSign{\Aca GM/48/}\HinterSignsSpace
\loneSign{\Aca GM/48/}\HinterSignsSpace
\loneSign{\Aca GA/32/}}\end{hieroglyph}] 
        \tl{@wy}, \td{Houy} %\\
        % \autocite[]{RK}
  \item [\begin{hieroglyph}{\leavevmode \Cadrat{\CadratLineI{\Aca GS/43/}\CadratLine{\Aca GN/64/}\CadratLine{\Aca GZ/36/}}\HinterSignsSpace
\loneSign{\Aca GF/66/}\HinterSignsSpace
\Cadrat{\CadratLineI{\Aca GD/52/}\CadratLine{\Aca GX/32/}}\HinterSignsSpace
\loneSign{\Aca GB/32/}}\end{hieroglyph}] 
        \tl{Nbw-nfr=t(j)}, \td{Nébounofré} \\
        \autocite[191.13]{RK}
  \item [\begin{hieroglyph}{\leavevmode \loneSign{\Aca GM/48/}\HinterSignsSpace
\Cadrat{\CadratLineI{\Aca GY/36/}\CadratLine{\Aca GN/66/}\CadratLine{\Aca GAa/43/}}\HinterSignsSpace
\loneSign{\Aca GM/48/}\HinterSignsSpace
\Cadrat{\CadratLineI{\Aca GQ/34/\hfill\Aca GX/32/}\CadratLine{\Aca GO/32/}}\HinterSignsSpace
\loneSign{\Aca GA/32/}}\end{hieroglyph}] 
        \tl{Jmn-m-jp.t}, \td{Aménémopé} \\
        \autocite[?]{RK}
  \item [\begin{hieroglyph}{\leavevmode \loneSign{\Aca GV/60/}\HinterSignsSpace
\loneSign{\Aca GV/59/}\HinterSignsSpace
\loneSign{\Aca GY/40/}\HinterSignsSpace
\loneSign{\Aca GM/48/}\HinterSignsSpace
\Cadrat{\CadratLineI{\Aca GY/36/}\CadratLine{\Aca GN/66/}}\HinterSignsSpace
\loneSign{\Aca GV/60/}\HinterSignsSpace
\loneSign{\Aca GV/59/}\HinterSignsSpace
\loneSign{\Aca GY/40/}}\end{hieroglyph}] 
        \tl{WAH-jmn-wAH}, \td{} \\
        \autocite[?]{RK}
  \item [\begin{hieroglyph}{\leavevmode \Cadrat{\CadratLineI{\Aca GY/36/}\CadratLine{\Aca GN/66/}\CadratLine{\Aca GX/32/\hfill\Aca GZ/40/}}\HinterSignsSpace
\Cadrat{\CadratLineI{\Aca GAa/43/}\CadratLine{\Aca GR/53/}\CadratLine{\Aca GR/43/}}\HinterSignsSpace
\loneSign{\Aca GM/48/}\HinterSignsSpace
\loneSign{\Aca GA/32/}}\end{hieroglyph}] 
        \tl{MnTw-m-Mnw}, \td{Montouemmin} \\
        \autocite[154.6]{RK}
  \item [\begin{hieroglyph}{\leavevmode \Cadrat{\CadratLineI{\Aca GN/59/}\CadratLine{\Aca GD/69/}\CadratLine{\Aca GY/32/}}\HinterSignsSpace
\loneSign{\Aca GA/32/}}\end{hieroglyph}] 
        \tl{\#aj}, \td{Khây} \\
        \autocite[263.7]{RK}
  \item [] 
        \tl{\%mn(w)-tA.wy}, \td{Séménoutaouy} \\
        \autocite[II.360 (150.24)]{RK}
  % \item [\begin{hieroglyph}{\leavevmode \HquarterSpace }\end{hieroglyph}] 
  %       \tl{}, \td{} \\
  %       \autocite[?]{RK}
\end{description}
\end{multicols}


\tikzset{every node/.style ={font=\smaller}}
\tikzset{lien/.style  = {rounded corners, thick}}
\tikzset{union/.style = {lien, draw = red}}
\tikzset{child/.style = {lien, draw = blue}}
\tikzset{roi/.style   = {homme, text = DarkGoldenrod}}
\tikzset{perso/.style = {%
  draw, 
  rectangle, rounded corners, 
  inner sep      = 1ex, 
  minimum height = 0.5cm, 
  minimum width  = 1.5cm,
  text width     = 1.5cm,
  text height    = 8pt,
  text depth     = 2pt,
  align          = center,
  % fill           = gray!18, 
}}
\tikzset{homme/.style = {%
  perso, 
  fill = gray!50!blue!20, 
}}
\tikzset{femme/.style = {%
  perso, 
  fill = gray!50!red!20, 
}}

\pgfmathsetmacro{\off}{1.00}
\begin{sidewaysfigure}
  \begin{tikzpicture}
    % \draw [help lines, step=0.5] (-9, 0) grid (10,-10) ;
    % \draw [help lines, red] (-9, 0) grid (10,-10) ;

    \node [homme] (nSj) {Néchi} ;
    \node [homme] (A)   [below=3.*\off of nSj.center] {A} ;
    \draw [child, dotted] (nSj) -- (A) 
          node [pos=0.66, fill=white] {\em8~générations} ;

    \node [femme] (CR)  [left=of A] {Chérytrê} ;
    \draw [union] (A) -- (CR) node [midway] (n) {} ;
    \node [femme] (wrl) [below=2*\off of CR.center] {Ourel} ;
    \node [femme] (txr) [left=of wrl] {Takharet} ;
    \node [homme] (F)   [left=of txr] {F} ;
    \node [homme] (G)   [right=of wrl] {G} ;
    \node [perso] (B)   [below right=\off and 0.5*\off of A.center] 
          {B} ;
    \node [perso] (C)   [right= of G] {C} ;
    \node [perso] (D)   [right=of B] {D} ;
    \node [perso] (E)   [right=of C] {E} ;

    \draw [child] (n.center) -- ++ (0,-0.66*\off) -| (wrl) ;
    \draw [child] (n.center) -- ++ (0,-0.66*\off) -| (txr) ;
    \draw [child] (n.center) -- ++ (0,-0.66*\off) -| (B) ;
    \draw [child] (n.center) -- ++ (0,-0.66*\off) -| (C) ;
    \draw [child] (n.center) -- ++ (0,-0.66*\off) -| (D) ;
    \draw [child] (n.center) -- ++ (0,-0.66*\off) -| (E) ;

    \draw [union] (txr) -- (F) node [midway] (n) {} ;
    \node [homme] (smn) [below=\off of n.center] {Sémentaouy} ;
    \draw [child] (n.center) -- (smn) ;

    \draw [union] (wrl) -- (G) node [midway] (n) {} ;
    \node [homme] (hwy) [below=\off of n.center] {Houy} ;
    \draw [child] (n.center) -- (hwy) ;
    \node [femme] (nbw) [right=of hwy] {Nébounofré} ;
    \draw [union] (hwy) -- (nbw) node [midway] (n) {} ;
    \node [homme] (msw) [below=\off of n.center] {Mosé} ;
    \draw [child] (n.center) -- (msw) ;
    \node [femme] (mwt) [left=of msw] {Moutnofré} ;
    \draw [union] (msw) -- (mwt) node [midway] (n) {} ;

    \node [homme] (mry) [below=\off of mwt.center] {Mérymaât} ;
    \node [homme] (amh) [left=of mry] {Amenemheb} ;
    \node [homme] (hat) [right=of mry] {Hatiay} ;
    \node [perso] (tje) [right=of hat] {Tjénéry} ;

    \draw [child] (n.center) -- ++ (0,-0.66*\off) -| (mry) ;
    \draw [child] (n.center) -- ++ (0,-0.66*\off) -| (amh) ;
    \draw [child] (n.center) -- ++ (0,-0.66*\off) -| (hat) ;
    \draw [child] (n.center) -- ++ (0,-0.66*\off) -| (tje) ;

    \node [perso] (H) [right=of tje] {H} ;
    \draw [union] (tje) -- (H) node [midway] (n) {} ;
    \node [homme] (ahb) [below=\off of n.center] {Amenemheb} ;
    \draw [child] (n.center) -- (ahb) ;

    \node [homme] (xay) [right=8*\off of msw.center] {Khây} ;
    \node [homme] (wsr) [above=\off of xay.center] {Ouserhat} ;
    \node [homme] (TAw) [above=\off of wsr.center] {Tchaouy} ;
    \node [homme] (htp) [above right=\off and 0.5*\off of TAw.center] 
          {Parâhotep} ;
    \node [homme] (hwi) [below right=\off and 0.5*\off of htp.center] 
          {Houy} ;
    \draw [child, dotted] (nSj) -- ++ (0,-\off) -| (htp) ;
    \draw [child] (htp) -- ++ (0,-0.66*\off) -| (TAw) ;
    \draw [child] (htp) -- ++ (0,-0.66*\off) -| (hwi) ;
    \draw [child] (TAw) -- (wsr) ;
    \draw [child] (wsr) -- (xay) ;
  \end{tikzpicture}
  \caption{Néchi et sa descendance}
  \label{genealogie}
\end{sidewaysfigure}


\begin{multicols}{2}[\section{Vocabulaire}]
\begin{vocab}
  \item [aro] \td{to swear (an oath)}, \td{to abjure}
  \item [aS] \td{to summon}
  \item [aDA] \td{falsehood}, \td{guilt}, \td{to be guilty of}
  \item [wAH.yt] \td{precinct}, \td{téménos}
  \item [wH.yt] \td{village}
  \item [psS.t] \td{partage}, \td{portion}
  \item [pgA] \td{?}
  \item [pt\trop{t}j] \td{voir}, \td{regarder}
  \item [mtrw] \td{témoin}
  \item [rwD] \td{to control}, \td{to administer}
  \item [hAw] \td{neighborhood}, \td{environment}, \td{time}, 
        \td{life-time}, \td{need}, \td{affairs}, \td{belongings}
  \item [xAa] \td{to turn one's back on}, \td{to negect}
  \item [SAa] \td{commencement}, \td{début}
  \item [swA] \td{break}, \td{cut}, \td{cut off}
  \item [smj] \td{to report}, \td{to complain}
  \item [smtr] \td{to examine}, \td{to make inquiry}, 
        \td{to bear witness}
  \item [Snw.ty] \td{double grenier}
  \item [Srj.t] \td{fille}
  \item [sxn] \td{to go to law}, \td{to contend}
  \item [on.yt] \td{les \frquote{Braves du roi}}
  \item [onb.t] \td{conseil}, \td{cour de justice}
  \item [dmj.t] \td{ville}
  \item [dny.t] \td{cadastre}
  % \item [] \td{}
\end{vocab}
\end{multicols}


\chapter{Mur nord}

\insertimg{425}{01}

\begin{hierobox}
  \tl{\lin{425}{01} \lacune}

  \td{\lacune}
\end{hierobox}

\insertimg{425}{02}

\insertimg{425}{03}

\begin{hierobox}
  \tl{\lin{425}{02} \lacune[\dots Hr] a.wy sr Hr jn nA rmT aA n(y) 
      pA dmj.t \lin{425}{03} r sDm r(A)=sn}

  \td{\lacune[\dots on a apporté un document] transmis par (\litt sur 
      les mains de) un magistrat citant à comparaître de nombreuses 
      personnes de la ville pour entendre leur témoignage}
\end{hierobox}

\insertimg{425}{04}

\section{Déposition de Mosé}

\begin{hierobox}
  \tl{Dd(w).t\col n \lacune n TAy-xaw Sd(w) rmT \lin{425}{04} \lacune 
      \crtch{Ra-ms(w)-sw-mry-Jn} \lacune}

  \td{Ce qu'a dit \lacune[\rem{titres et noms de Mosé}] au porteur 
      de flabellum, celui qui élève les gens \lacune[\rem{titres}] 
      Ramsès~II \lacune}
\end{hierobox}

\insertimg{425}{05}

\begin{hierobox}
  \tl{\og jr jnk \saispas[jnk] Srj n(y) @wy sA \lin{425}{05} 
      Wr\arc{nr} sA.t NSj}

  \td{\og je suis le fils de Houy, fils de Ourel, descendante de Néchi}
\end{hierobox}

\insertimg{425}{06}

\begin{hierobox}
  \tl{jw=tw Hr psS\trop{.t} n Wr\arc{nr} Hna snw=s\trop{.t} 
      \lin{425}{06} m tA onb.t aA.t m hAw (ny)-sw.t
      \crtch{+sr(=w)-xpr.w-Ra-stp\col n-Ra} d(=w) anx}

  \td{On fit des parts pour Ourel et sa fratrie dans la grande 
      \emph{qénébet} à l'époque du roi Djéseroukhépérourê Sétepenrê 
      doté de vie}
\end{hierobox}

\insertimg{425}{07}

\insertimg{425}{08}

\begin{hierobox}
  \tl{jw=tw Hr d.t jw.t wab \lin{425}{07} on.yt Jny nty m sr n(y) 
      tA onb.t aA.t r tA wH.yt \lin{425}{08} NSj}

  \td{On fit venir le prêtre \emph{ouab} de la chaise à porteur Iny 
      qui est un magistrat du grand conseil au village de Néchi}
\end{hierobox}

\begin{hierobox}
  \tl{jw=tw Hr psS\trop{.t} n=j Hna snw=j}

  \td{On fit des parts pour moi et ma fratrie}
\end{hierobox}

\insertimg{425}{09}

\begin{hierobox}
  \tl{jw=tw (Hr) d.t mw.t=j anx(.t)-(n(y.t)\trop{w})-njw.t 
      \lin{425}{09} Wr\arc{nr} m rwD n(y) snw=s\trop{.t}}

  \td{On nomma ma mère, la citadine Ourel, syndic de sa fratrie}
\end{hierobox}

\insertimg{425}{10}

\begin{hierobox}
  \tl{jw \&A-xAr.t tA sn.t \lin{425}{10} n(y) Wr\arc{nr} Hr sxn Hna 
      Wr\arc{nr} m tA onb.t aA.t}

  \td{Takharet, la s{\oe}ur d'Ourel vint avec Ourel faire une demande 
      dans la grande \emph{qénébet}}
\end{hierobox}

\insertimg{425}{11}

\begin{hierobox}
  \tl{jw=tw \lin{425}{11} Hr d.t jw.t sr n(y) onb.t}

  \td{On fit venir le magistrat de la \emph{qénébet} (au village)}
\end{hierobox}

\begin{hierobox}
  \tl{jw=tw Hr d.t rx s nb psS.t=f m pA 6 jwaw}

  \td{On fit en sorte que chacun apprenne sa part parmi les six 
      héritiers}
\end{hierobox}

\insertimg{425}{12}

\insertimg{425}{13}

\begin{hierobox}
  \tl{\lin{425}{12} xr m n(y)-sw.t \crtch{Nb-pH(.ty)-Ra} 
      \lacune[j-dw AH.t $x$ spA.t] m foA.w n \lin{425}{13} NSj 
      pAy=j jt}

  \td{D'ailleurs, c'est le roi Nebpéhtyrê qui avait donné $x$ arroures 
      de terres en récompense à Néchi mon ancêtre}
\end{hierobox}

\insertimg{425}{14}

\begin{hierobox}
  \tl{xr Dr (ny)-sw.t \crtch{Nb-pH(.ty)-Ra} jw tA AH.t \lin{425}{14} 
      Xr wa n wa r SAa pA hrw}

  \td{et depuis le roi Nebpéhtyrê, la terre était passée d'héritier 
      en héritier (\litt était soumise à l'un puis à l'autre) jusqu'à 
      ce jour}
\end{hierobox}

\insertimg{425}{15}

\insertimg{425}{16}

\begin{hierobox}
  \tl{jw @wy pAy=j jt \lin{425}{15} Hna mw.t=f Wr\arc{nr} Hr sxn Hna 
      nAy=w snw \lin{425}{16} m tA onb.t aA.t Hna tA onb.t Mn-nfr 
      \lacune sS}

  \td{Houy mon père et sa mère Ourel intentent une action avec / 
      contre \rem{(?)} leur fratrie dans la grande \emph{qénébet} et 
      la \emph{qénébet} de Memphis \lacune[\dots et ce fut jugé] par 
      écrit}
\end{hierobox}

\insertimg{426}{01}

\begin{hierobox}
  \tl{\lin{426}{01} jw pAy=j jt Hr mwt}

  \td{mon père est mort}
\end{hierobox}

\insertimg{426}{02}

\begin{hierobox}
  \begin{gramrule}%
    \tl{jw Nbw-nfr=t(j) tAy=j mw.t Hr \lin{426}{02} jj.y skA tA psS.t 
        \lacune[n(y) 
          \begin{possib}
            NSj\\
            @wy\\
          \end{possib}%
        ] pAy=j jt
    }%
  \end{gramrule}%

  \begin{gramrule}%
    \td{Nébounofré ma mère vint pour exploiter la part 
        \begin{possib}
          qui lui venait de Néchi mon ancêtre\\
          de Houy mon père\\
        \end{possib}
    }%
  \end{gramrule}%
\end{hierobox}

\insertimg{426}{03}

\begin{hierobox}
  \tl{\lin{426}{03} jw=tw tm Hr d.t skA=s.t}

  \td{On ne lui permit pas de l'exploiter}
\end{hierobox}

\begin{hierobox}
  \tl{jw=s\trop{.t} Hr smj rwDw \#ay}

  \td{Elle dénonça le syndic Khây}
\end{hierobox}

\insertimg{426}{04}

\insertimg{426}{05}

\begin{hierobox}
  \tl{jw\lin{426}{04}=tw Hr d.t jw.t=sn m-bAH TAt(y) Jwnw 
      m rnp.t sp \lacune[18] n(y) n(y)-sw.t bjt(y) 
      \crtch{Wsr-MAa.t-Ra-stp\col n-Ra} \lin{426}{05} sA Ra 
      \crtch{Ra-ms(w)-sw-mry-Jmn} d(=w) anx}

  \td{On les fit comparaître devant le vizir d'Héliopolis en l'an~18 
      du roi de \HBE Ousermaâtrê Sétepenrê le fils de Rê Ramsès~II 
      doté de vie}
\end{hierobox}

\insertimg{426}{06}

\begin{hierobox}
  En restituant la lacune après \tl{d(=w) anx} par \dots
  \begin{hieroglyph}{\leavevmode \loneSign{\Aca GM/48/}\HinterSignsSpace
\loneSign{\Aca GG/77/}\HwordSpace
\loneSign{\Aca GS/63/}\HinterSignsSpace
\Cadrat{\CadratLineI{\HquarterSpace }\CadratLine{\Aca GX/32/}}\HwordSpace
\Cadrat{\CadratLineI{\Aca GD/33/}\CadratLine{\Aca GZ/32/}}\HwordSpace
\Cadrat{\CadratLineI{\ligAROBD}\negAROBvspace\negAROBvspace\CadratLine{{\Hsmaller\Aca GD/79/}}}\HwordSpace
\loneSign{\Aca GS/63/}\HinterSignsSpace
\loneSign{\Aca GW/52/}\HinterSignsSpace
\loneSign{\Aca GM/48/}\HinterSignsSpace
\loneSign{\Aca GA/33/}\HinterSignsSpace
\loneSign{\Aca GB/32/}}\end{hieroglyph} \dots :

  \tl{\lacune[jw=s Hr Dd : \og smj=j] pA wn tw=j xAa\lin{426}{06}=kw 
      r-b\arc{nr} m tA AH.t n(y) NSj pAy(=j) jt \fg}

  \td{Elle dit : \frquote{C'est parce que j'ai été expulsée hors de la 
      terre de Néchi mon ancêtre que je veux porter plainte}}
\end{hierobox}

\insertimg{426}{07}

\insertimg{426}{08}

\begin{hierobox}
  \tl{\lin{426}{07} jw s.t Hr Dd \frquote{jm jnt=(t)w n=j tA dny.t 
      m pr-HD (m-)mjt.t tA s.t tA \lin{426}{08} Snw.ty pr aA a.w.s.}}

  \td{Elle ajouta : \og permets qu'on aille chercher pour moi 
      le parcellaire au Trésor ainsi qu'au siège du double grenier 
      de Pharaon v.s.f.}
\end{hierobox}

\begin{hierobox}
  \tl{jw jb=j mH(=w) r-Dd \frquote{jnk Srj.t n(y) NSj}}

  \td{car je suis certaine (\litt mon c{\oe}ur est rempli) que je suis 
      la descendante de Néchi}
\end{hierobox}

\insertimg{426}{09}

\begin{hierobox}
  \tl{jw=tw \lin{426}{09} Hr psS\trop{.t} n=j Hna=sn}

  \td{car on a fait des part pour moi et eux}
\end{hierobox}

\insertimg{426}{10}

\begin{hierobox}
  \tl{jw bw rx=j rwDw \#ay \lacune m \lin{426}{10} sn}

  \td{je ne reconnais pas le syndic Khây \lacune en tant que frère}
\end{hierobox}

\begin{hierobox}
  \tl{jw rwDw \#aj Hr smj m tA onb.t aA.t m rnp.t sp 18}

  \td{Le syndic Khây se plaignit devant la grande \emph{qénébet} en 
      l'an~18}
\end{hierobox}

\insertimg{426}{11}

\insertimg{426}{12}

\begin{hierobox}
  \tl{jw=tw Hr \lin{426}{11} d.t jw.t wab on.yt Jmn-[m]-jp.t nty 
      m sr n(y) tA onb.t aA.t Hna=f \lin{426}{12} Xr wa n dny.t n(y) 
      aDA m Dr.t=f}

  \td{On fit venir le prêtre \emph{ouab} de la chaise à porteurs 
      Aménémopé qui est un magistrat de la grande \emph{qénébet} 
      avec lui, apportant à la main un parcellaire falsifié}
\end{hierobox}

\insertimg{426}{13}

\begin{hierobox}
  \tl{jw=j rwj=kw m Srj.t \lin{426}{13} n(y) NSj}

  \td{J'ai perdu (\litt suis partie de) ma qualité de descendante de 
      Néchi}
\end{hierobox}

\insertimg{426}{14}

\begin{hierobox}
  \tl{jw=tw Hr d.t rwDw \#aj m rwDw n snw\lin{426}{14}=f r tA s.t 
      n(y) pAy=j jwaw}

  \td{puis on confirma le syndic Khây syndic de sa fratrie à la place 
      de mon héritier.}
\end{hierobox}

\insertimg{426}{15}

\begin{hierobox}
  \tl{jw=j m jwaw n(y) NSj pAy\lin{426}{15}=j jt \fg}

  \td{bien que je fusse une héritière de Néchi, mon ancêtre \fg}
\end{hierobox}

\insertimg{426}{16}

\begin{hierobox}
  \tl{xr ptr tw=j m tA wH.yt NSj pAy\lin{426}{16}=j jt nty tA Hnp.t 
      n(y) NSj pAy=j jt jm=f}

  \td{Or voyez, je suis dans le village de Néchi, mon ancêtre dans 
      lequel se trouve la parcelle de Néchi, mon ancêtre}
\end{hierobox}

\insertimg{427}{01}

\insertimg{427}{02}

\begin{hierobox}
  \tl{\lin{427}{01} jm smty=tw=j mtw=j ptr nA jr Wr\arc{nr}, tA mw.t 
      \lin{427}{02} n(y) sS @wy pAy=j jt \zero}

  \td{Permettez que je sois interrogé pour déterminer (\litt voir) si 
      Ourel était la mère du scribe Houy mon père}
\end{hierobox}

\insertimg{427}{03}

\insertimg{427}{04}

\begin{hierobox}
  \tl{Dd\lacune[=j \dots] n(y) NSj jw bn \lin{427}{03} sw mn=tj Hr tA 
      dny.t jr(w.t)\col n rwDw \#ay Hna \lin{427}{04} pA sr n(y) onb.t 
      jjy jr=j m-a=f}

  \td{\lacune[et je vous dirai qu'il était un descendant] de Néchi, 
      alors qu'il \rem{(l'acte de propriété)} n'est pas établit dans 
      le parcellaire fabriqué par le syndic Khây contre moi avec le 
      magistrat de la grande \emph{qénébet} venu contre moi avec lui}
\end{hierobox}

\insertimg{427}{05}

\begin{hierobox}
  \tl{jw(=j) Hr smj r-Dd : \og dny.t n(y) aDA \zero, \lin{427}{05} tA 
      jry.t r=j}

  \td{Je portai plainte en ces termes : \og C'est un parcellaire 
      falsifié, ce qui a été fait contre moi}
\end{hierobox}

\insertimg{427}{06}

\begin{hierobox}
  \tl{xr jw=j smty=kw Xr-HA.t tw=j gm=kw \lin{427}{06} Hr war.t}

  \td{En effet, j'ai été interrogé auparavant et j'ai été trouvé sur 
      un registre}
\end{hierobox}

\insertimg{427}{07}

\begin{hierobox}
  \tl{jm smty=tw=j Hna nAy=j jwaw.w m-bAH \trop{n} rmT \lin{427}{07} 
      aA.w n(y) pA dmj.t}

  \td{Confrontez-moi avec mes héritiers à de nombreuses personnes de 
      la ville, }
\end{hierobox}

\insertimg{427}{08}

\begin{hierobox}
  \tl{ptr nA jnk Srj n(y) NSj \lin{427}{08} nA m-bjA.t}

  \td{en leur demandant si je suis un descendant de Néchi ou bien non.}
\end{hierobox}

\section{Déposition de Khây}

\begin{hierobox}
  \tl{Dd(w).t\col n rwDw \#ay :}
  \td{\saispas ce qui a été dit par le syndic Khây :}
\end{hierobox}

\insertimg{427}{09}

\begin{hierobox}
  \tl{\og jnk Srj n(y) rwDw \lin{427}{09} Wsr-HA.t sA TAwy sA 
      PA-Ra-Htp}

  \td{\og je suis le fils du syndic Ouserhat fils de Tchaouy fils de 
      Parâhotep,}
\end{hierobox}

\insertimg{427}{10}

\insertimg{427}{11}

\begin{hierobox}
  \tl{jw=f Hr d.t n=j \lin{427}{10} tAy=f psS.t AH.t m sSw m hAw 
      n(y)-sw.t \crtch{+sr(=w)-xpr.w-Ra-stp\col n-Ra} d(=w) anx 
      \lin{427}{11} m-bAH mtrw.w}

  \td{et il m'a donné sa part de terre par écrit à l'époque du roi 
      Djéseroukhépérourê Sétepenrê doté de vie devant témoins}
\end{hierobox}

\insertimg{427}{12}

\begin{hierobox}
  \tl{jw Hry jH(.w) @wy sA PA-Ra-Htp pA wnw \lin{427}{12} Hr skA=s 
      Dr hAw \crtch{J\lacune} d(=w) anx \zero}

  \td{alors que c'était le supérieur des écuries Houy fils 
      de Parâhotep, celui qui la cultivait depuis l'époque 
      d'\lacune[Amenophis~III] \rem{(?)} doté de vie}
\end{hierobox}

\insertimg{427}{13}

\begin{hierobox}
  \tl{jw Ssp\col n=\lacune[j] (s.t) m hAw \lin{427}{13} 
      \crtch{@r-m-Hb-mry-Jmn} r SAa pA hrw}

  \td{Je l'ai reçue à l'époque d'Horemheb jusqu'à ce jour}
\end{hierobox}

\insertimg{427}{14}

\begin{hierobox}
  \tl{jw sS @wy anx(.t-ny.t)-njw.t Nbw-nfr=t(j) \lin{427}{14} Hr TA 
      tAy(=j) psS.t AH.t}

  \td{le scribe Houy et la citadine Nébounofré se sont emparé de ma 
      part de terres}
\end{hierobox}

\insertimg{427}{15}

\begin{hierobox}
  \tl{jw=s(n) Hr d.t \saispas \lin{427}{15} \saispas}

  \td{Ils firent \saispas}
\end{hierobox}

\begin{hierobox}
  \tl{jw=j Hr smj n TAt(y) m Jwnw}

  \td{j'ai porté plainte devant le vizir à Héliopolis}
\end{hierobox}

\insertimg{427}{16}

\begin{hierobox}
  \tl{jw=f Hr d.t sxn=j \lin{427}{16} Hna Nbw-nfr=t(j) m-bAH TAt(y) m 
      tA onb.t aA.t}

  \td{il m'a permis d'intenter une action contre Nébounofré devant le 
      vizir dans la grande \emph{qénébet}}
\end{hierobox}

\insertimg{428}{01}

\begin{hierobox}
  \tl{jw=j Hr jn(.t) nAy\lin{428}{01}=j mtrw.w \lacune m Dr.t=j Dr 
      \crtch{Nb-pH(.ty)-Ra}}

  \td{je suis allé chercher mes témoins \lacune en ma possession 
      depuis Nebpéhtyrê}
\end{hierobox}

\insertimg{428}{02}

\begin{hierobox}
  \tl{jw Nbw-nfr=t(j) Hr \lin{428}{02} jn(.t) nAy=s mtrw.w m-mjt.t}

  \td{Nébounofré alla chercher ses témoins également}
\end{hierobox}

\insertimg{428}{03}

\begin{hierobox}
  \tl{jw=tw Hr pgA m-bAH TAt(y) \lin{428}{03} m tA onb.t aA.t}

  \td{On \saispas[entra / s'assied / se leva] devant le vizir dans la 
      grande \emph{qénébet}}
\end{hierobox}

\begin{hierobox}
  \tl{jw TAt(y) Hr Dd n=s jr nn sS.w : \saispas}

  \td{le vizir lui dit : \saispas}
\end{hierobox}

\insertimg{428}{04}

\begin{hierobox}
  \tl{\lin{428}{04} \saispas wa m pA 2 s \saispas}

  \td{\saispas}
\end{hierobox}

\insertimg{428}{05}

\begin{hierobox}
  \tl{jw Nbw-nfr=t(j) Hr Dd n TAt(y) : \og jm jn=tw n=j \lin{428}{05} 
      tA (dny.t n(y) pr-HD n tA s.t tA Snw.ty pr-Aa a.w.s}

  \td{Nébounofré dit au vizir : \frquote{permets qu'on aille chercher 
      pour moi le parcellaire au Trésor ainsi qu'au siège du Double 
      Grenier de Pharaon v.s.f.}}
\end{hierobox}

\begin{hierobox}
  \tl{jw TAt(y) Hr Dd n=s : \og nfr jor}

  \td{Le vizir lui dit : \og \saispas}
\end{hierobox}

\insertimg{428}{06}

\begin{hierobox}
  \tl{\lin{428}{06} pA j-Dd(w)}
\end{hierobox}

\begin{hierobox}
  \tl{jw=tw Hr jTA=n m xd r pr-\crtch{Ra-ms(w)-sw-mry-Jmn}}

  \td{On nous emmena vers le nord à Pi-Ramsès}
\end{hierobox}

\insertimg{428}{07}

\begin{hierobox}
  \tl{jw\lin{428}{07}=tw Hr ao r pr-HD n(y) pr-aA a.w.s m-mjt.t 
      tA s.t tA Snw.ty pr-aA a.w.s.}

  \td{On entra au Trésor de Pharaon v.s.f. \saispas la place du double 
      grenier de Pharaon v.s.f.}
\end{hierobox}

\insertimg{428}{08}

\begin{hierobox}
  \tl{jw=tw \lin{428}{08} Hr jn(.t) tA dny.t nw.t m-bAH TAt(y) m (tA) 
      onb.t aA.t}

  \td{On alla chercher le deuxième registre du cadastre devant le 
      vizir dans le grand conseil}
\end{hierobox}

\begin{hierobox}
  \tl{jw TAt(y) Hr Dd n Nbw-nfr=t(j) :}

  \td{Le vizir dit à Nébounofré :}
\end{hierobox}

\insertimg{428}{09}

\begin{hierobox}
  \tl{\lin{428}{09} \saispas pAy=j jwaw m \saispas nA jwaw.w 
      nty Hr tA}

  \td{\saispas mon héritier \saispas les héritiers \saispas}
\end{hierobox}

\insertimg{428}{10}

\begin{hierobox}
  \tl{\lin{428}{10} dny.t nw.t nty m Dr.t=n}

  \td{le deuxième registre du cadastre qui est dans notre main}
\end{hierobox}

\begin{hierobox}
  \tl{jw Nbw-nfr=t(j) Hr Dd :}

  \td{Nébounofré dit :}
\end{hierobox}

\insertimg{428}{11}

\begin{hierobox}
  \tl{nn \zero wn=w jwaw \lin{428}{11} jm=sn}

  \td{il n'y a pas d'héritiers dedans}
\end{hierobox}

\begin{hierobox}
  \tl{\saispas aDA \saispas m TAt(y)}
\end{hierobox}

\insertimg{428}{12}

\begin{hierobox}
  \tl{jw \lin{428}{12} sS wdHw (ny)-sw(.t) \saispas}

  \td{le scribe de la table d'offrande du roi \saispas}
\end{hierobox}

\begin{hierobox}
  \tl{\saispas sA MnTw-m-Mnw Hr Dd n TAt(y) jx pA sxr jrr=k}

  \td{\saispas fils de Montouemmin dit au vizir \saispas}
\end{hierobox}

\insertimg{428}{13}

\begin{hierobox}
  \tl{\lin{428}{13} n Nbw-nfr=t(j)}

  \td{Nébounofré}
\end{hierobox}

\begin{hierobox}
  \tl{jw TAt(y) Hr Dd n \saispas}

  \td{le vizir dit à \saispas}
\end{hierobox}

\begin{hierobox}
  \tl{tw=k n Xnw jx Sm=k r pr-HD}

  \td{\saispas on s'en va au Trésor}
\end{hierobox}

\insertimg{428}{14}

\begin{hierobox}
  \tl{\lin{428}{14} mtw=k ptr pAy=s sxrw}
\end{hierobox}

\begin{hierobox}
  \tl{jw \#aj pr}

  \td{Khây sort}
\end{hierobox}

\begin{hierobox}
  \tl{jw=f Hr Dd n=s jry}

  \td{il lui dit \saispas}
\end{hierobox}

\insertimg{428}{15}

\begin{hierobox}
  \tl{\lin{428}{15} \saispas smtr nA sS.w \saispas m sSw}

  \td{\saispas par écrit}
\end{hierobox}

\insertimg{428}{16}

\begin{hierobox}
  \tl{jw=tw Hr \lin{428}{16} aS r wab on.yt Jmn-m-jp.t}

  \td{on convoque le prêtre \emph{ouab} de la chaise à porteurs 
      Aménémopé}
\end{hierobox}

\begin{hierobox}
  \tl{jw=tw Hr d.t Sm=f r-Dd nw}
\end{hierobox}

\insertimg{429}{01}

\begin{hierobox}
  \tl{\lin{429}{01} nA jwa.w m}
\end{hierobox}

\begin{hierobox}
  \tl{tw=k d.t ptr=sn nA AH.t m}

  \td{tu fais qu'ils voient les terres \saispas}
\end{hierobox}

\insertimg{429}{02}

\begin{hierobox}
  \tl{tw=k \lin{429}{02} psS.t=sn j\col n=tw n=f Hna tA onb.t 
      Mn-nfr}

  \td{leur portion \saispas pour lui et le conseil de Memphis}
\end{hierobox}

\begin{hierobox}
  \tl{jw=j Hr d.t jw.t swA}
\end{hierobox}

\insertimg{429}{03}

\begin{hierobox}
  \tl{\lin{429}{03} \saispas wnw \saispas jmy-rA ssm.wt}

  \td{\saispas directeur des chevaux}
\end{hierobox}

\insertimg{429}{04}

\begin{hierobox}
  \tl{jw \lin{429}{04} sr n(y) onb.t Jmn-m-jp.t r aS r Ms-mn 
      r-Dd \saispas}

  \td{le magistrat du conseil Aménémopé va convoquer Mesmen \saispas}
\end{hierobox}

\insertimg{429}{05}

\begin{hierobox}
  \tl{\lin{429}{05} \lacune}
\end{hierobox}

\begin{hierobox}
  \tl{jw=tw Hr aS n=f r \saispas jmn.t}

  \td{on le convoque \saispas}
\end{hierobox}

\insertimg{429}{06}

\begin{hierobox}
  \tl{\lin{429}{06} jw=tw Hr d.t n=j AH.t jwty 13}

  \td{on me donne les terres \saispas}
\end{hierobox}

\begin{hierobox}
  \tl{jw=tw Hr d.t AH.t jwty}

  \td{on me donne les terres \saispas}
\end{hierobox}

\insertimg{429}{07}

\begin{hierobox}
  \tl{\lin{429}{07} \lacune n nA jwa.w \lacune rmT}

  \td{\saispas pour les héritiers \saispas}
\end{hierobox}

\insertimg{429}{08}

\begin{hierobox}
  \tl{\lin{429}{08} aA m pA dmj.t \saispas Ms-mn \lacune}

  \td{\saispas dans la ville \saispas}
\end{hierobox}

\insertimg{429}{09}

\begin{hierobox}
  \tl{\lin{429}{09} \lacune wAH Jmn wAH pA HoA j-Dd(w) m mAa.t 
      n pr-aA a.w.s.}

  \td{\saispas le seigneur qui parle en vérité pour Pharaon v.s.f.}
\end{hierobox}

\insertimg{429}{10}

\begin{hierobox}
  \tl{\lin{429}{10} bn Dd aDA mtw=j Dd aDA \saispas}
\end{hierobox}

\insertimg{429}{11}

\begin{hierobox}
  \tl{\lin{429}{11} \saispas}
\end{hierobox}

\begin{hierobox}
  \tl{jw=j r KS jr sS @wy Srj n(y) Wr\arc{nr}}

  \td{je vais à Kouch \saispas le scribe Houy, fils de Ourel}
\end{hierobox}

\insertimg{429}{12}

\begin{hierobox}
  \tl{\lin{429}{12} tw=tw Hr Dd r-Dd Srj n(y) NSj}

  \td{\saispas fils de Néchi}
\end{hierobox}

\begin{hierobox}
  \tl{tw=j ptr}
\end{hierobox}

\insertimg{429}{13}

\begin{hierobox}
  \tl{\lin{429}{13} nA jr Wr\arc{nr} \lacune AH.t}

  \td{\saispas Ourel \saispas terres}
\end{hierobox}

\insertimg{429}{14}

\begin{hierobox}
  \tl{\lin{429}{14} rwD(w) \#aj wAH Jmn wAH pA HoA}

  \td{le syndic Khây \saispas}
\end{hierobox}

\insertimg{429}{15}

\begin{hierobox}
  \tl{jr sS @wy \lin{429}{15} Srj n(y) Wr\arc{nr} sA.t NSj mtw 
      \lacune r-Dd}

  \td{le scribe Houy fils de Ourel, fille de Néchi \saispas}
\end{hierobox}

\insertimg{429}{16}

\begin{hierobox}
  \tl{\lin{429}{16} bn \saispas nA \saispas}

  \tl{jw=j \saispas wAH Jmn wAH pA HoA}
\end{hierobox}

\insertimg{430}{01}

\begin{hierobox}
  \tl{\lin{430}{01} jw bn \lacune \saispas sDm=tw=sn}
\end{hierobox}

\insertimg{430}{02}

\begin{hierobox}
  \tl{\lin{430}{02} \saispas r(A)=sn Sd=tw pAy=sn skA}
\end{hierobox}

\insertimg{430}{03}

\begin{hierobox}
  \tl{\lin{430}{03} \saispas Dd t n f}
\end{hierobox}

\insertimg{430}{04}

\begin{hierobox}
  \tl{\lin{430}{04} wAH Jmn wAH pA HoA}
\end{hierobox}

\begin{hierobox}
  \tl{mtw=tw smtr mtw=tw gmH}
\end{hierobox}

\insertimg{430}{05}

\begin{hierobox}
  \tl{\lin{430}{05} jw skA(w) \lacune psS.t n(y) \lacune Hr=j}
\end{hierobox}

\insertimg{430}{06}

\begin{hierobox}
  \tl{jw \lin{430}{06}=j \saispas wab \saispas n pr ptH wAH Jmn 
      wAH pA HoA}
\end{hierobox}

\insertimg{430}{07}

\begin{hierobox}
  \tl{\lin{430}{07} j-Dd(w) m mAa.t}

  \td{qui parle en vérité}
\end{hierobox}

\begin{hierobox}
  \tl{bn Dd=j aDA \saispas Dd aDA \saispas}
\end{hierobox}

\insertimg{430}{08}

\begin{hierobox}
  \tl{\lin{430}{08} \saispas}
\end{hierobox}

\begin{hierobox}
  \tl{jw=j r KS}

  \td{je vais à Kouch}
\end{hierobox}

\begin{hierobox}
  \tl{tw=j rx=kw}

  \td{je sais}
\end{hierobox}

\insertimg{430}{09}

\begin{hierobox}
  \tl{\lin{430}{09} \lacune sS @wy Srj n(y) Wr\arc{nr}}

  \td{\saispas scribe Houy, fils de Ourel}
\end{hierobox}

\begin{hierobox}
  \tl{jw=f Hr skA}

  \td{il laboure}
\end{hierobox}

\insertimg{430}{10}

\begin{hierobox}
  \tl{\lin{430}{10} nAy=f AH.t rnp.t n rnp.t}

  \td{ses champs année après année}
\end{hierobox}

\begin{hierobox}
  \tl{jw jr=f Hr skA s.t r-Dd}
\end{hierobox}

\begin{hierobox}
  \tl{jnk}

  \td{je suis}
\end{hierobox}

\insertimg{430}{11}

\begin{hierobox}
  \tl{\lin{430}{11} Srj n(y) Wr\arc{nr} sA.t NSj Dd t n \saispas}

  \td{le fils de Ourel, fille de Néchi \saispas}
\end{hierobox}

\insertimg{430}{12}

\begin{hierobox}
  \tl{\lin{430}{12} \saispas n(y) pr-HD n(y) pr-aA a.w.s. 
      wAH Jmn wAH pA HoA}

  \td{\saispas Trésor de Pharaon v.s.f. \saispas}
\end{hierobox}

\begin{hierobox}
  \tl{mtw=j Dd aDA \saispas}

  \td{\saispas}
\end{hierobox}

\insertimg{430}{13}

\begin{hierobox}
  \tl{\lin{430}{13} \saispas}
\end{hierobox}

\begin{hierobox}
  \tl{jw=j r KS jr sS @wy}

  \td{je vais à Kouch \saispas scribe Houy}
\end{hierobox}

\insertimg{430}{14}

\begin{hierobox}
  \tl{\lin{430}{14} Srj n(y) Wr\arc{nr}}

  \td{fils de Ourel}
\end{hierobox}

\begin{hierobox}
  \tl{xr jr Wr\arc{nr} Srj.t n(y) NSj}

  \td{Ourel fille de Néchi}
\end{hierobox}

\insertimg{430}{15}

\begin{hierobox}
  \tl{\lin{430}{15} Dd t n Hry jH.w Nb-nfr m-mjt.t r-Dd}

  \td{\saispas supérieur des écuries Nebnéfer \saispas}
\end{hierobox}

\begin{hierobox}
  \tl{jr sS @wy wn=f Hr skA}

  \td{quant au scribe Houy, il laboure}
\end{hierobox}

\insertimg{430}{16}

\begin{hierobox}
  \tl{\lin{430}{16} nAy=f AH.t rnp.t n rnp.t}

  \td{ses terres année après année}
\end{hierobox}

\begin{hierobox}
  \tl{jw=f jr t n pA nty nb Hr=f}
\end{hierobox}

\begin{hierobox}
  \tl{jw=sn Hr=f Atp n=f}
\end{hierobox}

\insertimg{431}{01}

\begin{hierobox}
  \tl{\lin{431}{01} pA skA nAy=f AH.t rnp.t n rnp.t}

  \td{\saispas ses terres année après année}
\end{hierobox}

\insertimg{431}{02}

\begin{hierobox}
  \tl{xr wnw=f Hr sxn \lin{431}{02} Hna anx-njw.t \&A-xAr.t tA mw.t 
      \saispas \%mn(w)-tA.wy}

  \td{il intente un procès contre \saispas Takharet, la mère \saispas}
\end{hierobox}

\insertimg{431}{03}

\begin{hierobox}
  \tl{\lin{431}{03} xr sxn=f Hna \%mn(w)-tA.wy pAy=s Srj}

  \td{il intente un procès contre \saispas son fis}
\end{hierobox}

\insertimg{431}{04}

\begin{hierobox}
  \tl{mtw=tw \lin{431}{04} d.t nA AH.t n @wy}

  \td{et on donne les terres à Houy}
\end{hierobox}

\begin{hierobox}
  \tl{jw=sn mn \saispas}
\end{hierobox}

\insertimg{431}{05}

\begin{hierobox}
  \tl{\lin{431}{05} \saispas m-mjt.t r-Dd}
\end{hierobox}

\begin{hierobox}
  \tl{jr sS @wy Srj n(y) wr\arc{nr}}

  \td{le sribe Houy fils de Ourel}
\end{hierobox}

\insertimg{431}{06}

\begin{hierobox}
  \tl{jr \lin{431}{06} Wr\arc{nr} sA.t NSj Dd t n \saispas}

  \td{Ourel, fille de Néchi \saispas}
\end{hierobox}

\insertimg{431}{07}

\begin{hierobox}
  \tl{\lin{431}{07} \saispas wAH Jmn wAH pA HoA}
\end{hierobox}

\begin{hierobox}
  \tl{mtw=j Dd aDA}
\end{hierobox}

\begin{hierobox}
  \tl{jw=j r pH.wy pr}
\end{hierobox}

\insertimg{431}{08}

\begin{hierobox}
  \tl{\lin{431}{08} jr sS @wy Srj n(y) wr\arc{nr} xr jr wr\arc{nr}}

  \td{le scribe Houy, fils de Ourel \saispas}
\end{hierobox}

\insertimg{431}{09}

\begin{hierobox}
  \tl{\lin{431}{09} sA.t NSj Dd t n anx-njw.t \saispas m-mjt.t 
      Dd t n anx-njw.t}

  \td{file de Néchi \saispas}
\end{hierobox}

\insertimg{431}{10}

\begin{hierobox}
  \tl{tw\lin{431}{10}y m-mjt.t}
\end{hierobox}

\chapter{Mur sud}

\begin{bloc}{432}{01}
  \tl{\lin{432}{01} }
\end{bloc}

% \begin{hierobox}
% \end{hierobox}

\begin{bloc}{432}{02}
  \tl{\lin{432}{02} }
\end{bloc}

\begin{bloc}{432}{03}
  \tl{\lin{432}{03} }
\end{bloc}

\begin{bloc}{432}{04}
  \tl{\lin{432}{04} }
\end{bloc}

\begin{bloc}{432}{05}
  \tl{\lin{432}{05} }
\end{bloc}

\begin{bloc}{432}{06}
  \tl{\lin{432}{06} }
\end{bloc}

\begin{bloc}{432}{07}
  \tl{\lin{432}{07} }
\end{bloc}

\begin{bloc}{432}{08}
  \tl{\lin{432}{08} }
\end{bloc}

\begin{bloc}{432}{09}
  \tl{\lin{432}{09} }
\end{bloc}

\begin{bloc}{432}{10}
  \tl{\lin{432}{10} }
\end{bloc}

\begin{bloc}{432}{11}
  \tl{\lin{432}{11} }
\end{bloc}

\begin{bloc}{432}{12}
  \tl{\lin{432}{12} }
\end{bloc}

\begin{bloc}{432}{13}
  \tl{\lin{432}{13} }
\end{bloc}

\begin{bloc}{432}{14}
  \tl{\lin{432}{14} }
\end{bloc}

\begin{bloc}{432}{15}
  \tl{\lin{432}{15} }
\end{bloc}

\begin{bloc}{432}{16}
  \tl{\lin{432}{16} }
\end{bloc}

\begin{bloc}{433}{01}
  \tl{\lin{433}{01} }
\end{bloc}

\begin{bloc}{433}{02}
  \tl{\lin{433}{02} }
\end{bloc}

\begin{bloc}{433}{03}
  \tl{\lin{433}{03} }
\end{bloc}

\begin{bloc}{433}{04}
  \tl{\lin{433}{04} }
\end{bloc}

\begin{bloc}{433}{05}
  \tl{\lin{433}{05} }
\end{bloc}

\begin{bloc}{433}{06}
  \tl{\lin{433}{06} }
\end{bloc}

\begin{bloc}{433}{07}
  \tl{\lin{433}{07} }
\end{bloc}

\begin{bloc}{433}{08}
  \tl{\lin{433}{08} }
\end{bloc}

\begin{bloc}{433}{09}
  \tl{\lin{433}{09} }
\end{bloc}

\begin{bloc}{433}{10}
  \tl{\lin{433}{10} }
\end{bloc}

\begin{bloc}{433}{11}
  \tl{\lin{433}{11} }
\end{bloc}

\begin{bloc}{433}{12}
  \tl{\lin{433}{12} }
\end{bloc}

\begin{bloc}{433}{13}
  \tl{\lin{433}{13} }
\end{bloc}

\begin{bloc}{433}{14}
  \tl{\lin{433}{14} }
\end{bloc}

\begin{bloc}{433}{15}
  \tl{\lin{433}{15} }
\end{bloc}

\begin{bloc}{433}{16}
  \tl{\lin{433}{16} }
\end{bloc}

\begin{bloc}{434}{01}
  \tl{\lin{434}{01} }
\end{bloc}

\begin{bloc}{434}{02}
  \tl{\lin{434}{02} }
\end{bloc}

\begin{bloc}{434}{03}
  \tl{\lin{434}{03} }
\end{bloc}

\begin{bloc}{434}{04}
  \tl{\lin{434}{04} }
\end{bloc}

\begin{bloc}{434}{05}
  \tl{\lin{434}{05} }
\end{bloc}

\begin{bloc}{434}{06}
  \tl{\lin{434}{06} }
\end{bloc}

\begin{bloc}{434}{07}
  \tl{\lin{434}{07} }
\end{bloc}

\begin{bloc}{434}{08}
  \tl{\lin{434}{08} }
\end{bloc}

\chapter{Fragments non localisés}

\begin{bloc}{434}{11}
  \tl{\lin{434}{11} }
\end{bloc}

\begin{bloc}{434}{12}
  \tl{\lin{434}{12} }
\end{bloc}

\begin{bloc}{434}{13}
  \tl{\lin{434}{13} }
\end{bloc}

\chapter{Victoire finale de Mosé devant la cour de justice}

\section{Au-dessus des juges}

\begin{bloc}{435}{02}
  \tl{\lin{435}{02} }
\end{bloc}

\begin{bloc}{435}{03}
  \tl{\lin{435}{03} }
\end{bloc}

\begin{bloc}{435}{04}
  \tl{\lin{435}{04} }
\end{bloc}

\section{Au-dessus du scribe}

\begin{bloc}{435}{05}
  \tl{\lin{435}{05} }
\end{bloc}

\begin{bloc}{435}{06}
  \tl{\lin{435}{06} }
\end{bloc}

\begin{bloc}{435}{07}
  \tl{\lin{435}{07} }
\end{bloc}

\begin{bloc}{435}{08}
  \tl{\lin{435}{08} }
\end{bloc}

\chapter{Texte Au-dessus de la cour de justice}

\begin{bloc}{435}{12}
  \tl{\lin{435}{12} }
\end{bloc}

\begin{bloc}{435}{13}
  \tl{\lin{435}{13} }
\end{bloc}

\begin{bloc}{435}{14}
  \tl{\lin{435}{14} }
\end{bloc}

\begin{bloc}{435}{15}
  \tl{\lin{435}{15} }
\end{bloc}

\begin{bloc}{435}{16}
  \tl{\lin{435}{16} }
\end{bloc}








% \include{Kheops_Grandet_Histoire_Chap0_Intro}
% \include{Kheops_Grandet_Histoire_Chap1_PD}
% \include{Kheops_Grandet_Histoire_Chap2_PT}
% \include{Kheops_Grandet_Histoire_Chap3_AE}
% \include{Kheops_Grandet_Histoire_Chap4_PPI}
% \include{Kheops_Grandet_Histoire_Chap5_ME}
% \include{Kheops_Grandet_Histoire_Chap6_DPI}
% \include{Kheops_Grandet_Histoire_Chap7_NE_a}
% \include{Kheops_Grandet_Histoire_Chap7_NE_b}
% \include{Kheops_Grandet_Histoire_Chap7_NE_c}
% \include{Kheops_Grandet_Histoire_Chap8_PO}

\appendix

% \appendixpage
% \phantomsection
% \addcontentsline{toc}{chapter}{\appendixpagename}
% \chapter*{\appendixpagename}

% \book{\appendixpagename}
% \book*{\appendixpagename}
% \addcontentsline{toc}{book}{\appendixpagename}

% \include{Kheops_Grandet_Histoire_KVTombs}
% \include{Kheops_Grandet_Histoire_TT}
% %!TEX root = Kheops_Grandet_Histoire.htx

\chapter{Carte de l'\kmt}

\clearpage

%!TEX root = Kheops_Grandet.htx


\definecolor{sand}{RGB}{233,221,175}
\definecolor{sea}{RGB}{147,157,172}
\definecolor{sealegend}{RGB}{219,222,227}

\tikzset{ville/.style = {%
  ultra thin, 
  DarkRed, 
  font=\scriptsize, 
  text=Sepia
}}

\tikzset{terre/.style = {fill=sand}}
\tikzset{mers/.style  = {ultra thin, draw, sea!66}}

\tikzset{frontiere/.style = {%
  font=\itshape\tiny, 
  gray, 
  ultra thick, 
  line cap  = round,
  line join = round
}}
\tikzset{oldfront/.style = {%
  frontiere,
  dash pattern=on 20.80pt off 20.80pt
}}
\tikzset{newfront/.style = {%
  frontiere,
  dash pattern=on 38.40pt off 9.60pt on 4.80pt off 9.60pt
}}

\tikzset{boussole/.style = {%
  font=\bfseries, 
  black, 
  very thick, 
  line cap  = round,
  line join = round
}}
\tikzset{legende/.style = {%
  font=\scriptsize, 
  black,
  line cap  = round,
  line join = round
}}
\tikzset{region/.style  = {%
  DarkGreen, 
  font=\itshape\sffamily
}}

% .. Fond de carte ..
\newcommand{\fondterre}{%
  \fill [terre] (A) rectangle (D) ;
}

% .. Villes ..
% #1 Optionnel : options du label
% #2 Nom
% #3 Position de la puce 
% #4 Position des légende et options
% #5 node distance
\newcommand{\ville}[5][]{{%
  \tikzset{villpuce/.style = {%
    circle, 
    fill = DarkRed, 
    minimum size = 3, 
    % fill = blue, 
    % opacity = 0.5
  }}
  \tikzset{villabel/.style = {%
    font=\scriptsize, 
    Sepia 
    % opacity=0.5, 
    % blue
  }}
  \begin{scope}[node distance = #5]
    \node [villpuce, alias=puce] at (#3) {} ;
    \node [villabel, #4=of puce, #1] {#2} ;
  \end{scope}
}}
\newcommand{\villes}{%
  \ville{Méroé}               {102.30, -279.68}{right}{20}
  \ville{Gebel Barkal}        { 45.85, -229.73}{left}{20}
  \ville{Kaoua}               {  1.65, -209.68}{left}{20}
  \ville{Kerma}               { -2.05, -200.02}{right}{20}
  \ville{Sesebi}              {  1.55, -186.42}{left}{20}
  \ville{Soleb}               { -7.30, -175.22}{left}{20}
  \ville{Amara}               {  -.70, -161.88}{left}{20}
  \ville{Bouhen}              { 12.50, -146.73}{left}{20}
  \ville{Faras}               { 27.80, -121.08}{left}{20}
  \ville{Abou Simbel}         { 35.80, -115.69}{left}{20}
  \ville{Qasr Ibrim}          { 45.50, -108.23}{below right}{20}
  \ville{Phil\ae}             { 72.90,  -66.13}
        {above right}{-10 and 15}
  \ville{Éléphantine}         { 71.30,  -61.68}{left}{20}
  \ville{Assouan}             { 73.50,  -61.37}{above right}{10}
  \ville{Gebel Silsileh}      { 72.50,  -46.98}{left}{20}
  \ville{Edfou}               { 69.15,  -34.72}{below right}{10 and -5}
  \ville{Hiérakonpolis}       { 66.05,  -30.89}{left}{20}
  \ville{Armant}              { 58.45,  -17.33}{left}{20}
  \ville{Thèbes}              { 66.35,  -14.18}{below right}{10}
  \ville{Médamoud}            { 71.40,  -11.63}{right}{20}
  \ville{Coptos}              { 68.25,   -3.98}{below left}{10}
  \ville{Diospolis Parva}     { 49.89,   -2.73}{left}{20}
  \ville{Dendera}             { 61.19,    2.22}
        {above right}{15 and -50}
  \ville{Abydos}              { 35.29,    8.22}{left}{20}
  \ville{Naga el-Deir}        { 41.15,   12.72}{right}{20}
  \ville{Akhmim}              { 31.90,   21.62}{left}{20}
  \ville{El-Haouaouich}       { 34.65,   24.52}{below right}{-5 and 10}
  \ville{Matmar}              { 32.90,   27.47}{above right}{10}
  \ville{Mostagedda}          { 26.35,   32.37}{above right}{5 and -50}
  \ville{Assiout}             { 17.15,   34.22}{left}{20}
  \ville{Amarna}              { 12.10,   46.82}{right}{20}
  \ville[align=right, text width=65]%
        {Achmounein (\mbox{Hermopolis Magna})}%
        {  9.00,   56.28}{below left}{-25 and 15}
  \ville{Dechacheh}           { 11.85,   88.52}{below left}{-30 and 10}
  \ville{Hérakléopolis}       { 14.95,   92.47}
        {below right}{-15 and 10}
  \ville{Sedment}             { 14.45,   95.98}{below left}{-15 and 10}
  \ville{Hawara}              { 18.10,   97.32}{right}{20}
  \ville{Tebtynis}            {  5.05,   97.47}{above left}{-15 and 15}
  \ville{El-Roubaiyât}        { 14.65,  104.07}{above left}{-5 and 10}
  \ville{Meidoum}             { 20.75,  107.52}{right}{20}
  \ville{Dahchour}            { 21.05,  111.97}{right}{20}
  \ville{Saqqara}             { 18.95,  116.77}{below left}{-10 and 10}
  \ville{Memphis}             { 21.10,  119.07}{right}{20}
  \ville{Abousir}             { 18.60,  121.17}{below left}{-15 and 10}
  \ville{Giza}                { 20.16,  124.62}{left}{20}
  \ville{Le Caire}            { 22.16,  125.52}{above left}{15 and 5}
  \ville{Héliopolis}          { 29.80,  126.97}
        {below right}{10 and -40}
  \ville{Tell el-Maskhouta (Pithôm)}{ 50.85,  132.57}{below right}{10}
  \ville{\TeD[Sepia] (Avaris)}{ 43.15,  136.57}{right}{20}
  \ville{Athribis}            { 25.20,  137.67}{left}{20}
  \ville{Kôm Abou Billou (Térénouthis)}{  9.65,  142.67}{left}{20}
  \ville{Boubastis}           { 33.95,  150.27} {left}{20}
  \ville{Tanis}               { 41.55,  156.22} {right}{20}
  \ville{Saïs}                {  8.35,  158.57} {left}{20}
  \ville{Alexandrie}          {-11.75,  161.37} {left}{20}
  \ville{Tell el-Balamoun}    { 31.65,  162.57} {above right}{10}
}

% .. Boussole ..
\newcommand{\boussole}{%
  \node [boussole, circle, draw, inner sep=3] 
        (boussole) at ( 0.00, 0.00) {N} ;
  \draw [boussole] (boussole.south) -- ++ (  0.00,-10.00) ;
  \draw [boussole] (boussole.east)  -- ++ ( 10.00,  0.00) ;
  \draw [boussole] (boussole.west)  -- ++ (-10.00,  0.00) ;
  \draw [boussole, ->, >=latex] 
                   (boussole.north) -- ++ (  0.00, 20.00) ;
}

% .. Échelle ..
\newcommand{\echelle}{%
  % ... Kilomètres ...
  \begin{scope}[shift={( 0.00, 1.50)}]
     \draw [legende] (   0.00, 0.00) -- ++ ( 107.00, 0.00) 
          node [above left=0 and 50, pos=0] {Kilomètres} ;

    \foreach \x\v in { 0.00/0, 13.38/50, 26.75/100, 
                      53.50/200, 80.25/300, 107.00/400}{%
      \draw [legende] ( \x, 0.00) -- ++ ( 0.00, 2.60) 
            node [above=30] {\v} ;
    }
  \end{scope}
  % ... Miles ...
  \begin{scope}[shift={( 0.00,-1.50)}]
    \draw [legende] (   0.00, 0.00) -- ++ ( 129.40, 0.00) 
          node [below left=0 and 50, pos=0] {Miles} ;

    \foreach \x\v in { 0.00/0, 21.57/50, 43.13/100, 
                      86.27/200, 129.40/300}{%
      \draw [legende] ( \x, 0.00) -- ++ ( 0.00,-2.60) 
            node [below=30] {\v} ;
    }
  \end{scope}
}


% .. Frontières ..
% ... Frontière antique ...
\newcommand{\frontold}{%
  \begin{scope}[oldfront]
    \draw (   0.00,-2.24) node (L) {} .. 
          controls (  28.76,-2.96) and (  49.49,-2.83) ..
          (  84.40,-2.80) .. 
          controls ( 115.02,-2.46) and ( 144.30,-1.55) .. 
          ( 173.12,-0.00) ;
    \node [left, xshift=-5, align=right, text width=80] at (L) 
          {Frontière sud de l'Ancienne \kmt\par} ;
  \end{scope}
}
% ... Frontière moderne ...
\newcommand{\frontnew}{%
  \begin{scope}[newfront]
    \draw ( 309.09,-273.13) --  ( 297.67,-284.66) -- 
          ( 289.99,-280.93) --  ( 282.53,-299.70) -- 
          ( 282.53,-299.70) --  ( 282.53,-299.70) -- 
          ( 268.06,-302.18) --  ( 263.99,-315.86) -- 
          ( 246.24,-318.01) --  ( 235.73,-308.29) .. controls 
          ( 209.71,-311.33) and (  56.12,-309.25) .. 
          (   4.68,-308.29) node [pos=0.8] (L) {} --  
          (   5.25, -96.12) --  (   1.41, -85.16) ;
    \draw ( 0.73,-61.31) .. 
          controls ( 1.48,-60.66) and ( 1.50,-60.08) ..
          ( 1.80,-58.88) -- ( 6.38,-41.07) ..
          controls ( 0.30,-35.36) and ( 0.00,-25.51) .. 
          ( 0.73,-22.48) -- ( 0.11,-19.71) ;
    \draw (  0.01,-6.22) .. 
          controls ( 4.03,-4.33) and ( 7.03,-2.98) .. 
          ( 10.41, 0.00) ;
    \node [above, yshift=5]  at (L) {\kmt};
    \node [below, yshift=-5] at (L) {Soudan};
  \end{scope}
}

% .. Noms des régions ..
\newcommand{\regions}{%
  \node [region] at ( 45.00,-200.00) {KOUSH} ;
  \node [region] at ( 50.00,-150.00) {HAUTE-NUBIE} ;
  \node [region] at ( 30.00,-100.00) {BASSE-NUBIE} ;
  \node [region] at (100.00, 100.00) {SINAÏ} ;
  \node [region] at (-40.00,  10.00) {HAUTE-ÉGYPTE} ;
  \node [region] at (-35.00, 115.00) {FAYOUM} ;
  \node [region] at ( 55.00, 145.00) {BASSE-ÉGYPTE} ;
}

% .. Légende des masses d'eau ..
\newcommand{\legendemasseseau}{%
\begin{scope}[every path/.style={%
                decorate, decoration={text along path}
              }]
  % ... Méditerranée ...
  \path [shift={(-75.00, 190.00)}, 
         decoration={%
           text={|\small\color{sealegend}|Mer M{\'e}diterran{\'e}e ||}
         }]
        (  0.00, 0.00) to [out=-24, in=-156] ( 54.00, 0.00) ;

  % ... Delta ...
  \draw [shift={((11.50, 170.00))}, 
         decoration={text={|\small\color{sealegend}|Delta~|| }}] 
        (  0.00, 3.91) to [out=30, in=-210] ( 23.82, 0.00) ;

  % ... Mer Rouge ...
  \draw [shift={((127.00, 28.76))}, 
         decoration={text={|\small\color{sealegend}|Mer Rouge~||}}]
        (  0.00,  0.00) -- ( 16.00,-28.00) ;            

  % ... Bahr Yousouf ...
  \draw [shift={(4.69, 82.26)}, 
         decoration={text={|\miniscule\color{sea!75!black}|Bahr ~||}}]
        ( 2.00,-24.50) to ( 1.75,-20.00) to ( 0.00,-15.00) ;
  \draw [shift={(4.69, 82.26)}, 
         decoration={  text={|\miniscule\color{sea!75!black}|Y{\kern-1pto}us{\kern+1pts}ouf ~||}}]
        ( 0.00,-14.70) to ( 0.30,-10.00) to ( 2.60,-5.00)
                       to ( 4.50,  0.00) to ( 5.00, 5.00) ;

  % ... Nil ...
  \draw [shift={(12.00, 66.30)}, 
         decoration={text={|\miniscule\color{sea!75!black}|Nil ~||}}]
        ( 0.00, 0.00) to ( 2.72, 4.05) ;
  \draw [shift={(16.40,-140.00)}, 
         decoration={text={|\miniscule\color{sea!75!black}|Nil~||}}]
        ( 0.00, 0.00) to ( 4.19, 4.11) ;
\end{scope}
}

% .. Lacs ..
\newcommand{\lacalexandrie}{%
  \fill [mers]
        ( -22.30, 151.23) .. controls ( -22.60, 151.54) and ( -22.60, 151.68) ..
        ( -22.36, 152.21) .. controls ( -22.08, 152.83) and ( -21.41, 153.37) ..
        ( -20.08, 154.06) .. controls ( -19.06, 154.58) and ( -18.67, 154.86) ..
        ( -17.68, 155.75) .. controls ( -17.03, 156.34) and ( -16.12, 157.09) ..
        ( -15.65, 157.43) .. controls ( -14.73, 158.10) and ( -14.27, 158.56) ..
        ( -13.19, 159.85) .. controls ( -12.81, 160.31) and ( -12.37, 160.75) ..
        ( -12.20, 160.84) .. controls ( -12.01, 160.94) and ( -11.53, 161.01) ..
        ( -10.89, 161.04) .. controls ( -10.59, 161.05) and ( -10.41, 161.05) ..
        ( -10.05, 160.98) .. controls (  -9.91, 160.93) and (  -9.82, 160.91) ..
        (  -9.71, 160.81) .. controls (  -9.58, 160.70) and (  -9.52, 160.34) ..
        (  -9.90, 159.26) .. controls ( -10.15, 158.54) and ( -10.16, 158.48) ..
        ( -10.02, 158.14) .. controls (  -9.91, 157.88) and (  -9.75, 157.72) ..
        (  -9.48, 157.59) .. controls (  -9.25, 157.48) and (  -9.09, 157.33) ..
        (  -9.07, 157.20) .. controls (  -9.03, 156.95) and (  -9.61, 156.20) ..
        ( -10.48, 155.35) .. controls ( -11.45, 154.41) and ( -11.80, 154.42) ..
        ( -12.56, 155.46) .. controls ( -13.49, 156.74) and ( -13.94, 156.64) ..
        ( -14.85, 154.97) .. controls ( -15.44, 153.89) and ( -15.99, 153.22) ..
        ( -16.45, 153.04) .. controls ( -16.59, 152.99) and ( -17.25, 152.90) ..
        ( -17.91, 152.86) .. controls ( -18.58, 152.81) and ( -19.28, 152.71) ..
        ( -19.47, 152.63) .. controls ( -19.66, 152.55) and ( -20.15, 152.15) ..
        ( -20.57, 151.73) .. controls ( -21.24, 151.05) and ( -21.37, 150.96) ..
        ( -21.69, 150.96) .. controls ( -21.96, 150.96) and ( -22.13, 151.04) ..
        ( -22.30, 151.23) -- 
        cycle ;
k}
\newcommand{\lacavaris}{%
  \fill [mers]
        (  54.70, 135.35) .. controls (  53.80, 135.87) and (  53.48, 136.40) ..
        (  53.84, 136.76) .. controls (  54.26, 137.17) and (  55.51, 136.94) ..
        (  56.12, 136.34) .. controls (  56.51, 135.96) and (  56.54, 135.64) ..
        (  56.20, 135.31) .. controls (  55.86, 134.97) and (  55.34, 134.98) ..
        (  54.70, 135.35) -- 
        cycle ;
}
\newcommand{\lacsinai}{%
  \fill [mers]
        (  77.37, 153.91) .. controls (  77.09, 154.15) and (  77.05, 154.25) ..
        (  77.05, 154.66) .. controls (  77.05, 154.92) and (  77.14, 155.39) ..
        (  77.25, 155.71) .. controls (  77.49, 156.42) and (  77.51, 156.85) ..
        (  77.29, 156.93) .. controls (  77.06, 157.02) and (  76.06, 156.72) ..
        (  74.72, 156.15) .. controls (  72.30, 155.26) and (  72.21, 155.16) ..
        (  69.16, 154.72) .. controls (  68.43, 154.62) and (  68.06, 154.60) ..
        (  67.93, 154.67) .. controls (  67.48, 154.91) and (  67.98, 155.25) ..
        (  69.84, 156.00) .. controls (  70.60, 156.31) and (  72.03, 156.93) ..
        (  73.02, 157.40) .. controls (  75.87, 158.75) and (  76.96, 159.06) ..
        (  78.73, 159.06) .. controls (  79.98, 159.06) and (  81.02, 158.87) ..
        (  82.52, 158.35) .. controls (  85.01, 157.49) and (  85.43, 157.38) ..
        (  86.42, 157.27) .. controls (  87.17, 157.18) and (  87.89, 157.18) ..
        (  89.14, 157.27) .. controls (  90.08, 157.33) and (  90.95, 157.35) ..
        (  91.07, 157.31) .. controls (  91.69, 157.12) and (  91.82, 156.59) ..
        (  91.39, 156.07) .. controls (  91.25, 155.90) and (  90.79, 155.60) ..
        (  90.37, 155.39) .. controls (  89.62, 155.02) and (  89.59, 155.01) ..
        (  88.50, 155.02) .. controls (  87.44, 155.03) and (  87.34, 155.05) ..
        (  85.89, 155.55) .. controls (  83.66, 156.31) and (  83.14, 156.26) ..
        (  81.00, 155.02) .. controls (  79.57, 154.18) and (  78.71, 153.79) ..
        (  78.14, 153.71) .. controls (  77.74, 153.65) and (  77.65, 153.68) ..
        (  77.37, 153.91) -- 
        cycle ;
}
\newcommand{\lacpithom}{%
  \fill [mers]
        (  59.10, 128.03) .. controls (  58.39, 128.13) and (  57.73, 128.39) ..
        (  57.17, 128.81) .. controls (  56.39, 129.38) and (  55.87, 129.64) ..
        (  55.35, 129.73) .. controls (  54.60, 129.85) and (  53.98, 130.58) ..
        (  54.08, 131.23) .. controls (  54.17, 131.74) and (  54.55, 132.45) ..
        (  54.93, 132.77) .. controls (  55.32, 133.12) and (  55.63, 133.14) ..
        (  56.42, 132.87) .. controls (  56.74, 132.76) and (  57.29, 132.62) ..
        (  57.65, 132.56) .. controls (  58.47, 132.42) and (  58.90, 132.24) ..
        (  59.04, 131.98) .. controls (  59.10, 131.87) and (  59.15, 131.40) ..
        (  59.15, 130.94) .. controls (  59.15, 130.17) and (  59.17, 130.08) ..
        (  59.41, 129.86) .. controls (  59.55, 129.72) and (  59.94, 129.48) ..
        (  60.28, 129.31) .. controls (  61.06, 128.93) and (  61.17, 128.76) ..
        (  60.85, 128.42) .. controls (  60.49, 128.03) and (  59.97, 127.92) ..
        (  59.10, 128.03) -- 
        cycle ;
}
\newcommand{\lactanis}{%
  \fill [mers]
        (  53.32, 151.16) .. controls (  52.75, 151.16) and (  52.12, 151.62) ..
        (  51.30, 152.59) .. controls (  51.17, 152.75) and (  51.08, 152.88) ..
        (  50.94, 153.07) .. controls (  50.78, 153.31) and (  50.52, 153.60) ..
        (  50.39, 153.71) .. controls (  50.36, 153.74) and (  50.16, 153.91) ..
        (  49.95, 154.06) .. controls (  49.42, 154.41) and (  49.04, 154.34) ..
        (  47.95, 153.91) .. controls (  46.93, 153.51) and (  46.28, 153.46) ..
        (  46.02, 153.75) .. controls (  45.92, 153.85) and (  45.85, 154.00) ..
        (  45.85, 154.09) .. controls (  45.85, 154.28) and (  46.40, 154.77) ..
        (  47.02, 155.12) .. controls (  48.06, 155.72) and (  48.09, 155.90) ..
        (  47.26, 156.41) .. controls (  46.11, 157.11) and (  45.90, 157.69) ..
        (  46.40, 158.81) .. controls (  46.77, 159.66) and (  46.81, 159.98) ..
        (  46.56, 160.21) .. controls (  46.44, 160.31) and (  46.16, 160.43) ..
        (  45.92, 160.47) .. controls (  44.72, 160.70) and (  43.82, 160.97) ..
        (  43.05, 161.33) .. controls (  42.58, 161.56) and (  42.02, 161.79) ..
        (  41.80, 161.86) .. controls (  41.09, 162.09) and (  40.23, 162.60) ..
        (  40.04, 162.91) .. controls (  39.55, 163.71) and (  39.93, 164.63) ..
        (  41.33, 166.00) .. controls (  42.22, 166.87) and (  42.37, 167.06) ..
        (  42.60, 167.67) .. controls (  42.93, 168.53) and (  43.23, 168.98) ..
        (  43.59, 169.15) .. controls (  43.99, 169.34) and (  44.26, 169.19) ..
        (  45.48, 168.15) .. controls (  46.02, 167.69) and (  46.64, 167.17) ..
        (  46.86, 166.99) .. controls (  47.09, 166.81) and (  47.38, 166.45) ..
        (  47.51, 166.19) .. controls (  47.85, 165.50) and (  48.58, 164.62) ..
        (  49.13, 164.22) .. controls (  49.86, 163.69) and (  52.20, 162.42) ..
        (  53.20, 162.00) .. controls (  54.54, 161.45) and (  54.54, 161.45) ..
        (  54.53, 160.18) .. controls (  54.53, 159.46) and (  54.58, 158.95) ..
        (  54.69, 158.61) .. controls (  54.89, 157.99) and (  54.89, 157.34) ..
        (  54.70, 156.97) .. controls (  54.62, 156.81) and (  54.21, 156.40) ..
        (  53.80, 156.06) .. controls (  52.81, 155.26) and (  52.57, 154.79) ..
        (  52.86, 154.23) .. controls (  52.93, 154.11) and (  53.19, 153.77) ..
        (  53.45, 153.46) .. controls (  54.05, 152.77) and (  54.14, 152.57) ..
        (  54.15, 152.10) .. controls (  54.09, 151.49) and (  53.88, 151.18) ..
        (  53.32, 151.16) -- 
        cycle ;
}
\newcommand{\lacfayoum}{%
  \fill [mers]
        (  11.77, 106.45) -- 
        (  11.72, 105.94) .. controls (  11.58, 105.38) and (  10.90, 104.67) ..
        (  10.18, 104.31) .. controls (   9.71, 104.08) and (   9.42, 104.02) ..
        (   8.65, 103.97) -- 
        (   7.70, 103.91) -- 
        (   6.90, 103.78) -- 
        (   6.60, 103.71) .. controls (   6.44, 103.68) and (   5.74, 103.61) ..
        (   5.05, 103.57) -- 
        (   3.80, 103.50) -- 
        (   2.82, 102.98) .. controls (   2.28, 102.70) and (   1.74, 102.46) ..
        (   1.61, 102.46) .. controls (   1.49, 102.46) and (   1.16, 102.58) ..
        (   0.89, 102.72) .. controls (  -0.07, 103.23) and (  -0.31, 103.33) ..
        (  -0.96, 103.50) .. controls (  -1.84, 103.74) and (  -2.19, 104.02) ..
        (  -2.19, 104.51) .. controls (  -2.19, 105.33) and (  -1.39, 105.50) ..
        (  -0.29, 104.91) .. controls (   0.49, 104.50) and (   0.77, 104.49) ..
        (   1.69, 104.81) .. controls (   2.47, 105.09) and (   2.72, 105.28) ..
        (   3.26, 106.08) .. controls (   3.47, 106.38) and (   3.78, 106.76) ..
        (   3.95, 106.90) .. controls (   4.30, 107.19) and (   4.55, 107.18) ..
        (   5.10, 106.85) .. controls (   5.51, 106.60) and (   5.57, 106.61) ..
        (   6.05, 107.01) .. controls (   6.72, 107.58) and (   7.23, 107.63) ..
        (   8.45, 107.27) .. controls (   9.04, 107.09) and (   9.69, 107.14) ..
        (  10.27, 107.41) .. controls (  10.88, 107.70) and (  11.13, 107.72) ..
        (  11.41, 107.52) .. controls (  11.51, 107.44) and (  11.58, 107.37) ..
        (  11.72, 107.20) .. controls (  11.89, 106.93) and (  11.79, 106.64) ..
        (  11.77, 106.45) -- 
        cycle ;
}
\newcommand{\lacdelta}{%
  \fill [mers]
        (  20.89, 170.00) -- 
        (  20.33, 170.01) .. controls (  19.66, 170.01) and (  19.50, 169.90) ..
        (  19.04, 169.03) .. controls (  18.61, 168.21) and (  17.47, 167.08) ..
        (  16.70, 166.71) .. controls (  16.19, 166.46) and (  15.94, 166.41) ..
        (  15.00, 166.36) .. controls (  14.05, 166.31) and (  13.83, 166.26) ..
        (  13.40, 166.02) .. controls (  13.03, 165.82) and (  12.65, 165.75) ..
        (  12.39, 165.78) .. controls (  11.57, 165.83) and (  11.08, 166.30) ..
        (  10.46, 166.57) .. controls (   9.77, 166.83) and (   9.64, 166.69) ..
        (   8.22, 166.05) .. controls (   7.62, 165.78) and (   7.01, 165.56) ..
        (   6.85, 165.56) .. controls (   6.15, 165.56) and (   5.75, 166.84) ..
        (   6.28, 167.35) .. controls (   6.48, 167.53) and (   6.63, 167.56) ..
        (   7.37, 167.56) .. controls (   8.23, 167.56) and (   8.24, 167.57) ..
        (   9.40, 168.16) .. controls (  10.68, 168.82) and (  10.88, 168.86) ..
        (  11.80, 168.60) .. controls (  13.39, 168.16) and (  13.94, 168.14) ..
        (  14.22, 168.52) .. controls (  14.28, 168.61) and (  14.36, 168.96) ..
        (  14.40, 169.30) .. controls (  14.50, 170.13) and (  14.64, 170.44) ..
        (  15.18, 170.98) .. controls (  15.92, 171.72) and (  16.02, 171.74) ..
        (  18.32, 171.72) .. controls (  19.95, 171.70) and (  20.39, 171.66) ..
        (  20.58, 171.54) .. controls (  20.96, 171.29) and (  21.45, 170.76) ..
        (  21.45, 170.60) .. controls (  21.52, 170.19) and (  21.24, 170.07) ..
        (  20.89, 170.00) -- 
        cycle ;
}

\newcommand{\lenil}{%
  \fill [mers]
        (  20.01, 128.93) .. controls (  20.39, 129.13) and (  20.71, 129.30) ..
        (  20.73, 129.31) .. controls (  20.74, 129.32) and (  20.59, 129.60) ..
        (  20.40, 129.92) .. controls (  19.94, 130.68) and (  19.76, 131.31) ..
        (  19.69, 132.35) .. controls (  19.62, 133.19) and (  19.62, 133.19) ..
        (  19.17, 133.60) .. controls (  18.61, 134.12) and (  18.26, 134.66) ..
        (  18.11, 135.25) .. controls (  17.96, 135.85) and (  17.58, 136.11) ..
        (  16.62, 136.26) .. controls (  14.82, 136.54) and (  14.33, 137.07) ..
        (  14.20, 138.91) .. controls (  14.12, 139.94) and (  13.92, 140.78) ..
        (  13.70, 140.96) .. controls (  13.58, 141.06) and (  13.52, 141.02) ..
        (  13.36, 140.75) .. controls (  13.10, 140.34) and (  12.85, 139.23) ..
        (  12.85, 138.56) .. controls (  12.85, 137.59) and (  13.00, 137.32) ..
        (  14.27, 135.99) .. controls (  15.58, 134.61) and (  15.62, 134.52) ..
        (  15.52, 133.34) .. controls (  15.45, 132.52) and (  15.61, 132.01) ..
        (  15.97, 131.88) .. controls (  16.10, 131.83) and (  16.47, 131.76) ..
        (  16.81, 131.73) .. controls (  17.90, 131.62) and (  17.91, 131.61) ..
        (  17.99, 130.59) .. controls (  18.07, 129.51) and (  18.18, 129.10) ..
        (  18.47, 128.80) .. controls (  18.78, 128.46) and (  19.10, 128.50) ..
        (  20.01, 128.93) -- cycle 
        (  11.57, 107.33) .. controls (  11.62, 107.35) and (  11.81, 107.40) ..
        (  11.95, 107.30) .. controls (  12.09, 107.20) and (  12.39, 106.92) ..
        (  12.55, 106.77) .. controls (  13.24, 106.11) and (  13.30, 106.06) ..
        (  14.36, 106.54) .. controls (  15.23, 106.93) and (  15.62, 107.02) ..
        (  16.03, 106.85) .. controls (  16.48, 106.66) and (  18.80, 104.17) ..
        (  18.95, 103.71) .. controls (  19.09, 103.29) and (  19.01, 102.66) ..
        (  18.77, 102.19) .. controls (  18.58, 101.83) and (  17.80, 101.14) ..
        (  16.91, 100.54) .. controls (  16.57, 100.31) and (  16.11,  99.98) ..
        (  15.88,  99.81) -- 
        (  15.46,  99.49) -- 
        (  15.88,  99.30) .. controls (  16.41,  99.06) and (  16.82,  98.72) ..
        (  17.00,  98.36) .. controls (  17.20,  97.97) and (  17.19,  97.28) ..
        (  16.97,  96.84) .. controls (  16.70,  96.32) and (  15.61,  95.13) ..
        (  14.85,  94.53) .. controls (  14.49,  94.25) and (  14.15,  93.95) ..
        (  14.08,  93.86) .. controls (  13.89,  93.63) and (  13.93,  93.30) ..
        (  14.19,  93.00) .. controls (  14.59,  92.51) and (  14.75,  92.10) ..
        (  14.75,  91.54) .. controls (  14.75,  90.85) and (  14.44,  90.35) ..
        (  13.48,  89.46) .. controls (  12.29,  88.37) and (  11.84,  87.70) ..
        (  11.50,  86.52) .. controls (  11.35,  86.02) and (  11.33,  85.69) ..
        (  11.39,  84.47) .. controls (  11.50,  82.02) and (  11.34,  81.33) ..
        (  10.15,  79.31) .. controls (   9.43,  78.09) and (   7.88,  74.82) ..
        (   7.47,  73.66) .. controls (   6.60,  71.17) and (   6.33,  67.93) ..
        (   6.85,  66.22) .. controls (   6.96,  65.88) and (   7.31,  65.04) ..
        (   7.64,  64.36) .. controls (   7.97,  63.67) and (   8.30,  62.87) ..
        (   8.39,  62.56) .. controls (   8.74,  61.35) and (   8.83,  59.16) ..
        (   8.65,  56.45) .. controls (   8.54,  54.96) and (   8.45,  54.44) ..
        (   7.61,  50.81) .. controls (   7.09,  48.57) and (   7.18,  47.26) ..
        (   7.95,  45.69) .. controls (   8.50,  44.58) and (   9.33,  43.56) ..
        (  10.46,  42.61) .. controls (  10.98,  42.18) and (  11.32,  41.91) ..
        (  11.23,  42.02) .. controls (  10.67,  42.67) and (  10.25,  44.25) ..
        (  10.25,  45.68) .. controls (  10.25,  47.16) and (  10.56,  48.07) ..
        (  11.48,  49.33) .. controls (  11.73,  49.67) and (  12.00,  50.08) ..
        (  12.09,  50.25) .. controls (  12.45,  50.93) and (  12.17,  52.32) ..
        (  11.50,  53.46) .. controls (  10.42,  55.11) and (  10.26,  55.92) ..
        (  10.69,  57.48) .. controls (  11.10,  58.98) and (  11.07,  59.21) ..
        (  10.28,  60.77) .. controls (   9.64,  62.05) and (   9.61,  62.13) ..
        (   9.53,  63.10) .. controls (   9.37,  65.01) and (   9.28,  65.38) ..
        (   8.74,  66.51) .. controls (   7.82,  68.38) and (   7.91,  68.82) ..
        (   9.54,  70.66) .. controls (  10.63,  71.90) and (  11.05,  72.60) ..
        (  11.05,  73.16) .. controls (  11.05,  73.35) and (  10.98,  73.85) ..
        (  10.89,  74.26) .. controls (  10.69,  75.23) and (  10.69,  77.08) ..
        (  10.90,  78.27) .. controls (  11.31,  80.64) and (  12.10,  83.21) ..
        (  12.95,  84.91) .. controls (  13.73,  86.49) and (  14.51,  87.45) ..
        (  16.23,  88.94) .. controls (  17.46,  90.01) and (  17.66,  90.23) ..
        (  18.00,  90.88) .. controls (  18.80,  92.42) and (  19.47,  93.14) ..
        (  21.10,  94.24) .. controls (  22.56,  95.21) and (  23.04,  96.35) ..
        (  23.05,  98.84) .. controls (  23.05,  99.94) and (  23.40, 101.20) ..
        (  24.12, 102.60) .. controls (  24.73, 103.86) and (  24.77, 104.18) ..
        (  24.41, 105.04) .. controls (  23.93, 106.20) and (  23.84, 106.67) ..
        (  23.76, 108.46) .. controls (  23.69, 109.96) and (  23.64, 110.32) ..
        (  23.43, 110.91) .. controls (  22.83, 112.56) and (  22.81, 112.97) ..
        (  23.04, 116.75) .. controls (  23.11, 117.98) and (  23.14, 119.21) ..
        (  23.11, 119.50) .. controls (  23.07, 119.78) and (  22.81, 120.55) ..
        (  22.53, 121.21) .. controls (  22.09, 122.24) and (  22.01, 122.53) ..
        (  21.97, 123.23) .. controls (  21.92, 124.17) and (  22.05, 124.54) ..
        (  22.75, 125.41) .. controls (  23.43, 126.27) and (  23.48, 126.36) ..
        (  23.53, 126.85) .. controls (  23.59, 127.47) and (  23.34, 127.75) ..
        (  22.45, 128.12) .. controls (  22.09, 128.27) and (  21.68, 128.47) ..
        (  21.53, 128.58) -- 
        (  21.26, 128.77) -- 
        (  20.37, 128.32) .. controls (  19.66, 127.95) and (  19.38, 127.86) ..
        (  18.97, 127.86) .. controls (  18.50, 127.86) and (  18.41, 127.90) ..
        (  18.01, 128.29) .. controls (  17.52, 128.76) and (  17.40, 129.09) ..
        (  17.29, 130.23) .. controls (  17.23, 130.92) and (  17.22, 130.94) ..
        (  16.93, 131.00) .. controls (  16.76, 131.04) and (  16.50, 131.06) ..
        (  16.35, 131.06) .. controls (  16.00, 131.06) and (  15.44, 131.29) ..
        (  15.23, 131.53) .. controls (  14.85, 131.95) and (  14.73, 132.45) ..
        (  14.80, 133.30) -- 
        (  14.87, 134.11) -- 
        (  14.49, 134.61) .. controls (  14.28, 134.89) and (  13.78, 135.47) ..
        (  13.37, 135.90) .. controls (  12.40, 136.92) and (  12.15, 137.46) ..
        (  12.15, 138.51) .. controls (  12.15, 139.34) and (  12.35, 140.38) ..
        (  12.59, 140.84) .. controls (  12.73, 141.12) and (  12.74, 141.11) ..
        (  11.68, 143.37) .. controls (  11.24, 144.32) and (  11.23, 144.65) ..
        (  11.63, 145.46) .. controls (  11.81, 145.81) and (  11.95, 146.19) ..
        (  11.95, 146.31) .. controls (  11.95, 146.44) and (  11.71, 146.77) ..
        (  11.35, 147.13) .. controls (  10.77, 147.71) and (  10.75, 147.76) ..
        (  10.75, 148.24) .. controls (  10.75, 148.71) and (  10.80, 148.82) ..
        (  11.33, 149.53) .. controls (  11.82, 150.18) and (  12.15, 150.75) ..
        (  12.15, 150.91) .. controls (  12.15, 150.93) and (  11.97, 150.98) ..
        (  11.75, 151.01) .. controls (  11.16, 151.11) and (  10.65, 151.62) ..
        (  10.65, 152.11) .. controls (  10.65, 152.41) and (  10.74, 152.57) ..
        (  11.15, 153.00) .. controls (  11.89, 153.79) and (  11.81, 154.10) ..
        (  10.68, 154.73) .. controls (   9.78, 155.23) and (   9.39, 155.65) ..
        (   8.80, 156.73) .. controls (   8.24, 157.75) and (   7.92, 158.02) ..
        (   7.10, 158.13) .. controls (   6.51, 158.21) and (   5.82, 158.66) ..
        (   5.37, 159.26) .. controls (   5.18, 159.51) and (   4.89, 159.79) ..
        (   4.71, 159.90) .. controls (   3.89, 160.39) and (   3.75, 160.81) ..
        (   4.22, 161.37) .. controls (   4.37, 161.55) and (   4.77, 161.81) ..
        (   5.11, 161.97) .. controls (   5.63, 162.20) and (   5.75, 162.30) ..
        (   5.84, 162.59) .. controls (   6.09, 163.35) and (   5.65, 163.75) ..
        (   4.48, 163.83) .. controls (   3.37, 163.91) and (   3.14, 164.08) ..
        (   2.76, 165.06) .. controls (   2.59, 165.50) and (   2.45, 165.88) ..
        (   2.45, 165.91) .. controls (   2.45, 165.94) and (   2.20, 166.47) ..
        (   1.90, 167.09) .. controls (   1.36, 168.20) and (   1.17, 169.36) ..
        (   1.17, 169.66) -- 
        (   1.17, 170.12) -- 
        (   1.75, 170.23) .. controls (   1.74, 170.00) and (   1.76, 169.83) ..
        (   1.78, 169.60) .. controls (   1.78, 168.89) and (   2.07, 168.37) ..
        (   2.64, 167.16) .. controls (   2.97, 166.48) and (   3.31, 165.68) ..
        (   3.40, 165.39) .. controls (   3.62, 164.69) and (   3.77, 164.56) ..
        (   4.33, 164.56) .. controls (   5.85, 164.55) and (   6.77, 163.80) ..
        (   6.60, 162.69) .. controls (   6.52, 162.13) and (   6.06, 161.60) ..
        (   5.40, 161.31) .. controls (   4.85, 161.07) and (   4.51, 160.76) ..
        (   4.72, 160.69) .. controls (   5.02, 160.59) and (   5.52, 160.17) ..
        (   5.83, 159.76) .. controls (   6.21, 159.26) and (   6.78, 158.87) ..
        (   7.12, 158.86) .. controls (   7.50, 158.86) and (   8.29, 158.55) ..
        (   8.67, 158.24) .. controls (   8.92, 158.05) and (   9.17, 157.69) ..
        (   9.41, 157.18) .. controls (   9.86, 156.23) and (  10.17, 155.81) ..
        (  10.58, 155.60) .. controls (  11.27, 155.25) and (  11.95, 154.73) ..
        (  12.15, 154.42) .. controls (  12.27, 154.23) and (  12.35, 153.92) ..
        (  12.35, 153.67) .. controls (  12.35, 153.30) and (  12.29, 153.18) ..
        (  11.83, 152.70) .. controls (  11.21, 152.04) and (  11.24, 151.88) ..
        (  12.06, 151.69) .. controls (  12.64, 151.57) and (  12.85, 151.37) ..
        (  12.85, 150.97) .. controls (  12.85, 150.63) and (  12.45, 149.78) ..
        (  12.08, 149.36) .. controls (  11.24, 148.38) and (  11.24, 148.16) ..
        (  12.05, 147.38) .. controls (  12.61, 146.84) and (  12.65, 146.77) ..
        (  12.65, 146.34) .. controls (  12.65, 146.03) and (  12.56, 145.71) ..
        (  12.35, 145.31) .. controls (  11.99, 144.63) and (  11.98, 144.40) ..
        (  12.29, 143.79) .. controls (  12.42, 143.53) and (  12.67, 142.96) ..
        (  12.85, 142.54) .. controls (  13.29, 141.33) and (  13.58, 142.16) ..
        (  14.19, 141.45) .. controls (  14.63, 140.94) and (  14.81, 140.34) ..
        (  14.90, 139.11) .. controls (  14.98, 137.98) and (  15.21, 137.41) ..
        (  15.67, 137.22) .. controls (  15.85, 137.15) and (  16.34, 137.03) ..
        (  16.75, 136.96) .. controls (  18.03, 136.75) and (  18.53, 136.38) ..
        (  18.85, 135.40) .. controls (  19.06, 134.78) and (  19.57, 133.96) ..
        (  19.74, 133.96) .. controls (  19.88, 133.96) and (  19.98, 134.54) ..
        (  19.90, 134.94) .. controls (  19.85, 135.23) and (  19.54, 135.60) ..
        (  18.32, 136.86) .. controls (  16.13, 139.09) and (  15.81, 139.65) ..
        (  15.50, 141.61) .. controls (  15.33, 142.68) and (  15.32, 144.79) ..
        (  15.49, 145.17) .. controls (  15.54, 145.28) and (  15.34, 145.54) ..
        (  14.80, 146.07) .. controls (  13.95, 146.89) and (  13.55, 147.56) ..
        (  13.55, 148.14) .. controls (  13.55, 148.34) and (  13.67, 148.83) ..
        (  13.81, 149.21) .. controls (  14.22, 150.31) and (  14.22, 150.38) ..
        (  13.81, 151.10) .. controls (  13.35, 151.93) and (  13.34, 152.29) ..
        (  13.80, 153.11) .. controls (  14.20, 153.83) and (  14.23, 154.15) ..
        (  13.95, 154.87) .. controls (  13.83, 155.17) and (  13.72, 155.74) ..
        (  13.69, 156.20) .. controls (  13.66, 156.65) and (  13.62, 157.12) ..
        (  13.59, 157.25) .. controls (  13.56, 157.38) and (  13.19, 157.86) ..
        (  12.76, 158.30) .. controls (  12.33, 158.75) and (  11.90, 159.26) ..
        (  11.81, 159.44) .. controls (  11.57, 159.91) and (  11.61, 160.52) ..
        (  11.96, 161.52) .. controls (  12.41, 162.80) and (  12.38, 163.24) ..
        (  11.81, 164.56) .. controls (  11.37, 165.60) and (  11.35, 165.89) ..
        (  11.41, 166.17) -- 
        (  11.99, 166.00) -- 
        (  12.21, 165.36) .. controls (  12.31, 165.12) and (  12.56, 164.62) ..
        (  12.72, 164.26) .. controls (  13.09, 163.38) and (  13.09, 162.51) ..
        (  12.70, 161.46) .. controls (  12.17, 160.04) and (  12.23, 159.79) ..
        (  13.30, 158.76) .. controls (  14.17, 157.92) and (  14.31, 157.63) ..
        (  14.38, 156.51) .. controls (  14.41, 155.96) and (  14.51, 155.41) ..
        (  14.64, 155.08) .. controls (  14.96, 154.28) and (  14.92, 153.66) ..
        (  14.50, 152.91) .. controls (  14.31, 152.56) and (  14.15, 152.21) ..
        (  14.15, 152.11) .. controls (  14.15, 152.02) and (  14.31, 151.66) ..
        (  14.51, 151.30) .. controls (  14.94, 150.54) and (  14.94, 150.15) ..
        (  14.51, 148.97) .. controls (  14.13, 147.94) and (  14.18, 147.71) ..
        (  15.07, 146.80) -- 
        (  15.77, 146.08) -- 
        (  16.04, 146.57) .. controls (  16.18, 146.84) and (  16.51, 147.29) ..
        (  16.76, 147.58) -- 
        (  16.76, 147.58) .. controls (  17.07, 147.94) and (  17.57, 148.32) ..
        (  18.28, 148.74) .. controls (  18.87, 149.09) and (  19.41, 149.47) ..
        (  19.50, 149.59) .. controls (  19.60, 149.73) and (  19.65, 150.04) ..
        (  19.65, 150.50) .. controls (  19.65, 151.19) and (  19.65, 151.19) ..
        (  20.09, 151.60) .. controls (  20.34, 151.83) and (  20.72, 152.16) ..
        (  20.94, 152.35) .. controls (  21.44, 152.76) and (  21.51, 153.09) ..
        (  21.31, 153.96) .. controls (  21.10, 154.85) and (  21.11, 155.44) ..
        (  21.35, 155.83) .. controls (  21.64, 156.32) and (  22.05, 156.48) ..
        (  23.24, 156.56) .. controls (  24.82, 156.66) and (  24.88, 156.73) ..
        (  24.64, 158.02) .. controls (  24.49, 158.83) and (  24.27, 159.23) ..
        (  23.45, 160.10) .. controls (  22.81, 160.80) and (  22.75, 160.90) ..
        (  22.75, 161.29) .. controls (  22.75, 161.54) and (  22.83, 161.83) ..
        (  22.95, 162.01) .. controls (  23.26, 162.48) and (  23.21, 162.87) ..
        (  22.77, 163.43) .. controls (  22.24, 164.10) and (  22.15, 164.29) ..
        (  22.15, 164.76) .. controls (  22.15, 165.12) and (  22.21, 165.22) ..
        (  22.66, 165.65) .. controls (  23.06, 166.04) and (  23.18, 166.22) ..
        (  23.22, 166.55) .. controls (  23.30, 167.17) and (  22.88, 167.91) ..
        (  21.79, 169.05) .. controls (  21.79, 169.05) and (  20.99, 169.93) ..
        (  20.86, 170.08) -- 
        (  21.26, 170.57) .. controls (  21.33, 170.53) and (  21.75, 170.13) ..
        (  22.13, 169.73) .. controls (  23.01, 168.79) and (  23.37, 168.33) ..
        (  23.69, 167.69) .. controls (  24.16, 166.75) and (  23.99, 165.85) ..
        (  23.24, 165.25) .. controls (  22.73, 164.85) and (  22.74, 164.63) ..
        (  23.30, 163.89) .. controls (  24.01, 162.94) and (  24.08, 162.53) ..
        (  23.64, 161.80) .. controls (  23.33, 161.30) and (  23.40, 161.12) ..
        (  24.19, 160.27) .. controls (  24.94, 159.46) and (  25.35, 158.70) ..
        (  25.35, 158.09) .. controls (  25.35, 157.53) and (  25.50, 157.52) ..
        (  25.94, 158.03) .. controls (  26.78, 159.00) and (  27.12, 160.01) ..
        (  27.43, 162.39) .. controls (  27.74, 164.71) and (  28.24, 165.47) ..
        (  30.87, 167.61) .. controls (  30.87, 167.61) and (  31.37, 167.98) ..
        (  31.43, 168.01) -- 
        (  31.94, 167.58) .. controls (  31.78, 167.44) and (  31.41, 167.14) ..
        (  31.00, 166.82) .. controls (  30.02, 166.04) and (  28.92, 164.88) ..
        (  28.60, 164.22) .. controls (  28.45, 163.92) and (  28.29, 163.35) ..
        (  28.23, 162.95) .. controls (  27.73, 159.24) and (  27.49, 158.64) ..
        (  25.98, 157.12) .. controls (  24.95, 156.09) and (  24.56, 155.90) ..
        (  23.35, 155.84) .. controls (  22.82, 155.82) and (  22.29, 155.74) ..
        (  22.17, 155.68) .. controls (  21.87, 155.51) and (  21.80, 155.12) ..
        (  21.94, 154.47) .. controls (  22.32, 152.76) and (  22.15, 152.25) ..
        (  20.96, 151.44) .. controls (  20.71, 151.27) and (  20.48, 151.11) ..
        (  20.45, 151.09) .. controls (  20.42, 151.07) and (  20.38, 150.70) ..
        (  20.34, 150.27) .. controls (  20.27, 149.24) and (  20.08, 148.99) ..
        (  18.73, 148.19) .. controls (  17.19, 147.28) and (  16.62, 146.56) ..
        (  16.24, 145.05) .. controls (  16.06, 144.35) and (  16.03, 144.03) ..
        (  16.08, 143.06) .. controls (  16.14, 141.87) and (  16.34, 140.72) ..
        (  16.60, 140.10) .. controls (  16.89, 139.42) and (  17.70, 138.41) ..
        (  19.07, 137.04) .. controls (  19.83, 136.29) and (  20.45, 135.62) ..
        (  20.45, 135.56) .. controls (  20.45, 135.43) and (  20.43, 135.42) ..
        (  21.71, 136.27) .. controls (  23.87, 137.71) and (  25.35, 139.16) ..
        (  25.74, 140.20) .. controls (  26.04, 141.02) and (  26.10, 141.79) ..
        (  25.90, 142.32) .. controls (  25.71, 142.83) and (  25.41, 143.07) ..
        (  24.63, 143.36) .. controls (  23.90, 143.63) and (  23.18, 144.27) ..
        (  23.10, 144.70) .. controls (  22.96, 145.44) and (  23.38, 146.21) ..
        (  24.34, 146.97) .. controls (  24.88, 147.40) and (  25.10, 147.76) ..
        (  25.21, 148.43) .. controls (  25.26, 148.77) and (  25.22, 148.89) ..
        (  24.79, 149.51) .. controls (  24.19, 150.39) and (  24.11, 150.86) ..
        (  24.48, 151.49) .. controls (  24.86, 152.13) and (  25.40, 152.59) ..
        (  27.43, 153.95) .. controls (  28.43, 154.63) and (  29.38, 155.34) ..
        (  29.53, 155.53) .. controls (  29.69, 155.71) and (  29.92, 156.17) ..
        (  30.05, 156.54) .. controls (  30.55, 158.00) and (  31.15, 158.84) ..
        (  31.92, 159.18) .. controls (  32.40, 159.40) and (  32.77, 159.36) ..
        (  33.60, 159.04) .. controls (  34.19, 158.81) and (  34.55, 158.86) ..
        (  35.35, 159.26) .. controls (  36.53, 159.86) and (  37.51, 160.90) ..
        (  37.77, 161.83) .. controls (  37.83, 162.04) and (  37.88, 162.96) ..
        (  37.89, 163.86) .. controls (  37.92, 166.49) and (  38.02, 166.68) ..
        (  40.38, 168.48) .. controls (  41.53, 169.36) and (  42.20, 170.03) ..
        (  42.58, 170.77) .. controls (  42.69, 171.00) and (  42.77, 171.17) ..
        (  42.82, 171.30) -- 
        (  43.45, 171.07) .. controls (  43.45, 171.07) and (  43.31, 170.67) ..
        (  43.10, 170.32) .. controls (  42.61, 169.48) and (  41.93, 168.78) ..
        (  40.68, 167.84) .. controls (  39.63, 167.04) and (  39.13, 166.50) ..
        (  38.80, 165.81) .. controls (  38.64, 165.48) and (  38.61, 165.14) ..
        (  38.59, 163.66) .. controls (  38.57, 161.81) and (  38.50, 161.46) ..
        (  38.02, 160.70) .. controls (  37.61, 160.05) and (  36.43, 159.02) ..
        (  35.66, 158.63) .. controls (  34.69, 158.15) and (  34.20, 158.08) ..
        (  33.46, 158.35) .. controls (  32.09, 158.85) and (  31.54, 158.46) ..
        (  30.80, 156.50) .. controls (  30.26, 155.07) and (  29.87, 154.69) ..
        (  27.18, 152.94) .. controls (  25.54, 151.87) and (  24.86, 151.14) ..
        (  24.99, 150.60) .. controls (  25.03, 150.47) and (  25.26, 150.06) ..
        (  25.50, 149.70) .. controls (  26.02, 148.95) and (  26.07, 148.59) ..
        (  25.75, 147.69) .. controls (  25.59, 147.21) and (  25.43, 146.99) ..
        (  24.80, 146.41) .. controls (  23.91, 145.58) and (  23.70, 145.11) ..
        (  24.02, 144.69) .. controls (  24.12, 144.55) and (  24.69, 144.19) ..
        (  25.30, 143.88) .. controls (  26.37, 143.33) and (  26.40, 143.30) ..
        (  26.58, 142.82) .. controls (  26.84, 142.11) and (  26.83, 140.87) ..
        (  26.56, 140.16) .. controls (  26.33, 139.56) and (  25.60, 138.38) ..
        (  25.17, 137.93) .. controls (  24.88, 137.62) and (  23.07, 136.25) ..
        (  22.10, 135.59) .. controls (  21.76, 135.37) and (  21.44, 135.12) ..
        (  21.37, 135.04) .. controls (  21.30, 134.95) and (  21.12, 134.83) ..
        (  20.97, 134.77) .. controls (  20.72, 134.67) and (  20.68, 134.57) ..
        (  20.51, 133.68) .. controls (  20.31, 132.63) and (  20.33, 132.10) ..
        (  20.60, 131.18) -- 
        (  20.76, 130.67) -- 
        (  21.37, 131.17) .. controls (  21.71, 131.44) and (  22.06, 131.67) ..
        (  22.14, 131.68) .. controls (  22.23, 131.69) and (  22.59, 131.70) ..
        (  22.95, 131.71) .. controls (  23.35, 131.72) and (  23.83, 131.81) ..
        (  24.20, 131.95) -- 
        (  24.81, 132.18) -- 
        (  25.48, 131.92) -- 
        (  26.16, 131.67) -- 
        (  26.39, 132.20) .. controls (  26.74, 132.99) and (  27.25, 133.22) ..
        (  28.11, 132.97) .. controls (  28.58, 132.84) and (  28.69, 132.96) ..
        (  28.56, 133.42) .. controls (  28.22, 134.66) and (  28.62, 135.08) ..
        (  30.32, 135.27) .. controls (  32.09, 135.47) and (  32.48, 135.80) ..
        (  32.06, 136.80) .. controls (  31.95, 137.09) and (  31.85, 137.37) ..
        (  31.85, 137.43) .. controls (  31.85, 137.62) and (  32.45, 138.06) ..
        (  33.04, 138.33) .. controls (  33.77, 138.65) and (  34.07, 138.95) ..
        (  34.41, 139.70) .. controls (  34.64, 140.22) and (  34.70, 140.51) ..
        (  34.76, 141.56) .. controls (  34.86, 143.13) and (  35.03, 143.88) ..
        (  35.50, 144.74) .. controls (  36.21, 146.05) and (  36.67, 146.36) ..
        (  38.30, 146.66) .. controls (  38.69, 146.73) and (  39.13, 146.91) ..
        (  39.60, 147.20) .. controls (  40.82, 147.95) and (  41.22, 148.01) ..
        (  42.56, 147.63) .. controls (  43.82, 147.27) and (  43.88, 147.26) ..
        (  44.31, 147.38) .. controls (  44.78, 147.51) and (  46.38, 149.07) ..
        (  47.90, 150.91) .. controls (  48.45, 151.57) and (  49.30, 152.54) ..
        (  49.74, 152.97) .. controls (  50.17, 153.40) and (  50.41, 153.56) ..
        (  50.57, 153.69) -- 
        (  50.96, 153.18) .. controls (  50.64, 152.91) and (  50.53, 152.76) ..
        (  50.31, 152.55) .. controls (  50.00, 152.25) and (  49.20, 151.39) ..
        (  48.55, 150.62) .. controls (  47.07, 148.86) and (  45.31, 147.04) ..
        (  44.85, 146.81) .. controls (  44.32, 146.54) and (  43.54, 146.57) ..
        (  42.40, 146.90) .. controls (  41.50, 147.16) and (  41.36, 147.18) ..
        (  40.95, 147.07) .. controls (  40.70, 147.01) and (  40.27, 146.80) ..
        (  39.99, 146.62) .. controls (  39.17, 146.08) and (  38.74, 145.90) ..
        (  37.95, 145.76) .. controls (  37.30, 145.65) and (  37.15, 145.58) ..
        (  36.85, 145.27) .. controls (  36.05, 144.44) and (  35.56, 143.04) ..
        (  35.55, 141.56) .. controls (  35.55, 140.40) and (  35.12, 139.19) ..
        (  34.44, 138.42) .. controls (  34.26, 138.22) and (  33.81, 137.91) ..
        (  33.39, 137.71) .. controls (  32.98, 137.52) and (  32.65, 137.34) ..
        (  32.65, 137.32) .. controls (  32.65, 137.30) and (  32.72, 137.13) ..
        (  32.80, 136.93) .. controls (  33.34, 135.63) and (  32.52, 134.78) ..
        (  30.52, 134.57) .. controls (  29.24, 134.43) and (  29.11, 134.36) ..
        (  29.21, 133.88) .. controls (  29.36, 133.21) and (  29.31, 132.68) ..
        (  29.08, 132.44) .. controls (  28.85, 132.22) and (  28.73, 132.21) ..
        (  27.55, 132.29) .. controls (  27.23, 132.31) and (  27.20, 132.29) ..
        (  27.04, 131.87) .. controls (  26.86, 131.37) and (  26.44, 130.96) ..
        (  26.12, 130.97) .. controls (  26.00, 130.97) and (  25.70, 131.06) ..
        (  25.46, 131.17) .. controls (  24.93, 131.41) and (  24.54, 131.42) ..
        (  24.12, 131.20) .. controls (  23.64, 130.95) and (  23.20, 130.87) ..
        (  22.78, 130.97) .. controls (  22.35, 131.07) and (  22.02, 130.91) ..
        (  21.47, 130.34) -- 
        (  21.14, 130.00) -- 
        (  21.54, 129.54) .. controls (  21.82, 129.24) and (  22.13, 129.02) ..
        (  22.52, 128.86) .. controls (  23.73, 128.38) and (  24.15, 127.99) ..
        (  24.23, 127.31) -- 
        (  24.28, 126.84) -- 
        (  25.34, 126.88) .. controls (  26.16, 126.92) and (  26.49, 126.98) ..
        (  26.80, 127.14) .. controls (  27.02, 127.25) and (  27.61, 127.69) ..
        (  28.10, 128.10) .. controls (  29.40, 129.20) and (  30.19, 129.57) ..
        (  32.42, 130.17) .. controls (  34.47, 130.72) and (  35.51, 131.15) ..
        (  36.12, 131.70) .. controls (  36.53, 132.07) and (  37.03, 132.91) ..
        (  37.25, 133.61) .. controls (  37.47, 134.28) and (  37.75, 134.65) ..
        (  38.53, 135.28) .. controls (  38.94, 135.62) and (  39.56, 136.18) ..
        (  39.89, 136.54) .. controls (  40.58, 137.28) and (  42.07, 138.31) ..
        (  42.58, 138.40) .. controls (  42.77, 138.44) and (  43.05, 138.42) ..
        (  43.20, 138.36) .. controls (  43.35, 138.31) and (  43.77, 138.25) ..
        (  44.13, 138.24) .. controls (  44.86, 138.22) and (  45.17, 138.34) ..
        (  45.93, 138.92) .. controls (  46.17, 139.10) and (  46.66, 139.35) ..
        (  47.03, 139.47) .. controls (  47.94, 139.77) and (  48.03, 139.85) ..
        (  48.55, 140.68) .. controls (  49.35, 141.95) and (  49.81, 142.35) ..
        (  51.80, 143.48) .. controls (  52.79, 144.04) and (  53.94, 144.65) ..
        (  54.35, 144.84) .. controls (  55.71, 145.46) and (  56.39, 146.13) ..
        (  56.66, 147.10) .. controls (  56.82, 147.68) and (  56.68, 148.18) ..
        (  55.43, 149.66) .. controls (  54.22, 151.26) and (  53.51, 151.18) ..
        (  53.35, 151.30) -- 
        (  53.86, 152.01) .. controls (  54.05, 151.75) and (  54.21, 151.68) ..
        (  54.65, 151.34) .. controls (  55.38, 150.85) and (  55.47, 150.64) ..
        (  56.01, 150.01) .. controls (  56.66, 149.28) and (  57.06, 148.70) ..
        (  57.29, 148.29) .. controls (  57.44, 147.76) and (  57.47, 147.87) ..
        (  57.42, 147.36) .. controls (  57.39, 146.99) and (  57.27, 146.59) ..
        (  57.09, 146.26) .. controls (  56.62, 145.40) and (  55.41, 144.41) ..
        (  54.47, 144.12) .. controls (  54.35, 144.08) and (  53.33, 143.53) ..
        (  52.22, 142.91) .. controls (  50.14, 141.73) and (  49.81, 141.47) ..
        (  49.43, 140.66) .. controls (  49.17, 140.13) and (  48.43, 139.31) ..
        (  48.01, 139.09) .. controls (  47.85, 139.01) and (  47.49, 138.86) ..
        (  47.20, 138.76) .. controls (  46.92, 138.67) and (  46.39, 138.38) ..
        (  46.02, 138.13) .. controls (  45.65, 137.88) and (  45.18, 137.63) ..
        (  44.98, 137.57) .. controls (  44.50, 137.44) and (  43.38, 137.43) ..
        (  43.30, 137.56) .. controls (  43.18, 137.77) and (  42.60, 137.65) ..
        (  41.96, 137.31) .. controls (  41.54, 137.08) and (  40.93, 136.59) ..
        (  40.30, 135.97) .. controls (  39.75, 135.44) and (  39.05, 134.79) ..
        (  38.75, 134.54) .. controls (  38.26, 134.14) and (  38.15, 133.98) ..
        (  37.87, 133.23) .. controls (  37.07, 131.13) and (  36.03, 130.37) ..
        (  32.77, 129.51) .. controls (  30.17, 128.83) and (  29.74, 128.63) ..
        (  28.30, 127.38) .. controls (  27.16, 126.39) and (  26.82, 126.24) ..
        (  25.43, 126.17) .. controls (  24.07, 126.10) and (  24.23, 126.21) ..
        (  23.19, 124.69) -- 
        (  22.62, 123.87) -- 
        (  22.67, 123.24) .. controls (  22.71, 122.78) and (  22.87, 122.27) ..
        (  23.24, 121.38) .. controls (  23.88, 119.87) and (  23.93, 119.36) ..
        (  23.74, 116.46) .. controls (  23.53, 113.33) and (  23.55, 112.82) ..
        (  24.00, 111.51) .. controls (  24.34, 110.48) and (  24.37, 110.31) ..
        (  24.45, 108.66) .. controls (  24.52, 106.98) and (  24.54, 106.87) ..
        (  24.91, 105.86) .. controls (  25.14, 105.21) and (  25.30, 104.57) ..
        (  25.33, 104.18) .. controls (  25.37, 103.58) and (  25.34, 103.49) ..
        (  24.81, 102.43) .. controls (  24.11, 101.01) and (  23.75,  99.74) ..
        (  23.75,  98.68) .. controls (  23.75,  97.62) and (  23.58,  96.46) ..
        (  23.34,  95.80) .. controls (  23.02,  94.96) and (  22.46,  94.35) ..
        (  21.20,  93.46) .. controls (  19.89,  92.54) and (  19.41,  92.03) ..
        (  18.86,  90.96) .. controls (  18.32,  89.93) and (  18.09,  89.67) ..
        (  16.60,  88.35) .. controls (  14.95,  86.89) and (  14.39,  86.21) ..
        (  13.65,  84.71) .. controls (  12.81,  83.03) and (  11.99,  80.39) ..
        (  11.60,  78.13) .. controls (  11.41,  77.01) and (  11.41,  75.20) ..
        (  11.60,  74.41) .. controls (  11.82,  73.50) and (  11.79,  72.70) ..
        (  11.51,  72.11) .. controls (  11.38,  71.84) and (  10.81,  71.07) ..
        (  10.23,  70.41) .. controls (   9.65,  69.75) and (   9.10,  69.06) ..
        (   9.00,  68.87) .. controls (   8.75,  68.38) and (   8.85,  67.88) ..
        (   9.45,  66.65) .. controls (   9.99,  65.54) and (  10.05,  65.28) ..
        (  10.23,  63.23) .. controls (  10.32,  62.28) and (  10.35,  62.18) ..
        (  10.95,  61.03) .. controls (  11.53,  59.92) and (  11.59,  59.75) ..
        (  11.63,  59.06) .. controls (  11.66,  58.46) and (  11.62,  58.12) ..
        (  11.40,  57.32) .. controls (  11.00,  55.85) and (  11.11,  55.32) ..
        (  12.15,  53.67) .. controls (  12.68,  52.82) and (  12.95,  51.93) ..
        (  12.95,  51.03) .. controls (  12.95,  50.24) and (  12.93,  50.20) ..
        (  11.97,  48.81) .. controls (  11.69,  48.39) and (  11.42,  47.93) ..
        (  11.39,  47.78) .. controls (  11.36,  47.63) and (  11.30,  47.42) ..
        (  11.25,  47.31) .. controls (  10.57,  45.78) and (  11.09,  43.02) ..
        (  12.27,  41.84) .. controls (  12.51,  41.60) and (  13.36,  40.97) ..
        (  14.15,  40.44) .. controls (  16.84,  38.66) and (  18.67,  36.92) ..
        (  20.07,  34.83) .. controls (  21.09,  33.31) and (  21.52,  32.89) ..
        (  23.02,  31.99) .. controls (  24.54,  31.07) and (  25.00,  30.66) ..
        (  25.76,  29.56) .. controls (  26.10,  29.07) and (  26.52,  28.54) ..
        (  26.70,  28.38) .. controls (  26.88,  28.22) and (  27.61,  27.81) ..
        (  28.31,  27.46) .. controls (  29.33,  26.96) and (  29.71,  26.72) ..
        (  30.18,  26.25) .. controls (  30.92,  25.51) and (  31.15,  25.00) ..
        (  31.65,  22.93) .. controls (  31.87,  22.01) and (  32.14,  21.09) ..
        (  32.24,  20.88) .. controls (  32.35,  20.67) and (  32.91,  19.99) ..
        (  33.50,  19.36) .. controls (  34.70,  18.08) and (  34.89,  17.77) ..
        (  35.10,  16.74) .. controls (  35.35,  15.56) and (  35.81,  14.97) ..
        (  37.04,  14.19) .. controls (  37.72,  13.76) and (  38.22,  13.31) ..
        (  38.41,  12.94) .. controls (  38.50,  12.78) and (  38.55,  12.37) ..
        (  38.55,  11.86) .. controls (  38.55,  10.54) and (  38.79,  10.23) ..
        (  40.81,   8.88) .. controls (  42.57,   7.70) and (  42.92,   7.26) ..
        (  43.64,   5.40) .. controls (  44.23,   3.87) and (  44.67,   3.49) ..
        (  46.45,   2.97) .. controls (  48.11,   2.48) and (  48.45,   2.27) ..
        (  49.33,   1.25) .. controls (  50.27,   0.14) and (  50.78,  -0.19) ..
        (  51.55,  -0.17) .. controls (  52.24,  -0.15) and (  55.09,   0.43) ..
        (  55.49,   0.63) .. controls (  55.66,   0.72) and (  56.09,   1.00) ..
        (  56.45,   1.27) .. controls (  57.39,   1.95) and (  57.82,   2.07) ..
        (  59.28,   2.05) .. controls (  60.50,   2.03) and (  61.12,   2.13) ..
        (  62.00,   2.52) .. controls (  63.37,   3.12) and (  64.09,   3.28) ..
        (  65.15,   3.24) .. controls (  66.00,   3.21) and (  66.18,   3.17) ..
        (  66.87,   2.83) .. controls (  67.73,   2.42) and (  68.72,   1.52) ..
        (  69.17,   0.76) .. controls (  69.83,  -0.36) and (  69.91,  -1.65) ..
        (  69.40,  -2.86) .. controls (  68.66,  -4.58) and (  68.42,  -5.32) ..
        (  68.37,  -5.95) .. controls (  68.33,  -6.55) and (  68.36,  -6.68) ..
        (  68.74,  -7.57) .. controls (  69.22,  -8.70) and (  69.29,  -9.48) ..
        (  68.96, -10.12) .. controls (  68.85, -10.32) and (  67.81, -11.41) ..
        (  66.64, -12.54) .. controls (  64.32, -14.78) and (  62.37, -16.92) ..
        (  62.02, -17.63) .. controls (  61.87, -17.95) and (  61.80, -18.27) ..
        (  61.80, -18.69) .. controls (  61.80, -19.23) and (  61.84, -19.35) ..
        (  62.25, -19.99) .. controls (  62.95, -21.07) and (  62.98, -21.16) ..
        (  63.02, -22.59) .. controls (  63.04, -23.30) and (  63.11, -24.01) ..
        (  63.17, -24.16) .. controls (  63.24, -24.34) and (  64.50, -25.48) ..
        (  66.56, -27.24) .. controls (  71.67, -31.60) and (  72.25, -32.26) ..
        (  72.25, -33.70) .. controls (  72.25, -34.34) and (  72.08, -35.21) ..
        (  71.90, -35.50) .. controls (  71.66, -35.87) and (  71.05, -37.74) ..
        (  70.99, -38.28) .. controls (  70.91, -38.91) and (  71.12, -39.79) ..
        (  71.55, -40.63) .. controls (  73.02, -43.54) and (  73.40, -46.10) ..
        (  72.66, -48.01) .. controls (  72.48, -48.46) and (  72.28, -48.87) ..
        (  72.21, -48.91) .. controls (  72.14, -48.95) and (  71.95, -49.21) ..
        (  71.80, -49.48) .. controls (  71.40, -50.16) and (  71.43, -50.64) ..
        (  71.91, -51.84) .. controls (  72.26, -52.69) and (  72.29, -52.87) ..
        (  72.29, -53.64) .. controls (  72.28, -54.35) and (  72.22, -54.69) ..
        (  71.86, -55.78) .. controls (  71.32, -57.45) and (  71.27, -58.02) ..
        (  71.54, -59.18) .. controls (  71.66, -59.67) and (  71.79, -60.12) ..
        (  71.84, -60.18) .. controls (  72.20, -60.66) and (  72.30, -62.79) ..
        (  72.06, -64.76) .. controls (  71.89, -66.11) and (  71.97, -66.63) ..
        (  72.43, -67.23) .. controls (  72.58, -67.42) and (  72.91, -67.92) ..
        (  73.17, -68.32) .. controls (  73.88, -69.45) and (  74.00, -70.08) ..
        (  74.00, -72.84) -- 
        (  74.00, -75.09) -- 
        (  73.51, -76.94) .. controls (  72.80, -79.61) and (  72.82, -79.99) ..
        (  73.75, -83.04) .. controls (  74.23, -84.62) and (  74.40, -85.73) ..
        (  74.30, -86.67) .. controls (  74.26, -87.07) and (  74.21, -87.54) ..
        (  74.20, -87.74) .. controls (  74.19, -87.93) and (  74.11, -88.22) ..
        (  74.02, -88.39) .. controls (  73.80, -88.81) and (  73.09, -89.63) ..
        (  72.72, -89.91) .. controls (  72.55, -90.04) and (  72.29, -90.26) ..
        (  72.14, -90.41) .. controls (  71.99, -90.56) and (  71.63, -90.92) ..
        (  71.32, -91.21) .. controls (  70.73, -91.77) and (  70.24, -92.64) ..
        (  69.65, -94.12) .. controls (  69.16, -95.37) and (  68.71, -95.97) ..
        (  67.34, -97.24) .. controls (  64.18,-100.17) and (  64.12,-100.26) ..
        (  64.10,-102.39) .. controls (  64.08,-104.74) and (  63.69,-105.22) ..
        (  61.65,-105.43) .. controls (  60.41,-105.56) and (  59.89,-105.77) ..
        (  59.33,-106.39) .. controls (  57.55,-108.35) and (  57.39,-108.47) ..
        (  56.34,-108.46) .. controls (  55.19,-108.45) and (  54.50,-108.02) ..
        (  53.15,-106.44) .. controls (  52.19,-105.31) and (  51.93,-105.08) ..
        (  51.40,-104.83) .. controls (  50.89,-104.59) and (  50.48,-104.59) ..
        (  49.87,-104.83) .. controls (  47.87,-105.62) and (  45.26,-107.98) ..
        (  41.75,-112.18) .. controls (  40.56,-113.61) and (  39.19,-115.21) ..
        (  38.72,-115.74) .. controls (  38.24,-116.26) and (  37.46,-117.13) ..
        (  36.98,-117.68) .. controls (  35.65,-119.19) and (  35.21,-119.59) ..
        (  33.55,-120.79) .. controls (  30.03,-123.31) and (  29.27,-123.94) ..
        (  28.49,-124.95) .. controls (  27.90,-125.72) and (  27.44,-126.74) ..
        (  27.10,-128.02) .. controls (  26.79,-129.23) and (  26.38,-130.06) ..
        (  25.62,-131.01) .. controls (  25.02,-131.76) and (  22.75,-135.07) ..
        (  22.33,-135.81) .. controls (  22.18,-136.07) and (  21.96,-136.68) ..
        (  21.84,-137.17) .. controls (  21.47,-138.68) and (  20.94,-139.30) ..
        (  18.63,-140.92) .. controls (  16.28,-142.57) and (  16.06,-142.93) ..
        (  15.45,-146.06) .. controls (  15.23,-147.24) and (  15.08,-147.62) ..
        (  14.54,-148.44) .. controls (  13.75,-149.64) and (  11.75,-151.08) ..
        (   9.05,-152.41) .. controls (   6.54,-153.64) and (   6.15,-154.18) ..
        (   6.15,-156.37) .. controls (   6.15,-157.44) and (   6.14,-157.49) ..
        (   5.79,-158.12) .. controls (   5.60,-158.47) and (   5.15,-159.10) ..
        (   4.79,-159.51) .. controls (   4.44,-159.91) and (   3.74,-160.86) ..
        (   3.23,-161.62) .. controls (   1.46,-164.29) and (   0.85,-164.84) ..
        (  -0.31,-164.84) .. controls (  -1.02,-164.84) and (  -1.81,-165.00) ..
        (  -2.35,-165.26) .. controls (  -3.42,-165.77) and (  -4.38,-167.17) ..
        (  -4.70,-168.69) .. controls (  -4.88,-169.56) and (  -4.89,-170.18) ..
        (  -4.71,-171.14) .. controls (  -4.63,-171.55) and (  -4.54,-172.27) ..
        (  -4.50,-172.73) .. controls (  -4.41,-173.80) and (  -4.19,-174.40) ..
        (  -3.54,-175.29) .. controls (  -2.97,-176.07) and (  -2.51,-176.93) ..
        (  -2.10,-178.02) .. controls (  -1.52,-179.56) and (  -0.92,-179.83) ..
        (   1.32,-179.53) .. controls (   2.29,-179.40) and (   2.80,-179.51) ..
        (   3.31,-179.96) .. controls (   4.22,-180.75) and (   4.42,-182.04) ..
        (   4.10,-185.08) .. controls (   3.92,-186.74) and (   3.95,-187.02) ..
        (   4.47,-188.66) .. controls (   4.73,-189.47) and (   4.79,-189.86) ..
        (   4.80,-190.54) .. controls (   4.80,-191.30) and (   4.77,-191.46) ..
        (   4.47,-192.05) .. controls (   4.02,-192.97) and (   3.25,-193.73) ..
        (   2.29,-194.20) -- 
        (   1.51,-194.59) -- 
        (  -0.35,-194.59) -- 
        (  -2.21,-194.59) -- 
        (  -2.63,-194.86) .. controls (  -3.28,-195.29) and (  -3.50,-195.80) ..
        (  -3.50,-196.89) .. controls (  -3.50,-197.68) and (  -3.47,-197.85) ..
        (  -3.21,-198.33) .. controls (  -3.06,-198.63) and (  -2.75,-199.10) ..
        (  -2.55,-199.38) .. controls (  -1.99,-200.11) and (  -1.61,-200.95) ..
        (  -1.30,-202.09) .. controls (  -0.81,-203.90) and (  -0.55,-204.37) ..
        (   0.74,-205.93) .. controls (   1.97,-207.41) and (   1.90,-207.05) ..
        (   1.93,-211.69) .. controls (   1.97,-215.93) and (   1.86,-218.91) ..
        (   1.59,-221.69) .. controls (   1.40,-223.56) and (   1.48,-224.89) ..
        (   1.80,-225.58) .. controls (   1.90,-225.78) and (   2.42,-226.42) ..
        (   2.96,-227.00) .. controls (   3.73,-227.82) and (   4.05,-228.26) ..
        (   4.38,-228.92) -- 
        (   4.80,-229.76) -- 
        (   4.85,-231.78) .. controls (   4.90,-233.68) and (   4.92,-233.81) ..
        (   5.17,-234.36) .. controls (   5.61,-235.30) and (   6.02,-235.81) ..
        (   7.60,-237.40) .. controls (   8.44,-238.23) and (   9.30,-239.19) ..
        (   9.53,-239.52) .. controls (  10.39,-240.82) and (  10.61,-242.04) ..
        (  10.29,-243.79) .. controls (  10.18,-244.34) and (  10.10,-245.21) ..
        (  10.10,-245.74) .. controls (  10.10,-246.60) and (  10.13,-246.75) ..
        (  10.42,-247.37) .. controls (  10.85,-248.27) and (  11.65,-249.05) ..
        (  12.89,-249.78) .. controls (  14.02,-250.44) and (  14.45,-250.81) ..
        (  15.04,-251.58) .. controls (  15.62,-252.32) and (  15.93,-252.43) ..
        (  17.26,-252.36) .. controls (  18.52,-252.30) and (  18.83,-252.41) ..
        (  19.89,-253.27) .. controls (  20.72,-253.96) and (  20.82,-254.02) ..
        (  21.43,-254.13) .. controls (  21.78,-254.20) and (  22.11,-254.19) ..
        (  22.58,-254.09) .. controls (  23.36,-253.92) and (  23.71,-253.75) ..
        (  24.45,-253.18) .. controls (  24.75,-252.95) and (  25.18,-252.70) ..
        (  25.40,-252.64) .. controls (  25.62,-252.58) and (  26.63,-252.49) ..
        (  27.65,-252.44) .. controls (  30.97,-252.26) and (  31.41,-252.03) ..
        (  33.22,-249.49) .. controls (  33.88,-248.57) and (  34.56,-247.81) ..
        (  36.14,-246.24) .. controls (  39.48,-242.92) and (  39.93,-242.26) ..
        (  40.71,-239.49) .. controls (  41.05,-238.28) and (  41.48,-237.51) ..
        (  42.20,-236.80) .. controls (  42.70,-236.30) and (  43.05,-236.08) ..
        (  44.45,-235.41) .. controls (  46.91,-234.21) and (  47.64,-233.71) ..
        (  48.08,-232.89) .. controls (  48.28,-232.53) and (  48.30,-232.37) ..
        (  48.26,-231.39) .. controls (  48.23,-230.78) and (  48.21,-229.77) ..
        (  48.21,-229.14) .. controls (  48.20,-228.06) and (  48.22,-227.96) ..
        (  48.47,-227.53) .. controls (  48.76,-227.04) and (  49.15,-226.69) ..
        (  49.80,-226.36) .. controls (  50.62,-225.94) and (  51.58,-225.75) ..
        (  55.00,-225.33) .. controls (  56.88,-225.09) and (  58.05,-224.83) ..
        (  58.80,-224.48) .. controls (  59.74,-224.04) and (  60.49,-223.09) ..
        (  62.49,-219.82) .. controls (  63.07,-218.87) and (  63.74,-218.11) ..
        (  64.30,-217.78) .. controls (  64.54,-217.64) and (  65.32,-217.33) ..
        (  66.03,-217.09) .. controls (  67.84,-216.49) and (  68.17,-216.23) ..
        (  68.95,-214.89) .. controls (  69.08,-214.67) and (  69.62,-214.04) ..
        (  70.15,-213.49) .. controls (  70.68,-212.94) and (  71.59,-211.90) ..
        (  72.16,-211.19) .. controls (  73.47,-209.56) and (  74.33,-208.71) ..
        (  75.32,-208.05) .. controls (  76.24,-207.44) and (  77.90,-206.61) ..
        (  79.30,-206.07) -- 
        (  80.30,-205.69) -- 
        (  81.90,-205.69) .. controls (  83.62,-205.69) and (  83.95,-205.75) ..
        (  86.06,-206.44) .. controls (  88.82,-207.34) and (  89.89,-208.42) ..
        (  90.80,-211.24) .. controls (  91.30,-212.79) and (  91.54,-213.13) ..
        (  93.48,-215.09) .. controls (  95.32,-216.96) and (  95.48,-217.15) ..
        (  95.93,-217.99) .. controls (  96.15,-218.41) and (  96.19,-218.61) ..
        (  96.19,-219.29) .. controls (  96.19,-220.21) and (  96.10,-220.47) ..
        (  95.09,-222.20) .. controls (  93.74,-224.52) and (  93.62,-225.93) ..
        (  94.58,-227.79) .. controls (  94.81,-228.22) and (  95.27,-228.76) ..
        (  96.27,-229.74) .. controls (  98.17,-231.62) and (  98.36,-232.02) ..
        (  99.05,-235.69) .. controls (  99.31,-237.04) and (  99.68,-237.87) ..
        ( 100.36,-238.59) .. controls ( 100.95,-239.21) and ( 101.54,-239.60) ..
        ( 103.39,-240.59) .. controls ( 104.86,-241.38) and ( 105.89,-242.26) ..
        ( 106.85,-243.54) .. controls ( 109.39,-246.96) and ( 109.30,-250.43) ..
        ( 106.55,-256.09) .. controls ( 105.65,-257.93) and ( 105.21,-259.24) ..
        ( 105.04,-260.53) .. controls ( 104.87,-261.86) and ( 104.98,-262.30) ..
        ( 106.12,-264.57) .. controls ( 107.07,-266.48) and ( 107.07,-266.48) ..
        ( 107.02,-267.10) .. controls ( 106.94,-268.12) and ( 106.63,-268.57) ..
        ( 104.60,-270.59) .. controls ( 103.60,-271.58) and ( 102.64,-272.58) ..
        ( 102.47,-272.82) .. controls ( 102.05,-273.41) and ( 101.88,-273.97) ..
        ( 101.70,-275.39) .. controls ( 101.56,-276.49) and ( 101.50,-276.68) ..
        ( 100.97,-277.77) .. controls ( 100.21,-279.35) and ( 100.21,-279.62) ..
        ( 100.90,-280.94) .. controls ( 101.35,-281.81) and ( 101.40,-281.94) ..
        ( 101.39,-282.59) .. controls ( 101.40,-283.79) and ( 100.84,-284.70) ..
        (  98.59,-287.17) .. controls (  97.18,-288.71) and (  96.26,-289.52) ..
        (  95.45,-289.92) .. controls (  94.91,-290.18) and (  94.85,-290.19) ..
        (  93.00,-290.20) .. controls (  90.76,-290.21) and (  90.63,-290.24) ..
        (  89.02,-291.32) .. controls (  88.46,-291.70) and (  87.62,-292.22) ..
        (  87.15,-292.49) .. controls (  85.31,-293.55) and (  83.02,-296.40) ..
        (  82.28,-298.58) .. controls (  81.96,-299.50) and (  81.96,-299.41) ..
        (  82.10,-304.94) .. controls (  82.18,-308.38) and (  82.11,-310.34) ..
        (  81.79,-313.09) .. controls (  81.56,-315.87) and (  80.66,-319.56) ..
        (  80.58,-319.65) -- 
        (  79.92,-319.45) .. controls (  79.93,-319.38) and (  80.04,-318.81) ..
        (  80.08,-318.71) .. controls (  80.11,-318.56) and (  80.16,-318.38) ..
        (  80.30,-317.77) .. controls (  80.64,-316.26) and (  81.01,-313.88) ..
        (  81.21,-311.80) .. controls (  81.39,-310.01) and (  81.45,-305.45) ..
        (  81.35,-301.19) -- 
        (  81.30,-299.09) -- 
        (  81.59,-298.34) .. controls (  82.55,-295.80) and (  85.12,-292.70) ..
        (  87.05,-291.73) .. controls (  87.30,-291.61) and (  87.92,-291.22) ..
        (  88.44,-290.87) .. controls (  89.43,-290.19) and (  90.29,-289.72) ..
        (  90.80,-289.58) .. controls (  90.97,-289.54) and (  91.91,-289.50) ..
        (  92.90,-289.49) .. controls (  94.67,-289.49) and (  94.71,-289.48) ..
        (  95.30,-289.20) .. controls (  95.77,-288.97) and (  96.21,-288.59) ..
        (  97.30,-287.50) .. controls (  99.74,-285.04) and ( 100.70,-283.65) ..
        ( 100.69,-282.60) .. controls ( 100.69,-282.19) and ( 100.59,-281.90) ..
        ( 100.20,-281.14) .. controls (  99.74,-280.26) and (  99.70,-280.13) ..
        (  99.70,-279.48) .. controls (  99.70,-278.82) and (  99.74,-278.68) ..
        ( 100.27,-277.58) .. controls ( 100.80,-276.46) and ( 100.85,-276.30) ..
        ( 101.03,-275.01) .. controls ( 101.32,-272.95) and ( 101.63,-272.44) ..
        ( 103.83,-270.41) .. controls ( 105.74,-268.65) and ( 106.28,-267.90) ..
        ( 106.29,-266.99) .. controls ( 106.30,-266.55) and ( 106.18,-266.23) ..
        ( 105.33,-264.49) .. controls ( 104.22,-262.21) and ( 104.14,-261.85) ..
        ( 104.34,-260.28) .. controls ( 104.50,-259.06) and ( 104.98,-257.69) ..
        ( 105.93,-255.74) .. controls ( 107.34,-252.83) and ( 107.84,-251.14) ..
        ( 107.92,-249.09) .. controls ( 107.96,-247.97) and ( 107.94,-247.67) ..
        ( 107.75,-246.99) .. controls ( 107.36,-245.50) and ( 106.48,-243.98) ..
        ( 105.37,-242.85) .. controls ( 104.67,-242.14) and ( 104.24,-241.85) ..
        ( 102.65,-240.99) .. controls (  99.68,-239.39) and (  98.81,-238.34) ..
        (  98.34,-235.84) .. controls (  97.71,-232.39) and (  97.52,-231.98) ..
        (  95.85,-230.33) .. controls (  94.53,-229.04) and (  93.92,-228.22) ..
        (  93.53,-227.25) .. controls (  93.34,-226.78) and (  93.31,-226.49) ..
        (  93.31,-225.49) .. controls (  93.31,-224.05) and (  93.44,-223.65) ..
        (  94.54,-221.73) .. controls (  95.44,-220.17) and (  95.62,-219.59) ..
        (  95.47,-218.84) .. controls (  95.29,-218.02) and (  94.78,-217.35) ..
        (  92.95,-215.54) .. controls (  91.01,-213.63) and (  90.75,-213.23) ..
        (  90.04,-211.27) .. controls (  89.26,-209.08) and (  88.71,-208.36) ..
        (  87.30,-207.69) .. controls (  86.50,-207.30) and (  84.66,-206.73) ..
        (  83.45,-206.48) .. controls (  82.18,-206.22) and (  80.76,-206.32) ..
        (  79.60,-206.73) .. controls (  77.64,-207.44) and (  75.63,-208.54) ..
        (  74.53,-209.53) .. controls (  74.27,-209.76) and (  73.51,-210.63) ..
        (  72.83,-211.47) .. controls (  72.16,-212.30) and (  71.19,-213.41) ..
        (  70.69,-213.94) .. controls (  70.18,-214.46) and (  69.61,-215.16) ..
        (  69.42,-215.49) .. controls (  68.70,-216.72) and (  67.98,-217.23) ..
        (  66.12,-217.83) .. controls (  65.58,-218.00) and (  64.97,-218.23) ..
        (  64.76,-218.33) .. controls (  64.20,-218.62) and (  63.56,-219.37) ..
        (  62.80,-220.64) .. controls (  61.24,-223.24) and (  60.43,-224.29) ..
        (  59.49,-224.89) .. controls (  58.54,-225.51) and (  57.90,-225.65) ..
        (  53.29,-226.28) .. controls (  50.49,-226.67) and (  49.53,-227.05) ..
        (  49.07,-227.94) .. controls (  48.84,-228.37) and (  48.84,-228.44) ..
        (  48.91,-229.64) .. controls (  49.06,-231.86) and (  49.02,-232.63) ..
        (  48.77,-233.12) .. controls (  48.22,-234.15) and (  47.55,-234.65) ..
        (  44.97,-235.93) .. controls (  43.93,-236.45) and (  43.01,-236.98) ..
        (  42.74,-237.22) .. controls (  42.14,-237.75) and (  41.75,-238.51) ..
        (  41.31,-240.02) .. controls (  40.59,-242.46) and (  40.07,-243.22) ..
        (  36.99,-246.34) .. controls (  35.28,-248.07) and (  34.42,-249.04) ..
        (  33.68,-250.04) .. controls (  31.73,-252.68) and (  31.19,-252.97) ..
        (  27.80,-253.14) .. controls (  26.75,-253.19) and (  25.74,-253.28) ..
        (  25.55,-253.35) .. controls (  25.36,-253.41) and (  25.00,-253.63) ..
        (  24.75,-253.84) .. controls (  24.12,-254.35) and (  23.55,-254.63) ..
        (  22.72,-254.80) .. controls (  21.47,-255.06) and (  20.65,-254.83) ..
        (  19.59,-253.93) .. controls (  18.59,-253.07) and (  18.38,-253.00) ..
        (  17.13,-253.07) .. controls (  15.63,-253.17) and (  15.26,-252.99) ..
        (  14.25,-251.69) .. controls (  13.96,-251.31) and (  13.57,-251.01) ..
        (  12.69,-250.49) .. controls (  11.19,-249.59) and (  10.31,-248.75) ..
        (   9.79,-247.68) .. controls (   9.29,-246.67) and (   9.23,-245.81) ..
        (   9.55,-243.87) .. controls (   9.75,-242.66) and (   9.77,-242.41) ..
        (   9.66,-241.76) .. controls (   9.47,-240.48) and (   9.05,-239.84) ..
        (   7.15,-237.89) .. controls (   5.21,-235.90) and (   4.96,-235.59) ..
        (   4.53,-234.67) .. controls (   4.21,-233.99) and (   4.20,-233.97) ..
        (   4.14,-231.99) .. controls (   4.06,-229.30) and (   4.00,-229.15) ..
        (   2.18,-227.15) .. controls (   1.13,-226.00) and (   0.96,-225.65) ..
        (   0.82,-224.42) .. controls (   0.73,-223.62) and (   0.74,-223.09) ..
        (   0.89,-221.52) .. controls (   1.16,-218.56) and (   1.24,-216.43) ..
        (   1.25,-212.17) .. controls (   1.25,-208.84) and (   1.23,-208.19) ..
        (   1.09,-207.76) .. controls (   0.98,-207.42) and (   0.62,-206.88) ..
        (  -0.01,-206.10) .. controls (  -1.06,-204.79) and (  -1.55,-203.92) ..
        (  -1.86,-202.77) .. controls (  -1.98,-202.34) and (  -2.11,-201.94) ..
        (  -2.15,-201.89) .. controls (  -2.19,-201.83) and (  -2.30,-201.56) ..
        (  -2.40,-201.29) .. controls (  -2.50,-201.01) and (  -2.78,-200.49) ..
        (  -3.01,-200.13) .. controls (  -3.25,-199.76) and (  -3.62,-199.11) ..
        (  -3.82,-198.68) .. controls (  -4.15,-197.99) and (  -4.20,-197.77) ..
        (  -4.23,-197.04) .. controls (  -4.27,-195.99) and (  -4.03,-195.25) ..
        (  -3.45,-194.65) .. controls (  -2.74,-193.91) and (  -2.63,-193.89) ..
        (  -0.55,-193.89) -- 
        (   1.30,-193.89) -- 
        (   1.96,-193.57) .. controls (   2.85,-193.15) and (   3.38,-192.65) ..
        (   3.77,-191.86) .. controls (   4.22,-190.95) and (   4.21,-190.20) ..
        (   3.72,-188.69) .. controls (   3.24,-187.18) and (   3.20,-186.76) ..
        (   3.40,-184.84) .. controls (   3.74,-181.61) and (   3.36,-180.24) ..
        (   2.11,-180.22) .. controls (   1.94,-180.21) and (   1.26,-180.25) ..
        (   0.59,-180.29) .. controls (  -0.74,-180.37) and (  -1.15,-180.30) ..
        (  -1.74,-179.86) .. controls (  -2.15,-179.55) and (  -2.59,-178.87) ..
        (  -2.81,-178.20) .. controls (  -3.07,-177.41) and (  -3.62,-176.36) ..
        (  -4.16,-175.62) .. controls (  -4.82,-174.73) and (  -5.04,-174.19) ..
        (  -5.16,-173.16) .. controls (  -5.22,-172.68) and (  -5.33,-171.88) ..
        (  -5.41,-171.37) .. controls (  -5.69,-169.58) and (  -5.53,-168.35) ..
        (  -4.84,-166.96) .. controls (  -4.01,-165.26) and (  -2.65,-164.30) ..
        (  -0.90,-164.18) .. controls (   0.69,-164.06) and (   0.83,-163.95) ..
        (   2.65,-161.24) .. controls (   3.19,-160.44) and (   3.91,-159.47) ..
        (   4.25,-159.09) .. controls (   5.38,-157.82) and (   5.38,-157.83) ..
        (   5.46,-156.15) .. controls (   5.53,-154.58) and (   5.58,-154.37) ..
        (   6.06,-153.67) .. controls (   6.50,-153.04) and (   7.35,-152.45) ..
        (   8.99,-151.65) .. controls (  11.74,-150.30) and (  13.68,-148.78) ..
        (  14.29,-147.51) .. controls (  14.43,-147.20) and (  14.68,-146.30) ..
        (  14.84,-145.47) .. controls (  15.16,-143.85) and (  15.33,-143.31) ..
        (  15.70,-142.68) .. controls (  16.03,-142.12) and (  17.13,-141.11) ..
        (  18.23,-140.35) .. controls (  20.32,-138.93) and (  20.79,-138.38) ..
        (  21.14,-136.98) .. controls (  21.40,-135.94) and (  21.60,-135.54) ..
        (  22.35,-134.49) .. controls (  22.66,-134.05) and (  23.27,-133.15) ..
        (  23.71,-132.49) .. controls (  24.14,-131.83) and (  24.75,-130.97) ..
        (  25.06,-130.59) .. controls (  25.85,-129.60) and (  25.93,-129.43) ..
        (  26.60,-127.19) .. controls (  27.29,-124.88) and (  28.34,-123.68) ..
        (  31.85,-121.16) .. controls (  32.81,-120.47) and (  33.98,-119.62) ..
        (  34.44,-119.26) .. controls (  35.38,-118.53) and (  37.51,-116.18) ..
        (  40.68,-112.39) .. controls (  44.19,-108.18) and (  44.97,-107.35) ..
        (  46.50,-106.14) .. controls (  48.22,-104.77) and (  49.59,-104.03) ..
        (  50.52,-103.96) .. controls (  51.61,-103.87) and (  52.15,-104.25) ..
        (  54.12,-106.51) .. controls (  54.89,-107.38) and (  55.53,-107.74) ..
        (  56.34,-107.74) .. controls (  57.04,-107.74) and (  57.58,-107.43) ..
        (  58.14,-106.70) .. controls (  58.73,-105.94) and (  59.48,-105.26) ..
        (  59.98,-105.04) .. controls (  60.21,-104.93) and (  60.87,-104.80) ..
        (  61.44,-104.74) .. controls (  63.24,-104.54) and (  63.37,-104.36) ..
        (  63.39,-102.19) .. controls (  63.40,-100.90) and (  63.43,-100.74) ..
        (  63.67,-100.22) .. controls (  63.98, -99.56) and (  64.99, -98.46) ..
        (  66.61, -97.03) .. controls (  67.23, -96.49) and (  67.90, -95.81) ..
        (  68.10, -95.53) .. controls (  68.31, -95.25) and (  68.74, -94.42) ..
        (  69.06, -93.68) .. controls (  69.37, -92.95) and (  69.78, -92.09) ..
        (  69.96, -91.77) .. controls (  70.33, -91.13) and (  71.54, -89.80) ..
        (  71.94, -89.61) .. controls (  72.32, -89.43) and (  73.23, -88.17) ..
        (  73.39, -87.59) .. controls (  73.77, -86.27) and (  73.69, -85.36) ..
        (  72.95, -82.89) .. controls (  72.39, -81.02) and (  72.20, -79.93) ..
        (  72.30, -79.11) .. controls (  72.33, -78.79) and (  72.57, -77.73) ..
        (  72.82, -76.76) .. controls (  73.25, -75.10) and (  73.28, -74.88) ..
        (  73.32, -73.39) .. controls (  73.39, -71.33) and (  73.25, -70.23) ..
        (  72.83, -69.30) .. controls (  72.52, -68.60) and (  72.09, -67.92) ..
        (  71.92, -67.85) .. controls (  71.87, -67.83) and (  71.70, -67.54) ..
        (  71.54, -67.20) .. controls (  71.19, -66.46) and (  71.17, -65.88) ..
        (  71.42, -64.02) .. controls (  71.58, -62.84) and (  71.58, -62.68) ..
        (  71.42, -61.82) .. controls (  71.33, -61.31) and (  71.12, -60.37) ..
        (  70.95, -59.74) .. controls (  70.52, -58.18) and (  70.56, -57.41) ..
        (  71.15, -55.59) .. controls (  71.72, -53.84) and (  71.73, -53.33) ..
        (  71.20, -52.04) .. controls (  70.70, -50.81) and (  70.66, -50.02) ..
        (  71.06, -49.34) .. controls (  71.68, -48.29) and (  71.89, -47.87) ..
        (  72.12, -47.20) .. controls (  72.34, -46.57) and (  72.37, -46.38) ..
        (  72.31, -45.59) .. controls (  72.24, -44.44) and (  71.89, -43.14) ..
        (  71.30, -41.82) .. controls (  70.25, -39.49) and (  70.31, -39.68) ..
        (  70.31, -38.59) .. controls (  70.32, -37.65) and (  70.35, -37.51) ..
        (  70.77, -36.34) .. controls (  71.53, -34.27) and (  71.58, -34.08) ..
        (  71.52, -33.46) .. controls (  71.41, -32.44) and (  70.72, -31.67) ..
        (  67.30, -28.78) .. controls (  63.59, -25.63) and (  62.80, -24.93) ..
        (  62.58, -24.56) .. controls (  62.38, -24.23) and (  62.35, -24.05) ..
        (  62.35, -23.09) .. controls (  62.34, -21.73) and (  62.24, -21.23) ..
        (  61.81, -20.58) .. controls (  61.21, -19.68) and (  61.05, -19.28) ..
        (  61.05, -18.71) .. controls (  61.05, -18.05) and (  61.18, -17.64) ..
        (  61.60, -16.99) .. controls (  62.30, -15.90) and (  64.69, -13.35) ..
        (  66.78, -11.46) .. controls (  68.60,  -9.81) and (  68.76,  -9.38) ..
        (  68.09,  -7.89) .. controls (  67.78,  -7.21) and (  67.72,  -6.94) ..
        (  67.68,  -6.19) .. controls (  67.63,  -5.39) and (  67.66,  -5.19) ..
        (  67.90,  -4.49) .. controls (  68.05,  -4.05) and (  68.33,  -3.39) ..
        (  68.51,  -3.04) .. controls (  68.95,  -2.18) and (  69.12,  -1.41) ..
        (  68.96,  -0.97) .. controls (  68.89,  -0.79) and (  68.86,  -0.59) ..
        (  68.90,  -0.54) .. controls (  68.93,  -0.49) and (  68.84,  -0.18) ..
        (  68.69,   0.14) .. controls (  68.31,   0.93) and (  67.43,   1.79) ..
        (  66.58,   2.19) .. controls (  65.96,   2.49) and (  65.82,   2.51) ..
        (  64.90,   2.51) .. controls (  64.02,   2.51) and (  63.83,   2.48) ..
        (  63.35,   2.25) .. controls (  61.67,   1.44) and (  60.75,   1.22) ..
        (  59.40,   1.31) .. controls (  58.10,   1.40) and (  57.61,   1.27) ..
        (  56.80,   0.66) .. controls (  55.88,  -0.05) and (  55.54,  -0.19) ..
        (  53.80,  -0.54) .. controls (  51.96,  -0.92) and (  50.97,  -0.96) ..
        (  50.45,  -0.70) .. controls (  50.26,  -0.60) and (  49.61,  -0.01) ..
        (  49.00,   0.61) .. controls (  47.81,   1.83) and (  47.88,   1.79) ..
        (  46.00,   2.35) .. controls (  44.31,   2.86) and (  43.66,   3.46) ..
        (  42.98,   5.13) .. controls (  42.23,   6.99) and (  42.02,   7.25) ..
        (  40.25,   8.41) .. controls (  39.09,   9.18) and (  38.28,   9.92) ..
        (  38.02,  10.48) .. controls (  37.90,  10.73) and (  37.85,  11.10) ..
        (  37.85,  11.65) .. controls (  37.85,  12.09) and (  37.80,  12.57) ..
        (  37.73,  12.71) .. controls (  37.66,  12.86) and (  37.24,  13.22) ..
        (  36.76,  13.54) .. controls (  35.77,  14.19) and (  35.05,  14.87) ..
        (  34.77,  15.42) .. controls (  34.67,  15.64) and (  34.50,  16.15) ..
        (  34.41,  16.56) .. controls (  34.19,  17.55) and (  33.94,  17.95) ..
        (  32.85,  19.00) .. controls (  32.35,  19.49) and (  31.83,  20.05) ..
        (  31.70,  20.25) .. controls (  31.45,  20.64) and (  31.04,  22.00) ..
        (  30.85,  23.11) .. controls (  30.78,  23.50) and (  30.59,  24.16) ..
        (  30.43,  24.58) .. controls (  30.04,  25.62) and (  29.55,  26.06) ..
        (  27.87,  26.90) .. controls (  27.15,  27.26) and (  26.41,  27.70) ..
        (  26.22,  27.86) .. controls (  26.03,  28.03) and (  25.60,  28.57) ..
        (  25.26,  29.06) .. controls (  24.51,  30.15) and (  24.04,  30.57) ..
        (  22.67,  31.38) .. controls (  21.29,  32.20) and (  20.53,  32.92) ..
        (  19.56,  34.35) .. controls (  18.30,  36.20) and (  17.38,  37.17) ..
        (  15.42,  38.70) .. controls (  14.78,  39.19) and (  14.03,  39.66) ..
        (  13.31,  40.00) .. controls (  11.59,  40.83) and (  10.46,  41.63) ..
        (   9.15,  42.95) .. controls (   8.27,  43.84) and (   7.89,  44.31) ..
        (   7.52,  44.96) .. controls (   6.34,  47.08) and (   6.28,  48.49) ..
        (   7.21,  52.31) .. controls (   7.51,  53.52) and (   7.70,  54.59) ..
        (   7.80,  55.56) .. controls (   8.25,  60.23) and (   8.07,  61.93) ..
        (   6.90,  64.26) .. controls (   6.65,  64.76) and (   6.33,  65.51) ..
        (   6.18,  65.94) .. controls (   5.93,  66.67) and (   5.91,  66.81) ..
        (   5.92,  68.46) .. controls (   5.92,  70.50) and (   6.15,  71.98) ..
        (   6.69,  73.61) .. controls (   7.09,  74.80) and (   8.76,  78.36) ..
        (   9.46,  79.53) .. controls (  10.61,  81.43) and (  10.79,  82.17) ..
        (  10.69,  84.49) .. controls (  10.62,  86.17) and (  10.73,  86.97) ..
        (  11.14,  87.81) .. controls (  11.51,  88.55) and (  12.30,  89.49) ..
        (  13.12,  90.16) .. controls (  13.47,  90.44) and (  13.82,  90.81) ..
        (  13.91,  90.98) .. controls (  14.15,  91.44) and (  14.03,  92.12) ..
        (  13.66,  92.51) .. controls (  13.27,  92.91) and (  13.15,  93.26) ..
        (  13.24,  93.68) .. controls (  13.37,  94.23) and (  13.50,  94.40) ..
        (  14.47,  95.24) .. controls (  15.85,  96.43) and (  16.25,  96.88) ..
        (  16.35,  97.32) .. controls (  16.40,  97.54) and (  16.42,  97.80) ..
        (  16.39,  97.91) .. controls (  16.21,  98.52) and (  15.40,  98.86) ..
        (  14.13,  98.86) .. controls (  13.07,  98.86) and (  12.68,  98.69) ..
        (  12.11,  97.96) .. controls (  11.59,  97.29) and (  11.31,  97.08) ..
        (  10.84,  97.00) .. controls (  10.35,  96.92) and (   9.99,  97.25) ..
        (   9.59,  98.12) .. controls (   9.31,  98.73) and (   9.16,  98.93) ..
        (   8.58,  99.39) .. controls (   7.07, 100.60) and (   6.81, 101.09) ..
        (   6.87, 102.71) -- 
        (   6.90, 103.78) -- 
        (   7.70, 103.91) -- 
        (   7.64, 103.61) .. controls (   7.60, 103.45) and (   7.57, 102.95) ..
        (   7.56, 102.51) .. controls (   7.55, 101.34) and (   7.79, 100.88) ..
        (   8.83, 100.07) .. controls (   9.54,  99.52) and (  10.25,  98.69) ..
        (  10.25,  98.40) .. controls (  10.25,  98.35) and (  10.34,  98.17) ..
        (  10.45,  97.99) .. controls (  10.71,  97.56) and (  10.83,  97.62) ..
        (  11.75,  98.57) .. controls (  12.58,  99.43) and (  12.85,  99.56) ..
        (  14.02,  99.61) .. controls (  14.60,  99.63) and (  14.68,  99.67) ..
        (  15.08, 100.03) .. controls (  15.32, 100.25) and (  15.93, 100.71) ..
        (  16.44, 101.06) .. controls (  18.00, 102.11) and (  18.49, 102.77) ..
        (  18.25, 103.51) .. controls (  18.18, 103.72) and (  17.70, 104.30) ..
        (  17.00, 105.02) .. controls (  15.68, 106.37) and (  15.70, 106.37) ..
        (  14.58, 105.88) .. controls (  14.09, 105.66) and (  13.71, 105.56) ..
        (  13.37, 105.56) .. controls (  12.92, 105.56) and (  12.82, 105.61) ..
        (  12.38, 106.00) -- 
        (  11.79, 106.55) -- 
        cycle ;
}

% .. Mediterranée ..
\newcommand{\mermed}{%
  \fill [mers] 
        (  42.02, 171.30) .. controls (  41.25, 171.19) and (  40.65, 170.74) ..
        (  39.99, 170.05) .. controls (  37.76, 167.70) and (  37.82, 167.76) ..
        (  37.00, 167.36) .. controls (  36.01, 166.87) and (  35.08, 166.72) ..
        (  34.03, 166.87) .. controls (  33.27, 166.97) and (  32.50, 167.19) ..
        (  32.03, 167.43) .. controls (  31.87, 167.51) and (  31.44, 167.74) ..
        (  31.29, 167.86) -- 
        (  29.85, 169.15) .. controls (  28.85, 170.04) and (  28.07, 170.62) ..
        (  26.62, 171.44) .. controls (  24.91, 172.42) and (  24.17, 172.76) ..
        (  22.92, 173.08) .. controls (  21.68, 173.39) and (  20.38, 173.46) ..
        (  19.26, 173.31) .. controls (  18.14, 173.16) and (  17.41, 173.01) ..
        (  16.17, 172.33) .. controls (  14.86, 171.60) and (  14.40, 171.48) ..
        (  12.61, 171.41) .. controls (  10.58, 171.33) and (   9.68, 171.07) ..
        (   8.35, 170.13) .. controls (   7.39, 169.46) and (   7.08, 169.41) ..
        (   6.15, 169.37) .. controls (   5.21, 169.37) and (   3.75, 169.79) ..
        (   3.05, 170.00) .. controls (   2.44, 170.17) and (   1.37, 170.43) ..
        (   0.71, 169.85) .. controls (   0.22, 169.46) and (  -0.19, 168.85) ..
        (  -0.35, 167.52) .. controls (  -0.43, 166.85) and (  -0.67, 166.07) ..
        (  -0.81, 165.77) .. controls (  -1.18, 165.00) and (  -2.35, 163.87) ..
        (  -3.21, 163.45) .. controls (  -4.46, 162.84) and (  -5.42, 162.95) ..
        (  -6.60, 163.86) .. controls (  -7.74, 164.74) and (  -8.25, 164.68) ..
        (  -9.47, 163.51) .. controls ( -10.26, 162.75) and ( -10.87, 162.27) ..
        ( -11.45, 161.97) .. controls ( -11.70, 161.84) and ( -11.97, 161.68) ..
        ( -12.05, 161.62) .. controls ( -12.21, 161.49) and ( -13.64, 160.66) ..
        ( -13.70, 160.66) .. controls ( -13.78, 160.66) and ( -16.10, 159.23) ..
        ( -16.92, 158.68) .. controls ( -17.41, 158.35) and ( -18.52, 157.51) ..
        ( -19.40, 156.82) .. controls ( -22.23, 154.59) and ( -24.08, 153.42) ..
        ( -26.73, 152.22) .. controls ( -27.48, 151.88) and ( -28.64, 151.30) ..
        ( -29.30, 150.93) .. controls ( -29.96, 150.56) and ( -30.86, 150.14) ..
        ( -31.30, 150.00) .. controls ( -31.97, 149.79) and ( -35.86, 149.08) ..
        ( -37.20, 148.93) .. controls ( -38.27, 148.81) and ( -39.06, 149.29) ..
        ( -39.75, 150.46) .. controls ( -40.25, 151.31) and ( -40.57, 151.59) ..
        ( -41.40, 151.86) .. controls ( -42.14, 152.11) and ( -42.31, 152.29) ..
        ( -42.41, 152.96) .. controls ( -42.55, 153.87) and ( -42.73, 154.02) ..
        ( -43.96, 154.21) .. controls ( -44.16, 154.24) and ( -44.50, 154.41) ..
        ( -44.71, 154.58) .. controls ( -45.35, 155.09) and ( -45.63, 155.16) ..
        ( -46.95, 155.16) .. controls ( -48.43, 155.16) and ( -48.60, 155.24) ..
        ( -49.03, 156.04) .. controls ( -49.21, 156.37) and ( -49.41, 156.69) ..
        ( -49.50, 156.76) .. controls ( -49.58, 156.83) and ( -50.09, 156.95) ..
        ( -50.62, 157.04) .. controls ( -51.16, 157.12) and ( -52.12, 157.36) ..
        ( -52.74, 157.57) .. controls ( -54.21, 158.06) and ( -55.73, 158.26) ..
        ( -57.93, 158.26) .. controls ( -58.85, 158.26) and ( -59.86, 158.31) ..
        ( -60.18, 158.36) .. controls ( -60.83, 158.47) and ( -61.60, 158.81) ..
        ( -62.20, 159.27) .. controls ( -62.67, 159.62) and ( -63.52, 159.80) ..
        ( -65.35, 159.95) .. controls ( -66.37, 160.04) and ( -66.61, 160.10) ..
        ( -67.49, 160.48) .. controls ( -68.16, 160.76) and ( -68.63, 161.04) ..
        ( -68.92, 161.32) -- 
        ( -69.35, 161.73) -- 
        ( -69.35, 162.75) .. controls ( -69.35, 163.95) and ( -69.44, 164.21) ..
        ( -69.83, 164.12) .. controls ( -69.98, 164.08) and ( -70.26, 163.95) ..
        ( -70.45, 163.82) .. controls ( -72.05, 162.78) and ( -73.61, 162.75) ..
        ( -77.50, 163.70) .. controls ( -78.71, 164.00) and ( -80.04, 164.32) ..
        ( -80.45, 164.42) .. controls ( -81.97, 164.76) and ( -82.85, 165.40) ..
        ( -82.85, 166.14) .. controls ( -82.85, 166.43) and ( -82.76, 166.63) ..
        ( -82.45, 167.00) .. controls ( -81.88, 167.69) and ( -81.93, 167.96) ..
        ( -82.64, 167.96) .. controls ( -82.86, 167.96) and ( -82.98, 168.06) ..
        ( -83.19, 168.37) .. controls ( -83.34, 168.59) and ( -83.58, 168.82) ..
        ( -83.73, 168.88) .. controls ( -84.20, 169.06) and ( -85.16, 168.98) ..
        ( -86.49, 168.66) .. controls ( -87.68, 168.37) and ( -87.88, 168.35) ..
        ( -89.19, 168.38) .. controls ( -90.51, 168.41) and ( -90.63, 168.43) ..
        ( -91.06, 168.68) .. controls ( -91.33, 168.84) and ( -91.68, 169.18) ..
        ( -91.93, 169.55) .. controls ( -92.71, 170.67) and ( -93.02, 170.77) ..
        ( -95.55, 170.86) .. controls ( -97.82, 170.94) and ( -98.04, 171.01) ..
        ( -98.77, 171.88) .. controls ( -99.47, 172.73) and ( -99.72, 172.93) ..
        (-100.21, 173.07) .. controls (-100.60, 173.18) and (-100.80, 173.18) ..
        (-101.34, 173.07) .. controls (-101.70, 173.00) and (-102.41, 172.92) ..
        (-102.92, 172.89) .. controls (-103.97, 172.84) and (-104.28, 172.91) ..
        (-106.30, 173.66) .. controls (-107.28, 174.03) and (-108.13, 174.25) ..
        (-109.75, 174.56) .. controls (-114.57, 175.49) and (-116.65, 176.21) ..
        (-118.65, 177.66) .. controls (-120.56, 179.04) and (-121.62, 179.32) ..
        (-123.08, 178.83) .. controls (-123.50, 178.68) and (-124.30, 178.34) ..
        (-124.85, 178.06) .. controls (-126.16, 177.41) and (-126.68, 177.28) ..
        (-128.55, 177.16) .. controls (-130.64, 177.03) and (-131.10, 176.95) ..
        (-133.23, 176.37) .. controls (-134.26, 176.09) and (-135.51, 175.82) ..
        (-136.08, 175.76) .. controls (-139.11, 175.43) and (-141.79, 176.45) ..
        (-143.11, 178.43) .. controls (-143.58, 179.14) and (-143.99, 180.32) ..
        (-144.16, 181.42) .. controls (-144.26, 182.11) and (-144.26, 182.80) ..
        (-144.15, 185.09) .. controls (-144.04, 187.43) and (-144.03, 189.71) ..
        (-144.14, 198.41) .. controls (-144.20, 204.19) and (-144.29, 209.87) ..
        (-144.32, 211.04) -- 
        (-144.38, 213.16) -- 
        (  -8.96, 213.16) .. controls (  80.23, 213.16) and ( 126.45, 213.13) ..
        ( 126.45, 213.06) .. controls ( 126.45, 213.01) and ( 126.36, 212.64) ..
        ( 126.25, 212.24) .. controls ( 125.83, 210.66) and ( 125.22, 206.96) ..
        ( 123.90, 197.76) .. controls ( 122.27, 186.44) and ( 122.19, 185.93) ..
        ( 121.70, 184.96) .. controls ( 121.14, 183.86) and ( 120.71, 182.76) ..
        ( 119.80, 180.06) .. controls ( 118.85, 177.30) and ( 118.36, 175.99) ..
        ( 117.97, 175.21) .. controls ( 117.29, 173.85) and ( 116.36, 172.86) ..
        ( 115.14, 172.19) .. controls ( 114.29, 171.72) and ( 114.00, 171.41) ..
        ( 113.00, 169.86) .. controls ( 110.56, 166.10) and ( 110.23, 165.70) ..
        ( 108.30, 164.27) .. controls ( 105.84, 162.44) and ( 105.51, 162.25) ..
        ( 103.15, 161.32) .. controls ( 102.35, 161.00) and ( 101.13, 160.45) ..
        ( 100.43, 160.10) .. controls (  97.89, 158.83) and (  97.01, 158.69) ..
        (  94.55, 159.12) .. controls (  92.57, 159.46) and (  91.81, 159.43) ..
        (  90.40, 158.92) .. controls (  88.86, 158.38) and (  88.32, 158.40) ..
        (  86.07, 159.16) .. controls (  85.10, 159.48) and (  83.60, 159.91) ..
        (  82.75, 160.10) .. controls (  78.99, 160.96) and (  77.98, 160.83) ..
        (  74.32, 158.96) .. controls (  71.89, 157.72) and (  71.18, 157.44) ..
        (  69.80, 157.16) .. controls (  68.48, 156.90) and (  68.21, 156.79) ..
        (  67.41, 156.23) .. controls (  66.31, 155.44) and (  66.00, 155.32) ..
        (  65.10, 155.33) .. controls (  63.31, 155.34) and (  60.91, 156.68) ..
        (  58.91, 158.78) .. controls (  56.61, 161.20) and (  56.83, 161.01) ..
        (  55.13, 161.87) .. controls (  54.27, 162.31) and (  53.08, 162.99) ..
        (  52.48, 163.39) .. controls (  50.47, 164.74) and (  48.04, 166.77) ..
        (  46.15, 168.67) .. controls (  44.75, 170.07) and (  44.43, 170.56) ..
        (  43.48, 171.04) .. controls (  43.48, 171.04) and (  42.67, 171.39) ..
        (  42.03, 171.30) -- 
        cycle ;
}

% .. Mer Rouge ..
\newcommand{\merrouge}{%
  \fill [mers]
        ( 230.05,-155.78) .. controls ( 226.13,-154.58) and ( 210.31,-149.14) ..
        ( 203.90,-146.79) .. controls ( 201.66,-145.97) and ( 200.85,-145.56) ..
        ( 199.96,-144.89) .. controls ( 198.69,-143.92) and ( 198.16,-143.11) ..
        ( 197.78,-141.62) .. controls ( 197.44,-140.25) and ( 197.27,-139.95) ..
        ( 196.27,-138.91) -- 
        ( 195.42,-138.02) -- 
        ( 195.48,-137.46) .. controls ( 195.51,-137.14) and ( 195.56,-136.81) ..
        ( 195.60,-136.72) .. controls ( 195.64,-136.61) and ( 195.52,-136.39) ..
        ( 195.21,-136.04) .. controls ( 194.64,-135.40) and ( 194.63,-135.17) ..
        ( 195.15,-134.42) .. controls ( 195.75,-133.55) and ( 195.74,-133.27) ..
        ( 194.86,-132.45) .. controls ( 194.18,-131.80) and ( 193.95,-131.37) ..
        ( 193.95,-130.78) .. controls ( 193.95,-130.33) and ( 194.26,-129.59) ..
        ( 194.77,-128.81) .. controls ( 195.45,-127.76) and ( 195.46,-127.47) ..
        ( 194.87,-126.49) .. controls ( 194.64,-126.11) and ( 194.45,-125.70) ..
        ( 194.45,-125.58) .. controls ( 194.45,-125.47) and ( 194.57,-125.15) ..
        ( 194.71,-124.88) .. controls ( 195.73,-122.98) and ( 195.78,-122.84) ..
        ( 195.52,-122.34) .. controls ( 195.33,-121.97) and ( 194.82,-121.67) ..
        ( 193.86,-121.38) .. controls ( 193.25,-121.19) and ( 192.81,-120.98) ..
        ( 192.34,-120.64) .. controls ( 191.43,-119.98) and ( 191.37,-119.95) ..
        ( 190.34,-119.73) .. controls ( 189.04,-119.45) and ( 188.86,-119.35) ..
        ( 187.88,-118.41) .. controls ( 187.40,-117.95) and ( 186.64,-117.31) ..
        ( 186.20,-116.99) .. controls ( 185.76,-116.66) and ( 185.00,-116.01) ..
        ( 184.50,-115.54) .. controls ( 184.01,-115.06) and ( 183.24,-114.41) ..
        ( 182.80,-114.09) .. controls ( 182.37,-113.77) and ( 181.63,-113.15) ..
        ( 181.18,-112.73) .. controls ( 180.57,-112.14) and ( 180.26,-111.92) ..
        ( 179.93,-111.84) .. controls ( 178.86,-111.57) and ( 178.68,-111.45) ..
        ( 178.26,-110.80) .. controls ( 177.99,-110.36) and ( 177.65,-110.01) ..
        ( 177.17,-109.65) .. controls ( 176.39,-109.07) and ( 176.21,-108.77) ..
        ( 176.11,-107.89) .. controls ( 176.07,-107.57) and ( 175.96,-107.10) ..
        ( 175.85,-106.86) .. controls ( 175.58,-106.26) and ( 174.57,-105.31) ..
        ( 173.80,-104.93) -- 
        ( 173.16,-104.62) -- 
        ( 171.58,-104.68) .. controls ( 169.60,-104.77) and ( 169.46,-104.73) ..
        ( 166.30,-103.22) .. controls ( 165.04,-102.61) and ( 163.71,-101.98) ..
        ( 163.36,-101.81) .. controls ( 162.08,-101.20) and ( 161.55,-100.48) ..
        ( 160.95, -98.53) .. controls ( 160.44, -96.89) and ( 160.04, -96.24) ..
        ( 159.53, -96.24) .. controls ( 159.44, -96.24) and ( 159.17, -96.49) ..
        ( 158.93, -96.81) .. controls ( 158.42, -97.48) and ( 158.08, -97.60) ..
        ( 157.69, -97.23) .. controls ( 157.33, -96.89) and ( 157.24, -96.39) ..
        ( 157.27, -94.97) .. controls ( 157.30, -93.74) and ( 157.29, -93.67) ..
        ( 157.03, -93.24) .. controls ( 156.89, -92.99) and ( 156.46, -92.43) ..
        ( 156.08, -92.00) .. controls ( 154.05, -89.66) and ( 153.58, -88.64) ..
        ( 153.35, -86.04) .. controls ( 153.29, -85.32) and ( 153.17, -84.63) ..
        ( 153.05, -84.30) .. controls ( 152.85, -83.77) and ( 152.74, -83.60) ..
        ( 151.81, -82.41) .. controls ( 151.21, -81.64) and ( 151.11, -81.24) ..
        ( 151.10, -79.69) .. controls ( 151.10, -78.50) and ( 151.08, -78.35) ..
        ( 150.85, -77.94) .. controls ( 150.45, -77.18) and ( 150.51, -76.61) ..
        ( 151.14, -75.19) .. controls ( 151.94, -73.41) and ( 152.09, -72.96) ..
        ( 152.09, -72.29) .. controls ( 152.09, -71.53) and ( 151.88, -71.12) ..
        ( 151.27, -70.71) .. controls ( 150.38, -70.09) and ( 150.11, -69.79) ..
        ( 150.12, -69.42) .. controls ( 150.13, -69.09) and ( 150.49, -68.04) ..
        ( 150.83, -67.34) .. controls ( 151.15, -66.68) and ( 152.53, -66.26) ..
        ( 153.52, -66.53) .. controls ( 153.75, -66.59) and ( 154.17, -66.84) ..
        ( 154.45, -67.08) .. controls ( 154.73, -67.32) and ( 155.07, -67.54) ..
        ( 155.20, -67.56) .. controls ( 155.52, -67.62) and ( 156.00, -67.44) ..
        ( 156.65, -66.99) .. controls ( 157.12, -66.67) and ( 157.26, -66.62) ..
        ( 157.61, -66.66) .. controls ( 157.83, -66.69) and ( 158.20, -66.83) ..
        ( 158.43, -66.99) .. controls ( 158.88, -67.28) and ( 159.36, -67.34) ..
        ( 159.61, -67.13) .. controls ( 159.68, -67.07) and ( 159.80, -66.84) ..
        ( 159.86, -66.64) .. controls ( 160.07, -65.93) and ( 159.81, -65.50) ..
        ( 158.46, -64.40) .. controls ( 157.47, -63.58) and ( 157.02, -63.42) ..
        ( 155.65, -63.34) .. controls ( 153.81, -63.23) and ( 153.45, -62.91) ..
        ( 153.45, -61.42) .. controls ( 153.45, -59.92) and ( 153.22, -59.72) ..
        ( 152.04, -60.18) .. controls ( 150.72, -60.70) and ( 150.36, -60.43) ..
        ( 149.43, -58.14) .. controls ( 149.23, -57.67) and ( 148.94, -56.86) ..
        ( 148.76, -56.34) .. controls ( 148.33, -55.04) and ( 147.98, -54.46) ..
        ( 147.19, -53.77) .. controls ( 146.82, -53.45) and ( 146.18, -52.87) ..
        ( 145.75, -52.49) .. controls ( 145.14, -51.92) and ( 144.84, -51.73) ..
        ( 144.21, -51.50) .. controls ( 143.31, -51.16) and ( 142.81, -50.82) ..
        ( 142.04, -50.04) .. controls ( 141.73, -49.73) and ( 141.10, -49.22) ..
        ( 140.64, -48.92) .. controls ( 139.49, -48.15) and ( 139.06, -47.52) ..
        ( 139.19, -46.76) .. controls ( 139.39, -45.64) and ( 139.39, -45.68) ..
        ( 138.98, -45.03) .. controls ( 138.57, -44.35) and ( 138.56, -44.33) ..
        ( 138.66, -42.54) .. controls ( 138.69, -41.78) and ( 138.67, -41.64) ..
        ( 138.46, -41.27) .. controls ( 138.33, -41.05) and ( 138.00, -40.69) ..
        ( 137.72, -40.47) .. controls ( 136.84, -39.81) and ( 135.97, -38.94) ..
        ( 135.29, -38.06) .. controls ( 134.21, -36.66) and ( 134.21, -36.67) ..
        ( 134.30, -34.97) -- 
        ( 134.37, -33.52) -- 
        ( 134.06, -32.91) .. controls ( 133.88, -32.55) and ( 133.40, -31.90) ..
        ( 132.91, -31.37) .. controls ( 132.42, -30.83) and ( 131.84, -30.06) ..
        ( 131.50, -29.47) .. controls ( 131.18, -28.93) and ( 130.73, -28.24) ..
        ( 130.50, -27.94) .. controls ( 129.71, -26.90) and ( 129.33, -26.21) ..
        ( 128.90, -25.03) .. controls ( 128.35, -23.53) and ( 128.09, -23.05) ..
        ( 127.34, -22.18) .. controls ( 126.29, -20.98) and ( 126.21, -20.78) ..
        ( 126.15, -19.33) .. controls ( 126.07, -17.66) and ( 125.90, -17.30) ..
        ( 124.46, -15.79) .. controls ( 122.15, -13.39) and ( 121.95, -13.10) ..
        ( 121.60, -11.74) .. controls ( 121.27, -10.45) and ( 121.14, -10.26) ..
        ( 120.20,  -9.78) .. controls ( 119.13,  -9.24) and ( 118.78,  -8.77) ..
        ( 118.29,  -7.24) .. controls ( 117.80,  -5.69) and ( 117.54,  -5.27) ..
        ( 116.24,  -3.88) .. controls ( 114.84,  -2.38) and ( 114.56,  -1.95) ..
        ( 113.94,  -0.44) .. controls ( 113.02,   1.82) and ( 112.15,   3.39) ..
        ( 111.21,   4.43) .. controls ( 110.56,   5.16) and ( 110.45,   5.43) ..
        ( 110.45,   6.38) .. controls ( 110.45,   8.08) and ( 110.10,   8.78) ..
        ( 108.25,  10.81) .. controls ( 107.27,  11.90) and ( 106.90,  12.49) ..
        ( 105.80,  14.71) .. controls ( 104.98,  16.39) and ( 104.75,  16.75) ..
        ( 103.94,  17.71) .. controls ( 103.44,  18.32) and ( 102.91,  19.06) ..
        ( 102.76,  19.36) .. controls ( 102.52,  19.88) and ( 102.50,  20.01) ..
        ( 102.50,  21.26) .. controls ( 102.51,  22.95) and ( 102.63,  23.26) ..
        ( 103.58,  24.10) .. controls ( 104.29,  24.73) and ( 104.55,  25.20) ..
        ( 104.55,  25.89) .. controls ( 104.55,  26.42) and ( 104.36,  26.70) ..
        ( 103.83,  26.97) .. controls ( 103.61,  27.09) and ( 103.21,  27.37) ..
        ( 102.96,  27.61) .. controls ( 102.71,  27.85) and ( 102.32,  28.13) ..
        ( 102.10,  28.24) .. controls ( 101.64,  28.47) and ( 101.14,  28.97) ..
        ( 100.92,  29.41) .. controls ( 100.84,  29.58) and ( 100.67,  30.07) ..
        ( 100.55,  30.51) .. controls ( 100.17,  31.86) and (  99.81,  32.35) ..
        (  98.65,  33.08) .. controls (  98.01,  33.49) and (  97.45,  34.08) ..
        (  97.14,  34.68) .. controls (  97.02,  34.92) and (  96.75,  35.34) ..
        (  96.54,  35.62) .. controls (  96.08,  36.22) and (  96.06,  36.41) ..
        (  96.40,  36.86) .. controls (  96.85,  37.48) and (  96.69,  37.84) ..
        (  95.75,  38.31) .. controls (  95.03,  38.68) and (  94.95,  38.91) ..
        (  95.24,  39.73) .. controls (  95.37,  40.08) and (  95.47,  40.53) ..
        (  95.47,  40.72) .. controls (  95.47,  41.19) and (  95.06,  41.99) ..
        (  94.46,  42.69) .. controls (  94.19,  43.01) and (  93.71,  43.63) ..
        (  93.40,  44.07) .. controls (  93.06,  44.55) and (  92.57,  45.08) ..
        (  92.17,  45.40) .. controls (  91.80,  45.69) and (  91.44,  46.01) ..
        (  91.36,  46.10) .. controls (  91.20,  46.29) and (  91.17,  46.67) ..
        (  91.28,  47.13) .. controls (  91.35,  47.41) and (  91.32,  47.48) ..
        (  90.91,  47.87) .. controls (  90.53,  48.22) and (  90.45,  48.36) ..
        (  90.45,  48.65) .. controls (  90.45,  48.97) and (  90.54,  49.09) ..
        (  91.22,  49.73) .. controls (  91.94,  50.40) and (  92.00,  50.49) ..
        (  92.00,  50.85) .. controls (  92.00,  51.34) and (  91.82,  51.55) ..
        (  90.94,  52.12) .. controls (  90.05,  52.70) and (  89.55,  53.27) ..
        (  89.55,  53.69) .. controls (  89.55,  53.88) and (  89.60,  54.08) ..
        (  89.67,  54.14) .. controls (  89.83,  54.31) and (  90.41,  54.29) ..
        (  91.03,  54.11) .. controls (  92.53,  53.67) and (  92.54,  54.84) ..
        (  91.05,  56.71) .. controls (  90.83,  56.99) and (  90.56,  57.48) ..
        (  90.45,  57.81) .. controls (  90.19,  58.60) and (  90.06,  58.78) ..
        (  88.80,  60.18) .. controls (  87.61,  61.48) and (  87.56,  61.51) ..
        (  86.26,  61.61) .. controls (  85.25,  61.68) and (  84.99,  61.81) ..
        (  84.43,  62.48) .. controls (  84.19,  62.77) and (  83.68,  63.34) ..
        (  83.29,  63.76) .. controls (  82.89,  64.18) and (  82.47,  64.73) ..
        (  82.35,  64.97) .. controls (  81.78,  66.10) and (  80.60,  67.43) ..
        (  79.84,  67.82) .. controls (  79.08,  68.21) and (  78.69,  68.64) ..
        (  78.50,  69.32) .. controls (  78.07,  70.91) and (  77.93,  71.09) ..
        (  76.57,  71.99) .. controls (  75.50,  72.70) and (  75.03,  73.20) ..
        (  74.08,  74.61) .. controls (  73.02,  76.20) and (  72.73,  76.50) ..
        (  71.53,  77.30) .. controls (  70.41,  78.04) and (  69.98,  78.41) ..
        (  69.76,  78.83) .. controls (  69.70,  78.96) and (  69.65,  79.61) ..
        (  69.65,  80.44) .. controls (  69.65,  82.56) and (  69.37,  83.22) ..
        (  68.16,  83.90) .. controls (  67.45,  84.30) and (  66.75,  84.99) ..
        (  66.48,  85.56) .. controls (  66.37,  85.81) and (  65.94,  86.71) ..
        (  65.54,  87.56) .. controls (  64.59,  89.56) and (  64.55,  89.92) ..
        (  65.09,  92.07) .. controls (  65.29,  92.85) and (  65.45,  93.59) ..
        (  65.45,  93.71) .. controls (  65.45,  93.82) and (  65.33,  94.23) ..
        (  65.20,  94.61) .. controls (  64.88,  95.52) and (  64.88,  96.07) ..
        (  65.21,  96.76) .. controls (  65.59,  97.56) and (  65.58,  98.04) ..
        (  65.18,  98.58) .. controls (  65.00,  98.82) and (  64.51,  99.55) ..
        (  64.08, 100.21) .. controls (  62.99, 101.93) and (  62.64, 102.35) ..
        (  61.09, 103.86) .. controls (  59.89, 105.03) and (  58.68, 106.35) ..
        (  56.87, 108.49) .. controls (  56.21, 109.26) and (  56.01, 109.72) ..
        (  56.01, 110.41) .. controls (  56.00, 111.43) and (  56.58, 112.31) ..
        (  58.25, 113.80) .. controls (  60.02, 115.40) and (  59.98, 116.06) ..
        (  57.98, 117.92) .. controls (  57.40, 118.46) and (  56.86, 119.05) ..
        (  56.79, 119.23) .. controls (  56.33, 120.31) and (  57.17, 121.59) ..
        (  58.52, 121.87) .. controls (  58.88, 121.94) and (  59.58, 121.97) ..
        (  60.35, 121.94) .. controls (  61.81, 121.89) and (  62.43, 121.68) ..
        (  62.99, 121.06) .. controls (  63.54, 120.45) and (  63.62, 119.89) ..
        (  63.35, 118.37) .. controls (  63.04, 116.56) and (  63.15, 116.15) ..
        (  64.33, 114.57) .. controls (  65.35, 113.21) and (  65.39, 113.01) ..
        (  65.31, 110.47) .. controls (  65.22, 107.53) and (  65.39, 107.00) ..
        (  66.83, 105.84) .. controls (  67.48, 105.31) and (  67.60, 105.15) ..
        (  67.99, 104.33) .. controls (  68.72, 102.81) and (  70.11,  99.66) ..
        (  70.41,  98.86) .. controls (  70.76,  97.92) and (  71.40,  97.26) ..
        (  72.75,  96.42) .. controls (  74.10,  95.60) and (  74.43,  95.30) ..
        (  75.61,  93.88) .. controls (  76.33,  93.02) and (  76.81,  92.32) ..
        (  77.16,  91.64) .. controls (  77.81,  90.38) and (  78.12,  90.02) ..
        (  78.78,  89.81) .. controls (  80.05,  89.39) and (  80.21,  89.16) ..
        (  80.11,  87.80) .. controls (  80.06,  87.22) and (  80.08,  86.64) ..
        (  80.16,  86.25) .. controls (  80.22,  85.90) and (  80.28,  85.16) ..
        (  80.29,  84.60) -- 
        (  80.30,  83.59) -- 
        (  80.73,  83.08) .. controls (  81.45,  82.23) and (  81.45,  82.17) ..
        (  80.86,  80.56) .. controls (  80.49,  79.57) and (  80.64,  78.78) ..
        (  81.55,  76.76) .. controls (  81.81,  76.19) and (  82.16,  75.25) ..
        (  82.34,  74.69) .. controls (  82.71,  73.48) and (  82.90,  73.23) ..
        (  84.20,  72.25) .. controls (  84.69,  71.87) and (  85.55,  71.22) ..
        (  86.10,  70.81) .. controls (  87.41,  69.81) and (  88.12,  69.42) ..
        (  89.40,  69.01) .. controls (  91.06,  68.47) and (  91.29,  68.27) ..
        (  92.28,  66.57) .. controls (  92.81,  65.66) and (  95.45,  62.46) ..
        (  96.41,  61.56) .. controls (  96.68,  61.31) and (  97.40,  60.77) ..
        (  98.00,  60.36) .. controls (  99.36,  59.45) and (  99.90,  58.95) ..
        ( 100.66,  57.90) .. controls ( 101.44,  56.83) and ( 101.90,  56.43) ..
        ( 103.43,  55.54) .. controls ( 105.04,  54.60) and ( 105.07,  54.60) ..
        ( 106.44,  54.31) .. controls ( 107.92,  54.01) and ( 107.96,  53.99) ..
        ( 108.80,  53.33) .. controls ( 109.62,  52.67) and ( 109.87,  52.59) ..
        ( 110.44,  52.82) .. controls ( 110.99,  53.04) and ( 111.33,  53.50) ..
        ( 112.03,  54.98) .. controls ( 112.73,  56.46) and ( 113.30,  57.33) ..
        ( 114.78,  59.15) .. controls ( 115.38,  59.90) and ( 115.96,  60.71) ..
        ( 116.07,  60.94) .. controls ( 116.17,  61.17) and ( 116.41,  61.91) ..
        ( 116.60,  62.59) .. controls ( 116.78,  63.26) and ( 117.03,  64.08) ..
        ( 117.15,  64.41) .. controls ( 117.27,  64.74) and ( 117.39,  65.31) ..
        ( 117.42,  65.68) .. controls ( 117.47,  66.27) and ( 117.44,  66.46) ..
        ( 117.11,  67.39) .. controls ( 116.66,  68.72) and ( 116.65,  69.01) ..
        ( 117.10,  69.81) .. controls ( 117.43,  70.40) and ( 117.45,  70.48) ..
        ( 117.45,  71.39) -- 
        ( 117.45,  72.35) -- 
        ( 118.01,  73.17) .. controls ( 118.69,  74.16) and ( 118.91,  74.64) ..
        ( 119.40,  76.21) .. controls ( 119.86,  77.68) and ( 119.88,  77.73) ..
        ( 120.68,  78.74) .. controls ( 121.72,  80.05) and ( 122.08,  81.42) ..
        ( 122.06,  84.03) -- 
        ( 122.05,  85.64) -- 
        ( 122.44,  86.48) .. controls ( 122.92,  87.48) and ( 123.03,  88.18) ..
        ( 123.11,  90.61) .. controls ( 123.14,  91.55) and ( 123.21,  92.46) ..
        ( 123.26,  92.63) .. controls ( 123.32,  92.81) and ( 123.52,  93.14) ..
        ( 123.70,  93.38) .. controls ( 124.17,  93.96) and ( 124.22,  94.31) ..
        ( 123.96,  95.08) .. controls ( 123.68,  95.90) and ( 123.69,  96.07) ..
        ( 124.04,  96.41) .. controls ( 124.44,  96.78) and ( 124.71,  97.45) ..
        ( 124.85,  98.40) .. controls ( 125.06,  99.82) and ( 125.30, 100.34) ..
        ( 126.82, 102.78) .. controls ( 127.49, 103.85) and ( 128.01, 104.83) ..
        ( 128.31, 105.56) .. controls ( 128.97, 107.22) and ( 129.55, 107.99) ..
        ( 130.22, 108.12) .. controls ( 130.49, 108.18) and ( 130.59, 108.14) ..
        ( 130.85, 107.88) .. controls ( 131.09, 107.64) and ( 131.15, 107.49) ..
        ( 131.15, 107.17) .. controls ( 131.15, 106.64) and ( 130.98, 106.32) ..
        ( 130.40, 105.76) .. controls ( 129.62, 105.01) and ( 129.52, 104.66) ..
        ( 129.79, 103.58) .. controls ( 129.94, 103.01) and ( 129.99, 102.53) ..
        ( 129.96, 102.05) .. controls ( 129.96, 101.91) and ( 129.66, 101.35) ..
        ( 129.30, 100.82) .. controls ( 128.71,  99.94) and ( 128.45,  99.37) ..
        ( 128.45,  98.96) .. controls ( 128.45,  98.88) and ( 128.59,  98.27) ..
        ( 128.75,  97.61) .. controls ( 129.22,  95.75) and ( 129.16,  95.35) ..
        ( 128.35,  94.51) .. controls ( 127.61,  93.76) and ( 127.54,  93.53) ..
        ( 127.61,  91.90) -- 
        ( 127.68,  90.51) -- 
        ( 127.31,  90.01) .. controls ( 126.64,  89.09) and ( 126.62,  89.00) ..
        ( 126.70,  87.26) -- 
        ( 126.77,  85.71) -- 
        ( 126.48,  85.17) .. controls ( 126.24,  84.69) and ( 126.21,  84.54) ..
        ( 126.23,  83.85) .. controls ( 126.26,  83.14) and ( 126.23,  83.02) ..
        ( 125.98,  82.59) .. controls ( 125.74,  82.19) and ( 125.70,  82.01) ..
        ( 125.70,  81.41) .. controls ( 125.71,  80.78) and ( 125.76,  80.61) ..
        ( 126.20,  79.71) .. controls ( 126.65,  78.79) and ( 126.69,  78.66) ..
        ( 126.68,  77.96) .. controls ( 126.68,  76.93) and ( 126.42,  76.32) ..
        ( 125.27,  74.63) .. controls ( 124.14,  72.96) and ( 124.01,  72.68) ..
        ( 123.74,  71.35) .. controls ( 123.50,  70.16) and ( 123.45,  70.06) ..
        ( 122.76,  69.12) .. controls ( 122.05,  68.17) and ( 121.96,  67.82) ..
        ( 122.24,  67.11) .. controls ( 122.66,  66.00) and ( 122.45,  65.65) ..
        ( 121.24,  65.42) .. controls ( 120.19,  65.21) and ( 119.85,  65.07) ..
        ( 119.75,  64.81) .. controls ( 119.64,  64.52) and ( 120.44,  62.04) ..
        ( 120.76,  61.71) .. controls ( 121.02,  61.42) and ( 121.35,  61.36) ..
        ( 121.73,  61.52) .. controls ( 122.10,  61.68) and ( 122.37,  62.05) ..
        ( 122.71,  62.92) .. controls ( 122.92,  63.43) and ( 123.14,  63.76) ..
        ( 123.55,  64.18) .. controls ( 124.24,  64.87) and ( 124.68,  65.01) ..
        ( 125.25,  64.70) .. controls ( 125.68,  64.47) and ( 125.73,  64.16) ..
        ( 125.50,  63.26) .. controls ( 125.23,  62.23) and ( 125.42,  62.23) ..
        ( 126.90,  63.22) .. controls ( 127.43,  63.57) and ( 127.97,  63.86) ..
        ( 128.09,  63.86) .. controls ( 128.34,  63.86) and ( 128.70,  63.51) ..
        ( 129.05,  62.93) .. controls ( 129.62,  61.97) and ( 130.70,  62.25) ..
        ( 131.99,  63.70) .. controls ( 132.36,  64.11) and ( 132.77,  64.47) ..
        ( 132.91,  64.51) .. controls ( 133.27,  64.60) and ( 133.66,  64.35) ..
        ( 134.31,  63.61) .. controls ( 135.00,  62.82) and ( 135.19,  62.71) ..
        ( 136.10,  62.62) .. controls ( 136.49,  62.58) and ( 136.91,  62.48) ..
        ( 137.04,  62.40) .. controls ( 137.24,  62.26) and ( 137.27,  62.19) ..
        ( 137.23,  61.76) .. controls ( 137.17,  61.19) and ( 137.17,  61.19) ..
        ( 137.72,  60.84) .. controls ( 138.09,  60.60) and ( 138.14,  60.51) ..
        ( 138.20,  60.09) .. controls ( 138.34,  59.08) and ( 138.30,  59.10) ..
        ( 139.29,  59.15) .. controls ( 140.09,  59.18) and ( 140.18,  59.16) ..
        ( 140.49,  58.93) .. controls ( 141.22,  58.37) and ( 141.96,  56.94) ..
        ( 142.15,  55.71) .. controls ( 142.36,  54.31) and ( 142.67,  53.88) ..
        ( 143.60,  53.72) .. controls ( 144.46,  53.56) and ( 144.72,  53.30) ..
        ( 145.04,  52.29) .. controls ( 145.28,  51.50) and ( 145.57,  51.11) ..
        ( 146.48,  50.39) .. controls ( 147.75,  49.38) and ( 148.25,  48.71) ..
        ( 148.25,  48.00) .. controls ( 148.25,  47.70) and ( 148.18,  47.57) ..
        ( 147.85,  47.26) .. controls ( 147.63,  47.06) and ( 147.45,  46.82) ..
        ( 147.45,  46.73) .. controls ( 147.45,  46.32) and ( 147.77,  45.85) ..
        ( 148.34,  45.40) .. controls ( 149.16,  44.77) and ( 149.46,  44.28) ..
        ( 149.95,  42.82) .. controls ( 150.17,  42.15) and ( 150.42,  41.50) ..
        ( 150.51,  41.38) .. controls ( 150.76,  41.03) and ( 151.15,  40.89) ..
        ( 152.20,  40.77) .. controls ( 153.78,  40.59) and ( 153.97,  40.42) ..
        ( 154.19,  38.93) .. controls ( 154.38,  37.65) and ( 154.61,  37.21) ..
        ( 155.35,  36.63) .. controls ( 155.68,  36.37) and ( 155.95,  36.10) ..
        ( 155.95,  36.02) .. controls ( 155.95,  35.94) and ( 155.97,  35.82) ..
        ( 156.01,  35.74) .. controls ( 156.11,  35.47) and ( 155.57,  34.98) ..
        ( 154.99,  34.80) .. controls ( 153.97,  34.47) and ( 153.82,  34.06) ..
        ( 154.36,  32.98) .. controls ( 154.80,  32.09) and ( 155.39,  31.68) ..
        ( 157.27,  30.96) .. controls ( 158.88,  30.34) and ( 159.69,  29.86) ..
        ( 159.98,  29.37) .. controls ( 160.32,  28.79) and ( 160.68,  28.45) ..
        ( 161.26,  28.16) .. controls ( 161.90,  27.84) and ( 162.36,  27.31) ..
        ( 162.55,  26.68) .. controls ( 162.62,  26.48) and ( 162.93,  25.75) ..
        ( 163.25,  25.06) .. controls ( 163.57,  24.38) and ( 163.97,  23.32) ..
        ( 164.15,  22.72) .. controls ( 164.32,  22.11) and ( 164.55,  21.48) ..
        ( 164.66,  21.30) .. controls ( 164.94,  20.84) and ( 165.52,  20.35) ..
        ( 166.33,  19.88) .. controls ( 167.21,  19.36) and ( 167.47,  19.05) ..
        ( 167.75,  18.19) .. controls ( 168.25,  16.67) and ( 169.01,  15.68) ..
        ( 170.16,  15.07) .. controls ( 170.87,  14.69) and ( 171.41,  14.24) ..
        ( 171.59,  13.87) .. controls ( 171.66,  13.73) and ( 171.80,  13.27) ..
        ( 171.89,  12.85) .. controls ( 172.07,  12.05) and ( 172.38,  11.46) ..
        ( 173.09,  10.61) .. controls ( 173.73,   9.85) and ( 173.90,   9.58) ..
        ( 174.25,   8.73) .. controls ( 174.68,   7.71) and ( 175.12,   7.13) ..
        ( 176.33,   6.06) .. controls ( 177.40,   5.10) and ( 177.90,   4.45) ..
        ( 178.24,   3.53) .. controls ( 178.73,   2.19) and ( 178.95,   2.04) ..
        ( 180.71,   1.87) .. controls ( 182.22,   1.72) and ( 182.76,   1.55) ..
        ( 183.30,   1.03) .. controls ( 184.09,   0.27) and ( 184.04,  -0.21) ..
        ( 182.98,  -1.77) .. controls ( 181.62,  -3.76) and ( 181.62,  -4.00) ..
        ( 182.87,  -5.64) .. controls ( 183.23,  -6.11) and ( 183.73,  -6.82) ..
        ( 183.99,  -7.22) .. controls ( 184.59,  -8.16) and ( 184.81,  -8.26) ..
        ( 186.04,  -8.17) .. controls ( 186.84,  -8.11) and ( 186.98,  -8.13) ..
        ( 187.37,  -8.33) .. controls ( 188.86,  -9.11) and ( 190.81, -11.38) ..
        ( 191.60, -13.25) .. controls ( 192.06, -14.34) and ( 192.20, -14.59) ..
        ( 192.70, -15.19) .. controls ( 193.27, -15.86) and ( 193.42, -16.18) ..
        ( 193.55, -17.00) .. controls ( 193.82, -18.64) and ( 193.88, -18.90) ..
        ( 194.10, -19.33) .. controls ( 194.37, -19.85) and ( 195.13, -20.62) ..
        ( 196.00, -21.24) .. controls ( 197.31, -22.17) and ( 198.28, -23.34) ..
        ( 198.79, -24.59) .. controls ( 198.92, -24.92) and ( 199.10, -25.68) ..
        ( 199.20, -26.29) .. controls ( 199.41, -27.70) and ( 199.61, -28.14) ..
        ( 200.39, -28.94) .. controls ( 201.49, -30.05) and ( 201.56, -30.46) ..
        ( 200.86, -31.38) .. controls ( 200.65, -31.66) and ( 200.42, -31.98) ..
        ( 200.36, -32.10) .. controls ( 200.17, -32.48) and ( 200.24, -33.11) ..
        ( 200.55, -33.85) .. controls ( 201.07, -35.08) and ( 200.94, -35.44) ..
        ( 199.72, -36.15) .. controls ( 198.85, -36.66) and ( 198.55, -36.95) ..
        ( 198.55, -37.28) .. controls ( 198.55, -37.66) and ( 198.86, -37.90) ..
        ( 199.66, -38.14) .. controls ( 200.18, -38.29) and ( 200.49, -38.46) ..
        ( 200.73, -38.70) -- 
        ( 201.07, -39.04) -- 
        ( 201.00, -39.97) .. controls ( 200.92, -41.05) and ( 201.00, -41.20) ..
        ( 201.76, -41.48) .. controls ( 202.02, -41.57) and ( 202.47, -41.82) ..
        ( 202.76, -42.04) .. controls ( 203.83, -42.82) and ( 204.21, -42.98) ..
        ( 214.60, -47.24) .. controls ( 226.78, -52.22) and ( 230.97, -54.05) ..
        ( 231.48, -54.61) .. controls ( 231.64, -54.79) and ( 231.65, -57.32) ..
        ( 231.63,-105.44) .. controls ( 231.60,-156.00) and ( 231.60,-156.09) ..
        ( 231.40,-156.11) .. controls ( 231.29,-156.12) and ( 230.68,-155.97) ..
        ( 230.05,-155.78) -- 
        cycle ;
}

\newcommand{\fullmap}{%
  % .. Terre ..
  \fondterre

  % .. Mers ..
  % ... Mediterranée ...
  \mermed
  % ... Mer Rouge ...
  \merrouge

  % .. Nil ..
  \lenil

  % .. Lacs ..
  \lacfayoum
  \lacdelta
  \lactanis
  \lacpithom
  \lacavaris
  \lacalexandrie
  \lacsinai

  % .. Frontières ..
  \begin{scope}[shift={(-19.38,-66.23)}]
    \frontold
  \end{scope}
  \begin{scope}[shift={(-155.28, 181.32)}]
    \frontnew
  \end{scope}

  % .. Légendes des masses d'eau ..
  \legendemasseseau

  % .. Villes ..
  \villes

  % .. Régions ..
  \regions
}



\pgfmathsetmacro{\xmin}{-155.74}
\pgfmathsetmacro{\xmax}{ 231.49}
\pgfmathsetmacro{\ymin}{-319.18}
\pgfmathsetmacro{\ymax}{ 213.34}

\thispagestyle{empty}
\begin{tikzpicture}[remember picture, overlay, anchor=north west,
                    inner sep=0, outer sep=0]

  \node at (current page.north west) {%

    % \rule{\paperwidth}{\paperheight}

    \begin{tikzpicture}[%
          remember picture, overlay,
          framed, tight background,
          background rectangle/.style={ultra thick, draw},
          scale=13.5, y=0.118pt, x=0.118pt, 
          inner sep=0pt, outer sep=0pt,
          shift={(-\xmin,-\ymax)}
    ]

      % .. Define the four angles of the map ..
      \coordinate (A) at (\xmin, \ymin) ;
      \coordinate (B) at (\xmax, \ymin) ;
      \coordinate (C) at (\xmin, \ymax) ;
      \coordinate (D) at (\xmax, \ymax) ;

      % \clip ([shift={(0,5)}] A) rectangle ([shift={(0,-5)}] D) ;

      \fullmap

      % .. Boussole ..
      \begin{scope}[shift={(-120.00,-250.00)}]
        \boussole
      \end{scope}

      % .. Echelle ..
      \begin{scope}[shift={(-110.00,-290.00)}]
        \echelle
      \end{scope}

      % \fill [opacity=0.25, red] ([shift={(0,5)}] A) rectangle ([shift={(0,-5)}] D) ;

      % \draw [help lines, step=1] (A) grid (D) ;
      % \draw [help lines, step=5] (A) grid (D) ;
      % \draw [help lines, step=10] (A) grid (D) ;
      % \node [red] at (0.00, 0.00) {O} ;
      % \node [red, above right] at (A) {A} ;
      % \node [red, above left] at (B) {B} ;
      % \node [red, below right] at (C) {C} ;
      % \node [red, below left] at (D) {D} ;
    \end{tikzpicture}

  } ;
\end{tikzpicture}

% %!TEX root = Kheops_Grandet_Histoire.htx

\chapter{Chronologie}
%=====================================================================

Par \DF, avec la collaboration de \gens[Raphaële]{Meffre} et 
\gens[Vincent]{Razanajao} (version de novembre 2012).

\section{Prédynastique}
%---------------------------------------------------------------------

\begin{listerois}
  \item [\HE Badari \datesregne{4500}{3800}] ~\\
        \nouveaute{Premières traces de cuivre.}
  \item [\BE Maadi Bouto \datation{(environ \anorange{4000}{3400})}]
  \item [\HE Nagada~I Nagada~IIA-B 
         \datation{(environ \anorange{3800}{3600})}]
  \item [\HE Nagada~IIC-D \datation{(environ \anorange{3600}{3300})}]
  \item [\HE Nagada~IIIA \datation{(environ \anorange{3300}{3250})}]
\end{listerois}

Nagada~IIIA-B correspond en \BE à Bouto~III-IV. Mais on ne doit pas 
les considérer comme des cultures différentes. En effet, à partir de 
la fin de Nagada~II, la culture de Nagada s'étend sur tout le 
territoire égyptien (et jusqu'à Gaza).

\section{Dynastie~0}
%---------------------------------------------------------------------

Pas une vraie dynastie, artificielle.

\datation{Environ~\anorange{3250}{3100}}

\begin{listerois}
  \item [\frquote{Scorpion~I}] Tombe à Abydos~U-j.
  \item [Iry-Hor] Tombes à Abydos~B1-B2-B0.
  \item [Scorpion~\frquote{II}] Tombe à Abydos~B50 ou à Hiérakonpolis ?
  \item [Sékhen ou Ka (\nospace{?})] tombes à Abydos~B7-B9.
  \item [Nârmer] Tombes à Abydos~B17-B18. \\
        Un événement : \frquote{Frapper les Tjéhénous} (peut-être 
        dans le Delta à l'époque) sur un sceau-cylindre de 
        Hiérakonpolis et une étiquette de jarre de la tombe de 
        Âha et la \source{palette de Nârmer}.
        C'est un événement marqueur de son règne, mais pas une 
        datation.
\end{listerois}

\section{Époque Thinite}
%---------------------------------------------------------------------

\datation{Environ~\anorange{3100}{2700}}

\subsection{\texorpdfstring{\dyn{1}}{Ie dynastie}}
%~~~~~~~~~~~~~~~~~~~~~~~~~~~~~~~~~~~~~~~~~~~~~~~~~~~~~~~~~~~~~~~~~~~~~

Nous savons maintenant que les tombes des rois de cette dynastie 
se trouvent à Abydos, celles de Saqqara appartiennent à de hauts 
fonctionnaires.

événements marquants pour Âha et Djer \donc{} plus d'un par règne, 
on glisse d'une épithète vers un découpage événementiel \donc{} 
conscience d'un devoir de datation.
Éponymie sous Serpent.
Sous Qâ, événements récurrents, répétitifs.
Pas d'indice sur leur régularité \donc{} problème pour datation.
Première attestation d'Apis, sous Âha.

\begin{listerois}
  \item [Âha] Tombes à Abydos~B1O-B15-B19-B13-B14.\\
        Présenté comme le fondateur de Memphis ds les annales.
  \item [Djer] Tombe~O à Abydos.
  \item [Serpent (Djet, Ouadji)] Tombe~Z à Abydos.
  \item [Den (Oudimou)] Tombe~T à Abydos.
  \item [Ândjib] Tombe~X à Abydos.
  \item [Sémerkhet] Tombe à Abydos.
  \item [Qâ] Tombe~Q à Abydos.
\end{listerois}

\subsection{\texorpdfstring{\dyn{2}}{IIe dynastie}}
%~~~~~~~~~~~~~~~~~~~~~~~~~~~~~~~~~~~~~~~~~~~~~~~~~~~~~~~~~~~~~~~~~~~~~

Cette dynastie a laissé très peu d'informations. Deux rois ont une 
tombe à Saqqara, deux une tombe à Abydos, nous ne savons rien quant à 
la sépulture des autres. On a proposé de voir dans les galeries 
souterraines sous le bâtiment nord du complexe funéraire de Djoser, 
les restes de tombes royales de la~\dyn{2} ; de même pour les 
galeries inachevées de la cour nord.

\noindent\nouveaute{Première attestation de divinité à tête animale.}

Aucune tablette \donc{} sûrement à cause d'un changement de pratique, 
réforme du système d'enregistrement des datations. Seules infos : 
\source{Pierre de Palerme}
\donc{} événements en cycles bisannuels : années d'escorte d'Horus, 
avec recensement, sans précision mais avec un chiffre. Restitution ou 
interprétation ?

\begin{listerois}
  \item [Hétepsékhemouy] Tombe à Saqqara.
  \item [Nebrê / Râneb]
  \item [Nynétjer] Tombe à Saqqara.
  \item [Ounegnebty]
  \item [Séned]
  \item [Sékhemib]
  \item [Péribsen] Tombe à Abydos.
  \item [Khâsékhemouy] Tombe à Abydos. Il a deux enclos 
        d'\SI{1}{\hectare} à Abydos et Hiérakonpolis
\end{listerois}

\section{\OK}
%---------------------------------------------------------------------

\datation{Environ~\anorange{2700}{2200}}

\subsection{\texorpdfstring{\dyn{3}}{IIIe dynastie}}
%~~~~~~~~~~~~~~~~~~~~~~~~~~~~~~~~~~~~~~~~~~~~~~~~~~~~~~~~~~~~~~~~~~~~~

Nous ne connaissons pas avec certitude les noms de tous les souverains 
de cette dynastie. En effet, elle comprendrait cinq ou six rois, or 
nous avons trop de noms et pas assez de tombes. Nous ne connaissons 
avec certitude que la position des deux premiers et du dernier, 
l'ordre chronologique des autres est sujet à réflexion. Les listes 
d'époques pharaoniques ultérieures se contredisent \emph{a priori} 
concernant l'ordre des règnes, le nombre de rois et leurs noms. Elles 
semblent même inventer des rois fictifs, à moins que ce ne soient des 
noms inventés correspondant à des rois que nous ne reconnaissons pas. 
Leurs données ne concordent pas avec les témoins contemporains de la 
\dyn{3}. Par exemple, la liste de \gens{Manéthon} donne neuf noms, 
chiffre certainement fictif.

\begin{listerois}
  \item [Nétjérykhet (Djoser)] Pyramide à Saqqara.
  \item [Sékhemkhet] = Djéserti(-ânkh) / Djéserttéti ? Pyramide 
        inachevée à Saqqara.
  \item [Sanakht] = Nebka ? Canon de Turin, liste d'Abydos.
  \item [Khaba] Pyramide à Zawiyet el-Aryan.
  \item [Qahedjet] Stèle du Louvre. Serait peut-être le roi suivant, 
        \textbf{\sffamily Houny (Hounysout / Nysouthou)}. 
        Tombe inconnue.
  \item [Houny] construction de pyr provinciales = marqueurs du pouvoir
\end{listerois}

Les listes postérieures se contredisent : invention de rois (ou rois 
existants mais qu'on ne connaît pas), les données ne concordent pas 
avec les témoins de la \dyn{3} (\gens{Manéthon} : neuf rois ! chiffre 
fictif = Ennéade, Palerme : escortes d'Horus et apparition royale, sans
numérotation mais avec mention d'événements).

Djoser : nouveau type de complexe funéraire royal : pyramide 
en lits déversés, par tranches, mur d'enceinte, tout en pierre, 
\SI{15}{\hectare} pour l'ensemble. Maître d'{\oe}uvre = Imhotep, 
grand prêtre d'Héliopolis, chancelier du roi, directeur des artisans. 
A fini comme patron des scribes, puis dieu guérisseur et fils de Ptah, 
et après le paganisme grand maître des alchimistes et des philosophes. 
Pas la première architecture de pierre, mais la première bâtie 
entièrement en pierre et utilisation aussi massive.

\subsection{\texorpdfstring{\dyn{4}}{IVe dynastie}}
%~~~~~~~~~~~~~~~~~~~~~~~~~~~~~~~~~~~~~~~~~~~~~~~~~~~~~~~~~~~~~~~~~~~~~

Nous connaissons sept rois, la tradition en donne huit. Du huitième 
(Thamphthis), nous n'avons aucune trace. Le nom du propriétaire de la 
pyramide de Zawiyet el-Aryan est douteux. Sa position chronologique 
n'est pas absolument certaine, mais avec une bonne probabilité. 

\noindent\nouveaute{Le cartouche apparaît à cette époque.} \\
\nouveaute{Dans les tombes de particuliers, apparition de la formule 
d'offrande, des adresses aux passants et des autobiographies.}

Mise en place du complexe type.
                  
\begin{listerois}
  \item [Snéfrou] Une pyramide à Meïdoum (lits déversés, en tranches, 
        parement en lits horizontaux, famille à \SI{1}{\km}, notamment 
        Néfermaât (invente la gravure en creux rempli de pâte)), deux 
        pyramides à Dahchour (sud = l'inverse, nord = tout horizontal) 
        famille à mi-distance des deux, courtisans à plus 
        d'\SI{1}{\km} de chacune des pyramides). \\
        \nouveaute{Apparition de la datation chiffrée bisannuelle} ; \\
        \nouveaute{Première représentation d'acclimatation d'arbres 
        exotiques (pins de Syrie verts et arbres à encens verts)} ; \\
        \nouveaute{Première formule d'offrande, premier récit 
        autobiographique à la troisième personne, première 
        représentation de domaines funéraires, peut-être le premier 
        serdab, premières scènes de vie quotidienne (chez Metjen).}
  \item [Khnoumkhoufoui / Khéops] Fils de Snéfrou. Pyramide à Giza 
        (tout horizontal sur un inselberg) . \\
        \nouveaute{Première couverture en chevrons ;} \\
        \nouveaute{Premières formules de fondation royale.}
  \item [Rêdjédef (Djidoufri / Ratoisès)] Fils de Khéops. 
        Pyramide à Abou Rawach (inselberg, lits horizontaux, parement 
        en lits déversés, partie inférieure en granit, la plus longue 
        chaussée d'\kmt, avec un coude, comme Snéfrou, sûrement temple 
        médian, dans la première partie ;
        courtisans à \SI{1.5}{\km}, sur une autre éminence, temple à 
        l'est, une fosse naviforme, statues des enfants et famille 
        dans le temple funéraire).
        Elle a bien été finie, mais détruite par les romains. 
        Céramiques de la \dyn{6}, culte continu durant tout le \MK.
  \item [Khéphren] À lire probablement Râkhâf. Frère cadet de 
        Rêdjédef. Pyramide à Giza (lits plats, partie inférieure en 
        granit). Plan type du complexe. Temple du Sphinx.
  \item [Bakarê] = Bikheris pour Manéthon ? Fils de Rêdjédef ? 
        de Khéphren ? 
        Pyramide inachevée à Zawiyet el-Aryan.
  \item [Mykérinos] Fils de Khéphren. Pyramide à Giza (mm genre que 
        Rêdjédef avec granit, finie par Chépsèskaf, courtisans 
        répartis autour des trois pyramides).
  \item [Chépsèskaf] Fils de Mykérinos. Tombe à Saqqara sud 
        (Mastabat Faraoun, en forme de sarcophage ou de sanctuaire 
        \emph{per-nou}, calcaire + granit pour le bas). \\
        \nouveaute{Premier décret royal.}
\end{listerois}

Khentykaous : c'est peut-être elle ce 8\ieme roi ???
Elle serait la mère de deux rois. Peut-être considérée comme roi après.

\subsection{\texorpdfstring{\dyn{5}}{Ve dynastie}}
%~~~~~~~~~~~~~~~~~~~~~~~~~~~~~~~~~~~~~~~~~~~~~~~~~~~~~~~~~~~~~~~~~~~~~

Les complexes funéraires de six rois de la dynastie comprennent un 
temple solaire. Deux seulement ont été localisés.
                  
\noindent\nouveaute{Premières mentions d'Osiris.} \\
\nouveaute{Apparition du second cartouche.} 
nouveau titre : \titre{doyen de la salle d'audience} : changement 
dans les institutions, l'organisation de l'administration \donc{} 
effort important en littérature et architecture sur l'infaillibilité 
du roi,~\dots (avant le \MK, finalement)

\begin{listerois}
  \item [Ouserkaf] Pyramide à Saqqara, au nord-est de Djoser, 
        temple solaire à Abousir.
  \item [Sahourê] Fils d'Ouserkaf, pyramide à Abousir, temple 
        solaire. \\
        \nouveaute{Première \emph{Königsnovelle} ;} \\
        \nouveaute{Premier serment royal (chez lui et chez 
        Nyankhsakhmet, son médecin-chef. on trouve chez lui aussi les 
        premiers exemples d'enseignement loyaliste !).}
  \item [Néferirkarê Kakaï] Pyramide à Abousir, temple solaire
  \item [Chépsèskarê Ousernétjérou] Pyramide inachevée à Abousir ?
  \item [Rânéferef Izi] Pyramide à Abousir, temple solaire
  \item [Niouserrê Iny] Pyramide à Abousir, temple solaire à 
        Abou Ghourob (= \frquote{père des corbeaux})
  \item [Menkaouhor Akaouhor] Pyramide et temple solaire non localisés
  \item [Djedkarê Isési] Pyramide à Saqqara sud, pas de temple solaire
  \item [Ounas] Pyramide à Saqqara, pas de temple solaire. Chaussée 
        fait un coude (suivant le talweg). L'organisation du complexe 
        ressemble à ceux de la \dyn{6} \\
        \nouveaute{Premiers \TP.}
\end{listerois}

Incertitude sur l'ordre de Chépsèskarê et Rânéferef.

Néferkarê et Rânéferef : archives dans les temples funéraires \donc{} 
plusieurs centaines de personnes y travaillent et appartiennent aux 
institutions de l'état : association entre titres religieux et titres 
civils. Domaines funéraires : appartiennent aux morts et font vivre 
les vivants. Ça fonctionne jusqu'à la fin du \MK.
Production de nourriture surtout au temple solaire, ravitaillement 
deux fois par jour (viande,~\dots pas ou peu d'encens !)

\subsection{\texorpdfstring{\dyn{6}}{VIe dynastie}}
%~~~~~~~~~~~~~~~~~~~~~~~~~~~~~~~~~~~~~~~~~~~~~~~~~~~~~~~~~~~~~~~~~~~~~

\begin{listerois}
  \item [Téti] Pyramide à Saqqara.
  \item [Ouserkarê] Connu seulement par des annales postérieures, 
        mais existence certaine (meurt l'année suivant son 
        couronnement).
  \item [Méryrê Pépi~I\ier] Pyramide à Saqqara sud.
        \TP dans les pyramides de reines, mais au nom du roi.
  \item [Mérenrê Nemtyemsaf] Pyramide à Saqqara sud.
  \item [Néferkarê Pépi~II] Pyramide à Saqqara sud.
  \item [Nemtyemsaf~II]
  \item [Nitocris] Cette reine est fictive.
\end{listerois}

Développement des récits événementiels (expéditions,~\dots).
Développement des nécropoles provinciales. Grâce à l'affermissement 
de l'administration centrale. 
Le défunt est désormais qualifié d'Osiris.
De plus en plus de \titre{prêtres ritualistes en chef}, en parallèle 
avec le développement des appels aux vivants et la montée du culte des 
intercesseurs (qui perdure jusqu'au début du \NK). Ce sont des titres 
de tombe !
Nubie, Libye, Sinaï.

\section{\PPI}
%---------------------------------------------------------------------

\datation{Environ~\anorange{2200}{2050}}

La \dyn{7} est fictive. \gens{Manéthon} dit que soixante-dix rois 
ont régné pendant soixante-dix jours (période violente de la fin de 
la \dyn{6}).
Des \dynum{8}, \dynlist{9}{10}, on ne sait pas grand-chose. Nous 
ne donnons ici que quelques noms de rois de cette période.

\subsection{\texorpdfstring{\dyn{8}}{VIIIe dynastie}}
%~~~~~~~~~~~~~~~~~~~~~~~~~~~~~~~~~~~~~~~~~~~~~~~~~~~~~~~~~~~~~~~~~~~~~

D'après les listes postérieures, la \dyn{8} serait composée de 
vingt-sept rois.

\begin{listerois}
  \item [Qararê Aba] 14\ieme~roi de la dynastie ? 
        Pyramide à Saqqara sud. \\
        \nouveaute{Dernier exemple de \TP dans une tombe royale.}
  \item [Néferkaouhor Khououihap] 16\ieme~roi de la dynastie ? 
        Pyramide à Dara en \ME.
\end{listerois}

\subsection{\texorpdfstring{\dynum{9}-\dyn{10}}{IXe-Xe dynastie}}
%~~~~~~~~~~~~~~~~~~~~~~~~~~~~~~~~~~~~~~~~~~~~~~~~~~~~~~~~~~~~~~~~~~~~~

La \dyn{10} de \gens{Manéthon} semble être en fait un double de la 
\dynum{9}. 
La capitale de ce royaume se trouve à Hiérakléopolis. Le nombre de 
rois est incertain. La tradition en donnerait dix-huit, les sources
contemporaines en donne six.

\begin{listerois}
  \item [Méryibrê Khéty]
  \item [Mérykarê] Pyramide à Saqqara ?
\end{listerois}

\subsection{\texorpdfstring{\dyn{11}}{XIe dynastie} (thébaine)}
%~~~~~~~~~~~~~~~~~~~~~~~~~~~~~~~~~~~~~~~~~~~~~~~~~~~~~~~~~~~~~~~~~~~~~

Cette dynastie, dont la capitale se trouve à Thèbes, est contemporaine 
et concurrente de la précédente. La tradition donne sept rois, mais en 
fait le premier était un prince local et la royauté du deuxième est 
certainement postérieure.

\noindent\nouveaute{Premières mentions d'Amon.}

\begin{listerois}
  \item [Montouhotep~I] N'est pas roi.
  \item [Antef~I] Tombe à Thèbes, El-Tarif.
  \item [Antef~II le grand] Tombe à Thèbes, El-Tarif. \\
        \nouveaute{Début de la construction de Karnak.}
  \item [Antef~III] Tombe à Thèbes, El-Tarif.
\end{listerois}

\section{\MK}
%---------------------------------------------------------------------

\noindent\nouveaute{Désormais les événements sont enregistrés suivant 
une datation chiffrée annuelle.} \\
\nouveaute{Rédaction des \TS.}


\subsection{\texorpdfstring{\dyn{11}}{XIe dynastie} (suite)}
%~~~~~~~~~~~~~~~~~~~~~~~~~~~~~~~~~~~~~~~~~~~~~~~~~~~~~~~~~~~~~~~~~~~~~

\begin{listerois}
  \item [Nebhépetrê Montouhotep~II] Unifie le pays, change 
        de titulature au cours de son règne (au fur et à mesure de ses 
        victoires). Tombe à Thèbes, \DeB.
  \item [Séânkhkarê Montouhotep~III]
  \item [Nebtaouyrê Montouhotep~IV] seulement info sur les carrières : 
        en l'\ar{2}, améthyste, \SI{30}{\km} au sud d'Assouan (Ouadi 
        el-Houdi) ; 
        expédition au Ouadi Hammamat (\num{10000}~hommes, miracle de 
        la gazelle, miracle de la pluie) ;
        expédition à Ayn Soukhna
  \item [Séhétepibrê Amenemhat~I] Tombe à Thèbes, vallée au sud 
        de \DeB dans la \frquote{vallée du dernier Montouhotep}
        (séhétepibtaouy).
\end{listerois}

\subsection{\texorpdfstring{\dyn{12}}{XIIe dynastie}}
%~~~~~~~~~~~~~~~~~~~~~~~~~~~~~~~~~~~~~~~~~~~~~~~~~~~~~~~~~~~~~~~~~~~~~

\datation{Environ~\anorange{2000}{1780}}

Le changement de dynastie est purement politique et intervient 
au cours du règne d'Amenemhat~I\ier. La question de la corégence 
éventuelle de certains rois est encore débattue.

Nouveau type statuaire : statue cube ;
Stèles cintrées (dès la \PPI), dans monument en briques crues, vouté, 
parvis, cour du temple, voies processionnelles \donc{} être dans la 
suite du dieu.

\TS ;
Formules des pyramidions ;
Textes funéraires royaux plus dans la pierre.

Développement du Fayoum (Sésostris~III - Amenemhat~III) ; 
Expédition à Pount et en Syrie, par la mer.

\noindent\nouveaute{Nouvelle formule d'offrandes.} 

\begin{listerois}
  \item [Séhétepibrê Amenemhat~I] même roi que le précédent. 
        Pyramide à Licht. (type \dyn{6}, avec éléments type \dyn{11})
        entame une grande réforme, mais est assassiné.
  \item [Khéperkarê Sésostris~I] Pyramide à Licht. (il termine la 
        pyramide de son père). Beaucoup de temples en pierre (Satis à 
        Éléphantine, Tôd, Amon à Karnak, Min à Coptos, Khentyimentyou 
        à Abydos, grand temple d'Héliopolis,~\dots) Poussée jusqu'à la 
        deuxième cataracte.
        Intensification des expéditions aux carrières, dont la grande 
        de l'\ar{38} au Ouadi Hammamat (\num{17000}~corvéables, des 
        militaires, des spécialistes, deux mois, \num{150}~sphinx, 
        \num{80}~statues), \SeK.
        Première attestation de Kouch.
        Beaucoup de fondations royales (= villes à plan carré, rues à 
        angles droits), céramique homogène dans tout le pays. Temples 
        en pierre : modif ds la PPI. Développement des nécropoles de 
        provinces : dans la capitale, c'est le roi le dieu de la 
        nécropole, là c'est le dieu local. L'économie est maintenant 
        centrée sur le temple, plus sur la tombe (\source{contrats 
        d'Assiout}).
        Le temple devient une statue de palais, et plus un palais 
        (pétrification).
        Diminution du nombre de titres (d'un facteur~\num{4} 
        à~\num{5}, standardisation).
        Littérature \donc{} montrer la légitimité du roi.
  \item [Néboukaourê Amenemhat~II] Pyramide à Dahchour.
  \item [Khâkhéperrê Sésostris~II] Pyramide à Illahoûn.
  \item [Khâkaourê Sésostris~III] Pyramide à Dahchour, complexe 
        funéraire à Abydos.
  \item [Nymaâtrê Amenemhat~III] Une pyramide à Dahchour, une 
        Pyramide à Hawara.
  \item [Maâkhérourê Amenemhat~IV] Pyramide à Mazghouna.
  \item [Kasobekrê Néfrousobek] Pyramide à Mazghouna.
\end{listerois}

Derniers \TP : Aba

On commence à avoir du bronze \donc{} grès

\subsection{\texorpdfstring{\dyn{13}}{XIIIe dynastie}}
%~~~~~~~~~~~~~~~~~~~~~~~~~~~~~~~~~~~~~~~~~~~~~~~~~~~~~~~~~~~~~~~~~~~~~

Elle comprend de nombreux rois, la position chronologique de beaucoup 
d'entre eux est incertaine. Certains égyptologues font commencer la 
\DPI avec l'avènement de cette dynastie, d'autres après le règne de 
Sébekhotep~IV. Nous suivrons cette seconde proposition.

Des divinités apparaissent désormais sur les stèles

\section{\DPI}
%---------------------------------------------------------------------

Il s'agit d'une période extrêmement confuse et avare en documents 
historiques. Une dynastie originaire d'Asie, les Hyksos, règne en 
\BE et a la suzeraineté sur le petit royaume de \HE. L'\kmt a perdu la
Basse-Nubie (Ouaouat). Le royaume soudanais de Kouch, dont la capitale 
est Kerma, a pris possession de toute la Nubie et même par moments, 
du sud de la \HE.

Pourrait être l'époque de l'invention (ferment) du proto-sinaïtique.


\subsection{\texorpdfstring{\dyn{14}}{XIVe dynastie}}
%~~~~~~~~~~~~~~~~~~~~~~~~~~~~~~~~~~~~~~~~~~~~~~~~~~~~~~~~~~~~~~~~~~~~~

Dans le Delta oriental, contemporaine de la fin de la \dyn{13}.

\subsection{\texorpdfstring{\dyn{15}}{XVe dynastie}}
%~~~~~~~~~~~~~~~~~~~~~~~~~~~~~~~~~~~~~~~~~~~~~~~~~~~~~~~~~~~~~~~~~~~~~

Rois hyksos, capitale Avaris\footnote{À peu près à \SI{30}{\km} au sud 
de Tanis}.

Noms sémitiques de l'ouest

\subsection{\texorpdfstring{\dyn{16}}{XVIe dynastie}}
%~~~~~~~~~~~~~~~~~~~~~~~~~~~~~~~~~~~~~~~~~~~~~~~~~~~~~~~~~~~~~~~~~~~~~

Dynastie hyksos qui régnerait en Palestine.

\subsection{\texorpdfstring{\dyn{17}}{XVIIe dynastie}}
%~~~~~~~~~~~~~~~~~~~~~~~~~~~~~~~~~~~~~~~~~~~~~~~~~~~~~~~~~~~~~~~~~~~~~

Capitale Thèbes.

La \HE est très riche (commerce Nord-Sud).

\begin{listerois}
  \item [Seqenenrê Taâa (Djéhoutyâa)] Il meurt au combat contre 
        les Hyksos.
  \item [Kamosis] Guerre contre les Hyksos (et Kouch).
\end{listerois}

\section{\NK}
%---------------------------------------------------------------------

\datation{c.~\anorange{1550}{1070}}

Roue, char de guerre, chevaux (tombe d'Ahmosis), du \PO 

En face, confédération de Khabour, avec \Mtn (Naharina pour les 
égyptiens) (indo-européen), toute la Syrie du nord

\subsection{\texorpdfstring{\dyn{18}}{XVIIIe dynastie}}
%~~~~~~~~~~~~~~~~~~~~~~~~~~~~~~~~~~~~~~~~~~~~~~~~~~~~~~~~~~~~~~~~~~~~~

Chronologie relative très précise. Chronologie absolue : dix ans 
maximum d'incertitude.

L'empire égyptien au \PO s'étend jusqu'en Syrie actuelle, à la 
frontière sud d'Ougarit et jusqu'au nord de la \Bqa. 
Au sud, le pays de Ouaouat et le pays de Kouch sont gouvernés par un 
vice-roi égyptien : le fils royal de Kouch. 

C'est l'époque de rédaction du \LM et des livres funéraires royaux.

De nouveau des textes funéraires dans les tombes royales (Amdouat, 
vache du ciel, cavernes, portes,~\dots).

Le roi suit un cycle solaire.

Les saints intercesseurs périclitent au début de la \dyn{18}.

Nouvelle iconographie, à partir d'Hatchepsout-Thoutmosis III, on 
s'adresse directement aux dieux.
Statues portant des images divines \donc{} \frquote{culte divin 
personnel}.
Sous Akhénaton : chapelles de culte personnel dans les maisons.

Retour à Amon, développement des oracles.

Fin \NK : théocratie.

Divines adoratrices.

Talatates $\Longleftrightarrow$ taylorisme.

\begin{listerois}
  \item [Ahmosis~I\ier \datation{(c.~\anorange{1550}{1525})}] 
        Tombe à Abydos sud.
        (victoire en l'\ar{2}, par sa mère et sa grand-mère)
  \item [Aménophis~I\ier \datation{(c.~\anorange{1525}{1504})}]
  \item [Thoutmosis~I\ier \datation{(c.~\anorange{1504}{1492})}] 
        poussée jusqu'à la 5\ieme cataracte
  \item [Thoutmosis~II \datation{(c.~\anorange{1492}{1479})}]
  \item [Hatchepsout \datation{(c.~\anorange{1479}{1458})}] 
        Corégente du jeune Thoutmosis~III. Expédition à Pount.
        À sa mort : le \Mtn attaque ! \donc{} Qadesh, Megiddo.
        Demi-victoire.
        Ils recommencent.
        \num{17}~campagnes militaires au \PO, \num{17}~victoires 
        (massacres et pillages), une campagne en Nubie.
        Grande quantité de bronze et de bois \donc{} constructions 
        partout
  \item [Thoutmosis~III \datation{(c.~\anorange{1479}{1425})}] 
        Tombe à Thèbes, \kv{34}. Campagnes militaires 
        en Syro-Palestine contre le royaume du \Mtn et ses alliés.
  \item [Aménophis~II \datation{(c.~\anorange{1425}{1397})}] 
        Tombe à Thèbes, \kv{35}.
  \item [Thoutmosis~IV \datation{(c.~\anorange{1397}{1388})}] 
        Tombe à Thèbes, \kv{43}.
  \item [Aménophis~III \datation{(c.~\anorange{1388}{1350})}] 
        Tombe à Thèbes, Vallée des Singes~WV\,22.
  \item [Aménophis~IV Akhénaton \datation{(c.~\anorange{1350}{1334})}] 
        Tombe à Amarna. Hérésie amarnienne. \\
        Le roi Soupilouliouma, vainqueur du royaume du \Mtn, 
        étend l'hégémonie hittite sur le nord du \PO. \\
        attaque le nord des possessions de l'\kmt et fomente des 
        révoltes dans l'\kmt elle-même.
        \nouveaute{Première attestation du chadouf.}
  \item [Smenekhkarê \datation{(c.~\anorange{1337}{1333})}]
  \item [Toutânkhamon \datation{(c.~\anorange{1333}{1323})}] 
        Tombe à Thèbes, \kv{62}. Retour au culte d'Amon.
  \item [Ay \datation{(c.~\anorange{1323}{1319})}] 
        Tombe à Thèbes, Vallée des Singes~WV\,23.
  \item [Horemheb \datation{(c.~\anorange{1319}{1292})}] 
        Tombe à Thèbes, \kv{57}.
\end{listerois}

Dieux dynastiques = Thot et Amon.

\subsection{\texorpdfstring{\dyn{19}}{XIXe dynastie}}
%~~~~~~~~~~~~~~~~~~~~~~~~~~~~~~~~~~~~~~~~~~~~~~~~~~~~~~~~~~~~~~~~~~~~~

\datation{c.~\anorange{1292}{1185}}

\begin{listerois}
  \item [Ramsès~I\ier \datation{(c.~\anorange{1292}{1290})}] 
        Tombe à Thèbes, \kv{16}. 
  \item [Séthi~I\ier \datation{(c.~\anorange{1290}{1279})}] 
        Tombe à Thèbes, \kv{17}. 
  \item [Ramsès~II \datation{(c.~\anorange{1279}{1213})}] 
        Tombe à Thèbes, \kv{7}. Bataille de Qadech contre les 
        Hittites. Traité égypto-hittite.
  \item [Merenptah \datation{(c.~\anorange{1213}{1203})}] 
        Tombe à Thèbes, \kv{8}. 
  \item [Amenmessé \datation{(c.~\anorange{1203}{1199})}] 
        Tombe à Thèbes, \kv{10}. 
  \item [Séthi~II \datation{(c.~\anorange{1199}{1193})}] 
        Tombe à Thèbes, \kv{15}. 
  \item [Siptah \datation{(c.~\anorange{1193}{1187})}] 
        Tombe à Thèbes, \kv{47}. 
  \item [Taousert \datation{(c.~\anorange{1187}{1185})}] Reine.
        Tombe à Thèbes, \kv{14}. 
\end{listerois}

\subsection{\texorpdfstring{\dyn{20}}{XXe dynastie}}
%~~~~~~~~~~~~~~~~~~~~~~~~~~~~~~~~~~~~~~~~~~~~~~~~~~~~~~~~~~~~~~~~~~~~~

\datation{c.~\anorange{1185}{1069}}

\begin{listerois}
  \item [Sethnakht \datation{(c.~\anorange{1185}{1183})}] 
        Tombe à Thèbes, \kv{14}. 
  \item [Ramsès~III \datation{(c.~\anorange{1183}{1152})}] 
        Tombe à Thèbes, \kv{11}. 
        Guerre contre les \frquote{Peuples de la Mer}.
  \item [Ramsès~IV \datation{(c.~\anorange{1152}{1145})}] 
        Tombe à Thèbes, \kv{2}. 
  \item [Ramsès~V \datation{(c.~\anorange{1145}{1142})}] 
        Tombe à Thèbes, \kv{9}. 
  \item [Ramsès~VI \datation{(c.~\anorange{1142}{1134})}] 
        Tombe à Thèbes, \kv{9}. 
  \item [Ramsès~VII \datation{(c.~\anorange{1134}{1126})}] 
        Tombe à Thèbes, \kv{1}. 
  \item [Ramsès~VIII \datation{(c.~\anorange{1126}{1125})}] 
  \item [Ramsès~IX \datation{(c.~\anorange{1125}{1107})}] 
        Tombe à Thèbes, \kv{6}. 
  \item [Ramsès~X \datation{(c.~\anorange{1107}{1099})}] 
        Tombe à Thèbes, \kv{18}. 
  \item [Ramsès~XI \datation{(c.~\anorange{1099}{1069})}] 
        Tombe à Thèbes, \kv{4}. 
\end{listerois}

\section{\TPI}
%---------------------------------------------------------------------

La fin du \NK est très embrouillée et, à vrai dire, aucun égyptologue 
ne peut décrire avec certitude la chronologie des événements. On doit 
donc considérer les dates des derniers rois ramessides et de ceux de 
la~\dyn{21} comme des outils pratiques mais en partie arbitraires. 
Les dates des \dynum{22}, \dynlist{23}{24} ne sont pas beaucoup plus 
assurées. Il faut attendre la \dyn{25} pour avoir des informations 
chronologiques fiables.

\subsection{\texorpdfstring{\dyn{21}}{XXIe dynastie}}
%~~~~~~~~~~~~~~~~~~~~~~~~~~~~~~~~~~~~~~~~~~~~~~~~~~~~~~~~~~~~~~~~~~~~~

\datation{environ~\anorange{1070}{945}}

\begin{listerois}
  \item [Smendès~I\ier] \num{26}~ans de règne.
  \item [Amenménisou / Nephercheres] \num{4}~ans en corégence avec 
        Psousennès I\ier.
  \item [Psousennès~I\ier] \num{49}~ans de règne, en incluant les 
        \num{4}~ans de corégence. Tombe à Tanis.
  \item [Amenemopé \datation{(c.~\anorange{993}{984})}] 
        Environ \num{10}~ans de règne. 
        Tombe à Tanis.
  \item [Osorkon / Osochor] Plus ou moins \num{6}~ans de règne.
  \item [Siamon] \num{17}~ans de règne.
  \item [Psousennès~II] (c.~\anorange{959}{945}) \num{13}~ans de 
        règne. Tombe à Tanis.
\end{listerois}

\subsection[Dynastie des GPA de Thèbes]{Dynastie parallèle des Grands 
            Prêtres d'Amon (GPA) de Thèbes}
%~~~~~~~~~~~~~~~~~~~~~~~~~~~~~~~~~~~~~~~~~~~~~~~~~~~~~~~~~~~~~~~~~~~~~

L'ordre de succession de Piânkh et Hérihor est totalement incertain.

\begin{listerois}
  \item [Hérihor]
  \item [Piânkh]
  \item [Pinedjem~I\ier \datation{(c.~\anorange{1070}{1032})}] 
        GPA pendant \num{16}~ans puis roi \num{24}~ans. Ses fils sont 
        alors GPA : Masaharta, Djedkhonsouiouefânkh, puis Menkheperrê.
        Momie dans la cachette de \DeB DB\,320, à Thèbes.
  \item [Menkhéperrê \datation{(c.~\anorange{1045}{992})}] 
        GPA presque \num{50}~ans.
        Momie dans la cachette de \DeB DB\,320, à Thèbes.
  \item [Pinedjem~II] GPA presque \num{30}~ans.
  \item [Psousennès~III] GPA un peu plus de \num{20}~ans.

\end{listerois}

\subsection[\texorpdfstring{\dynlist{22}{23}}
                           {XXIIe et XXIIIe dynasties}]
           {\dyn{22} (libyenne ou boubastite, fondée en \ano{945}, 
            \XIIA) et \dyn{23} \frquote{thébaine} (\XIIB{} et \XIIC)}
%~~~~~~~~~~~~~~~~~~~~~~~~~~~~~~~~~~~~~~~~~~~~~~~~~~~~~~~~~~~~~~~~~~~~~

La \dyn{23} est parfois appelée \dyn{22} \frquote{thébaine}.

Au cours du règne d’Osorkon~II, Harsiésis puis Takélot~II revendiquent 
la royauté à Thèbes. Durant le règne de Takélot~II, une autre figure 
royale apparaît à Thèbes, Pétoubastis~I\ier. 

\nouveaute{Fin \dyn{22}, premières momies de chat dédiées 
à une divinité.}

\begin{listerois}
  \item [\XIIA{} Chéchonq~I\ier] \num{21}~ans de règne.
  \item [\XIIA{} Osorkon~I\ier] \num{35}~ans de règne.
  \item [\XIIA{} Takélot~I\ier] \num{15}~ans de règne.
  \item [\XIIA{} Osorkon~II] Très long règne, entre \num{23} et 
        \num{44}~ans. Tombe à Tanis.
  \item [\XIIB{} Harsiésis] Thèbes. Règne entre l'an~\num{4} et 
        l'an~\num{16} d'Osorkon~II. Tombe à \MH.
  \item [\XIIB{} Takélot~II] Thèbes. \num{25}~ans de règne. 
        Contemporain de Chéchonq~III et de Pétoubastis~I\ier.
  \item [\XIIC{} Pédoubastis~I\ier] \num{23}~ans de règne. 
        En parallèle avec \XIIB Takélot~II et \XIIA Chéchonq~III.
  \item [\XIIC{} Ioupout~I\ier] \num{12}~ans de règne. Contemporain de 
        Pédoubastis~I\ier.
  \item [\XIIA{} Chéchonq~III] \num{39}~ans de règne.
  \item [\XIIB{} Osorkon~III] \num{28}~ans de règne, dont cinq en 
        corégence avec son fils Takélot~III. 
        Contemporain de Chéchonq~IV, Pami et Chéchonq~V.
  \item [\XIIA{} Chéchonq~IV] \num{14}~ans de règne.
  \item [\XIIA{} Pami] \num{7}~ans de règne.
  \item [\XIIA{} Chéchonq~V] \num{38}~ans de règne.
  \item [\XIIB{} Takélot~III] \num{14}~ans de règne ? Les cinq 
        premiers en corégence avec son père. 
        Contemporain de Chéchonq~V.
  \item [\XIIB{} Roudamon] Règne bref.
\end{listerois}

En \ano{753}, fondation de Rome.

\subsubsection{Les rois contemporains de la campagne militaire 
               de Piânkhy :}
%......................................................................
\begin{listerois}
  \item [Osorkon~IV] Tanis-Bubastis. \num{9}~ans de règne.
  \item [Ioupout~II] Léontopolis. \num{23}~ans de règne.
  \item [Peftjaouaouybastet] Hérakléopolis. \num{10}~ans de règne.
  \item [Nimlot D] Hermopolis.
\end{listerois}

\subsection{\texorpdfstring{\dyn{24}}{XXIVe dynastie} (Saïs)}
%~~~~~~~~~~~~~~~~~~~~~~~~~~~~~~~~~~~~~~~~~~~~~~~~~~~~~~~~~~~~~~~~~~~~~

\begin{listerois}
  \item [Tefnakht \datation{(c.~728/7-720)}]
  \item [Bocchoris / Bakenrenef \datation{(c.~\anorange{720}{715})}]
        Règne au moins \num{6}~ans. Couronné après la campagne de 
        Piânkhy, il est éliminé par Chabaka.
\end{listerois}

\subsection{Pré-\texorpdfstring{\dyn{25}}{XXVe dynastie} (Napata)}
%~~~~~~~~~~~~~~~~~~~~~~~~~~~~~~~~~~~~~~~~~~~~~~~~~~~~~~~~~~~~~~~~~~~~~

\begin{listerois}
  \item [Alara \datation{(c.~\anorange{785}{760})}]
  \item [Kachta \datation{(c.~\anorange{760}{750})}] 
        Pyramide à El-Kourrou~Ku\,8, au Soudan.
  \item [Piânkhy \datation{(c.~\anorange{750}{716})}] 
        Pyramide à El-Kourrou~Ku\,17, au Soudan. Invasion de 
        l'ensemble de l'\kmt, début de la \dyn{25} égyptienne.
\end{listerois}

\subsection{\texorpdfstring{\dyn{25}}{XXVe dynastie}}
%~~~~~~~~~~~~~~~~~~~~~~~~~~~~~~~~~~~~~~~~~~~~~~~~~~~~~~~~~~~~~~~~~~~~~

\begin{listerois}
  \item [Chabaka \datation{(c.~\anorange{716}{702})}] 
        Pyramide à El-Kourrou~Ku\,15, au Soudan.
  \item [Chabataka \datation{(c.~\anorange{702}{690})}] 
        Pyramide à El-Kourrou~Ku\,18, au Soudan.
  \item [Taharqa \datation{(c.~\anorange{690}{664})}] 
        Pyramide à Nouri~\num{1}, au Soudan.
  \item [Tanoutamen \datation{(c.~\anorange{664}{656})}] 
        Pyramide à El-Kourrou~Ku\,16, au Soudan.
\end{listerois}

En \ano[0]{664}/\ano{3}, sac de Thèbes par les Assyriens 
d'Assourbanipal.

À cause des Néo-assyriens, Tanoutamen quitte l'\kmt. 
Ses successeurs dirigent le royaume napatéen en Nubie et au Soudan 
jusqu'en \ano{400} environ.
Ils se nomment Atlanersa, Senkamanisken, Anlamani, Aspelta, 
Aramatelqo, Irike-Amanote\dots

\section{\LP}
%---------------------------------------------------------------------

Désormais, la chronologie égyptienne, tant relative qu'absolue, 
devient assez sûre. Nous pouvons analyser l'enchaînement des 
événements dans toute la région : l'ensemble du Moyen-Orient et 
la Méditerranée.

\subsection{\texorpdfstring{\dyn{26}}{XXVIe dynastie}}
%~~~~~~~~~~~~~~~~~~~~~~~~~~~~~~~~~~~~~~~~~~~~~~~~~~~~~~~~~~~~~~~~~~~~~

\datation{c.~\anorange{663}{525}}

Les trois premiers sont contemporains des rois éthiopiens : dynastie 
\frquote{proto-saïte} (Saïs).

\begin{listerois}
  \item [Stéphinatès / Tefnakht] \num{8}~ans (Manéthon).
  \item [Néchepso] \num{7}~ans (Manéthon).
  \item [Néchao~I\ier] \num{9}~ans (Manéthon).
  \item [Psammétique~I\ier \datation{(\anorange{663}{609})}] 
        Réunification du pays en l'an~\num{9}. Avant cette date, 
        Tanoutamon est reconnu dans le sud du pays.
  \item [Nékao~II \datation{(\anorange{609}{594})}]
        Royaume néo-babylonien : règne de Nabopolassar jusqu'en 605, 
        puis de Nabuchodonosor.
  \item [Psammétique~II \datation{(\anorange{594}{588})}] 
        Campagne contre Napata en 591.
  \item [Apriès \datation{(\anorange{588}{568})}]
  \item [Amasis (Ahmosis~II) \datation{(\anorange{568}{526})}]
  \item [Psammétique~III \datation{(\anorange{526}{525})}]
\end{listerois}

\subsection{\texorpdfstring{\dyn{27}}{XXVIIe dynastie} (Première 
            Domination perse)}
%~~~~~~~~~~~~~~~~~~~~~~~~~~~~~~~~~~~~~~~~~~~~~~~~~~~~~~~~~~~~~~~~~~~~~

\datation{\anorange{525}{405}}

Cambyse conquiert l'\kmt et meurt sur le chemin du retour en~\ano{522} 
(tombe à Takht-i-Rustam, en Iran).

\begin{listerois}
  \item [Darius~I\ier \datation{(\anorange{522}{485})}] 
        Tombe à Naqch-i-Rustam, en Iran. \\
        Première guerre médique. \\
        En \ano[0]{490} bataille de Marathon. \\
        Révolte en \kmt du satrape Aryandès.
  \item [Xerxès~I\ier \datation{(\anorange{485}{464})}] 
        Tombe à Naqch-i-Rustam, en Iran. \\
        Deuxième guerre médique. \\
        En \ano[0]{480} bataille des Thermopyles et bataille de 
        Salamine. \\
        En \ano[0]{480} expédition carthaginoise contre Agrigente et 
        Syracuse. \\
        En \ano[0]{479} batailles de Platées et du cap Mycale. \\
        Révolte en \kmt contre les Perses.
  \item [Artaxerxès~I\ier \datation{(\anorange{464}{424})}]  
        Révolte contre les Perses.
  \item [Darius~II \datation{(\anorange{424}{404})}] 
\end{listerois}

Pas d'attestation {\hiero}ique de Xerxès~II, de Darius~II, ni 
d'Artaxerxès~II en \kmt.

\subsection{Rois égyptiens rebelles}
%~~~~~~~~~~~~~~~~~~~~~~~~~~~~~~~~~~~~~~~~~~~~~~~~~~~~~~~~~~~~~~~~~~~~~

\begin{listerois}
  \item [Pedoubastis~III]
  \item [Amyrtée \datation{(\anorange{404}{398})}] \dyn{29} \\
        Contemporain d'Artaxerxès~II.
\end{listerois}

Au Soudan, dynastie méroïtique de \ano{400} jusqu'en \ano[1]{400} 
(Harsiotef, Akhratan, Nastasen, Arkamani~I\ier, Amanichakheto, 
Natakamani, Amanitore).

\subsection{\texorpdfstring{\dyn{29}}{XXIXe dynastie}}
%~~~~~~~~~~~~~~~~~~~~~~~~~~~~~~~~~~~~~~~~~~~~~~~~~~~~~~~~~~~~~~~~~~~~~

\begin{listerois}
  \item [Nepheritès~I\ier \datation{(\anorange{398}{392})}] 
        Tombe à Mendès.
  \item [Psamouthis]
  \item [Achoris \datation{(\anorange{391}{379})}] Tombe à Mendès.
  \item [Nepheritès~II]
\end{listerois}

\subsection{\texorpdfstring{\dyn{30}}{XXXe dynastie}}
%~~~~~~~~~~~~~~~~~~~~~~~~~~~~~~~~~~~~~~~~~~~~~~~~~~~~~~~~~~~~~~~~~~~~~

\datation{\anorange{378}{341}}

\begin{listerois}
  \item [Nactenabo~I\ier \datation{(\anorange{378}{360})}]
  \item [Téos \datation{(\anorange{361}{359})}]
  \item [Nactenabo~II \datation{(\anorange{359}{341})}]
\end{listerois}

\subsection{\texorpdfstring{\dyn{31}}{XXXIe dynastie} (seconde 
domination perse)}
%~~~~~~~~~~~~~~~~~~~~~~~~~~~~~~~~~~~~~~~~~~~~~~~~~~~~~~~~~~~~~~~~~~~~~

\begin{listerois}
  \item [Artaxerxès~III (Ochos) \datation{(\anorange{358}{338})}] 
        Tombeau au Kuh-e-Rahmat en Iran. \\ 
        Contemporain de Nactenabo~II, il essaie de reprendre l'\kmt.
        Il réussit en \ano[0]{343}/\ano{342} à envahir le pays et 
        contenir Nactenabo en \HE.
  \item [Arsès] \datation{(\anorange{338}{335})}
  \item [Darius~III Codoman \datation{(\anorange{335}{330})}] 
        Défaite perse d'Issos en \ano{333}.
  \item [Khabbach] Roi indigène rebelle.
\end{listerois}

\subsection{Invasion d'Alexandre le Grand (\ano{332})}
%~~~~~~~~~~~~~~~~~~~~~~~~~~~~~~~~~~~~~~~~~~~~~~~~~~~~~~~~~~~~~~~~~~~~~

\begin{listerois}
  \item [Philippe Arrhidée]
  \item [Alexandre~II (IV)]
  \item [Ptolémée fils de Lagos] Satrape d'\kmt en \ano{323}.
\end{listerois}

\subsection{Dynastie ptolémaïque / lagide}
%~~~~~~~~~~~~~~~~~~~~~~~~~~~~~~~~~~~~~~~~~~~~~~~~~~~~~~~~~~~~~~~~~~~~~

\begin{listerois}
  \item [Ptolémée~I\ier Sote 
         \datation{(\ano[0]{305}/\ano[0]{4}-\ano{282})}]]
  \item [Ptolémée~II Philadelphe 
         \datation{(\ano[0]{285}/\ano[0]{4}-\ano{246})}]
  \item [Arsinoé~II]
  \item [Ptolémée~III Evergète 
         \datation{(\ano{246}-\ano{222}/\ano{1})}]
  \item [Bérénice~II]
  \item [Ptolémée~IV Philopator \datation{(\anorange{221}{204})}]
  \item [Rois rebelles en \HE{} 
         \datation{(\ano[0]{206}/\ano[0]{5}-\ano{186})}]~\\
        \begin{itemize}
          \item Horounnéfer ;
          \item Khâounnefer, vaincu en~\ano{186} par Ptolémée~V.
        \end{itemize}
  \item [Ptolémée~V Épiphane \datation{(\anorange{204}{180})}] 
        Sosibios et Agathocle assument la tutelle du jeune roi.
  \item [Cléopâtre~I\iere{} \datation{(morte en \ano{176})}]
  \item [Ptolémée~VI Philométor \datation{(\anorange{180}{145})}]
  \item [Ptolémée~VII Néos Philopator \datation{(mort en \ano{130})}]
  \item [Ptolémée~VIII Evergète~II Tryphon 
         \datation{(\anorange[0]{170}{163} et \anorange{145}{116})}]
  \item [Cléopâtre~II \datation{(morte en \ano{115})}]
  \item [Cléopâtre~III \datation{(morte en \ano{101})}]
  \item [Ptolémée~IX Philométor Soter 
        \datation{(\anorange[0]{116}{107} et \anorange{88}{80})}]
  \item [Ptolémée~X Alexandre~I\ier{} \datation{(\anorange{107}{88})}]
  \item [Bérénice~III \datation{(\ano{80})}]
  \item [Ptolémée~XI Alexandre~II \datation{(\ano{80})}]
  \item [Ptolémée~XII Néos Dionysos \datation{(\anorange{80}{51})}]
  \item [Bérénice~IV \datation{(\anorange{58}{55})}]
  \item [Cléopâtre~VII Philopator \datation{(\anorange{51}{30})}]
  \item [Ptolémée~XIII \datation{(\anorange{51}{47})}]
  \item [Ptolémée~XIV Philadelphe \datation{(\anorange{47}{44})}]
  \item [Ptolémée~XV Césarion \datation{(\anorange{44}{30})}]
\end{listerois}

\ano{31}, les troupes égypto-romaines de Cléopâtre et Antoine sont 
vaincues à la bataille d'Actium.

\section{L'\kmt, province romaine}
%---------------------------------------------------------------------

\begin{itemize}\setlength\itemsep{0.8\baselineskip}
  \item \ano{30}, l'\kmt tombe aux mains d'Octave, qui devient, 
        en \ano{27}, le premier empereur romain sous le nom d'Auguste.
  \item \ano[1]{222} : édit de Caracalla
  \item \ano[1]{267} : Zénobie fille de Zabbai, reine de Palmyre à 
        la mort de son époux Odenat. Elle prend le titre d'Augusta en 
        \ano[1]{270} et envahit l'\kmt. Elle est vaincue par Aurélien.
  \item 29 août \ano[1]{284} : début du calendrier copte, le 
        \frquote{calendrier des martyrs} (répression de Dioclétien).
  \item avril \ano[1]{313} : édit de Milan (Constantin~I\ier{} 
        \datation{(\anorange[1]{306}{337})} et Licinus de Nicomédie) 
        Liberté de culte et autorisation aux Chrétiens de ne pas 
        vénérer l'empereur.
  \item Théodose~I\ier{} \datation{(\anorange[1]{346}{395})}.
  \item \ano[1]{380} : édit établissant la religion chrétienne comme 
        religion officielle de l'empire.
  \item \ano{391}-\ano[1]{393} : édit de Théodose imposant la 
        fermeture des temples païens et interdisant les cultes 
        correspondants et dans l'Empire romain.
  \item \ano[1]{641} : invasion de l'\kmt par les armées musulmanes.
  \item \ano[1]{642} : fondation de Fostat.
  \item \ano[1]{751} : avènement de la dynastie Abbasside. 
        Interdiction progressive d'utiliser le copte dans les 
        documents officiels.

  \item \ano[1]{909} : Fatimides
  \item \ano[1]{969} : envahissent l'\kmt
  \item \ano[1]{1171} : Saladdin vizir
  \item \ano[1]{1173} : Saladdin sultan
\end{itemize}

% 393\,AD : édit de Théodose interdisant les cultes païens.


\backmatter
\newpage
\listoffigures
% \listoftables

\nocite{*}
\printbibliography[heading=memoir,title=Bibliographie]

%%%%%%%%%%%%%%%%%%%%%%%%%%%%%%%%%%%%%%%%%%%%%%%%%%%%%%%%%%%%%%%%%%%%%%%
\end{document}
