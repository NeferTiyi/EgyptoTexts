\documentclass[dvipsnames,a4paper,twoside,10pt,openany,article]{memoir}
\usepackage{pgf,tikz}
\usetikzlibrary{%
  calc,
  patterns,
  arrows,
  positioning,
  shadows,
  decorations.text
}
\usepackage{multicol}
\usepackage{ragged2e}
\usepackage{pict2e}
\usepackage{supertabular}
\usepackage{nefertiyi}
\usepackage{neferhiero}

% Numérotation des division et prise en compte dans la TOC
\maxsecnumdepth{subsection}
\maxtocdepth{subsection}

% \counterwithin{chapter}{part}
% \counterwithin{subsection}{section}
% % \counterwithout{figure}{part}
% \counterwithout{figure}{chapter}


% Apparence des numéros de divisions
\renewcommand{\thechapter}{\arabic{chapter}}
\renewcommand{\thesubsection}{\alph{subsection})}

% Indentation et largeur des numéros dans la TOC
\cftsetindents{part}{\cftpartindent}{2.0em}
\cftsetindents{chapter}{\cftchapterindent}{1.8em}
\cftsetindents{section}{\cftsectionindent}{2.7em}
\cftsetindents{subsection}{\cftsubsectionindent}{1.4em}
\cftsetindents{figure}{\cftfigureindent}{2.6em}
% Pointillés dans la TOC
\renewcommand{\cftsectiondotsep}{9}
\renewcommand{\cftsubsectiondotsep}{1}



% Entry         Level        Standard        memoir class
%                         indent numwidth   indent numwidth 
% ---------------------------------------------------------
% book             -2       —      —          0.0    —
% part             -1       0.0    —          0.0    1.5 
% chapter           0       0.0    1.5        0.0    1.5
% section           1       1.5    2.3        1.5    2.3
% subsection        2       3.8    3.2        3.8    3.2
% subsubsection     3       7.0    4.1        7.0    4.1
% paragraph         4      10.0    5.0       10.0    5.0
% subparagraph      5      12.0    6.0       12.0    6.0
% figure/table     (1)      1.5    2.3        0.0    1.5
% subfigure/table  (2)      —      —          1.5    2.3


% Macro               Default           Usage
% ---------------------------------------------------------------------
% \abstractname       Abstract          title for abstract environment
% \alsoname           see also          used by \seealso
% \amname             am                used in printing time of day
% \appendixname       Appendix          name for an appendix heading
% \appendixpagename   Appendices        name for an \appendixpage
% \appendixtocname    Appendices        ToC entry announcing appendices
% \bibname            Bibliography      title for \thebibliography
% \bookname           Book              name for \book heading
% \bookrefname        Book              used by \Bref
% \chaptername        Chapter           name for \chapter heading
% \chapterrefname     Chapter           used by \Cref
% \contentsname       Contents          title for \tableofcontents
% \figurename         Figure            name for figure \caption
% \figurerefname      Figure            used by \fref
% \glossaryname       Glossary          title for \theglossary
% \indexname          Index             title for \theindex
% \lcminusname        minus             used in named number formatting
% \listfigurename     List of Figures   title for \listoffigugres
% \listtablename      List of Tables    title for \listoftables
% \minusname          minus             used in named number formatting
% \namenumberand      and               used in named number formatting
% \namenumbercomma    ,                 used in named number formatting
% \notesname          Notes             title of \notedivision
% \pagename           page              for your use
% \pagerefname        page              used by \pref
% \partname           Part              name for \part heading
% \partrefname        Part              used by \Pref
% \pmnane             pm                used in printing time of day
% \sectionrefname     §                 used by \Sref
% \seename            see               used by \see
% \tablename          Table             name for table \caption
% \tablerefname       Table             used by \tref
% \ucminusname        Minus             used in named number formatting
% ---------------------------------------------------------------------
% \renewcommand{\partname}{Part}
% \renewcommand{\partrefname}{Part~}
% ---------------------------------------------------------------------
% \newcommand*{\nNamexi}{\iflowertonumname e\else E\fi leven}
\renewcommand{\chapterrefname}{Chapitre~}


\setpnumwidth{2.55em}
\setrmarg{3.55em}

\addtolength{\columnsep}{15pt}

\settypeblocksize{22cm}{15cm}{*}
\setulmargins{*}{*}{1}
\setlrmargins{*}{*}{1.3}
% \setmarginnotes{17pt}{51pt}{\onelineskip}
\setheadfoot{\onelineskip}{5\onelineskip}
\setheaderspaces{*}{2\onelineskip}{*}
\checkandfixthelayout


\newcommand{\separation}{%
  {\noi\hspace*{\fill}\rule{.33\textwidth}{1pt}\hspace*{\fill}}%
}

\newcommand{\DirUtils}{../../utils}
\newcommand{\DirImage}{../../images}

\graphicspath{%
  {\DirUtils/}%
  {\DirImage/Barbotin/}%
}

\setfloatadjustment{figure}{\centerfloat}
\setfloatadjustment{table}{\centerfloat}

\newsubfloat{figure}

% \subcaptionstyle{}
\subcaptionfont{\sffamily}
\subcaptionlabelfont{\bfseries\sffamily}

\captionnamefont{\bfseries\sffamily}
\captiontitlefont{\sffamily}
\captiondelim{~-- }

\newcommand{\CaptionNormal}{%
  \captionwidth{0.75\linewidth}
  \changecaptionwidth
  \hangcaption
}
\newcommand{\CaptionPetit}{%
  \normalcaptionwidth
  \normalcaption
}


\setlength\fboxsep{0.5mm}
%\setlength\tabcolsep{0mm}
%\setlength\parskip{1.0\baselineskip}


\makeatletter
  \renewcommand\@memfront@floats{%
    \counterwithout{figure}{chapter}
    \counterwithout{table}{chapter}
  }
  \renewcommand\@memmain@floats{%
    \counterwithout{figure}{chapter}
    \counterwithout{table}{chapter}
  }
  \renewcommand\@memback@floats{%
    \counterwithout{figure}{chapter}
    \counterwithout{table}{chapter}
    \setcounter{figure}{0}
    \setcounter{table}{0}
  }
\makeatother


%======================================================================
\title{La statuaire du\\\NK\\d'après les\\collections du Louvre}
\short{La statuaire du \NK au Louvre}
\subtitle{}
\lecturer{Christophe~\bsc{Barbotin}}
\orga{Organisé par \textsf{la \SFE} avec}
\author{Sonia~\bsc{Labetoulle}}
\date{vendredi 3 février 2012}

\hypersetup{%
  pdftitle  = {La statuaire du \NK d'après les collections du Louvre}, 
  pdfauthor = {Sonia Labetoulle}
}
% pdfsubject % pdfcreator % pdfproducer % pdfkeywords

%======================================================================

\input{\DirUtils/ChapterStyles}
\input{\DirUtils/TitlepageKheops}
\input{\DirUtils/HeadingsKheops}

% \chapterstyle{monchap}
\pagestyle{ruledarticle}
% \pagestyle{myheadings}

\addbibresource{\DirUtils/Hiero.bib}

%%%%%%%%%%%%%%%%%%%%%%%%%%%%%%%%%%%%%%%%%%%%%%%%%%%%%%%%%%%%%%%%%%%%%%%
\begin{document}

\makeatletter
\def\@Fpt{%
{
  \ifcase\value{part}%
    \or Premi{\FBegrave}re%
    \or Deuxi{\FBegrave}me%
    \or Troisi{\FBegrave}me%
    \or Quatri{\FBegrave}me%
    \or Cinqui{\FBegrave}me%
    \or Sixi{\FBegrave}me%
    \or Septi{\FBegrave}me%
    \or Huiti{\FBegrave}me%
    \or Neuvi{\FBegrave}me%
    \or Dixi{\FBegrave}me%
    \or Onzi{\FBegrave}me%
    \or Douzi{\FBegrave}me%
    \or Treizi{\FBegrave}me%
    \or Quatorzi{\FBegrave}me%
    \or Quinzi{\FBegrave}me%
    \or Seizi{\FBegrave}me%
    \or Dix-septi{\FBegrave}me%
    \or Dix-huiti{\FBegrave}me%
    \or Dix-neuvi{\FBegrave}me%
    \or Vingti{\FBegrave}me%
    \or Vingt-et-uni{\FBegrave}me%
    \or Vingt-Deuxi{\FBegrave}me%
    \or Vingt-Troisi{\FBegrave}me%
    \or Vingt-Quatri{\FBegrave}me%
    \or Vingt-Cinqui{\FBegrave}me%
    \or Vingt-Sixi{\FBegrave}me%
    \or Vingt-Septi{\FBegrave}me%
    \or Vingt-Huiti{\FBegrave}me%
    \or Vingt-Neuvi{\FBegrave}me%
    \or Trenti{\FBegrave}me%
    \or Trente-et-uni{\FBegrave}me%
    \or Trente-Deuxi{\FBegrave}me%
  \fi%
}\space\def\thepart{}}%
\makeatother


\thispagestyle{empty}
\maketitle

%%%%%%%%%%%%%%%%%%%%%%%%%%%%%%%%%%%%%%%%%%%%%%%%%%%%%%%%%%%%%%%%%%%%%%%

%##HieroDef.txt

%%%%%%%%%%%%%%%%%%%%%%%%%%%%%%%%%%%%%%%%%%%%%%%%%%%%%%%%%%%%%%%%%%%%%%%

% \vspace{\fill}

\frontmatter
\tableofcontents

% \newpage

\mainmatter

\begin{encadre}
  Visite exceptionnelle organisée par la \SFE, conduite par 
  M.~\bsc{CB}, conservateur en chef au département des antiquités 
  égyptiennes du musée du Louvre.

  \emph{\og Le thème proposé, celui de la statuaire du \NK d'après 
  les collections du Louvre, sera peut-être l'occasion de réévaluer 
  certains chefs-d'\oe uvre et, inversement, de mettre en lumière 
  d'autres pièces moins connues. \fg}

  Rendez-vous à 18h45, ce vendredi 3~février 2012, avec douze autres 
  chanceux, à l'accueil des groupes sous la pyramide. Pour 1h30 d'une 
  visite qui durera finalement près de 2h30\dots \textbf{*\_*}
\end{encadre}


\chapter{Le prince Iâhmes (\inv{E\,15682})}
\label{sec:E15682}
%======================================================================

\puceb{} \emph{voir \fref{fig:E15682}, \pref{fig:E15682} et 
\fref{fig:E15682_det}, \pref{fig:E15682_det}}
\bigskip

\begin{figure}[!h]
  \includegraphics[width=4cm]{E15682_1}
  \caption{Le prince Iâhmes \rem{(\inv{E\,15682})}}
  \label{fig:E15682}
\end{figure}

\begin{figure}
  \noi\begin{minipage}[b]{0.33\textwidth}%
    \centerfloat
    \includegraphics[width=4cm]{E15682_2}
    \subcaption{Pieds\label{fig:E15682_pieds}}
  \end{minipage}%
  \hfill%
  \begin{minipage}[b]{0.33\textwidth}%
    \centerfloat
    \includegraphics[width=4cm]{E15682_3}
    \subcaption{Jambe droite\label{fig:E15682_jambe}}%
  \end{minipage}%
  \hfill%
  \begin{minipage}[b]{0.33\textwidth}%
    \centerfloat
    \includegraphics[width=4cm]{E15682_4}
    \subcaption{Arrière de la tête\label{fig:E15682_tete}}%
  \end{minipage}%
  \caption{Le prince Iâhmes \rem{(\inv{E\,15682})} : détails}%
  \label{fig:E15682_det}%
\end{figure}


\begin{itemize}
  \item \datation{vers \ano{1550}} (fin \dyn{xvii})
  \item calcaire autrefois peint, incrusté, et doré
  \item dimensions : \SI{1.03x0.36x0.58}{\m}
  \item \emph{Don Gilly}
\end{itemize}

Il s'agit peut-être du futur roi Ahmosis.

\separation

\begin{itemize}
  \item traces de dorures (élément de divinisation, comme les cheveux 
        bleus) !
  \item bras et jambes cassés mais nez intact \donc~cassures 
        volontaires ?
  \item inscriptions : noms, mais \emph{très} courants\dots
  \item tous les textes sont écrits de droite à gauche, pas de symétrie
  \item deux coudées de haut
\end{itemize}

\quad\donc~Comment interpréter cette statue~\nospace{???}

Un beau jour Christophe~\bsc{Barbotin} a remarqué un trou avec de 
la peinture rouge derrière le mollet droit, puis d'autres à l'épaule, 
\dots Ces trous sont antérieurs aux cassures !?\\
\quad\donc~magie ? fixation magique de la statue 
(cf. mutilation des hiéros) ?

inscriptions latérales = lettres au mort

Il existe un relief montrant une reine adorant une statue de ce type, 
avec un ornement (diadème) attaché à l'arrière de la tête.
Notre statue comporte justement un trou à l'arrière de la tête !

Ce serait la statue originale ayant donné naissance à ce culte. Un 
prince mort très jeune ?


\chapter{Homme tenant une fleur (\inv{N\,5404})}
\label{sec:N5404}
%======================================================================

\puceb{} \emph{voir \fref{fig:N5404}, \pref{fig:N5404}}
\bigskip

\begin{figure}[!h]
  \includegraphics[width=4cm]{N5404_1}
  \quad
  \includegraphics[width=4cm]{N5404_2}
  \caption{Homme tenant une fleur \rem{(\inv{N\,5404})}}
  \label{fig:N5404}
\end{figure}

\begin{itemize}
  \item diorite
  \item dimensions : \SI{56.00x31.50}{\cm}
\end{itemize}

\separation

\begin{itemize}
  \item diorite \donc~personnage de très haut rang
  \item traces d'inscriptions effacées
  \item le personnage tient une fleur : c'est nouveau !
\end{itemize}
\quad\donc~Sénenmout ? Car c'est un habitué des nouveautés (premières 
statues naophore, sistrophore,~\dots)


\chapter{Sénenmout, conseiller de la reine Hatchepsout et 
         \og architecte \fg de son temple à \DeB (\inv{E\,11057})}
\label{sec:E11057}
%======================================================================

\puceb{} \emph{voir \fref{fig:E11057}, \pref{fig:E11057}}
\bigskip

\begin{figure}
  \noi\begin{minipage}[b]{0.33\textwidth}%
    \centerfloat
    \includegraphics[height=6cm]{E11057_1}
    \subcaption{Vue d'ensemble\label{fig:E11057_vue}}
  \end{minipage}%
  \hfill%
  \begin{minipage}[b]{0.33\textwidth}%
    \centerfloat
    \includegraphics[height=6cm]{E11057_2}
    \subcaption{Partie supérieure\label{fig:E11057_haut}}%
  \end{minipage}%
  \hfill%
  \begin{minipage}[b]{0.33\textwidth}%
    \centerfloat
    \includegraphics[height=6cm]{E11057_3}
    \subcaption{Partie inférieure\label{fig:E11057_bas}}%
  \end{minipage}%
  \caption{Sénenmout \rem{(\inv{E\,11057})}}%
  \label{fig:E11057}%
\end{figure}

\begin{itemize}
  \item quartzite
  \item dimensions : \SI{20.70x8.20x11.50}{\cm}
\end{itemize}

Il présente une corde d'arpenteur.

\separation

\begin{itemize}
  \item nom presque effacé
  \item rouleau de corde d'arpenteur sur les genoux
  \item tête humaine sur le rouleau : pas top, remplace quelque chose, 
        peut-être une tête de bélier d'Amon
\end{itemize}


\chapter{Manakhtef, chef des approvisionnements du roi 
         (\inv{E\,12926})}
\label{sec:E12926}
%======================================================================

\puceb{} \emph{voir \fref{fig:E12926}, \pref{fig:E12926}}
\bigskip

\begin{figure}[!h]
  \noi\begin{minipage}[m]{0.33\textwidth}
    \centerfloat
    \includegraphics[width=4cm]{E12926_1}
    \subcaption{Vue de face\label{fig:E12926_1}}
  \end{minipage}%
  \quad%
  \begin{minipage}[m]{0.33\textwidth}
    \centerfloat
    \includegraphics[width=4cm]{E12926_2}
    \subcaption{Vue de dessus\label{fig:E12926_2}}
  \end{minipage}%
  \quad%
  \begin{minipage}[m]{0.33\textwidth}
    \centerfloat
    \includegraphics[width=4cm]{E12926_3}
    \subcaption{Vue de dos\label{fig:E12926_3}}
  \end{minipage}%

  \noi\begin{minipage}[m]{0.50\textwidth}
    \centerfloat
    \includegraphics[width=4cm]{E12926_4}
    \subcaption{Profil gauche\label{fig:E12926_4}}
  \end{minipage}%
  \hspace{\fill}%
  \begin{minipage}[m]{0.50\textwidth}
    \centerfloat
    \includegraphics[width=4cm]{E12926_5}
    \subcaption{Vue de trois-quarts\label{fig:E12926_5}}
  \end{minipage}%
  \caption{Manakhtef \rem{(\inv{E\,12926})}}
  \label{fig:E12926}
\end{figure}

\begin{itemize}
  \item règne d'Aménophis~II (\datation{\anorange{1427}{1401}})
  \item trouvé à Médamoud
  \item diorite
  \item dimensions : \SI{50.00x23.00x29.50}{\cm}
\end{itemize}

\emph{L'échanson royal Manakhtef dit : \og Ô temple de Montou, 
conserve cette statue de l'échanson royal Manakhtef à l'intérieur 
de la cour de fête, pour qu'il respire l'odeur de la myrrhe et de 
l'encens, qu'il recueille sur le sol de la cour l'eau versée sur 
l'autel ; qu'il se nourrisse des restes des offrandes divines 
présentées par les prêtres, qu'il contemple le disque solaire au 
matin, dans la maison où l'on passe l'éternité, et qu'il accompagne 
son dieu quand il fait le tour de son temple lors de sa fête de la 
montagne sacrée\dots \fg}

\separation

\begin{itemize}
  \item intact, sauf le nez
  \item texte, avec date ! \textbf{*\_*}
  \item trouvé à Médamoud
  \item texte en trois parties :
  \begin{itemize}
    \item En haut : colonnes avec début de formule d'offrande 
          (\tl[black]{d n(y)-sw.t Htp})
    \item Puis deux lignes : suite des formules d'offrande
    \item Enfin, colonnes à nouveau, à l'aplomb des premières : 
          les titres du défunt
  \end{itemize}
\end{itemize}

\begin{description}
  \item[Côté droit :] mode d'emploi de la statue dans le temple. 
        Elle est le substitut du défunt dans le temple, son héritier, 
        elle participe au culte et prend des offrandes
  \item[Côté gauche :] chapitre~\num{60} ou~\num{61} du livre des 
        morts \donc~rappel à la fonction originale de la state cube :
        statue funéraire
\end{description}

\rem{(cf. catalogue de Mons, sur la hiérarchie des pierres)}

\begin{encadre}[font=\em, frametitle={Traduction 
                \autocite[doc.~\num{86}]{ChB}}]
  lgn{1} [L'échanson royal] Manakhtef déclare : 

  \og (ô) la grande cour (de justice) de Montou \lgn{2} [qui est face] 
  à son seigneur, daigne accorder que dure la statue que voici de 
  l'échanson royal \lgn{3} Manakhtef au sein de la grande salle des 
  cérémonies, qu'elle hume \lgn{4} la myrrhe et l'encens sur la 
  flamme, qu'elle verse l'eau lors de l'aspersion de l'autel sur 
  \lgn{5} le sol de la grande salle, qu'elle mange ce que l'on 
  présente \nospace{(?)} sur les mains des prêtres-purs au moment de 
  \lgn{6} la divine offrande, qu'elle contemple le soleil du matin 
  dans la demeure de \emph{Celui qui traverse l'éternité}, \lgn{7} 
  qu'elle accompagne son dieu lorsqu'il parcourt son domaine lors de 
  sa fête de cette fameuse butte isolée\footnotemark{}, à l'instar de 
  ce que j'accomplissais (moi-même) lorsque j'étais sur terre ! \fg
\end{encadre}
\footnotetext{lieu mythique de la création du monde}


\chapter{Fragment de statue du scribe royal Méniou (\inv{E\,11519})}
\label{sec:E11519}
%======================================================================

\puceb{} \emph{voir \fref{fig:E11519}, \pref{fig:E11519}}
\bigskip

\begin{figure}[!h]
  \noi\begin{minipage}[m]{0.50\textwidth}
    \centerfloat
    \includegraphics[width=4cm]{E11519_1}
    \subcaption{Vue de trois-quarts\label{fig:E11519_1}}
  \end{minipage}%
  \hspace{\fill}%
  \begin{minipage}[m]{0.50\textwidth}
    \centerfloat
    \includegraphics[width=4cm]{E11519_2}
    \subcaption{Vue de dos\label{fig:E11519_2}}
  \end{minipage}%

  \noi\begin{minipage}[m]{0.50\textwidth}
    \centerfloat
    \includegraphics[width=4cm]{E11519_3}
    \subcaption{Profil droit\label{fig:E11519_3}}
  \end{minipage}%
  \hspace{\fill}%
  \begin{minipage}[m]{0.50\textwidth}
    \centerfloat
    \includegraphics[width=4cm]{E11519_4}
    \subcaption{Profil gauche\label{fig:E11519_4}}
  \end{minipage}%
  \caption{Le scribe royal Méniou \rem{(\inv{E\,11519})}}
  \label{fig:E11519}
\end{figure}

\begin{itemize}
  \item calcaire peint
  \item dimensions : \SI{45}{\cm}
  \item \emph{Don \bsc{Peytel}}
\end{itemize}

\separation

Complète à son entrée au Louvre, en \ano[0]{1915} 
(\ano[0]{1905} ?)~\dots mais aux $\slashfrac{3}{4}$~fausse ! 
\donc~dérestauration

\begin{itemize}
  \item typique Amenophis~III : yeux en amande, perruque à frisons
  \item deux colliers : collier \tl[black]{wsx}, + collier à deux 
        rangs de perles jaunes au ras du cou
  \item bras : plis d'une tunique, galons festonnés d'un manteau
\end{itemize}

\begin{itemize}
  \item manteau \donc~personnage surement assis
  \item le premier collier = or de la récompense \donc~personnage 
        important
\end{itemize}

Méniou était à la tête du trésor et de l'armée

Il est longtemps resté sans identité : \tl[black]{mnjw} = berger, 
mais ça ne colle pas avec ses autres titres prestigieux\dots 
En fait, c'est son nom !

Pas d'autres documents à son nom, mais il est probablement le 
\emph{Mané} cité dans la correspondance amarnienne en akkadien, 
ambassadeur (au \Mtn ?), chargé de l'organisation des mariages 
diplomatiques.


\chapter{Fragment d'un groupe royal d'époque amarnienne (\inv{N\,831})}
\label{sec:N831}
%======================================================================

\puceb{} \emph{voir \fref{fig:N831}, \pref{fig:N831}}
\bigskip

\begin{figure}[!h]
  \includegraphics[width=4cm]{N831_1}%
  \quad%
  \includegraphics[width=4cm]{N831_2}%
  \caption{Groupe royal d'époque amarnien \rem{(\inv{N\,831})}}
  \label{fig:N831}
\end{figure}

\begin{itemize}
  \item pierre
  \item dimensions : \SI{64.00x17.20x35.00}{\cm}
\end{itemize}

De la reine autrefois assise à côté du roi, il ne subsiste que le bras 
dans son dos ; les pieds et le siège sont refaits.

\separation

Pas de provenance connue, mais menton, ventre, poitrine,~\dots{} 
typiques de la période amarnienne.

Contrairement à ce qu'on entend fréquemment, il n'y pas 
d'adoucissement progressif du style au cours du règne d'Akhénaton. 
L'aspect plus ou moins marqué n'est pas un critère de datation, mais 
est lié à des différences fonctionnelles.

Tout le bas est faux ! Mais la statue est trop connue pour envisager 
une dérestauration.

La pierre jaune indique probablement que la scène se situe en plein 
air, sous le soleil.

Du personnage présent à l'origine à droite du roi (certainement la 
reine), il ne reste qu'un bras dans le dos, et un doigt -- indice 
d'une main -- sous le coude.

Cette pose symbolise un lien intime entre les deux personnages.

\begin{remarque}
  cf.~scène de ce type sur une talatate d'Amarna publiée par 
  Claude~\bsc{Traunecker}.
\end{remarque}

\chapter{Corps de femme, sans doute Néfertiti (\inv{E\,25409})}
\label{sec:E25409}
%======================================================================

\puceb{} \emph{voir \fref{fig:E25409}, \pref{fig:E25409}}
\bigskip

\begin{figure}[!h]
  \noi\begin{minipage}[m]{0.33\textwidth}
    \centerfloat
    \includegraphics[height=6cm]{E25409_1}
    \subcaption{Vue de face\label{fig:E25409_1}}
  \end{minipage}%
  \hspace{\fill}%
  \begin{minipage}[m]{0.33\textwidth}
    \centerfloat
    % \includegraphics[trim=2cm 1cm 1cm 0, clip, width=4cm]{E25409_2}
    \includegraphics[width=4cm]{E25409_2}
    \subcaption{Détail de la restauration antique\label{fig:E25409_2}}
  \end{minipage}%
  \hspace{\fill}%
  \begin{minipage}[m]{0.33\textwidth}
    \centerfloat
    \includegraphics[height=6cm]{E25409_3}
    \subcaption{Vue de trois quarts\label{fig:E25409_3}}
  \end{minipage}%
  \caption{Néfertiti \rem{(\inv{E\,25409})}}
  \label{fig:E25409}
\end{figure}

\begin{itemize}
  \item quartzite
  \item dimensions : \SI{29}{\cm}
\end{itemize}

\separation

Il manque la tête, le bras droit, les jambes. Il n'y a aucune 
inscription, pas de nom.
Mais les épaules sont bien visibles : pas de retombées de coiffure, 
pas la mèche caractéristiques des princesses.

L'épaule droite présente des traces de restauration antique 
(traces vertes) et le bras droit, contrairement au gauche, n'était 
manifestement pas le long du corps (pas de traces de réserve de pierre).

\quad\donc~il était certainement levé, avec un sistre dans la 
main \rem{(cf. statue d'Akhénaton à Berlin)}.

Rare à cette époque ! Contrairement à ce qui se faisait à l'\OK, les 
égyptiens du \NK rechignent à décoller les membres du corps sur les 
statues en pierre.


\chapter{Le dieu Amon protège Toutânkhamon (\inv{E\,11609})}
\label{sec:E11609}
%======================================================================

\puceb{} \emph{voir \fref{fig:E11609}, \pref{fig:E11609}}
\bigskip

\begin{figure}[!h]
  \noi\begin{minipage}[m]{0.33\textwidth}
    \centerfloat
    \includegraphics[height=7cm]{E11609_1}
    \subcaption{Vue de trois-quarts\label{fig:E11609_1}}
  \end{minipage}%
  \hspace{\fill}%
  \begin{minipage}[m]{0.33\textwidth}
    \centerfloat
    \includegraphics[height=7cm]{E11609_2}
    \subcaption{Vue de dos\label{fig:E11609_2}}
  \end{minipage}%
  \hspace{\fill}%
  \begin{minipage}[m]{0.33\textwidth}
    \centerfloat
    \includegraphics[height=7cm]{E11609_3}
    \subcaption{Vue de face\label{fig:E11609_3}}
  \end{minipage}
  \caption{Amon protège Toutânkhamon \rem{(\inv{E\,11609})}}
  \label{fig:E11609}
\end{figure}

\begin{itemize}
  \item \datation{\anorange{1336}{1327}}
  \item diorite
  \item dimensions : \SI{2.20x0.44x0.78}{\m}
\end{itemize}

La tête, les bras et le nom du roi ont été volontairement détruits.

\separation

Typiquement post-amarnien : ventre, visage très doux.

Toutânkhamon est représenté entre les jambes d'Amon, immobile. Il a 
les bras le long du corps, devant lui, les mains sur le devanteau du 
pagne : position de prière.

Il est vêtu de la peau de léopard du prêtre \emph{sem} (fête d'Opet). 

La tête et les bras du roi sont cassés. Les avant-bras d'Amon 
également. Et les cartouches ont été martelés (à l'exception des 
noms de Rê et Amon). Il s'agissait de détruire l'identité du roi et 
d'empêcher le dieu de lui apporter sa protection. Mais le travail a 
été baclé ! Il reste, à droite du pagne, deux cartouches intacts.

Cette statue est un bel exemple d'antéposition honorifique et d'aspective \rem{(\og multiplicité des points de vue \fg, voir 
aussi \inv{A\,65}, \Cref{sec:A65} \pref{sec:A65} ; à comparer avec 
le \inv{E\,11005}, \Cref{sec:E11005} \pref{sec:E11005})} : il faut comprendre que le roi fait en fait face au dieu.

La différence de taille entre le roi et le dieu reflète le rapport 
hiérarchique qui existe entre eux.


\chapter{Piaÿ, portier du palais (\inv{E\,124})}
\label{sec:E124}
%======================================================================

\puceb{} \emph{voir \fref{fig:E124}, \pref{fig:E124}}
\bigskip

\begin{figure}[!h]
  \noi\begin{minipage}[m]{0.50\textwidth}
    \centerfloat
    \includegraphics[height=7cm]{E124_1}
    \subcaption{Vue de face\label{fig:E124_1}}
  \end{minipage}%
  \hspace{\fill}%
  \begin{minipage}[m]{0.50\textwidth}
    \centerfloat
    \includegraphics[height=7cm]{E124_2}
    \subcaption{Vue de dos\label{fig:E124_2}}
  \end{minipage}%

  \noi\begin{minipage}[m]{0.50\textwidth}
    \centerfloat
    \includegraphics[height=7cm]{E124_3}
    \subcaption{Profil gauche\label{fig:E124_3}}
  \end{minipage}%
  \hspace{\fill}%
  \begin{minipage}[m]{0.50\textwidth}
    \centerfloat
    \includegraphics[height=7cm]{E124_4}
    \subcaption{Vue de trois-quarts\label{fig:E124_4}}
  \end{minipage}%
  \caption{Piaÿ, portier du palais \rem{(\inv{E\,124})}}
  \label{fig:E124}
\end{figure}

\begin{itemize}
  \item \datation{vers \ano{1300}}
  \item bois de karité, socle en acacia
  \item dimensions : \SI{54.50x10.90x31.00}{\cm}
\end{itemize}

\separation

Typiquement post-amarnien : ventre, nombreux plis sur le vêtement, et 
le devanteau du pagne, qui remonte très haut dans le dos.

Texte sur le pilier dorsal : formule d'offrande avec épithètes d'Amon.

\quad\donc~couleur (\tl[black]{jnm}) $\equiv$ lumière !

\begin{remarque}
  cf. le \og faucon au plumage multicolore \fg
\end{remarque}

Les lignes de main sont visibles sur la paume droite !


\chapter{Statue du dieu Amon (\inv{E\,10377})}
\label{sec:E10377}
%======================================================================

\puceb{} \emph{voir \fref{fig:E10377}, \pref{fig:E10377}}
\bigskip

\begin{figure}[!h]
  \includegraphics[width=4cm]{E10377}
  \caption{Amon \rem{(\inv{E\,10377})}}
  \label{fig:E10377}
\end{figure}

\begin{itemize}
  \item règne de Toutânkhamon (\datation{\anorange{1336}{1327}})
  \item diorite
  \item dimensions : \SI{72.00x31.30x28.00}{\cm}
\end{itemize}

\separation

Basalte ?

Cassures extrêmement propres, c'est une \og statue en kit à 
monter soi-même \fg ;)

\begin{itemize}
  \item mortaise pour les plumes
  \item appui dorsal qui n'en est pas un, pour encastrer dans 
        une plaque plus grande
  \item \dots
\end{itemize}

\begin{remarque}
  cf. Mélanges \uwave{\bsc{Bonnet}}
\end{remarque}


\chapter{Dieux de Thèbes : Amon et Mout (\inv{N\,3566})}
\label{sec:N3566}
%======================================================================

\puceb{} \emph{voir \fref{fig:N3566}, \pref{fig:N3566}}
\bigskip

\begin{figure}[!h]
  \noi\begin{minipage}[t]{0.33\textwidth}%
    \centerfloat
    \includegraphics[height=7cm]{N3566_1}%
    \subcaption{Vue de face\label{fig:N3566_1}}%
  \end{minipage}%
  \quad%
  \begin{minipage}[t]{0.33\textwidth}%
    \centerfloat
    \includegraphics[height=7cm]{N3566_2}%
    \subcaption{Vue de trois-quarts\label{fig:N3566_2}}%
  \end{minipage}%
  \quad%
  \begin{minipage}[t]{0.33\textwidth}%
    \centerfloat
    \includegraphics[height=7cm]{N3566_3}%
    \subcaption{Vue de dos\label{fig:N3566_3}}%
  \end{minipage}%
  \caption{Amon et Mout \rem{(\inv{N\,3566})}}%
  \label{fig:N3566}
\end{figure}

\begin{itemize}
  \item grauwacke
  \item dimensions : \SI{16.00x8.00x7.50}{\cm}
\end{itemize}

Statue dédiée par Mérymaât, préposé à la balance.

\begin{encadre}[font=\em, frametitle={Traduction}]
  \noi\begin{itemize}
    \item Mout la grande, dame de l'Achérou, Ouadjet, dame de Karnak, 
          Sekhmet la grande, oeil de Rê, souveraine du Sud et du Nord, 
          qui repousse le mal, chasse le trouble, apporte 
          l'anéantissement à qui (l')attaque en sa ville. Comme il est 
          prospère son protégé ! Le mal ne l'atteindra pas.
    \item Amon-Rê, seigneur des trônes des Deux-Terres, seigneur du 
          ciel, roi des dieux, au grand prestige dans l'Héliopolis du 
          sud (Thèbes), au parfait visage dans l'Héliopolis du nord, 
          dieu vivant issu de l'océan primordial, qui illumine les 
          Deux-Terres de ses rayons, qui rajeunit de lui-même, en 
          perfection, qui crée la semence des hommes et des dieux.
    \item Le préposé à la balance du seigneur des Deux-Terres, 
          Mérymaât, acquitté.
  \end{itemize}
\end{encadre}

\separation

Stéatite ?

Statue votive.

Il y a un trou sous le socle, pour fixer la statue sur un support.

Le bouquet de fleur à l'avant est offert à Amon et Mout. Celui sur 
le côté gauche est offert, en même temps qu'Amon et Mout, aux dieux 
de Karnak.

Le texte associé à Amon comporte ses épithètes, il est assez générique.

Le texte associé à Mout, en revanche, est offensif, guerrier, et 
comporte une malédiction à l'encontre d'Akhénaton, \og qui l'attaque 
en sa ville \fg. Elle est le bras armée d'Amon. C'est une déesse 
dangereuse, au même titre que Sekhmet ou Ouadjet.


\chapter{Dieux de Memphis et de sa nécropole : Osiris, Sekhmet, 
         Hathor, Apis, Ptah, Néfertoum, Anubis (\inv{A\,64})}
\label{sec:A64}
%======================================================================

\puceb{} \emph{voir \fref{fig:A64}, \pref{fig:A64}}
\bigskip

\begin{figure}[!h]
  \includegraphics[height=6cm]{A64_1}%
  \quad%
  \includegraphics[height=6cm]{A64_2}%
  \caption{Dieux de Memphis et de sa nécropole \rem{(\inv{A\,64})}}
  \label{fig:A64}
\end{figure}

\begin{itemize}
  \item quartzite
  \item dimensions : \SI{76.50x48.00x34.00}{\cm}
\end{itemize}

\og Statue \fg au nom de Mériounou, intendant de Memphis.

\emph{\dots il dit : \og ô vous tous les prêtres et scribes du 
temple d'Osiris qui lirez cette stèle érigée en l'honneur du Maître 
de l'éternité, puissiez-vous réciter la formule d'offrande funéraire 
et verser de l'eau sur le sol, pour le chef des dessinateurs d'Amon, 
Dédia, et pour son épouse ! Alors le dieu Ounnéfer vous récompensera, 
vous pourrez transmettre vos charges à vos enfants après une longue 
vieillesse et vous bénéficierez de ce qui a été offert aux dieux Amon, 
Mout et Khonsou \dots \fg}

\separation

Substitut de naos avec toutes les divinités.

Le défunt est représenté deux fois, c'est la \og \uwave{définition 
par accumulation} \fg.


\chapter{Khâ, scribe de la table d'offrandes du roi (\inv{A\,65})}
\label{sec:A65}
%======================================================================

\puceb{} \emph{voir \fref{fig:A65}, \pref{fig:A65}}
\bigskip

\begin{figure}[!h]
  \noi\begin{minipage}[m]{4cm}
    \centerfloat
    \includegraphics[width=\textwidth]{A65_1}
    \subcaption{Vue de face\label{fig:A65_1}}
  \end{minipage}%
  \qquad%
  \begin{minipage}[m]{4cm}
    \centerfloat
    \includegraphics[width=\textwidth]{A65_2}
    \subcaption{Vue de trois-quarts dos\label{fig:A65_2}}
  \end{minipage}%
  \qquad%
  \begin{minipage}[m]{4cm}
    \centerfloat
    \includegraphics[width=\textwidth]{A65_4}
    \subcaption{Profil gauche\label{fig:A65_4}}
  \end{minipage}

  \bigskip

  \noi\begin{minipage}[m]{4cm}
    \centerfloat
    \includegraphics[width=\textwidth]{A65_3}
    \subcaption{Détail du naos\label{fig:A65_3}}
  \end{minipage}%
  \qquad%
  \begin{minipage}[m]{4cm}
    \centerfloat
    \includegraphics[width=\textwidth]{A65_5}
    \subcaption{Vue de dessus\label{fig:A65_5}}
  \end{minipage}%
  \caption{Khâ, scribe de la table d'offrandes du roi 
           \rem{(\inv{A\,65})}}
  \label{fig:A65}
\end{figure}

\begin{itemize}
  \item règne de Ramsès~II (\datation{\anorange{1279}{1213}}), 
        \dyn{xix}
  \item quartzite
  \item dimensions : \SI{63.50x23.80x37.00}{\cm}
\end{itemize}

Devant lui, une chapelle du dieu Thot, patron des scribes.

\separation

Provient probablement de Saqqara.

Statue cube et naophore ! Avec le babouin de Thot et une table 
d'offrande en vue de dessus.

Khâ est mentionné dans le testament de Mès (document juridique).

Là encore, illustration de l'aspectivité \rem{(voir E\,11609, 
p.\,\pageref{sec:E11609})} : le babouin fait en fait face à la table 
d'offrande et l'ensemble fait face au défunt.

C'est une statue d'attente : le défunt est immobile, il donne les 
offrandes au dieu et attend sa récompense en échange.


\chapter{La déesse Nephthys, soeur d'Isis et d'Osiris (\inv{E\,25389})}
\label{sec:E25389}
%======================================================================

\puceb{} \emph{voir \fref{fig:E25389}, \pref{fig:E25389}}
\bigskip

\begin{figure}[!h]
  \includegraphics[height=6cm]{E25389_1}%
  \qquad%
  \includegraphics[height=6cm]{E25389_2}%
  \caption{La déesse Nephthys \rem{(\inv{E\,25389})}}
  \label{fig:E25389}
\end{figure}

\begin{itemize}
  \item règne d'Aménophis~III (\datation{\anorange{1391}{1353}}), 
        \dyn{xviii}
  \item diorite
  \item dimensions : \SI{1.81x0.35x69.50}{\m}
\end{itemize}

\separation

Poli très brillant, typique de l'époque.

Le nom d'Amon dans le cartouche d'Aménophis et l'épithète de Nephthys 
ont été martelés. Seule l'épithète a été regravée.

\quad\donc~La re-gravure de l'épithète divine date du règne 
d'Aménophis~III lui-même. La statue, à l'origine dédiée à la fête 
\emph{sed} (jubilé), a été ensuite réaffectée (dans le temple de 
?\nospace{??}), et le texte a été réactualisé pour refléter cela 
\rem{(voir aussi \inv{A\,20}, \Cref{sec:A20} \pref{sec:A20})}.

\quad\donc~Le nom d'Amon, en revanche, a subi un effacement 
politique pendant la période amarnienne. On n'a pas pris soin de 
remédier à cela par la suite, certainement parce que la statue n'avait 
plus d'utilité.


\chapter{Le roi Ramsès~II (\inv{A\,20})}
\label{sec:A20}
%======================================================================

\puceb{} \emph{voir \fref{fig:A20}, \pref{fig:A20}}
\bigskip

\begin{figure}[!h]
  \includegraphics[width=4cm]{A20_1}%
  \qquad%
  \includegraphics[width=4cm]{A20_2}%
  \caption{Ramsès~II \rem{(\inv{A\,20})}}
  \label{fig:A20}
\end{figure}

\begin{itemize}
  \item \datation{\anorange{1279}{1213}} (\dyn{xix})
  \item statue trouvée à Tanis
  \item diorite
  \item dimensions : \SI{2.59x0.80x1.20}{\m}
\end{itemize}

\separation

Plusieurs regravures :
\begin{itemize}
  \item Un \tl[black]{smA-\&A.wy} a été remplacé par un texte relatif 
        au jubilé sur le côté du siège ;
  \item le némès et la ceinture ont un aspect \og grumeleux \fg{};
  \item \dots
\end{itemize}

\quad\donc~On a supposé que Ramsès avait usurpé cette statue. 
Elle a été attribuée à à peu près tous les rois possibles, notamment 
Aménophis~III.

En fait, il s'agit probablement d'une statue inachevée 
d'Aménophis~III, terminée par Ramsès. L'aspect grumeleux du némès est 
lié à l'application d'une dorure, il y a des reprises d'accidents. 
Quant au remplacement du \tl[black]{smA-\&A.wy}, il s'agit là encore, 
d'un changement de fonction en cours de règne \rem{(voir 
\inv{E\,25389}, \Cref{sec:E25389} \pref{sec:E25389})} : la statue a 
été ré-utilisée par Ramsès pour les fêtes du jubilé.

Pour une fois, ce n'est pas une usurpation !


\chapter{Statue du dieu Amon dédiée par le roi Toutânkhamon 
         (\inv{E\,11005})}
\label{sec:E11005}
%======================================================================

\puceb{} \emph{voir \fref{fig:E11005}, \pref{fig:E11005}}
\bigskip

\begin{figure}[!h]
  \includegraphics[height=8cm]{E11005_1}%
  \qquad%
  \includegraphics[height=8cm]{E11005_2}%
  \caption{Statue d'Amon dédiée par Toutânkhamon 
           \rem{(\inv{E\,11005})}}
  \label{fig:E11005}
\end{figure}

\begin{itemize}
  \item \datation{\anorange{1336}{1327}}, \dyn{xviii}
  \item diorite
  \item dimensions : \SI{1.11x0.29x0.70}{\m}
\end{itemize}

Le roi était figuré en petite taille aux pieds du dieu.

\separation

Même type de représentation que le \inv{E\,11609}, \Cref{sec:E11609}
\pref{sec:E11609}, à ceci près que le roi -- dont il ne reste que les 
talons -- fait réellement face à Amon.

Le dieu tient dans sa main gauche un \emph{ankh} et dans sa main 
droite une tige de millions d'années. Il s'agit de la représentation 
en ronde-bosse d'une scène très fréquente en relief. 

\tl[black]{Dd-mdw} sur le pilier dorsal : très rare !


\chapter{Le dieu-bélier Khnoum (\inv{AF\,2577})}
\label{sec:AF2577}
%======================================================================

\puceb{} \emph{voir \fref{fig:AF2577}, \pref{fig:AF2577}}
\bigskip

\begin{figure}[!h]
  \includegraphics[height=8cm]{AF2577}
  \caption{Le dieu-bélier Khnoum \rem{(\inv{AF\,2577})}}
  \label{fig:AF2577}
\end{figure}


\begin{itemize}
  \item quartzite
  \item dimensions : \SI{1.26x0.35x0.99}{\m}
\end{itemize}

Sur le socle, traces d'un roi agenouillé.

Le museau est restauré.

\separation

\begin{itemize}
  \item Pas de \no d'inventaire historique \donc~d'où vient 
        cette statue ?
  \item Couleur solaire
  \item Amon-Rê ?
  \item museau, disque solaire cassés \donc~détruire l'identité du dieu
  \item avant-bras cassés \donc~empêcher le dieu d'agir
  \item le fauteuil du dieu repose sur un socle en biseau 
        \begin{hieroglyph}{\leavevmode \loneSign{\Aca GAa/41/}}\end{hieroglyph} 
        \donc~représentation de Maât
\end{itemize}


\chapter{Stèle du chef des artisans, scribe et sculpteur Irtysen 
         (\inv{C\,14})}
\label{sec:C14}
%======================================================================

\puceb{} \emph{voir \fref{fig:C14}, \pref{fig:C14}}
\bigskip

\begin{figure}[!h]
  \noi\begin{minipage}[m]{6cm}
    \centerfloat
    \includegraphics[height=9cm]{C14_1}
  \end{minipage}%
  \qquad%
  \begin{minipage}[m]{6cm}
    \centerfloat
    \includegraphics[width=\textwidth]{C14_2}

    \bigskip

    \includegraphics[width=\textwidth]{C14_3}
  \end{minipage}
  \caption{Stèle d'Irtysen \rem{(\inv{C\,14})}}
  \label{fig:C14}
\end{figure}

\begin{itemize}
  \item règne de Nebhépetrê Montouhotep, 
        \datation{\anorange{2033}{1982}}, \dyn{xi}
  \item calcaire
  \item dimensions : \SI{1.17x0.56}{\m}
\end{itemize}

Sur sa stèle, Irtysen se vante des prouesses techniques qu'il est 
capable d'accomplir. C'est un des très rares textes égyptiens qui 
parlent d'art.

\begin{encadre}[font=\em, frametitle={Traduction partielle 
                \autocite[doc.~\num{19}]{ChB}}]
  Le directeur des ouvriers, le scribe sculpteur (?) Irtysen (qui) 
  déclare : 

  \og Je connais \lgn{7} le secret des hiéroglyphes et les directives 
  des rituels de fête. Toute espèce de magie, je l'ai acquise sans que 
  rien ne m'en échappe. \lgn{8} Je suis donc un artisan expert en son 
  art, le premier entre tous par ce que je connais. 
  Je connais les formules de la solidification \rem{(du métal en 
  fusion ?)}, \lgn{9} la pesée \rem{(du métal ?)} selon le compte 
  exact, (le fait d')enlever et d'ajouter \rem{(le métal ?)} lorsqu'il 
  sort et lorsqu'il entre \rem{(dans le moule ?)} jusqu'à ce que le 
  membre \rem{(de la statue de métal ?)} prenne sa place. Je connais 
  le pas de la statue \lgn{10} (d'homme), la démarche de la statue de 
  femme, l'attitude des onze oiseaux (?), la posture du prisonnier 
  isolé, le strabisme des yeux, l'expression de la terreur chez les 
  vaincus, \lgn{11} le port de bras du harponneur d'hippopotame, et 
  la démarche de la course. Je sais faire ce qui est en argile (?) 
  et les choses \lgn{12} qui descendent (?) sans laisser la flamme 
  les consumer, sans qu'ils soient lavés à l'eau, assurément !
  \lgn{13} Personne ne les divulguera \rem{(ces secrets)}, excepté 
  moi seul et mon propre fils aîné car le dieu a déclaré qu'il agirait
  \lgn{14} et que (je) les lui transmettrai. J'ai constaté son 
  efficacité dans le rôle de directeur des travaux en matière de 
  toutes sortes de pierres dures et vénérables, depuis l'argent et 
  l'or \lgn{15} jusqu'à l'ivoire et l'ébène. \fg
\end{encadre}


\separation

Les pieds des porteurs d'offrandes avaient été attachés pour qu'ils 
ne puissent pas se sauver de la stèle en emportant les victuailles !


%=============================================

\backmatter
\newpage
\listoffigures
% \listoftables
% \nocite{ChB}
\printbibliography

%%%%%%%%%%%%%%%%%%%%%%%%%%%%%%%%%%%%%%%%%%%%%%%%%%%%%%%%%%%%%%%%%%%%%%%%
\end{document}
